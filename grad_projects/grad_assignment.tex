%!TEX TS-program = lualatex
%!TEX encoding = UTF-8 Unicode

\documentclass[12pt]{article}
%\usepackage{graphicx}
%	\graphicspath{{/Users/goby/Pictures/teach/153/lab/}} % set of paths to search for images

\usepackage{geometry}
\geometry{letterpaper}                   
\geometry{bottom=1in, left=1.5in}
%\geometry{landscape}                % Activate for for rotated page geometry
%\usepackage[parfill]{parskip}    % Activate to begin paragraphs with an empty line rather than an indent
%\usepackage{amssymb}
%\usepackage{mathtools}
%	\everymath{\displaystyle}

%\pagenumbering{gobble}

\usepackage{fontspec}
\setmainfont[Ligatures={Common,TeX}, BoldFont={* Bold}, ItalicFont={* Italic}, Numbers={Proportional}]{Linux Libertine O}
\setsansfont[Scale=MatchLowercase,Ligatures=TeX, Numbers=OldStyle]{Linux Biolinum O}
%\setmonofont[Scale=MatchLowercase]{Inconsolata}
\usepackage{microtype}

\usepackage{unicode-math}
\setmathfont[Scale=MatchLowercase]{Asana-Math.otf}
%\setmathfont{XITS Math}

% To define fonts for particular uses within a document. For example, 
% This sets the Libertine font to use tabular number format for tables.
%\newfontfamily{\tablenumbers}[Numbers={Monospaced}]{Linux Libertine O}
%\newfontfamily{\libertinedisplay}{Linux Libertine Display O}


%\usepackage{booktabs}
%\usepackage{multicol}
%\usepackage{tabularx}
%\usepackage{longtable}
%\usepackage{siunitx}
%\usepackage[justification=raggedright, singlelinecheck=off]{caption}
%\captionsetup{labelsep=period} % Removes colon following figure / table number.
%\captionsetup{tablewithin=none}  % Sequential numbering of tables and figures instead of
%\captionsetup{figurewithin=none} % resetting numbers within each chapter (Intro, M&M, etc.)
%\captionsetup[table]{skip=0pt}

\usepackage{array}
\newcolumntype{L}[1]{>{\raggedright\let\newline\\\arraybackslash\hspace{0pt}}p{#1}}
\newcolumntype{C}[1]{>{\centering\let\newline\\\arraybackslash\hspace{0pt}}p{#1}}
\newcolumntype{R}[1]{>{\raggedleft\let\newline\\\arraybackslash\hspace{0pt}}p{#1}}

\usepackage{enumitem}
%\usepackage{hyperref}
%\usepackage{placeins} %PRovides \FloatBarrier to flush all floats before a certain point.
%\usepackage{hanging}
%\usepackage{color}
%\usepackage{calc}

%\usepackage{titling}
%\setlength{\droptitle}{-60pt}
%\posttitle{\par\end{center}}
%\predate{}\postdate{}

\usepackage[sc]{titlesec}


\usepackage{fancyhdr}
\setlength{\headheight}{14.5pt}
\fancyhf{}
\pagestyle{fancy}
\lhead{}
\chead{}
\rhead{\footnotesize pg. \thepage }
\renewcommand{\headrulewidth}{0.4pt}

\fancypagestyle{plain}{%
	\fancyhf{}
	\lhead{\textsc{bi} 638: Biogeography}
	\rhead{Annotated Bibliography}
	\renewcommand{\headrulewidth}{0pt}
}

%\newcommand{\VSpace}{\vspace{0.5\baselineskip}}
%\newcommand{\BigVSpace}{\vspace{2\baselineskip}}

\title{Annotated Bibliography}
\author{Biogeography}
\date{}                                           % Activate to display a given date or no date

\begin{document}
%\maketitle
\thispagestyle{plain}



\subsection*{Annotated Bibliography}

You are to write an annotated bibliography of the recent biogeographic literature.  A list of ``The Frontiers'' of biogeography is provided on the next page.  I have dropped a couple of topics and added a couple of topics. You may choose any topic from the list but the choices are first come, first serve. Your goal is to find eight peer-reviewed scientific publications related to your topic.  The papers must be from 2018 or later.  You will upload \textsc{pdf} files of the eight papers to the drop box. In addition, you must write a two paragraph synopsis of \textit{each} paper that you choose. Upload your final annotated bibliography to the drop box also.

For the topic that you choose, read about that topic on the provided handout for ideas. Use Google Scholar (http://scholar.google.com) to search for papers. The left side of the search results page will allow you to quickly limit your results to 2018 or more recent.  I'd really like 2020 or later, if the papers are there.

Look for papers with the following qualities:

\begin{itemize}

\item Information that you feel would be valuable to include in lectures.

\item Papers with accessible data that I might be able to turn into a future analysis exercise.

\item Papers that test some of the concepts we discussed in class.

\end{itemize}

I am willing to consider another topic if you have an idea for something that I may be interested in. However, it cannot be about the biogeography of a particular group (e.g., biogeography of fishes). I am interested in broad, biogeographical concepts.  If you have an idea, do not hesitate to ask. It would be a good idea if you send along a paper or two showing me what you have in mind, along with an explanation of why you would like to do that topic.

\subsection*{Due Date}

This assignment is due on Friday, 06 December, 11:59 pm.

\newpage

``Frontier'' topics: topics listed on the handout but \emph{not} listed below are not allowed as a choice.

\bigskip

Scale of space and time

\bigskip

The comparative approach

\bigskip

Integrations in biogeography

\bigskip

Biogeography of elusive and novel biotas

\bigskip

Evolution

\bigskip

Dispersal and immigration

\bigskip

Extinction

\bigskip

The geography of humanity

\bigskip

Conservation biogeography

\bigskip


\end{document}  