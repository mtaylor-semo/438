%!TEX TS-program = lualatex
%!TEX encoding = UTF-8 Unicode

%\documentclass[t]{beamer}

%%%% HANDOUTS For online Uncomment the following four lines for handout
%\documentclass[t,handout]{beamer}  %Use this for handouts.
%\usepackage{handoutWithNotes}
%\pgfpagesuselayout{3 on 1 with notes}[letterpaper,border shrink=5mm]

%\includeonlylecture{student}

%% For students, use \lecture{student}{student}
%% For mine, use \lecture{instructor}{instructor}


% FONTS
\usepackage{fontspec}
\def\mainfont{Linux Biolinum O}
\setmainfont[Ligatures={Common,TeX}, Contextuals={NoAlternate}, BoldFont={* Bold}, ItalicFont={* Italic}, Numbers={Proportional}]{\mainfont}
\setmonofont[Scale=0.82]{Linux Libertine Mono O} 
\setsansfont[Scale=MatchLowercase]{Linux Biolinum O} 
\usepackage{microtype}

\usepackage{graphicx}
	\graphicspath{%
	{/Users/goby/Pictures/teach/438/lectures/}%
	{/Users/goby/Pictures/teach/common/}} % set of paths to search for images

\usepackage{amsmath,amssymb}

%\usepackage{units}

\usepackage{booktabs}
\usepackage{multicol}
%	\setlength{\columnsep=1em}

\usepackage{textcomp}
\usepackage{setspace}
\usepackage{tikz}
	\tikzstyle{every picture}+=[remember picture,overlay]

\mode<presentation>
{
  \usetheme{Lecture}
  \setbeamercovered{invisible}
  \setbeamertemplate{items}[square]
}

\usepackage{calc}
\usepackage{hyperref}

\newcommand\HiddenWord[1]{%
	\alt<handout>{\rule{\widthof{#1}}{\fboxrule}}{#1}%
}



\begin{document}
%\lecture{instructor}{instructor}
%\lecture{student}{student}

\lecture{student}{student}
\begin{frame}[t,plain]{The \highlight{geographic template} explains biogeographic patterns.}
	\begin{center}
		\includegraphics[width=0.95\textwidth]{geographic_template}
	\end{center}
\end{frame}

\begin{frame}[t,plain]
	\hangpara What climate factors affect distribution of terrestrial organisms on a global scale?

	\hangpara Would these same factors affect freshwater organisms?

	\hangpara Would these same factors affect marine organisms?
\end{frame}


\begin{frame}[t,plain]{What patterns do you observe for climate variation?}

	\vspace{-0.5\baselineskip}
	\begin{center}
		\includegraphics[width=0.9\textwidth]{climate_global_regimes}
	\end{center}
	
	\vspace{-1.5\baselineskip}
	\hangpara Do rainforests or deserts occur at specific latitudes?\\Are mountain ranges (gray) related to climate zones?\\Do you find other latitudinal or east vs west side of continents?
	
\end{frame}

\begin{frame}[t,plain]{Predict the relationship between precipitation and latitude.}

	\begin{center}
		\includegraphics[width=0.9\textwidth]{climate_precipitation_latitude_blank}
	\end{center}

\end{frame}

\lecture{instructor}{instructor}
\begin{frame}[t,plain]

	\begin{center}
		\includegraphics[width=0.9\textwidth]{climate_precipitation_latitude}
	\end{center}

\end{frame}


\lecture{student}{student}
\begin{frame}[t,plain]{What patterns do you observe for climate variation?}

	\begin{center}
		\includegraphics[width=0.9\textwidth]{climate_global_regimes}
	\end{center}
	
\end{frame}


\begin{frame}[t,plain]{The \highlight{geographic template} regulates all biotic distributions.}
	\hangpara The spatial variation of environmental factors is non-random and predictable.
	
	\hangpara Geographic gradients are strongly correlated with climate variation.
	
	\hangpara What determines the patterns of climate variability?
\end{frame}


\begin{frame}[t,plain]{Solar radiation is the primary forcer of climate.}
	\begin{minipage}{0.6\textwidth}%
		\includegraphics[width=0.95\textwidth]{climate_solar_radiation}
	\end{minipage}
%	\hfill
	\begin{minipage}{0.35\textwidth}%
		\vspace{-4\baselineskip}
		\hangpara Intensity decreases with higher latitudes.
		
		\hangpara Intensity is modified by atmosphere.

	\end{minipage}
\end{frame}


\begin{frame}[t,plain]{Solar input is strongest at the equator.}
	
	\begin{center}
		\includegraphics[width=0.95\textwidth]{climate_equinox_solstice}
	\end{center}
		
\end{frame}

\begin{frame}[t,plain]{Solar input at the equator creates global wind patterns.}
	\begin{minipage}{0.55\textwidth}%
		\includegraphics[width=0.95\textwidth]{climate_hadley_cells}
	\end{minipage}
%	\hfill
	\begin{minipage}{0.40\textwidth}%
		\vspace{-4\baselineskip}
		\hangpara Vertical circulation creates latitudinal gradient.
		
		\hangpara Horizontal deflection due to Coriolis effect.

	\end{minipage}
\end{frame}

{
\usebackgroundtemplate{\includegraphics[width=\paperwidth]{climate_global_wind_circulation1}}
\begin{frame}[t,plain]{Solar input creates global precipitation patterns.}
\end{frame}
}

{
\usebackgroundtemplate{\includegraphics[width=\paperwidth]{climate_pacific_northwest_novalues}}
\begin{frame}[t,plain]
\end{frame}
}

\lecture{instructor}{instructor}
{
\usebackgroundtemplate{\includegraphics[width=\paperwidth]{climate_pacific_northwest_values}}
\begin{frame}[t,plain]
\end{frame}
}

\lecture{student}{student}
{
\usebackgroundtemplate{\includegraphics[width=\paperwidth]{climate_precip_gradient}}
\begin{frame}[t,plain]{Why is precipitation so different west and east of the Cascade Mountains?}
\end{frame}
}


\begin{frame}[t,plain]{Mountain ranges can create \highlight{rain shadows.}}
	\begin{center}
		\includegraphics[width=\textwidth]{climate_rain_shadow}
	\end{center}
\end{frame}

{
\usebackgroundtemplate{\includegraphics[width=\paperwidth]{climate_precip_gradient}}
\begin{frame}[t,plain]{Why does precipitation increase east of the Rocky Mountains?}
\end{frame}
}

{
	\usebackgroundtemplate{\includegraphics[width=\paperwidth]{hurricane_storm_tracks}}
	\begin{frame}[t]
		
		\vfilll
		
		\hfill \tiny \copyright~\href{http://www.openhazards.com/blogs/wardsn/hurricane-wind-forecasting-1-what-and-how}{OpenHazards.com}
	\end{frame}
}

\begin{frame}[t,plain]{To summarize, the geographic template}

	\hangpara is the \highlight{non-random spatial variation of climate variables} due to global gradients and geographic features,
	
	\hangpara forms the basis for biotic distributions, and
	
	\hangpara operates at global, regional and local scales.
\end{frame}


\end{document}
