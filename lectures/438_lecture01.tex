%!TEX TS-program = lualatex
%!TEX encoding = UTF-8 Unicode

\documentclass[t]{beamer}

%%%% HANDOUTS For online Uncomment the following four lines for handout
%\documentclass[t,handout]{beamer}  %Use this for handouts.
%\usepackage{handoutWithNotes}
%\pgfpagesuselayout{3 on 1 with notes}[letterpaper,border shrink=5mm]
%	\setbeamercolor{background canvas}{bg=black!5}


%%% Including only some slides for students.
%%% Uncomment the following line. For the slides,
%%% use the labels shown below the command.
%\includeonlylecture{student}

%% For students, use \lecture{student}{student}
%% For mine, use \lecture{instructor}{instructor}


%\usepackage{pgf,pgfpages}
%\pgfpagesuselayout{4 on 1}[letterpaper,border shrink=5mm]

% FONTS
\usepackage{fontspec}
\def\mainfont{Linux Biolinum O}
\setmainfont[Ligatures={Common,TeX}, Contextuals={NoAlternate}, BoldFont={* Bold}, ItalicFont={* Italic}, Numbers={Proportional}]{\mainfont}
\setmonofont[Scale=MatchLowercase]{Inconsolata} 
\setsansfont[Scale=MatchLowercase]{Linux Biolinum O} 
\usepackage{microtype}

\usepackage{graphicx}
	\graphicspath{%
	{/Users/goby/Pictures/teach/438/lectures/}%
	{/Users/goby/Pictures/teach/common/}} % set of paths to search for images

\usepackage{amsmath,amssymb}

%\usepackage{units}

\usepackage{booktabs}
\usepackage{multicol}
%	\setlength{\columnsep=1em}

\usepackage{textcomp}
\usepackage{setspace}
\usepackage{tikz}
	\tikzstyle{every picture}+=[remember picture,overlay]

\mode<presentation>
{
  \usetheme{Lecture}
  \setbeamercovered{invisible}
  \setbeamertemplate{items}[square]
}

\usepackage{calc}
\usepackage{hyperref}

\newcommand\HiddenWord[1]{%
	\alt<handout>{\rule{\widthof{#1}}{\fboxrule}}{#1}%
}



\begin{document}
%\lecture{instructor}{instructor}
%\lecture{student}{student}


{
\usebackgroundtemplate{\includegraphics[width=\paperwidth]{tetons_landscape}}
\begin{frame}[b,plain]{\textcolor{white}{BI 438 / 638: Biogeography}}

\end{frame}
}

%% Contact Info
{
\usebackgroundtemplate{\includegraphics[width=\paperwidth]{mike_snake.jpg}}
\begin{frame}[t,plain]
\large
\vspace{5ex}
\hangpara\hspace{17em} Mike Taylor

\hangpara\hspace{17em} RH 217

\hangpara\hspace{17em} mtaylor@semo.edu

\end{frame}
}

%% Grades
\begin{frame}[t,plain]{You \highlight{earn} your grade with }

	\hangpara 3 exams @ 75--100 points, 

	\hangpara 3--5 critical analyses @ 30 points, 

		\hspace{2em} Grads / honors will lead a discussion,

	\hangpara Pre- and post-tests,
	
	\hangpara Data analyses and assignments, and
	
	\hangpara Grad / honors project and presentation.
	
\end{frame}

%% Software
{
\usebackgroundtemplate{\includegraphics[width=\paperwidth]{software}}
\begin{frame}[t,plain]{You will complete many analyses.}

	\hangpara You may bring a laptop.
	
	\hangpara Install free \href{https://www.r-project.org/}{R software}.
	
	\hangpara Analysis of fishes, crayfishes, and mussels.

\end{frame}
}


%% Textbook
\begin{frame}[t,plain]{The textbook is \highlight{required} for this course.}
	\begin{center}
		\includegraphics[height=0.7\textheight]{textbook}
	\end{center}
\end{frame}


%% First assignment
{
\usebackgroundtemplate{\includegraphics[width=\paperwidth]{first_assignment}}
\begin{frame}[t,plain]{Here is your first assignment. Maybe.}

	\hangpara You'll build a presence/absence data matrix for the fishes of Georgia.

	\hangpara \url{http://mtaylor4.semo.edu/~goby/biogeo/}
	
	\hangpara \url{http://fishesofgeorgia.uga.edu}
	
\end{frame}
}

%% What is biogeography?
{
\usebackgroundtemplate{\includegraphics[width=\paperwidth]{tetons_landscape}}
\begin{frame}[b,plain]{\textcolor{white}{What is biogeography?}}

\end{frame}
}


{
\usebackgroundtemplate{\includegraphics[width=\paperwidth]{tetons_landscape}}
\begin{frame}[t,plain]{\textcolor{white}{What explains the distribution of organisms living here?}}

	\hangpara\textcolor{white}{Consider ecological and geological time scales.}
\end{frame}
}

{
\usebackgroundtemplate{\includegraphics[width=\paperwidth]{teton_region_overview}}
\begin{frame}[t,plain]
\end{frame}
}

{
\usebackgroundtemplate{\includegraphics[width=\paperwidth]{craters_overview}}
\begin{frame}[t,plain]
\end{frame}
}

{
\usebackgroundtemplate{\includegraphics[width=\paperwidth]{craters_of_the_moon}}
\begin{frame}[t,plain]
\end{frame}
}

{
\usebackgroundtemplate{\includegraphics[width=\paperwidth]{twin_falls}}
\begin{frame}[t,plain]
\end{frame}
}

{
\usebackgroundtemplate{\includegraphics[width=\paperwidth]{teton_region_overview}}
\begin{frame}[t,plain]
\end{frame}
}



{
\usebackgroundtemplate{\includegraphics[width=\paperwidth]{grand_prismatic_spring}}
\begin{frame}[b,plain]{\textcolor{white}{Grand Prismatic Spring, Yellowstone NP.}}
\hfill\tiny\textcolor{white}{Clément Bardot, Wikimedia Commons}
\end{frame}
}

{
\usebackgroundtemplate{\includegraphics[width=\paperwidth]{teton_hotspots}}
\begin{frame}[b,plain]
\hfill\tiny{\textcolor{white}{Arjuno3, Wikimedia Commons}}
\end{frame}
}

{
\usebackgroundtemplate{\includegraphics[width=\paperwidth]{teton_fault_zone}}
\begin{frame}[t,plain]
\end{frame}
}

{
\usebackgroundtemplate{\includegraphics[width=\paperwidth]{teton_burned_area}}
\begin{frame}[t,plain]
\end{frame}
}

{
\usebackgroundtemplate{\includegraphics[width=\paperwidth]{tetons_landscape}}
\begin{frame}[t,plain]{\textcolor{white}{What explains the distribution of organisms living here?}}

	\hangpara\textcolor{white}{Consider ecological and geological time scales.}
\end{frame}
}


\end{document}
