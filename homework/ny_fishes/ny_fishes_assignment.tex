%!TEX TS-program = lualatex
%!TEX encoding = UTF-8 Unicode

\documentclass[12pt]{exam}
\usepackage{graphicx}
	\graphicspath{{/Users/goby/Pictures/teach/438/homework/}} % set of paths to search for images

\usepackage{geometry}
\geometry{letterpaper, bottom=0.9in}                   

\usepackage{afterpage}
\usepackage{pdflscape}

\newlength{\myindent}
\setlength{\myindent}{\parindent}
\newcommand{\ind}{\hspace*{\myindent}}

\newlength{\litindent}
\setlength{\litindent}{\parindent}

\usepackage[parfill]{parskip} 

\usepackage{fontspec}
\setmainfont[Ligatures={TeX}, BoldFont={* Bold}, ItalicFont={* Italic}, BoldItalicFont={* BoldItalic}, Numbers={OldStyle,Proportional}]{Linux Libertine O}
\setsansfont[Scale=MatchLowercase,Ligatures=TeX, Numbers=OldStyle]{Linux Biolinum O}
\setmonofont[Scale=MatchLowercase]{Inconsolatazi4}
\usepackage{microtype}

\usepackage{unicode-math}
\setmathfont[Scale=MatchLowercase]{Asana Math}
%\setmathfont[Scale=MatchLowercase]{XITS Math}

% To define fonts for particular uses within a document. For example, 
% This sets the Libertine font to use tabular number format for tables.
\newfontfamily{\tablenumbers}[Numbers={Monospaced}]{Linux Libertine O}
\newfontfamily{\libertinedisplay}{Linux Libertine Display O}

\usepackage{longtable}

\usepackage{booktabs}
\usepackage{multirow}
\usepackage{multicol}

\usepackage[justification=raggedright, labelsep=period]{caption}
\captionsetup{singlelinecheck=off}
\captionsetup{skip=0.2em}

%\usepackage{tabularx}
%\usepackage{siunitx}
\usepackage{array}
\newcolumntype{L}[1]{>{\raggedright\let\newline\\\arraybackslash\hspace{0pt}}p{#1}}
\newcolumntype{C}[1]{>{\centering\let\newline\\\arraybackslash\hspace{0pt}}p{#1}}
\newcolumntype{R}[1]{>{\raggedleft\let\newline\\\arraybackslash\hspace{0pt}}p{#1}}

\newcolumntype{M}[1]{>{\centering\let\newline\\\arraybackslash\hspace{0pt}}m{#1}}

\usepackage{tikz}

\usepackage{enumitem}
\setlist{leftmargin=*}
\setlist[1]{labelindent=\parindent}
\setlist[enumerate]{label=\textsc{\alph*}., ref=\textsc{\alph*}}

%\usepackage{hyperref}
\usepackage{hanging}

\usepackage[sc]{titlesec}


\renewcommand{\solutiontitle}{\noindent}
\unframedsolutions
\SolutionEmphasis{\bfseries}

\renewcommand{\questionshook}{%
	\setlength{\leftmargin}{-\leftskip}%
}
%Change \half command from 1/2 to .5
%\renewcommand*\half{.5}


\makeatletter
\def\SetTotalwidth{\advance\linewidth by \@totalleftmargin
\@totalleftmargin=0pt}
\makeatother

\pagestyle{headandfoot}
\firstpageheader{\textsc{bi} 438/638: Biogeography}{}{\ifprintanswers\textbf{KEY}\else Name: \enspace \makebox[2.5in]{\hrulefill}\fi}
\runningheader{New York Fishes}{}{\footnotesize{pg. \thepage}}
\footer{}{}{}
\runningheadrule

\newcommand*\AnswerBox[2]{%
    \parbox[t][#1]{0.92\textwidth}{%
    \begin{solution}#2\end{solution}}
    \vspace{\stretch{1}}
}

\newenvironment{AnswerPage}[1]
    {\begin{minipage}[t][#1]{0.92\textwidth}%
    \begin{solution}}
    {\end{solution}\end{minipage}
    \vspace{\stretch{1}}}

\newlength{\basespace}
\setlength{\basespace}{5\baselineskip}

\newcommand{\allele}[1]{\textit{#1}}

%\printanswers

\begin{document}

\textbf{This assignment is due by the start of the next class period.}

All scientific analyses begin with a set of data. Most data sets 
you will use in this course were assembled by a former student
of this course so that you can focus on analysis and interpretation. You 
will assemble a one small data set to gain some 
experience with this task.

You will assemble part of a presence / absence data matrix for 
the fishes of New York, using distribution maps created by 
scientists with the New York Department of Environmental 
Conservation and the New York State Museum. The presence/absence matrix shows whether a species is 
present~(1) or absent~(0) in a watershed. 
Below is a partial example matrix.

\begin{longtable}[]{@{}lccccc@{}}
\toprule
& Watershed 1 & Watershed 2 & Watershed 3 & \ldots{} & Watershed
X\tabularnewline
\midrule
\endhead
Species 1 & 1 & 1 & 0 & \ldots{} & 0\tabularnewline
Species 2 & 0 & 0 & 1 & \ldots{} & 1\tabularnewline
Species 3 & 1 & 0 & 0 & \ldots{} & 1\tabularnewline
\ldots{} & \ldots{} & \ldots{} & \ldots{} & \ldots{} &
\ldots{}\tabularnewline
Species \textsc{x} & 1 & 1 & 0 & \ldots{} & 0\tabularnewline
\bottomrule
\end{longtable}

You were given a short list of fishes. For each species, you 
will determine whether it is present or absent in a given 
watershed. A watershed includes all or parts of a large river and 
its tributaries. I will assemble the data obtained by each 
student into a single data set of all of the fishes that 
you will analyze later.

Follow these instructions carefully to complete this
exercise.

\begin{enumerate}
\item
  Obtain your list of fishes in class.
\item
  Download files from the course Canvas page:

  \begin{enumerate}
  \def\labelenumii{\arabic{enumii}.}
  \item 
  	Inland\_Fishes\_of\_New\_York.pdf
  	
  	This is the book that you will gather the data from.
  \item
    New\_York\_Fishes.xlsx
    
    This is the spreadsheet for your data.
    
  \item
    New\_York\_Watersheds.pdf (optional)
    
    This is the map included also on the last page of this handout, which you can tear off for reference as you gather your data. 
    If you tear off and then lose the map, this is here for you.
    
  \end{enumerate}
\item
  The watersheds map (see last page of this assignment) gives 
  the names of the watersheds. Each watershed is identified by 
  a thick black line.
\item
  Change the name of the spreadsheet to New\_York\_Fishes\_Lastname.xls.
  Use \emph{your} last name.
\item
  Type the scientific names of your list of species into the first
  column of the spreadsheet, one species per cell. Be certain to 
  type the names down the column and not across in the same row. 
  \emph{Type only the scientific names.}
\item
  Open the \textit{Inland Fishes of New York} \textsc{pdf}, and then scroll down until you find your fishes. Your list of fishes is
  arranged with the same order as the list of fishes in the text. As soon as you find the first species on your list, 
  continue sequentially until you complete your list. Do not skip 
  any species on your list. Do not add species not on your list.
\end{enumerate}

\begin{quote}
The text includes invasive species, which I deleted
before creating your lists. Make sure that you are only getting
distribution information for the species on your list and not 
others that may be intermingled in the book. These instances 
are few.
\end{quote}

\begin{enumerate}[resume]
\item \label{step1}
  For each species, you will see a distribution map on the 
  lower right of the web. The red, yellow, or
  green shading  
  represents the generalized area where the species' presence has 
  been documented.
\item \label{steptwo}
  For each watershed that is shaded (red, yellow, or green), change the 0 in the 
  spreadsheet to 1 to indicate the presence of the species.
\item
  Repeat steps \ref{step1} and \ref{steptwo} for each species on your list.
\item
  Upload your completed spreadsheet into the drop box available 
  from the course website.
\end{enumerate}
\newpage

\rotatebox{90}{\includegraphics[height=\textwidth]{ny_watershed_map}}

\end{document}  