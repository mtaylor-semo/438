%!TEX TS-program = lualatex
%!TEX encoding = UTF-8 Unicode

\documentclass[t]{beamer}

%%%% HANDOUTS For online Uncomment the following four lines for handout
%\documentclass[t,handout]{beamer}  %Use this for handouts.
%\includeonlylecture{student}
%\usepackage{handoutWithNotes}
%\pgfpagesuselayout{3 on 1 with notes}[letterpaper,border shrink=5mm]

%% For students, use \lecture{student}{student}
%% For mine, use \lecture{instructor}{instructor}


%\usepackage{pgf,pgfpages}
%\pgfpagesuselayout{4 on 1}[letterpaper,border shrink=5mm]

% FONTS
\usepackage{fontspec}
\def\mainfont{Linux Biolinum O}
\setmainfont[Ligatures={Common,TeX}, Contextuals={NoAlternate}, BoldFont={* Bold}, ItalicFont={* Italic}, Numbers={Proportional}]{\mainfont}
\setmonofont[Scale=0.75]{Linux Libertine Mono O} 
\setsansfont[Scale=MatchLowercase]{Linux Biolinum O} 
\usepackage{microtype}

\usepackage{xspace}
\newfontfamily\amperfont[Style=Alternate]{Linux Libertine O}    
\makeatletter
\DeclareRobustCommand{\amper}{{\amperfont\ifx\f@shape\scname\smaller[1.2]\fi\&}\xspace}
\makeatother

\usepackage{graphicx}
	\graphicspath{%
	{/Users/goby/Pictures/teach/438/lectures/}%
	{/Users/goby/Pictures/teach/438/homework/}
	{/Users/goby/Pictures/teach/163/common/}} % set of paths to search for images

\usepackage{amsmath,amssymb}

%\usepackage{units}

\usepackage{booktabs}
\usepackage{multicol}
%	\setlength{\columnsep=1em}

\usepackage{textcomp}
\usepackage{setspace}
\usepackage{tikz}
	\tikzstyle{every picture}+=[remember picture,overlay]

\mode<presentation>
{
  \usetheme{Lecture}
  \setbeamercovered{invisible}
  \setbeamertemplate{items}[square]
}

\usepackage{calc}
\usepackage{hyperref}

\newcommand\HiddenWord[1]{%
	\alt<handout>{\rule{\widthof{#1}}{\fboxrule}}{#1}%
}



\begin{document}
%\lecture{instructor}{instructor}
%\lecture{student}{student}


% First assignment
{
	\usebackgroundtemplate{\includegraphics[width=\paperwidth]{first_assignment}}
	\begin{frame}[t,plain]{Here is your first assignment.}
	
	\hangpara Build a presence/absence data matrix for the fishes of Georgia.
	
	\hangpara See “Files and link for Georgia fishes assignment” in \href{https://semo.instructure.com}{Canvas} for \textsc{xlsx} template, \amper\kern-3.33ptc. %\&\kern-.67ptc
	
	\hangpara \url{https://fishesofgeorgia.uga.edu}
	
\end{frame}
}

%\begin{frame}[t,plain]{Here is your first assignment.}
%
%\hangpara You will build a presence/absence data matrix for the inland fishes of \newline New York state.
%
%\hangpara Due next class meeting in drop box.
%
%\hangpara Get files from Moodle page.
%
%{\centering
%	\includegraphics[height=1.7in]{ny_brook_trout}\par
%}
%
%\vfilll
%
%\hfill \tiny Brook Trout, Carlson et al. 2016, \textit{Inland Fishes of New York}
%\end{frame}
%%}
%
%\begin{frame}{A species is present (1) or absent (0) from a watershed.}
%
%%\vspace{\baselineskip}
%
%\bigskip
%
%\includegraphics[width=0.48\textwidth]{ny_watershed_map} \hfill
%\includegraphics[width=0.48\textwidth]{ny_example_map}
%
%\vfilll
%
%\hfill \tiny Carlson et al. 2016, \textit{Inland Fishes of New York}
%\end{frame}

\begin{frame}
	\centering
	\includegraphics[height=0.95\textheight]{georgia_watershed_map}
\end{frame}

\end{document}
