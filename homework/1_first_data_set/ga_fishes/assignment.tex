All scientific analyses begin with a set of data. In this course, a
student employee and I have assembled most of the data sets for you so
that you can focus on the analysis and interpretation. You will,
however, assemble a few small data sets to gain some experience with
this task.

For this exercise, you will assemble part of a presence / absence data
matrix for the fishes of Georgia, using distribution maps created by
researchers at the University of Georgia Museum of Natural History. The
presence / absence matrix shows whether a species is present in a
watershed (1) or absent from the watershed (0). Below is a partial
example matrix.

\begin{longtable}[]{@{}llllll@{}}
\toprule
& Watershed 1 & Watershed 2 & Watershed 3 & \ldots{} & Watershed
X\tabularnewline
\midrule
\endhead
Species 1 & 1 & 1 & 0 & \ldots{} & 0\tabularnewline
Species 2 & 0 & 0 & 1 & \ldots{} & 1\tabularnewline
Species 3 & 1 & 0 & 0 & \ldots{} & 1\tabularnewline
\ldots{} & \ldots{} & \ldots{} & \ldots{} & \ldots{} &
\ldots{}\tabularnewline
Species X & 1 & 1 & 0 & \ldots{} & 0\tabularnewline
\bottomrule
\end{longtable}

You will be given a short list of fishes. For each species, you will
determine whether it is present or absent in a given watershed. A
watershed includes all or parts of a large river and its tributaries. I
will assemble the data obtained by each student into a single large
matrix for all of the fishes that we will analyze later.

Carefully follow these instructions to successfully complete this
exercise.

\begin{enumerate}
\def\labelenumi{\arabic{enumi})}
\item
  Obtain your list of fishes in class.
\item
  Download two files from
  http://mtaylor4.semo.edu/\textasciitilde{}goby/biogeo/

  \begin{enumerate}
  \def\labelenumii{\alph{enumii}.}
  \item
    Georgia\_Fishes.xlsx
  \item
    Georgia\_Watersheds.pdf
  \end{enumerate}
\item
  The watersheds map gives the names of the watersheds. It also has a
  thick black line that represents the fall line. The fall line is an
  important biogeographic feature in North America that you will learn
  more about later. (Unless, of course, you choose to go learn about it
  now.)
\item
  Change the name of the spreadsheet to Georgia\_Fishes\_Lastname.xls.
  Use \emph{your} last name.
\item
  Type the scientific names of your list of species into the first
  column of the spreadsheet, one species per cell. Be certain to type
  the names down the column and not across in the same row. Type only
  the scientific names.
\item
  Go to the Fishes of Georgia website at http://fishesofgeorgia.uga.edu
\item
  Click on the ``Fish List'' link near the top of the page.
\item
  Scroll down until you find your fishes. Your list of fishes is
  arranged with the same order as the list of fishes on the website. As
  soon as you find the first species on your list, continue sequentially
  until you complete your list. Do not skip any species on your list.
\end{enumerate}

\begin{quote}
The website did include invasive (exotic) species, which I deleted
before creating your lists. Make sure that you are only getting
distribution information for the species on your list and not others
that may be intermingled on the website. These instances are few.
\end{quote}

\begin{enumerate}
\def\labelenumi{\arabic{enumi})}
\item
  Click on the scientific name of your first species to bring up the
  species information. You will see a distribution map on the lower
  right of the web page. The green shading on the map represents the
  generalized area where the species' presence has been documented.
\item
  For each watershed with green shading, change the 0 in the spreadsheet
  to 1 to indicate the presence of the species.
\end{enumerate}

\begin{quote}
The fall line (the thick black line) on the map you downloaded divides
some of the watersheds into upper and lower parts. Be sure to pay close
attention to the location of this dividing fall line because it is not
drawn on the maps from the website. For a species that occurs in both
the upper and lower watersheds, be sure to change both 0s to 1s in the
spreadsheet.

If a species is widespread in one part of a watershed (for example, the
upper part) and just barely extend into the other part (e.g., lower),
you can treat the species as present in only the widespread part. As an
example, see \emph{Cyprinella xaenura} under the family Cyprinidae. The
range of this species includes the upper Ocmulgee Watershed. In the
Ocmulgee, its range extends down to and perhaps slightly past the fall
line. You would treat this species as present in the upper but not the
lower Ocmulgee watershed. If you are in doubt, don't hesitate to contact
me.
\end{quote}

\begin{enumerate}
\def\labelenumi{\arabic{enumi})}
\item
  Repeat steps 9 and 10 for each species on your list.
\item
  Upload the completed spreadsheet into the drop box available from the
  course website.
\end{enumerate}

\textbf{This exercise is due next Tuesday, one week from today.}
