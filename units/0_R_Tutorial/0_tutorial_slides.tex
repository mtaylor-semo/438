%!TEX TS-program = lualatex
%!TEX encoding = UTF-8 Unicode

\documentclass[t]{beamer}

%%%% HANDOUTS For online Uncomment the following four lines for handout
%\documentclass[t,handout]{beamer}  %Use this for handouts.
%\includeonlylecture{student}
%\usepackage{handoutWithNotes}
%\pgfpagesuselayout{3 on 1 with notes}[letterpaper,border shrink=5mm]

%% For students, use \lecture{student}{student}
%% For mine, use \lecture{instructor}{instructor}


%\usepackage{pgf,pgfpages}
%\pgfpagesuselayout{4 on 1}[letterpaper,border shrink=5mm]

% FONTS
\usepackage{fontspec}
\def\mainfont{Linux Biolinum O}
\setmainfont[Ligatures={Common,TeX}, Contextuals={NoAlternate}, BoldFont={* Bold}, ItalicFont={* Italic}, Numbers={Proportional}]{\mainfont}
\setmonofont[Scale=0.75]{Linux Libertine Mono O} 
\setsansfont[Scale=MatchLowercase]{Linux Biolinum O} 
\usepackage{microtype}

\usepackage{graphicx}
	\graphicspath{%
	{/Users/goby/Pictures/teach/438/lectures/}%
	{/Users/goby/Pictures/teach/438/homework/}
	{/Users/goby/Pictures/teach/438/lab/}} % set of paths to search for images

\usepackage{amsmath,amssymb}

%\usepackage{units}

\usepackage{booktabs}
\usepackage{multicol}
%	\setlength{\columnsep=1em}

\usepackage{textcomp}
\usepackage{setspace}
\usepackage{tikz}
	\tikzstyle{every picture}+=[remember picture,overlay]

\mode<presentation>
{
  \usetheme{Lecture}
  \setbeamercovered{invisible}
  \setbeamertemplate{items}[square]
}

\usepackage{calc}
\usepackage{hyperref}



\begin{document}
%\lecture{instructor}{instructor}
%\lecture{student}{student}

\begin{frame}[t,plain]{Set up your computer.}
	
	\hangpara Find the Documents folder on your computer. Make a “biogeo” folder inside the “Documents” folder. 
	
	\hangpara You will use this folder to store your all analysis files for this course.
	
	\hangpara If you want, make a shortcut to the biogeo folder and put the shortcut on your desktop or other convenient place for you.
	
	\hangpara Go to the Canvas page for this course. Scroll down to the R and R Studio module.
	
\end{frame}


\begin{frame}[t,plain]{Installing and using R.}

\begin{multicols}{2}

\hangpara Go to Canvas page for links to R and R Studio.

\hangpara \textbf{Download and install R first.} If prompted for the website to download from, scroll down to the United States, then choose Washington University. (You can search for wustl, too.)

\hangpara \textbf{Download and install R Studio second.} Download the free Desktop version.

\columnbreak

\noindent Launch R Studio.  You should see something like this:

\noindent\includegraphics[width=\linewidth]{rstudio_splash}
\end{multicols}

\end{frame}



\begin{frame}{Create and save an R Project file}

	\hangpara Choose “File $>$ New Project \dots\end{frame}

	\hangpara Choose ”Existing Directory”
	
	\hangpara Press the “Browse” button. 


\end{document}
