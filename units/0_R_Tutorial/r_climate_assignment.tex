%!TEX TS-program = lualatex
%!TEX encoding = UTF-8 Unicode

\documentclass[11pt]{article}
\usepackage{graphicx}
\graphicspath{{/Users/goby/Pictures/teach/438/lab/}} % set of paths to search for images

\usepackage{geometry}
\geometry{letterpaper}                   
\geometry{bottom=1in}
%\geometry{landscape}                % Activate for for rotated page geometry
\usepackage[parfill]{parskip}    % Activate to begin paragraphs with an empty line rather than an indent
%\usepackage{amssymb}
%\usepackage{mathtools}
%	\everymath{\displaystyle}

%\pagenumbering{gobble}

\usepackage{fontspec}
\setmainfont[Ligatures={Common, TeX}, BoldFont={* Bold}, ItalicFont={* Italic}, Numbers={Proportional}]{Linux Libertine O}
\setsansfont[Scale=MatchLowercase,Ligatures=TeX]{Linux Biolinum O}
\setmonofont[Scale=MatchLowercase]{Linux Libertine Mono O}
\usepackage{microtype}

\usepackage{unicode-math}
\setmathfont[Scale=MatchLowercase]{Asana-Math.otf}
%\setmathfont{XITS Math}

% To define fonts for particular uses within a document. For example, 
% This sets the Libertine font to use tabular number format for tables.
%\newfontfamily{\tablenumbers}[Numbers={Monospaced}]{Linux Libertine O}
%\newfontfamily{\libertinedisplay}{Linux Libertine Display O}


\usepackage{booktabs}
\usepackage{longtable}
%\usepackage{tabularx}
%\usepackage{siunitx}
%\usepackage[justification=raggedright, singlelinecheck=off]{caption}
%\captionsetup{labelsep=period} % Removes colon following figure / table number.
%\captionsetup{tablewithin=none}  % Sequential numbering of tables and figures instead of
%\captionsetup{figurewithin=none} % resetting numbers within each chapter (Intro, M&M, etc.)
%\captionsetup[table]{skip=0pt}

\usepackage{array}
\newcolumntype{L}[1]{>{\raggedright\let\newline\\\arraybackslash\hspace{0pt}}p{#1}}
\newcolumntype{C}[1]{>{\centering\let\newline\\\arraybackslash\hspace{0pt}}p{#1}}
\newcolumntype{R}[1]{>{\raggedleft\let\newline\\\arraybackslash\hspace{0pt}}p{#1}}

%\usepackage{enumitem}
%\usepackage{hyperref}
%\usepackage{placeins} %P4ovides \FloatBarrier to flush all floats before a certain point.

\usepackage[sc]{titlesec}

\usepackage{hanging}

\newcommand{\assignmentTitle}{Tutorial: climate and distribution}

\usepackage{fancyhdr}
\fancyhf{}
\pagestyle{fancy}
%\lhead{}
%\chead{}
%\rhead{Name: \rule{5cm}{0.4pt}}
%\renewcommand{\headrulewidth}{0pt}
\setlength{\headheight}{14pt}
\fancyhead[LE,RO]{\footnotesize \assignmentTitle\ \thepage}

\fancypagestyle{firstpage}
{
	\fancyhf{}
	\lhead{\textsc{bi}~438/638 Biogeography}
	\rhead{Name: \rule{5cm}{0.4pt}}
	\renewcommand{\headrulewidth}{0pt}
	%   \fancyfoot[C]{\footnotesize Page \thepage\ of \pageref{LastPage}}
}

\newcommand{\bigSpace}{\vspace{5\baselineskip}}

\newlength{\myLength}
\setlength{\myLength}{\parindent}

%\title{Tutorial: Climate and Distribution}
%\author{10 Points}
%\date{}                                           % Activate to display a given date or no date

\begin{document}
	%\maketitle
	\thispagestyle{firstpage}
	
	\subsection*{\assignmentTitle\ (10 points)}
	
	Temperature and precipitation are the two climate variables that most
	influence the distribution of ecosystems. This is a fundamental
	biogeographic concept that you will explore using a climate data set
	taken from Alberta, Canada. The data set contains mean annual
	temperature (MAT, in °C) and mean annual precipitation (MAP, in mm) from
	80 weather stations spread across eight different ecosystems (Ecosys,
	labeled A-H in the data set). For each ecosystem, the dominant plant species or group
	was recorded (Cw = Western Redcedar, Gr = mixed grasses, La = Subalpine
	Larch). The measurements are in five-year intervals from 1965--2010.
	
	You must recreate, as close as reasonably possible, the following figure
	(color version on screen and Canvas). Pay attention to the use of colors and symbols. Colors identify the
	ecosystems, while the symbols represent the dominant plant groups.
	
	\begin{center}
		\includegraphics[width=0.9\textwidth]{tutorial_assignment_climate}
	\end{center}
	
	
	
	\subsubsection*{Get the data}
	
	Go to the course Canvas page. Scroll down to the “Unit 0: Get Familiar with R” module. Download the “tutorial\_climate\_data.csv” file and put it in your biogeography folder. The data should be in the same folder as the R Project file you created for this course. If you do not do this, you will not be able to read the data into R.
	
	 \textsc{Important!} Be sure that your computer saves the file with the “.csv” extension. It must not be “.txt” or anything other than “.csv.”
	
	
	
	\subsubsection*{Read the data}
	
	R provides the \texttt{read.csv()} function to read data sets that you download for this course. You assign the data to a variable name, as you did in for other variables.
	You should assign the data to an informative variable name. To read the climate 
	data, type the following command into R:
	
	%\texttt{climate \textless{}-
		%read.csv('http://mtaylor4.semo.edu/\textasciitilde{}goby/biogeo/tutorial\_climate\_data.csv')}
	
	\texttt{climate \textless{}-
		read.csv('tutorial\_climate\_data.csv')}
	
	If you get an error, be \textit{sure} that the file ends with “.csv” and is located in the same folder as your open R Project file.
	
	\subsubsection*{Some hints:}
	
	Don't forget to use \texttt{attach()} and \texttt{detach()} functions.
	
	Type your commands into the Source pane. Press \texttt{Cmd/Ctrl + Enter} to execute them.
	
	Add only one argument at a time to the plot, as you
	did in the R Tutorial. If you make an error, it will be easier to find and fix
	before adding the next argument. Be careful with quotes and parentheses. You can use single or double quotes but they must match.
	
	Use \texttt{xlim} and \texttt{ylim} arguments in the \texttt{plot()} function to set the lower and
	upper values for the X and Y-axes, respectively. Type \texttt{?plot} to learn about
	the \texttt{xlim} and \texttt{ylim} arguments.
	
	
	You will need eight colors, one for each ecosystem. Instead of defining
	eight colors, use \texttt{rainbow(8)} to define your colors. See \texttt{?rainbow} for
	details.
	
	Use 21, 22 and 25 to specify the symbols for the \texttt{pch} argument in the
	\texttt{plot} function.
	
	\textsc{Important!} The command to assign the colors and shapes to variables is slightly different than the previous exercise. For example, to assign background color, you type:
	
	\texttt{bg = rainbow(8)[factor(varname)]}. You did not have to use the \texttt{factor()} function last time but you do this time, for reasons we won't cover here. For \texttt{varname}, use either \texttt{Ecosys} or \texttt{Species}. You will use one variable for color and the other variable for symbols. Use trial and error or logic to decide when to apply each variable. 
	
	Use the \texttt{text()} function to place the names of the major plant groups
	into the figure. You will need to use the \texttt{text()} function
	three times, one for each dominant plant group. Use the \texttt{font=3}
	argument to specify italics for the labels. Remember that the first two
	numbers should be the X and Y coordinates for your text. The Y value may
	be negative for one of the dominant plant group labels. Use trial and error to find a pleasing placement. You do not have to \emph{exactly} match the positions in the example.
	
	Technically, you should include °C in the Y-axis label but doing so might be
	more complex than I want to get into now. However, to encourage
	exploration, I will buy you an imaginary adult beverage of your choice if %add 50\% extra credit to this assignment (5 pts) if
	you figure it out. You may search the Internet. \textsc{Note:} The degree symbol is \emph{not} a superscript lower-case o or zero. It is a degree symbol. 
	
	Use the \texttt{main} argument in the \texttt{plot()} function to add
	a main title, which will appear above the graph (but is not shown
	above). The title should contain your name (and your partner's name, if you work with one).
	
	
	\subsection*{Upload the completed graph to the tutorial drop box.}
	
	Save the file as a PDF or png file (File menu \textgreater{} Save
	as\ldots{}).
	
	Give the file an obvious name like climate\_lastnames.pdf (or .jpg).
	Include both last names. 
	
\end{document}  