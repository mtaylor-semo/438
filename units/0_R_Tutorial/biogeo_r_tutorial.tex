%!TEX TS-program = lualatex
%!TEX encoding = UTF-8 Unicode

\documentclass[11pt]{article}
%\usepackage{graphicx}
%	\graphicspath{{/Users/goby/Pictures/teach/153/lab/}} % set of paths to search for images

\usepackage{geometry}
\geometry{letterpaper}                   
\geometry{bottom=1in}
%\geometry{landscape}                % Activate for for rotated page geometry
\usepackage[parfill]{parskip}    % Activate to begin paragraphs with an empty line rather than an indent
%\usepackage{amssymb}
%\usepackage{mathtools}
%	\everymath{\displaystyle}

%\pagenumbering{gobble}

\usepackage{fontspec}
\setmainfont[Ligatures={Common, TeX}, BoldFont={* Bold}, ItalicFont={* Italic}, Numbers={OldStyle,Proportional}]{Linux Libertine O}
\setsansfont[Scale=MatchLowercase,Ligatures=TeX]{Linux Biolinum O}
\setmonofont[Scale=0.8]{Linux Libertine Mono O}
\usepackage{microtype}

\usepackage{unicode-math}
\setmathfont[Scale=MatchLowercase]{Asana-Math.otf}
%\setmathfont{XITS Math}

% To define fonts for particular uses within a document. For example, 
% This sets the Libertine font to use tabular number format for tables.
%\newfontfamily{\tablenumbers}[Numbers={Monospaced}]{Linux Libertine O}
%\newfontfamily{\libertinedisplay}{Linux Libertine Display O}


\usepackage{booktabs}
\usepackage{longtable}
%\usepackage{tabularx}
%\usepackage{siunitx}
%\usepackage[justification=raggedright, singlelinecheck=off]{caption}
%\captionsetup{labelsep=period} % Removes colon following figure / table number.
%\captionsetup{tablewithin=none}  % Sequential numbering of tables and figures instead of
%\captionsetup{figurewithin=none} % resetting numbers within each chapter (Intro, M&M, etc.)
%\captionsetup[table]{skip=0pt}

\usepackage{array}
\newcolumntype{L}[1]{>{\raggedright\let\newline\\\arraybackslash\hspace{0pt}}p{#1}}
\newcolumntype{C}[1]{>{\centering\let\newline\\\arraybackslash\hspace{0pt}}p{#1}}
\newcolumntype{R}[1]{>{\raggedleft\let\newline\\\arraybackslash\hspace{0pt}}p{#1}}

%\usepackage{enumitem}
%\usepackage{hyperref}
%\usepackage{placeins} %P4ovides \FloatBarrier to flush all floats before a certain point.

%\usepackage{titling}
%\setlength{\droptitle}{-50pt}
%\posttitle{\par\end{center}}
%\predate{}\postdate{}

\usepackage[sc]{titlesec}

\usepackage{hanging}

\newcommand{\assignmentTitle}{R Tutorial}

\usepackage{fancyhdr}
\fancyhf{}
\pagestyle{fancy}
%\lhead{}
%\chead{}
%\rhead{Name: \rule{5cm}{0.4pt}}
%\renewcommand{\headrulewidth}{0pt}
\setlength{\headheight}{14pt}
\fancyhead[LE,RO]{\footnotesize \assignmentTitle\ \thepage}

\fancypagestyle{firstpage}
{
   \fancyhf{}
   \lhead{\textsc{bi}~438/638 Biogeography}
   \rhead{Name: \rule{5cm}{0.4pt}}
   \renewcommand{\headrulewidth}{0pt}
%   \fancyfoot[C]{\footnotesize Page \thepage\ of \pageref{LastPage}}
}

\newcommand{\bigSpace}{\vspace{5\baselineskip}}

\newlength{\myLength}
\setlength{\myLength}{\parindent}

%\title{R Tutorial}
%\author{Biogeography Unit 0}
%\date{}                                           % Activate to display a given date or no date

\begin{document}
% \maketitle
\thispagestyle{firstpage}

\subsection*{\assignmentTitle{}}

You will perform many types of biogeographic analyses during this
course. For most of these analyses, you will use a software package
called R. R is used often for statistical analyses but scientists also
use R to perform ecological and evolutionary analyses. R has several
advantages: it is powerful, it is free, it is available on all major
computer platforms (including PC, Mac, and linux), it is flexible, and
it is expandable. The main disadvantage of R is the steep learning
curve. The best way to learn R is through practice and repetition.
Because R is free, I encourage you to download R to your personal
computer and practice these exercises. The course website has
instructions on how to install R on your computer, as well as some of
the packages you will need and links to many tutorials to become
familiar with R. I strongly recommend that you, as a young biologist,
begin to learn R, as the skill may prove useful later in your career.

This exercise will introduce you to R. You will get exposure to the R
command line interface and syntax, learn some basic R functions, and get
a brief overview of R's powerful graphing features. \textbf{Study,
practice and memorize the commands and functions in this tutorial. You
will use them often in this course.}

Begin by launching R. After a moment, you will see a `\texttt{\textgreater{}}'
prompt. That is the command line. You type commands after the prompt.
Depending on the command, R may or may not return information to you, as
you will see.

At the prompt, type the following commands, one at a time, and then
observe the results that R returns to you. Commands that you type, the
results that you get, or the keys that you press will be \texttt{in this monospace
font.} Press the \texttt{Enter} or \texttt{Return} key after typing each command.

\begin{tabular}{@{}L{1in}L{4.5in}@{}}
\texttt{3 + 2} & \\

\texttt{A \textless{}- 3 + 2} & \texttt{\# ''\textless{}-''} is a less than sign, followed
by a dash, no space between them. \\
	& 	\texttt{\#} \textbf{NOTE}: Any text following a hash tag is a comment and ignored
by R.\\
 
\texttt{A} & \\

\texttt{a} & \# returns an error because the variable a has not been defined. R is
case sensitive.  \texttt{A} and \texttt{a} are different variables. Remember that R is case sensitive because
that is a common source of error.\\[2ex]

	
\texttt{B \textless{}- 4\^{}2}	&	 \# what does the caret do?\\ 
\texttt{B}	& \\[2ex]
	
\texttt{C \textless{}- A + B} & \\
\texttt{C}	& \\
	&	\\
\end{tabular}


What result did you get for variable \texttt{C}? \rule{5cm}{0.4pt}. Why?

You just learned a few things about R. Typing \texttt{3+2} returned \texttt{5} but did not
save the result. You will usually need to save results from your
analyses, so you assign them to variables. In the second example, \texttt{A} is
the variable and ``\texttt{\textless{}-}'' is the assignment operator. The
statement \texttt{A \textless{}- 3+2} evaluates \texttt{3+2}, then assigns the result (\texttt{5})
to the variable called \texttt{A}. When you typed \texttt{A} by itself, R returned the
value that was stored in variable \texttt{A}. For variable \texttt{B}, you assigned it
\texttt{4\textsuperscript{2}}, which is 16. The caret symbol means exponent.

\subsection*{Diving Deeper: R functions, the help function, and a simple
plot.}

Type the following commands:

\texttt{x \textless{}- rnorm(50)} \\
\texttt{x}

What did the \texttt{rnorm()} function do? To find out, type

\texttt{?rnorm}

\texttt{rnorm()} generates a set of random numbers that are normally distributed.
\texttt{rnorm()} is one of the many functions available in R. Typing the question
mark in front of a function name is a quick way to get help for the
function. You could also type \texttt{help(rnorm)}. The help window explains the
function, describes the arguments for the function, and provides some
examples of proper use. You can see the examples in action with the
\texttt{example()} function:

\texttt{example(rnorm)}

Here, you used the \texttt{rnorm()} function to generate 50 random numbers. To
confirm this, type

\begin{tabular}{@{}L{1in}L{4.5in}@{}}
\texttt{length(x)} 	&	\# \texttt{?length} \\
\end{tabular}

By default, \texttt{rnorm()} generates random numbers centered on a mean of 0 with
a standard deviation of 1. Because the numbers are random, the mean and
standard deviation won't equal exactly 0 and 1. To learn the mean and
standard deviation of your 50 random numbers, type:

\begin{tabular}{@{}L{1in}L{4.5in}@{}}
\texttt{mean(x)} 	&	 \# function to calculate the arithmetic \emph{mean} of a set of
values \\

\texttt{sd(x)} &	\# function to calculate the \emph{s}tandard \emph{d}eviation of a
set of values\\
\end{tabular}

To learn more about the \texttt{mean()} and \texttt{sd()} functions, type \texttt{?mean} and \texttt{?sd}.

Most functions accept more than one argument. Default values are
provided for most arguments so you only need to provide values when you
want to override the default values. For \texttt{rnorm()}, the default values
generate random numbers centered on a mean of 0 with a standard
deviation of 1. You can change the defaults for the arguments by
specifying values when you call the function, as shown below.

Assign 100 random numbers to \texttt{x} but use different values for the mean and
standard deviation arguments, such as a mean of 20 and a standard
deviation of 2.5:

\texttt{x \textless{}- rnorm(100, mean=20, sd=2.5)} \\
\texttt{x}

Repeat this one more time but assign the random numbers to the variable
\texttt{y}. Use slightly different values for the mean and standard deviation.

\texttt{mean(y)} \\
\texttt{sd(y)}

Now, plot the values of \texttt{x} and \texttt{y}:

\begin{tabular}{@{}L{1in}L{4.5in}@{}}
\texttt{plot(x, y)} & \# not a very informative graph. \\
\texttt{hist(x)} 	& \# frequency histogram for values in \texttt{x}. Note the similarity to a normal curve.\\
\end{tabular}

R has powerful plotting features to make complex figures.
You'll use some of the many plotting features throughout this course,
starting with the next section.

\subsection*{Plunge Into The Deep End: The Iris Data Set}

\texttt{data()} \# scroll through the list

R comes with many built in data sets. You will use the data set called
\texttt{iris}. The iris data set contains measurements from 50 individuals for
each of three species of iris: \emph{Iris setosa}, \emph{I. versicolor}
and \emph{I. virginica}. For each individual, the length and width of
the petals and sepals were measured in centimeters. For a typical data
set, the measured variables are in columns and the measurements for each
individual are in rows.

\begin{tabular}{@{}L{1in}L{4.5in}@{}}
\texttt{data(iris)} 	& \# load the data set \\

\texttt{iris} 			& \# view the data set. It scrolls off the screen. \\

\texttt{head(iris)} 	& \# view the column headers and first several rows. Very
handy.\\

\texttt{tail(iris)}		& \# view the last few rows. \\

\texttt{nrow(iris)} 	& \# what is the number of rows in the data set?\\

\texttt{ncol(iris)} 	& \# the number of columns? \\

\texttt{dim(iris)} 		& \# show number of rows and columns at the same time, aka the \emph{dim}ensions.\\
\end{tabular}

\begin{tabular}{@{}L{1.75in}L{3.75in}@{}}
\texttt{unique(iris\$Species)} & \# show only unique values in the Species column. The \$ selects a particular variable. In this case the Species variable is selected.\\

\texttt{iris\$Sepal.Length} & \# show only the sepal length variable.\\

\texttt{iris\$petal.length} & \# R is case sensitive. Fix the command to show the
petal length. If you don't remember the name of the column, use the \texttt{head()} function to find out.\\

\texttt{mean(iris\$Sepal.Length)} & \# what is the average length of the sepals in
the entire data set?\\

\end{tabular}

For the following command, notice that two equal signs are used. You must type 
both equal signs to indicate equality. A single equal sign
acts like the assignment operator (\texttt{\textless{}-}). You don't want to
assign the value `setosa' to a variable called Species. You want to
obtain the petal width data from all rows where Species is equal to
`setosa.' Not using two equal signs when necessary is another common
source of errors. Pay close attention to the number of equal signs when
you write and type commands. 

\begin{tabular}{@{}L{4in}L{2in}@{}}
\texttt{mean(iris\$Sepal.Width{[}iris\$Species == 'setosa'{]})} & \# mean for
\emph{I. setosa}. 
\end{tabular}

Modify the formula above to  find the mean petal length for all three species. Hint: Press the ``up'' cursor key to recall the last command. You can then click in the command line to edit it. Fill in the following table.


\begin{longtable}[l]{@{}lccc@{}}
\toprule
Species Name: & \emph{I. setosa} & \emph{I. versicolor} & \emph{I.
virginica}\tabularnewline
\midrule
&&&\tabularnewline[1ex]
Mean Petal Length: & \rule{1in}{0.4pt} & \rule{1in}{0.4pt}  & \rule{1in}{0.4pt} \tabularnewline
\bottomrule
\end{longtable}

One species clearly has shorter petals than the other two but are they
significantly different in a statistical sense? Perform a simple
analysis of variance (ANOVA) to find out:

\begin{tabular}{@{}L{4in}L{1.5in}@{}}
\texttt{iris.aov \textless{}- aov(Petal.Length \textasciitilde{} Species,
data=iris)} &  \# tilde between Petal Length and Species.\\
\end{tabular}

\texttt{aov} is one of several functions available for ANOVA in R. \texttt{aov()} analyzed
the mean petal length, using species as the categories for comparison.
The \texttt{data=iris} argument told the \texttt{aov()} function that the two variables to
be analyzed (Petal.Length, Species) were in the iris data set.

The command above did not show you the results. Instead, the results of
the analysis were assigned to the variable \texttt{iris.aov}. To see the results
of your analysis, type

\begin{tabular}{@{}L{1.5in}L{4in}@{}}
\texttt{summary(iris.aov)} & \# show the results of the analysis. \\
\end{tabular}

The \emph{p} value is less than 2$\times$ 10\textsuperscript{-16}
(\texttt{\textless{}2e-16} in the table), which is 2 with 15 zeros in front of
it. The probability that the mean petal lengths are different due to
random variation is about zero. The petal lengths differ significantly
among the three species.

Now, create a graph of sepal length (Y-axis) plotted against petal
length (X-axis) for all three species.

\begin{tabular}{@{}L{3.5in}L{2.25in}@{}}
\texttt{plot(iris\$Petal.Length, iris\$Sepal.Length)} & \# first argument is X
axis, the second argument is Y axis.\\
\end{tabular}

The cluster of points in the lower left is one species but the other
points are mixed together.

Another way you could obtain the same plot is

\begin{tabular}{@{}L{3.5in}L{2.25in}@{}}
\texttt{plot(iris\$Sepal.Length \textasciitilde{} iris\$Petal.Length)} & \# Plot
sepal length on Y axis as a function of petal length on the X axis. Use either method.\\
\end{tabular}

A good graph should distinguish among the three species, using different
colors, symbols, or preferably both. The plot function provides a way to
color the points and to use different symbols. You'll begin with color,
and then symbols.

Before we continue, however, you should learn a handy shortcut. Typing
\texttt{iris\$Petal.Length} and \texttt{iris\$Sepal.Length} every time would get tedious.
Use the \texttt{attach()} function to attach the iris data set to your commands.

\texttt{attach(iris)}

Now, any time you specify a variable, like \texttt{Petal.Length}, R will look
first in the attached data set. You must remember to \texttt{detach()} the data
set when you are finished. If you are doing analyses with multiple data
sets, you risk taking data from or assigning data to the wrong data set.
I'll remind you to detach the iris data set at the end of this exercise
but, if you use the \texttt{attach()} function, get in the habit of using the
\texttt{detach()} function when you have finished with a particular set of
analyses. Now, back to plotting\ldots{}.

First, define a set of colors for plotting. R has many built-in colors.
Type \texttt{colors()} to see the full list. The course website links to a PDF
color chart of the colors available in R. You can use any colors you
want, but the following three work well.

\begin{tabular}{@{}L{3.5in}L{2in}@{}}
\texttt{mycolors \textless{}- c('purple4','orange2','green4')} & \# \texttt{?c} \\
\end{tabular}

You created a variable called \texttt{mycolors} and assigned three of R's
built-in colors to it. When you want to use those colors, you can use
your \texttt{mycolors} variable, such as in the next call to \texttt{plot()}.

\begin{tabular}{@{}L{3.75in}L{2in}@{}}
\texttt{plot(Petal.Length, Sepal.Length, col=mycolors)} & \# add the \emph{col}ors
argument.\\
\end{tabular}

This graph is not much better. The colors are intermixed yet you know
that the petal lengths are significantly different. The plot function
used each color in the order of rows in the data set. That is, \texttt{purple4}
was assigned to the first row, \texttt{orange2} to the second row, and \texttt{green4} to
the third row. The colors repeated again for the next three rows and so
on to the end of the data set.

\texttt{plot(Petal.Length, Sepal.Length, col=mycolors{[}Species{]})}

Now you can distinguish among the three species. The \texttt{{[}Species{]}} part
of the argument tells the plot function to associate one color with one
species. However, the open circles are somewhat hard to see so try a
solid circle. Notice in the function below that \texttt{col} is changed to \texttt{bg}.

\texttt{plot(Petal.Length, Sepal.Length, bg=mycolors{[}Species{]}, pch=21)} \#
\texttt{?pch}

Notice the circles have a black edge and the interior (or the
\emph{b}ack\emph{g}round) is colored. Filled symbols use col to define
the edge and bg to define the \emph{b}ack\emph{g}round fill color. You
can add the \texttt{col=mycolors{[}Species{]}} argument (leaving in the \texttt{bg=}
argument) and have solid circles of a single color, without black
edging. I think the black edge helps the symbols to stand out but you
may use whatever appeals to you.

If you printed this graph on a non-color printer, you would not be able
to tell which points belonged to which species. A colorblind person may
also have difficulty distinguishing among the colors, depending on your
color choices. Follow good practice by using different symbols for each
species. 

\texttt{plot(Petal.Length, Sepal.Length, bg=mycolors{[}Species{]},
pch=c(21,22,23){[}Species{]})}

Now, even in grey scale, you can tell which points belong to which
species.

The labels for the X- and Y-axes are not informative. They should
include centimeters as the unit of measurement.

\texttt{plot(Petal.Length, Sepal.Length, bg=mycolors{[}Species{]},
pch=c(21,22,23){[}Species{]}, \\xlab='Petal Length (cm)', ylab='Sepal
Length (cm)')}

Many figures include a title. Add a title with the main argument
in the \texttt{plot()} function.

\texttt{plot(Petal.Length, Sepal.Length, bg=mycolors{[}Species{]}, pch=c(21,22,23){[}Species{]}, \\
xlab='Petal Length (cm)', ylab='Sepal Length (cm)', main = 'Sepal Length vs. \\
Petal Length in 3 Species of Irises')}

%Figures use legends to explain any colors and symbols. Add a legend to
%the figure. Following the \texttt{plot()} function, you will use the legend()
%function. \textbf{Important}: \textbf{After the plot window appears do
%not close it}. Move it to the back to type the command to build the
%legend. If you close the window, you will have to rebuild the plot.

%\texttt{plot(Petal.Length, Sepal.Length, bg=mycolors{[}Species{]}, pch=c(21,22,23){[}Species{]}, \\
%xlab='Petal Length (cm)', ylab='Sepal Length (cm)')}

%\texttt{legend(1.2, 7.8, legend=c('Iris setosa','I. versicolor','I. virginica'),
%pch=c(21,22,23), pt.bg=mycolors, text.font=3)}

%Placement of the legend is often a matter of trial and error. The first
%two numbers (1.2, 7.8) give the coordinates of the upper left corner of
%the legend box with respect to the X- and Y-axes. The text.font=3
%argument specifies italics, as appropriate for the scientific names. You
%should be able to figure out by now what pt.bg does.

You can also place text, arrows or other items anywhere on the graph.
For fun, add your name and your partner's name to the lower right of the
figure. Like the legend() function, the text() function needs X and Y
coordinates based on the two axes. For example:

\begin{tabular}{@{}L{4.25in}L{1.5in}@{}}
\texttt{text(5, 4.7, 'Your name\textbackslash{}nYour partner's name', pos=4)} & \# ?text\\

\end{tabular}

The first two numbers are the X and Y coordinates.  The \texttt{\textbackslash{}n} puts the second name on a \emph{n}ew line, which probably fits better in your graph. \texttt{pos=4} puts the text to the right of the specified coordinates.

You can save your graph by choosing ``Save As\ldots{}'' from the File
menu.

\begin{tabular}{@{}L{1in}L{4.5in}@{}}
\texttt{detach(iris)} & \# I said I would remind you to detach the iris data set
when you were finished. \\
\end{tabular}

To get a sense of the variety of plots and legends that can be created
type:

\texttt{demo(graphics)}

\texttt{demo(images)}

\texttt{example(legend)}

Most, if not all, of the exercises you will do in this course will
stress creating presentation quality figures using various plot()
functions and arguments so practice using them.

For further learning, for added practice, and for your first grade, you
will receive an assignment to complete.


\end{document}  