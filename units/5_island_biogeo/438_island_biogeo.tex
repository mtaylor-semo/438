%!TEX TS-program = lualatex
%!TEX encoding = UTF-8 Unicode

\documentclass[11pt, addpoints]{exam}

%\printanswers

\usepackage{graphicx}
	\graphicspath{{/Users/goby/Pictures/teach/438/exercises/}} % set of paths to search for images

\usepackage{geometry}
\geometry{letterpaper, bottom=1in}                   
%\geometry{landscape}                % Activate for for rotated page geometry
\usepackage[parfill]{parskip}    % Activate to begin paragraphs with an empty line rather than an indent
\usepackage{amssymb, amsmath}
\usepackage{mathtools}
	\everymath{\displaystyle}

\usepackage{fontspec}
\setmainfont[Ligatures={TeX}, BoldFont={* Bold}, ItalicFont={* Italic}, BoldItalicFont={* BoldItalic}, Numbers={Proportional}]{Linux Libertine O}
\setsansfont[Scale=MatchLowercase,Ligatures=TeX]{Linux Biolinum O}
\setmonofont[Scale=MatchLowercase]{Inconsolatazi4}
\usepackage{microtype}

\usepackage{unicode-math}
\setmathfont[Scale=MatchLowercase]{Asana Math}
%\setmathfont[Scale=MatchLowercase]{XITS Math}

% To define fonts for particular uses within a document. For example, 
% This sets the Libertine font to use tabular number format for tables.
\newfontfamily{\tablenumbers}[Numbers={Monospaced}]{Linux Libertine O}
\newfontfamily{\libertinedisplay}{Linux Libertine Display O}
 
\usepackage{booktabs}
%\usepackage{tabularx}
\usepackage{longtable}
%\usepackage{siunitx}
\usepackage{array}
\newcolumntype{L}[1]{>{\raggedright\let\newline\\\arraybackslash\hspace{0pt}}p{#1}}
\newcolumntype{C}[1]{>{\centering\let\newline\\\arraybackslash\hspace{0pt}}p{#1}}
\newcolumntype{R}[1]{>{\raggedleft\let\newline\\\arraybackslash\hspace{0pt}}p{#1}}

\usepackage{enumitem}
\setlist{leftmargin=*}
\setlist[1]{labelindent=\parindent}
\setlist[enumerate]{label=\textsc{\alph*}.}
\setlist[itemize]{label=\color{gray}\textbullet}

\usepackage{hyperref}
%\usepackage{placeins} %PRovides \FloatBarrier to flush all floats before a certain point.
%\usepackage{hanging}

\usepackage[sc]{titlesec}

\renewcommand{\solutiontitle}{\noindent}
\unframedsolutions
\SolutionEmphasis{\bfseries}

\pagestyle{headandfoot}
\firstpageheader{BI 438: Biogeography}{}{\ifprintanswers\textbf{KEY}\else Name: \enspace \makebox[2.5in]{\hrulefill}\fi}
\runningheader{}{}{\footnotesize{pg. \thepage}}
\footer{}{}{}
\runningheadrule

\unframedsolutions
\renewcommand{\solutiontitle}{}
\SolutionEmphasis{\bfseries}

\newcommand*\AnswerBox[3]{%
	\parbox[t][#1]{0.92\textwidth}{%
		\begin{solution}#3\end{solution}}
	\vspace*{\stretch{#2}}
}

\newenvironment{AnswerPage}[2]
{\begin{minipage}[t][#1]{0.92\textwidth}%
		\begin{solution}}
		{\end{solution}\end{minipage}
	\vspace*{\stretch{1}}}

\newlength{\basespace}
\setlength{\basespace}{4\baselineskip}

\begin{document}

\subsection*{Island Biogeography (\numpoints~points)}
The goal of this exercise is for you to explore the relationship of
island isolation and area with species richness. Along the way, you will
look at other island-based relationships as well.

The \textsc{url} to access the exercise is \url{https://semobio.shinyapps.io/island}. You will also find an “Island Biogeeography (Shiny)” link on our Canvas page that you can click instead.

\textbf{Important!} This exercise is different than the other online exercises you have completed. You will explore the data online but
the questions are listed below. Type your answers in to a Word document
and upload it to the drop box. You will not enter your answers online nor will you download a finished report.

You can use the Next and Prev buttons to navigate or choose a tab.

You will first explore basic island biogeography predictions based on island size, island distance (isolation). Then, you will explore similar relationships for birds of the Galapagos Islands.

\textit{Read the online information on linear regression on the Size and Distance tab very carefully so 
that you understand how to interpret the results.}

\subsubsection*{Species Richness}

\begin{questions}

\question[10]
For this section, you will view relationships between species richness and classic measures of island biogeography, such as area and distance. View each data set provided in the Data Set dropdown menu (and listed below). For each data set, tell whether there is a significant relationship between the variable shown on the x-axis and species richness. Report both the \textit{p} and \textit{R²} values. Answer any other questions provided. You will not need to report the results for the Caribbean Herps as that was given to you in the example.

Be sure to read the “About the data” section provided for each data set so that you can better interpret the results. 

\begin{enumerate}
	\item Florida Beetles

	\begin{enumerate}
		\item  Area \textit{R²:} 
	
		\item Area \textit{p:}
		\item  Distance \textit{R²:} 
		
		\item Distance \textit{p:}
		
		\item Is each variable a significant predictor of species richness for Long-horned Beetles in the Florida Keys?
	\end{enumerate}
	
	\item Raja Ampat Trees

	\begin{enumerate}
		\item  Area \textit{R²:} 
	
		\item Area \textit{p:}

		\item  Distance \textit{R²:} 
	
		\item Distance \textit{p:}
	
		\item Is each variable a significant predictor of species richness for trees in the Raja Ampat archipelago?
	\end{enumerate}

	\item Aleutian Island Plants. The Aleutian Island extend from Alaska to Katchatka Peninsula in Siberia so there are two possible mainland sources for distance.

	\begin{enumerate}
		\item  Area \textit{R²:} 
	
		\item Area \textit{p:}
	
		\item  Distance from Alaska \textit{R²:} 
	
		\item Distance from Alaska \textit{p:}
	
		\item  Distance from Kamchatka \textit{R²:} 

		\item Distance from Kamchatka \textit{p:}
		
		\item Which variable(s) are significant predictors of species richness? Which are not?
		
	\end{enumerate}

	\item Montaine Mammals. This study considered distance from two possible sources: nearby
	mountain tops and also much larger mountain ranges (mainland) to the east or west.

	\begin{enumerate}
		\item  Area \textit{R²:} 
	
		\item Area \textit{p:}
	
		\item  Distance between mountains \textit{R²:} 
	
		\item Distance between mountains \textit{p:}
	
		\item  Distance from mainland \textit{R²:} 
	
		\item Distance from mainland \textit{p:}
	
		\item Which variable(s) are significant predictors of species richness? Which are not?
	
	\end{enumerate}

	\item Arboreal Arthropods. Read the “About the data” section carefully.

\begin{enumerate}
	\item  Area \textit{R²:} 
	
	\item Area \textit{p:}
	
	\item Are the data consistent with the hypothesis of richness returning to equilibrium following a \textit{reduction} of island size?
	
\end{enumerate}

\end{enumerate}

\question[10]
For the variables you identified as non-significant predictors for Aleutian Island plants and Montaine Mammals, provided a reasonably explanation of why. Explain your reasoning.

Aleutian Plants: \bigskip

Montaine Mammals:\bigskip


\question
\textbf{What do you predict?} Where in the U.S. will species richness be highest? Will it be north of the glacial maximum? Will richness be highest in the interior highlands? The eastern or central highlands? Will richness be highest on the coastal plain? Explain why you think so. (Enter your answers in a separate document. You do not need to type the question. Just enter the question number and your answer.)


\subsubsection*{Galapagos Islands}

\fullwidth{%
	Begin with the “Birds” data set. View both the “Birds per Island” (number of bird species on each island) and “Islands per Bird” (number of islands occupied by each bird species). You can click on the diamonds next to each column header in the table to sort a column in ascending or descending order. Use the Previous, Next or numbered buttons below the table to see additional entries.
}

\question[2] 
Which islands have the fewest (10 or fewer) species?

\question[2]
Which islands have the most (15 or more) species? Do any islands have all 24 species?

\question[2]
Look at the islands on the provided map. Describe the patterns you detect, in terms of island size and distance from other islands, for islands with the fewest and greatest number of species.

\question[2]
How many species are found on Floreana and on Marchena? They are islands of similar size. Why do you think they have very different numbers of species? Discuss possibilities with your group.

\question[2]
How many species are found on North Seymour? What islands have the same number of species? North Seymour is the second smallest island listed. Why do you think North Seymour has so many species? Discuss possibilities with your group.

\question[2] Which species are endemic to a single island? 

\question[2]
Which species are found on the greatest number of islands? Are any species found on all 16 islands? (You can answer this question most easily by choosing the “Islands” data set.)


\fullwidth{%
	View the “Islands” data set if you haven't already switched. Notice the regression data appears.
}

\question[2]
Is island area a significant predictor for the number of species present on an island?

\question[2]
Is island elevation a significant predictor for the number of species present on an island?

\question[2]
The slopes of the area and elevation trend lines are very similar. Why do you think this is? (\textsc{Hint:} Study the map. Lighter brown on the islands indicate higher elevations. Darker green is closer to sea level.)

\question[5]
Do you think island area \textit{and} elevation could interact to increase species richness in a way that area alone would not explain? For example, assume you have two islands of similar size but one has much higher elevation (and richness) than the flatter island. How could elevation explain the additional richness than area alone would predict.  Explain with a reasoned hypothesis.

\end{questions}

\end{document}  