\documentclass[xcolor=svgnames]{beamer}
%\documentclass[xcolor=dvipsnames]{beamer}
%\usetheme{default}
\usepackage{pgf,pgfpages}
\usepackage{amsmath}
\usepackage{graphicx}
	\graphicspath{{img/}} % set of paths to search for images
\usepackage{units}
\usepackage{booktabs}
%\usepackage[utf8]{inputenc}
\usepackage[T1]{fontenc}
\usepackage{helvet}
\usepackage{tikz}
	\tikzstyle{every picture}+=[remember picture,overlay]


%\mode<handout> 	% Use this to make PDF with all overlays on one slide.
\mode<presentation>
{
  \usetheme{MST}
  \setbeamercovered{invisible}
  \setbeamertemplate{items}[square]
}

\usefonttheme[onlymath]{serif}
\usecolortheme[named=myBlue]{structure}

%% Uncomment the next two lines to make slides with note lines
%\usepackage{handoutWithNotes}
%\pgfpagesuselayout{3 on 1 with notes}[letterpaper,border shrink=5mm]


\title{Island Biogeography}

\begin{document}

\section{Insular Habitats}
\subsection{What are Islands?}

{
\usebackgroundtemplate{\includegraphics[width=\paperwidth]{island_intro.jpg}
}
\begin{frame}[b]{\textcolor{yellow} {Island Biogeography}}
	\hfill \textcolor{yellow}{Rock Islands, Republic of Palau}
\end{frame}
}

{
\usebackgroundtemplate{\includegraphics[width=\paperwidth]{patch_reefs.jpg}
}
\begin{frame}[b]{\textcolor{yellow} {Coral patch reefs are ``islands.''}}
	\hfill \textcolor{yellow}{Great Barrier Reef, Australia}
\end{frame}
}

{
\usebackgroundtemplate{\includegraphics[width=\paperwidth]{big_oak_tree_sp.jpg}
}
\begin{frame}[b]{\textcolor{white} {Fragmented landscapes are ``islands.''}}
	\hfill \textcolor{white}{Big Oak Tree State Park, Missouri}
\end{frame}
}

{
\usebackgroundtemplate{\includegraphics[width=\paperwidth]{pothole_lakes.jpg}
}
\begin{frame}[b]{\textcolor{white} {Lakes and springs are ``islands.''}}
	\hfill \textcolor{white}{Pleistocene pothole lakes, Siberia}
\end{frame}
}

{
\usebackgroundtemplate{\includegraphics[width=\paperwidth]{cave.jpg}
}
\begin{frame}[b]{\textcolor{white} {Caves are ``islands.''}}
	\hfill \textcolor{white}{Quadirikiri Cave, New Zealand}
\end{frame}
}

\begin{frame}{What do these have in common with islands?}
	\begin{columns}[T]
		\begin{column}{0.5\textwidth}
			\centering
			\includegraphics[height=0.4\textheight]{patch_reefs.jpg}\\
			\includegraphics[height=0.4\textheight]{big_oak_tree_sp.jpg}
		\end{column}
		\begin{column}{0.5\textwidth}
			\centering
			\includegraphics[height=0.4\textheight]{pothole_lakes.jpg}\\
			\includegraphics[height=0.4\textheight]{cave.jpg}
		\end{column}	
	\end{columns}
\end{frame}

\section{Early Studies}

\subsection{Relative Abundance}

\begin{frame}{Most species in a community are uncommon or rare.}
	\centering
		\includegraphics[width=0.9\textwidth]{species_relative_abundance}\\
	\pause
	\begin{block}{}
		\centering 
		Why does low abundance matter for taxa that live on islands?
	\end{block}
\end{frame}

\subsection{Species Richness}

\begin{frame}{Species richness increases with island area.}
	\centering
		\includegraphics[width=0.9\textwidth]{species_island_area}
%	\begin{tikzpicture}
%		\node at (-2.5,1.75) [black] {$S=\log(c)+z\log(A)$};
%		\node at (-6.5,5.2) [font=\footnotesize] {$S=cA^z$};
%	\end{tikzpicture}
\end{frame}

\begin{frame}{Species richness decreases with island isolation.}
	\centering
		\includegraphics[height=0.75\textheight]{species_island_isolation} \par
\end{frame}

%\begin{frame}{How do these patterns relate to island area and size?}
%	\begin{columns}[T] 
%		\begin{column}{0.5\textwidth}
%			\centering
%		\includegraphics[width=0.9\textwidth]{species_island_area}
%		\end{column}
%		\begin{column}{0.5\textwidth}
%			\centering
%		\includegraphics[width=0.75\textwidth]{species_island_isolation}
%		\end{column}
%	\end{columns}
%\end{frame}

\begin{frame}{How do area and isolation predict species richness?}
	\begin{columns}[T]
		\begin{column}{0.5\textwidth}
			\centering
			\includegraphics[height=0.6\textheight]{area_and_isolation_vert}\\
		\end{column}
		\begin{column}{0.5\textwidth}
			\begin{tikzpicture}
				[mainland/.style={rectangle,draw=black,fill=myBlue!10,thick,minimum size=15mm},
				 bigisland/.style={circle,draw=black,fill=black!10,thick,minimum size=7mm,text width=1cm,align=center},
				 tinyisland/.style={font=\tiny,circle,draw=black,fill=black!10,thick,minimum size=0mm,text width=0.5cm,align=center}]
				\path	node at (3,-0.8)	[mainland]		{Mainland Source}
						node at (1.2,-3)	[bigisland]		{Island 1}
						node at (5,-3)		[tinyisland]	{Island 2}
						node at (1.2,-5.6)	[tinyisland]	{Island 3}
						node at (5,-5.6)	[bigisland]		{Island 4};
			\end{tikzpicture}
		\end{column}		
	\end{columns}
\end{frame}

\section{Developing the Model}
\subsection{A Natural Experiment}

\begin{frame}{Volcanic activity destroyed life on Krakatau Islands.}
	\centering
		\includegraphics[height=0.85\textheight]{krakatau}
\end{frame}

\begin{frame}{Bird richness on the Krakatau Islands reached equilibrium 40 years after the eruption.}
	\centering
	\begin{tabular}{rccc}
		\toprule
		\textbf{Rakata} & Nonmigrant & Migrant & Total\\
		\midrule
		1908 & 13 & 0 & 13 \\
		1919--1921 & 27 & 4 & 31 \\
		1932--1934 & 27 & 3 & 30 \\
		\bottomrule
	\end{tabular}
	
	\vspace{1\baselineskip}
	
	\begin{tabular}{rccc}
		\toprule
		\textbf{Sertung}& Nonmigrant & Migrant & Total\\
		\midrule
		1908 & 1 & 0 & 1 \\
		1919--1921 & 27 & 2 & 29\\
		1932--1934 & 29 & 6 & 34\\
		\bottomrule
	\end{tabular}
\end{frame}

\begin{frame}{Species composition changed over time.}
	\centering
	\begin{tabular}{rccc}
		\toprule
		\textbf{Rakata} & Extinction & Immigration\\
		\midrule
		1908--1919 & 2 & 20 \\
		1921--1932 & 5 & 4  \\
		\bottomrule
	\end{tabular}

	\vspace{1\baselineskip}

	\begin{tabular}{rccc}
		\toprule
		\textbf{Sertung}& Extinction & Immigration\\
		\midrule
		1908--1919 & 0 & 28 \\
		1921--1932 & 2 & 7  \\
		\bottomrule
	\end{tabular}

	\vspace{1\baselineskip}

	\pause
	\begin{block}{}
		\textbf{Species turnover} occurs when immigrant taxa replace extinct taxa.
	\end{block}
\end{frame}

\section{Equilibrium Theory of MacArthur-Wilson}
\subsection{The Model}

\begin{frame}{Immigration and extinction will reach equilibrium.}
	\centering
		\includegraphics[height=0.8\textheight]{immigration_extinction_equilibrium}
\end{frame}

\begin{frame}{Equilibrium points vary by island area and isolation.}
	\centering
		\includegraphics[height=0.8\textheight]{island_equilibrium}
\end{frame}

\subsection{Testing the Model}

\begin{frame}{The model was tested  on tiny islands.}
	\centering
		\includegraphics[width=0.9\textwidth]{fumigation.jpg}
\end{frame}

\begin{frame}{Results showed rapid arthropod recovery.}
	\centering
		\includegraphics[width=0.9\textwidth]{arthropod_recovery}
\end{frame}

\subsection{Modifying the Model}

\begin{frame}{Species turnover was lower than predicted on ``near'' islands.}
	\centering
	\begin{columns}[T]
		\begin{column}{0.4\textwidth}
			\includegraphics[height=0.5\textheight]{island_equilibrium}
		\end{column}
		\begin{column}{0.6\textwidth}
			\includegraphics[height=0.5\textheight]{rescue_effect}
		\end{column}
	\end{columns}
	\pause
	\begin{tikzpicture}
		\draw	[<-,very thick]	(-3.2,1.36) -- (1.7,2) {};
		\draw	[<-,very thick,color=red!50!black]	(-3.5,1.6) -- (4.8,3) {};
	\end{tikzpicture}
	\pause
	\begin{block}{Rescue Effect}
		High immigration rates reduce extinction rates.
	\end{block}
\end{frame}

\begin{frame}{Species turnover was higher than predicted on ``large'' islands.}
	\centering
	\begin{columns}[T]
		\begin{column}{0.4\textwidth}
			\includegraphics[height=0.5\textheight]{island_equilibrium}
		\end{column}
		\begin{column}{0.6\textwidth}
			\includegraphics[height=0.5\textheight]{rescue_effect}
		\end{column}
	\end{columns}
	\begin{tikzpicture}
		\draw	[<-,very thick]	(-3.65,1.1) -- (0.25,2) {};
	\end{tikzpicture}
	\pause
	\begin{block}{Target Effect}
		Large islands have high immigration.
	\end{block}
\end{frame}

\begin{frame}{Islands have sequential equilibrium points over time.}
	\centering
		\includegraphics[width=0.9\textwidth]{sequential_equilibria}
\end{frame}


\section{New Models of Island Biogeography}
\subsection{A Model for Speciation}

\begin{frame}{New models of island biogeography are emerging.}
	\centering
			\includegraphics[height=0.8\textheight]{new_island_paradigm}
\end{frame}

\section{Island Communities}
\subsection{Community Assembly Rules}

{
\usebackgroundtemplate{\includegraphics[width=\paperwidth]{alligator_island.jpg}
}
\begin{frame}[b]{Are island communities random assemblages of species from the mainland source?}
\pause
If not, then what ecological abilities or traits should you consider?
\end{frame}
}

\section{Island Communities}
\subsection{Community Assembly Rules}

\begin{frame}{Filters prevent or allow dispersal of specific taxa.}
	\centering
		\includegraphics[width=0.9\textwidth]{filter_alaska}\\
\end{frame}

\begin{frame}{Filters prevent or allow dispersal of specific taxa.}
	\centering
		\includegraphics[width=0.9\textwidth]{filter_sunda}\\
\end{frame}

\begin{frame}{Species richness shows a hierarchical pattern of nesting.}
	\centering
		\includegraphics[width=0.9\textwidth]{nestedness_ants} \\
\end{frame}

\begin{frame}{Island isolation determines nestedness by selective immigration.}
	\centering
		\includegraphics[width=0.9\textwidth]{nestedness_selective_immigration} \\
\end{frame}

\begin{frame}{Island area determines nestedness by selective extinction.}
	\centering
		\includegraphics[width=0.9\textwidth]{nestedness_selective_extinction_before} \\
\end{frame}

\begin{frame}{Island area determines nestedness by selective extinction.}
	\centering
		\includegraphics[width=0.9\textwidth]{nestedness_selective_extinction_after} \\
\end{frame}

\begin{frame}{Selective immigration and selective extinction determine final nestedness pattern.}
	\centering
		\includegraphics[width=0.7\textwidth]{nestedness_selective_immigration_extinction} \\
\end{frame}

\begin{frame}{Immigration and extinction interact to determine presence or absence.}
	\centering
		\includegraphics[width=0.6\textwidth]{insular_distribution_function_top}\\
		\pause
		\begin{block}{}
			\textbf{Insular distribution function} shows interaction of immigration and extinction.
		\end{block}
\end{frame}

\begin{frame}{Insular distribution function of masked shrew.}
	\centering
		\includegraphics[width=0.7\textwidth]{insular_distribution_function_shrews}\\
	\pause
	\begin{tikzpicture}
		\draw [very thick] (-2.3,1.8) -- (3.6,5.3);
	\end{tikzpicture}
\end{frame}

\begin{frame}{Interspecific interactions influence presence or absence.}
\textbf{GET RID OF THIS SLIDE BECAUSE PICTURE SLIDES REPLACE IT?}
	\begin{itemize}
		\item Ecologically similar species have mutually exclusive distributions.
		\pause
		\vfill \item Insular species exhibit ecological release.
		\pause
		\vfill \item Populations on species-poor islands have high densities.
	\end{itemize}
\end{frame}

\begin{frame}{Ecologically similar species have non-overlapping distributions.}
	\centering
		\includegraphics[height=0.8\textheight]{checkerboard_distribution}\\
\end{frame}

\begin{frame}{Checkerboard pattern of various bird species on the Bismarck Islands.}
	\centering
		\includegraphics[width=0.8\textwidth]{insular_distribution_birds}\\
\end{frame}

\begin{frame}{Distribution of the Mangrove Golden Whistler on the Bismarck Islands.}
	\centering
		\includegraphics[width=0.8\textwidth]{insular_distribution_Pmelanura}\\
\end{frame}

\begin{frame}{Distribution of the Australian Golden Whistler on the Bismarck Islands.}
	\centering
		\includegraphics[width=0.8\textwidth]{insular_distribution_Ppectoralis}\\
\end{frame}

\begin{frame}{The two whistlers do not co-occur anywhere on the Bismarck Islands.}
	\centering
		\includegraphics[width=0.8\textwidth]{insular_distribution_pachychephala}\\
\end{frame}

\begin{frame}{Distribution of the Yellow-Bibbed Fruit Dove on the Bismarck Islands.}
	\centering
		\includegraphics[width=0.8\textwidth]{insular_distribution_Psolom}\\
\end{frame}

\begin{frame}{Distribution of the White-Bibbed Fruit Dove on the Bismarck Islands.}
	\centering
		\includegraphics[width=0.8\textwidth]{insular_distribution_Privoli}\\
\end{frame}

\begin{frame}{The yellow- and white-bibbed fruit doves rarely co-occur.}
	\centering
		\includegraphics[width=0.8\textwidth]{insular_distribution_ptilinopus}\\
\end{frame}

\begin{frame}{Given species A, M, N and R, how many unique arrangements, including single species, are possible?}
	\pause
	\centering
		\includegraphics[width=0.8\textwidth]{checkerboard_cuckoo_table}\\
\end{frame}

\begin{frame}{Elacatinus cleaner distribution}
\textbf{ADD GOBY DISTRIB SLIDE?}
\end{frame}

\begin{frame}{Ecological release broadens the population niche.}
	\begin{columns}[T]
		\begin{column}{0.5\textwidth}
		\centering
			\includegraphics[width=0.9\textwidth]{niche_shift_voles}\\
		\end{column}
		\begin{column}{0.5\textwidth}
		\centering
			\includegraphics[width=0.9\textwidth]{niche_shift_shrews}\\
		\end{column}
	\end{columns}
	\begin{block}{}
	Prey Vole: Ecological release in \textit{absence} of predator shrew.\\
	Predator Shrew: Ecological release in \textit{presence} of prey vole.
	\end{block}
\end{frame}

\begin{frame}{Summary of basic island biogeography concepts.}
	\begin{columns}[T]%
		\begin{column}{0.5\textwidth}%
		\begin{itemize}%
			\item<1-| alert@1> Insular habitats are any habitat surrounded by unsuitable habitat.
			\item<2-| alert@2> Communities assemble by complex interaction of
			\begin{itemize}%
				\item<3-| alert@3> island characteristics,
				\item<4-| alert@4> biogeographic and evolutionary processes,
				\item<5-| alert@5> and species interactions.
			\end{itemize}%
		\end{itemize}%
		\end{column}%
		\begin{column}{0.5\textwidth}%
			\includegraphics<1-2>[width=0.9\textwidth]{Global_islands.jpg}
			\includegraphics<3>[width=0.9\textwidth]{island_equilibrium}
			\includegraphics<4>[width=1\textwidth]{new_island_paradigm}
			\includegraphics<5>[width=0.9\textwidth]{insular_distribution_birds_b}
		\end{column}%
	\end{columns}%
\end{frame}


\end{document}
