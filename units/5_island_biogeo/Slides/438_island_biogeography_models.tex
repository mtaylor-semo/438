%!TEX TS-program = lualatex
%!TEX encoding = UTF-8 Unicode

%\documentclass[t]{beamer}

%%%% HANDOUTS For online Uncomment the following four lines for handout
\documentclass[t,handout]{beamer}  %Use this for handouts.
\usepackage{handoutWithNotes}
\pgfpagesuselayout{3 on 1 with notes}[letterpaper,border shrink=5mm]
%	\setbeamercolor{background canvas}{bg=black!5}


%%% Including only some slides for students.
%%% Uncomment the following line. For the slides,
%%% use the labels shown below the command.
%\includeonlylecture{student}

%% For students, use \lecture{student}{student}
%% For mine, use \lecture{instructor}{instructor}


%\usepackage{pgf,pgfpages}
%\pgfpagesuselayout{4 on 1}[letterpaper,border shrink=5mm]

% FONTS
\usepackage{fontspec}
\def\mainfont{Linux Biolinum O}
\setmainfont[Ligatures={Common,TeX}, Contextuals={NoAlternate}, BoldFont={* Bold}, ItalicFont={* Italic}, Numbers={Proportional}]{\mainfont}
%\setmonofont[Scale=MatchLowercase]{Inconsolata} 
\setsansfont[Scale=MatchLowercase]{Linux Biolinum O} 
\usepackage{microtype}

\usepackage{graphicx}
	\graphicspath{%
	{/Users/goby/Pictures/teach/438/lectures/}%
	{/Users/goby/Pictures/teach/common/}}%
%	{img/}} % set of paths to search for images

\usepackage{amsmath,amssymb}

%\usepackage{units}

\usepackage{booktabs}
\usepackage{multicol}
%	\setlength{\columnsep=1em}

%\usepackage{textcomp}
%\usepackage{setspace}
\usepackage{tikz}
	\tikzstyle{every picture}+=[remember picture,overlay]

\mode<presentation>
{
  \usetheme{Lecture}
  \setbeamercovered{invisible}
  \setbeamertemplate{items}[square]
}

\usepackage{calc}
\usepackage{hyperref}

\newcommand\HiddenWord[1]{%
	\alt<handout>{\rule{\widthof{#1}}{\fboxrule}}{#1}%
}


\begin{document}

%\lecture{instructor}{instructor}
\lecture{student}{student}

{
\usebackgroundtemplate{\includegraphics[width=\paperwidth]{island_intro.jpg}
}
\begin{frame}[b]{\textcolor{yellow} {Island Biogeography}}
	\hfill \textcolor{yellow}{Rock Islands, Republic of Palau}
\end{frame}
}

{
\usebackgroundtemplate{\includegraphics[width=\paperwidth]{patch_reefs.jpg}
}
\begin{frame}[b]{\textcolor{yellow} {Coral patch reefs are ``islands.''}}
	\hfill \textcolor{yellow}{Great Barrier Reef, Australia}
\end{frame}
}

{
\usebackgroundtemplate{\includegraphics[width=\paperwidth]{big_oak_tree_sp.jpg}
}
\begin{frame}[b]{\textcolor{white} {Fragmented landscapes are ``islands.''}}
	\hfill \textcolor{white}{Big Oak Tree State Park, Missouri}
\end{frame}
}

{
\usebackgroundtemplate{\includegraphics[width=\paperwidth]{pothole_lakes.jpg}
}
\begin{frame}[b]{\textcolor{white} {Lakes and springs are ``islands.''}}
	\hfill \textcolor{white}{Pleistocene pothole lakes, Siberia}
\end{frame}
}

{
\usebackgroundtemplate{\includegraphics[width=\paperwidth]{cave.jpg}
}
\begin{frame}[b]{\textcolor{white} {Caves are ``islands.''}}
	\hfill \textcolor{white}{Quadirikiri Cave, New Zealand}
\end{frame}
}

\begin{frame}{What do these have in common with islands?}
	\begin{columns}[T]
		\begin{column}{0.5\textwidth}
			\centering
			\includegraphics[height=0.4\textheight]{patch_reefs.jpg}\\
			\includegraphics[height=0.4\textheight]{big_oak_tree_sp.jpg}
		\end{column}
		\begin{column}{0.5\textwidth}
			\centering
			\includegraphics[height=0.4\textheight]{pothole_lakes.jpg}\\
			\includegraphics[height=0.4\textheight]{cave.jpg}
		\end{column}	
	\end{columns}
\end{frame}

\begin{frame}{Most species in a community are uncommon or rare.}
	\centering
		\includegraphics[width=0.9\textwidth]{species_relative_abundance}\\
	\bgroup
		\centering 
		Why does low abundance matter for taxa that live on islands?
	\egroup
\end{frame}

\begin{frame}{Species richness increases with island \highlight{area.}}
	\centering
		\includegraphics[width=0.9\textwidth]{species_island_area}
%	\begin{tikzpicture}
%		\node at (-2.5,1.75) [black] {$S=\log(c)+z\log(A)$};
%		\node at (-6.5,5.2) [font=\footnotesize] {$S=cA^z$};
%	\end{tikzpicture}
\end{frame}

\begin{frame}{Species richness decreases with island \highlight{isolation.}}
	\centering
		\includegraphics[height=0.75\textheight]{species_island_isolation} \par
\end{frame}

%\begin{frame}{How do these patterns relate to island area and size?}
%	\begin{columns}[T] 
%		\begin{column}{0.5\textwidth}
%			\centering
%		\includegraphics[width=0.9\textwidth]{species_island_area}
%		\end{column}
%		\begin{column}{0.5\textwidth}
%			\centering
%		\includegraphics[width=0.75\textwidth]{species_island_isolation}
%		\end{column}
%	\end{columns}
%\end{frame}

\begin{frame}{How do area and isolation predict species richness?}
	\begin{columns}[T]
		\begin{column}{0.5\textwidth}
			\centering
			\includegraphics[height=0.6\textheight]{area_and_isolation_vert}\\
		\end{column}
		\begin{column}{0.5\textwidth}
			\begin{tikzpicture}
				[mainland/.style={rectangle,draw=black,fill=blue5!10,thick,minimum size=15mm},
				 bigisland/.style={circle,draw=black,fill=black!10,thick,minimum size=7mm,text width=1cm,align=center},
				 tinyisland/.style={font=\tiny,circle,draw=black,fill=black!10,thick,minimum size=0mm,text width=0.6cm,align=center}]
				\path	node at (3,-0.8)	[mainland]		{Mainland Source}
						node at (1.2,-3)	[bigisland]		{Island 1}
						node at (5,-3)		[tinyisland]	{Island 2}
						node at (1.2,-5.6)	[tinyisland]	{Island 3}
						node at (5,-5.6)	[bigisland]		{Island 4};
			\end{tikzpicture}
		\end{column}		
	\end{columns}
\end{frame}

\begin{frame}{Volcanic activity destroyed life on Krakatau Islands.}
	\centering
		\includegraphics[height=0.85\textheight]{krakatau}
\end{frame}

\begin{frame}{Bird richness on the Krakatau Islands reached equilibrium 40 years after the eruption.}
	\centering
	\begin{tabular}{rccc}
		\toprule
		\textbf{Rakata} & Nonmigrant & Migrant & Total\\
		\midrule
		1908 & 13 & 0 & 13 \\
		1919--1921 & 27 & 4 & 31 \\
		1932--1934 & 27 & 3 & 30 \\
		\bottomrule
	\end{tabular}
	
	\vspace{1\baselineskip}
	
	\begin{tabular}{rccc}
		\toprule
		\textbf{Sertung}& Nonmigrant & Migrant & Total\\
		\midrule
		1908 & 1 & 0 & 1 \\
		1919--1921 & 27 & 2 & 29\\
		1932--1934 & 29 & 6 & 34\\
		\bottomrule
	\end{tabular}
\end{frame}

\begin{frame}{\highlight{Species turnover} occurs when immigrant taxa replace extinct taxa.}
	\centering
	\begin{tabular}{rccc}
		\toprule
		\textbf{Rakata} & Extinction & Immigration\\
		\midrule
		1908--1919 & 2 & 20 \\
		1921--1932 & 5 & 4  \\
		\bottomrule
	\end{tabular}

	\vspace{1\baselineskip}

	\begin{tabular}{rccc}
		\toprule
		\textbf{Sertung}& Extinction & Immigration\\
		\midrule
		1908--1919 & 0 & 28 \\
		1921--1932 & 2 & 7  \\
		\bottomrule
	\end{tabular}

%	\vspace{1\baselineskip}

%	\pause
%	\bgroup
		
%	\egroup
\end{frame}

\begin{frame}{Immigration and extinction should reach equilibrium.}
	\centering
		\includegraphics[height=0.8\textheight]{immigration_extinction_equilibrium}
\end{frame}

\begin{frame}{Equilibrium points vary by island \highlight{area and isolation.}}
	\centering
		\includegraphics[height=0.8\textheight]{island_equilibrium}
\end{frame}

\begin{frame}{The model was tested  on tiny Bahamian islands.}
	\centering
		\includegraphics[width=0.9\textwidth]{fumigation.jpg}
\end{frame}

\begin{frame}{Arthropods rapidly recolonized the islands.}
	\centering
		\includegraphics[width=0.9\textwidth]{arthropod_recovery}
\end{frame}

\begin{frame}{The \highlight{rescue effect} can decrease turnover rate on near islands because high immigration rates reduce extinction rates.}
	\centering
	\begin{columns}[T]
		\begin{column}{0.4\textwidth}
			\includegraphics[height=0.5\textheight]{island_equilibrium}
		\end{column}
		\begin{column}{0.6\textwidth}
			\includegraphics[height=0.5\textheight]{rescue_effect}
		\end{column}
	\end{columns}
%	\pause
	\begin{tikzpicture}
		\draw	[<-,very thick]	(-4.0,1.25) -- (3.3,2) {};
		\draw	[<-,very thick,color=red!50!black]	(-3.5,1.6) -- (4.8,3) {};
	\end{tikzpicture}
%	\pause

	\raggedright
	\hangpara Small-near islands are predicted to have the highest turnover rates (left panel) but actual rate is lower than small-far islands (right panel).

\end{frame}

\begin{frame}{The \highlight{target effect} can increase turnover rate on large islands due to high immigration.}
	\centering
	\begin{columns}[T]
		\begin{column}{0.4\textwidth}
			\includegraphics[height=0.5\textheight]{island_equilibrium}
		\end{column}
		\begin{column}{0.6\textwidth}
			\includegraphics[height=0.5\textheight]{rescue_effect}
		\end{column}
	\end{columns}
	\begin{tikzpicture}
		\draw	[<-,very thick]	(-3.65,1.1) -- (0.25,2) {};
	\end{tikzpicture}

	\raggedright
	
	\hangpara Large-far islands are predicted to have the lowest turnover rate (left panel) but actual rate is higher than large-near islands (right panel).
	

\end{frame}

\begin{frame}{Islands have sequential equilibrium points over time.}
	\centering
		\includegraphics[width=0.9\textwidth]{sequential_equilibria}
\end{frame}


\begin{frame}{New models of island biogeography are emerging.}
	\centering
			\includegraphics[height=0.8\textheight]{new_island_paradigm}
\end{frame}

\end{document}
