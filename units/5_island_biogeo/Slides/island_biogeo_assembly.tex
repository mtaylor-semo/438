%!TEX TS-program = lualatex
%!TEX encoding = UTF-8 Unicode

\documentclass[t]{beamer}

%%%% HANDOUTS For online Uncomment the following four lines for handout
%\documentclass[t,handout]{beamer}  %Use this for handouts.
%\usepackage{handoutWithNotes}
%\pgfpagesuselayout{3 on 1 with notes}[letterpaper,border shrink=5mm]
%	\setbeamercolor{background canvas}{bg=black!5}


%%% Including only some slides for students.
%%% Uncomment the following line. For the slides,
%%% use the labels shown below the command.
%\includeonlylecture{student}

%% For students, use \lecture{student}{student}
%% For mine, use \lecture{instructor}{instructor}

% FONTS
\usepackage{fontspec}
\def\mainfont{Linux Biolinum O}
\setmainfont[Ligatures={Common,TeX}, Contextuals={NoAlternate}, BoldFont={* Bold}, ItalicFont={* Italic}, Numbers={Proportional}]{\mainfont}
\setmonofont[Scale=MatchLowercase]{Inconsolatazi4} 
\setsansfont[Scale=MatchLowercase]{Linux Biolinum O} 
\usepackage{microtype}

\usepackage{graphicx}
	\graphicspath{%
	{/Users/goby/Pictures/teach/438/lectures/}%
	{/Users/goby/Pictures/teach/common/}}%
%	{img/}} % set of paths to search for images

\usepackage{amsmath,amssymb}

%\usepackage{units}

\usepackage{booktabs}
\usepackage{multicol}
%	\setlength{\columnsep=1em}

%\usepackage{textcomp}
%\usepackage{setspace}
\usepackage{tikz}
	\tikzstyle{every picture}+=[remember picture,overlay]

\mode<presentation>
{
  \usetheme{Lecture}
  \setbeamercovered{invisible}
  \setbeamertemplate{items}[square]
}

\usepackage{calc}
\usepackage{hyperref}

\newcommand\HiddenWord[1]{%
	\alt<handout>{\rule{\widthof{#1}}{\fboxrule}}{#1}%
}

\begin{document}

{
\usebackgroundtemplate{\includegraphics[width=\paperwidth]{island_community_assembly}
}
\begin{frame}[b,plain]
	\hfill\color{white}\tiny{Whitsunday Island courtesy Damien Dempsey. Wikimedia Commons.}
\end{frame}
}

\begin{frame}{\highlight{Filters} prevent or allow dispersal of specific taxa.}
	\centering
		\includegraphics[width=0.9\textwidth]{filter_alaska}\\
\end{frame}

\begin{frame}{\highlight{Filters} prevent or allow dispersal of specific taxa.}
	\centering
		\includegraphics[width=0.9\textwidth]{filter_sunda}\\
\end{frame}

\begin{frame}{Species richness shows a hierarchical pattern of \highlight{nesting.}}
	\centering
		\includegraphics[width=0.9\textwidth]{nestedness_ants} \\
\end{frame}

\begin{frame}{Island isolation determines nestedness by selective immigration.}
	\centering
		\includegraphics[width=0.9\textwidth]{nestedness_selective_immigration} \\
\end{frame}

\begin{frame}{Island area determines nestedness by selective extinction.}
	\centering
		\includegraphics[width=0.9\textwidth]{nestedness_selective_extinction_before} \\
\end{frame}

\begin{frame}{Island area determines nestedness by selective extinction.}
	\centering
		\includegraphics[width=0.9\textwidth]{nestedness_selective_extinction_after} \\
\end{frame}

\begin{frame}{Selective immigration and selective extinction determine final nestedness pattern.}
	\centering
		\includegraphics[width=0.8\textwidth]{nestedness_selective_immigration_extinction} \\
\end{frame}

\begin{frame}{Immigration and extinction interact to determine presence or absence.}
	\centering
		\includegraphics[width=0.8\textwidth]{insular_distribution_function_top}\\
\end{frame}

\begin{frame}{\highlight{Insular distribution function} shows interaction of immigration and extinction.}
	\centering
		\includegraphics[width=0.7\textwidth]{insular_distribution_function_top}\\
\end{frame}

\begin{frame}{Insular distribution function of masked shrew.}
	\centering
		\includegraphics[width=0.7\textwidth]{insular_distribution_function_shrews}\\
	\pause
	\begin{tikzpicture}
		\draw [very thick] (-2.3,1.8) -- (3.6,5.3);
	\end{tikzpicture}
\end{frame}

{
\usebackgroundtemplate{\includegraphics[width=\paperwidth]{species_interact.jpg}
}
\begin{frame}[b,plain]
\color{white}\tiny{\textit{Euglossa dilemma} (Green Orchid Bee) photo by Bob Peterson. Flickr Creative Commons.}
\end{frame}
}

\begin{frame}{Ecologically similar species tend to have non-overlapping distributions, like these whistlers.}
	\centering
		\includegraphics[height=0.8\textheight]{checkerboard_distribution}\\
\end{frame}

\begin{frame}{Checkerboard pattern of two whistler bird species on the Bismarck Islands.}
	\centering
		\includegraphics[width=0.9\textwidth]{insular_distribution_birds}\\
\end{frame}

\begin{frame}{Distribution of the Mangrove Golden Whistler on the Bismarck Islands.}
	\centering
		\includegraphics[width=0.9\textwidth]{insular_distribution_Pmelanura}\\
\end{frame}

\begin{frame}{Distribution of the Australian Golden Whistler on the Bismarck Islands.}
	\centering
		\includegraphics[width=0.9\textwidth]{insular_distribution_Ppectoralis}\\
\end{frame}

\begin{frame}{The two whistlers do not co-occur anywhere on the Bismarck Islands.}
	\centering
		\includegraphics[width=0.9\textwidth]{insular_distribution_pachychephala}\\
\end{frame}

\begin{frame}{Distribution of the Yellow-Bibbed Fruit Dove on the Bismarck Islands.}
	\centering
		\includegraphics[width=0.9\textwidth]{insular_distribution_Psolom}\\
\end{frame}

\begin{frame}{Distribution of the White-Bibbed Fruit Dove on the Bismarck Islands.}
	\centering
		\includegraphics[width=0.9\textwidth]{insular_distribution_Privoli}\\
\end{frame}

\begin{frame}{The yellow- and white-bibbed fruit doves rarely co-occur.}
	\centering
		\includegraphics[width=0.9\textwidth]{insular_distribution_ptilinopus}\\
\end{frame}

\begin{frame}{Given species A, M, N and R, how many unique arrangements, including single species, are possible?}
	\pause
	\centering
		\includegraphics[width=0.8\textwidth]{checkerboard_cuckoo_table}\\
\end{frame}

\begin{frame}{\textit{Elacatinus} cleaner gobies do not overlap.}
	\centering
		\includegraphics[width=0.8\textwidth]{elacatinus_cleaner_distrib.pdf}\\
\end{frame}

\begin{frame}{\textit{Elacatinus} sponge-dwelling gobies do not overlap.}
	\centering
		\includegraphics[width=0.8\textwidth]{elacatinus_sponger_distrib.pdf}\\
\end{frame}

\begin{frame}{Ecological release broadens the population niche.}
	\vspace{-1\baselineskip}
	\begin{center}
		\includegraphics[width=0.7\textwidth]{voles}
	\end{center}
	Prey Vole: Ecological release in \textit{absence} of predator shrew.
\end{frame}

\begin{frame}{Summary of island biogeography concepts.}
	\begin{columns}[T]%
		\begin{column}{0.5\textwidth}%
		\begin{itemize}%
			\item<1-| alert@1> Insular habitats are any habitat surrounded by unsuitable habitat.
			\item<2-| alert@2> Communities assemble by complex interaction of
			\begin{itemize}%
				\item<3-| alert@3> island characteristics,
				\item<4-| alert@4> biogeographic and evolutionary processes, and
				\item<5-| alert@5> species interactions.
			\end{itemize}%
		\end{itemize}%
		\end{column}%
		\begin{column}{0.5\textwidth}%
			\includegraphics<1-2>[width=0.9\textwidth]{Global_islands.jpg}
			\includegraphics<3>[width=0.9\textwidth]{island_equilibrium}
			\includegraphics<4>[width=1\textwidth]{new_island_paradigm}
			\includegraphics<5>[width=0.9\textwidth]{insular_distribution_birds_b}
		\end{column}%
	\end{columns}%
\end{frame}

\end{document}
