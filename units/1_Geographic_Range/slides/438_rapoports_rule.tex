%!TEX TS-program = lualatex
%!TEX encoding = UTF-8 Unicode

\documentclass[t]{beamer}

%%%% HANDOUTS For online Uncomment the following four lines for handout
%\documentclass[t,handout]{beamer}  %Use this for handouts.
%\usepackage{handoutWithNotes}
%\pgfpagesuselayout{3 on 1 with notes}[letterpaper,border shrink=5mm]
%	\setbeamercolor{background canvas}{bg=black!5}


%%% Including only some slides for students.
%%% Uncomment the following line. For the slides,
%%% use the labels shown below the command.
%\includeonlylecture{student}

%% For students, use \lecture{student}{student}
%% For mine, use \lecture{instructor}{instructor}


%\usepackage{pgf,pgfpages}
%\pgfpagesuselayout{4 on 1}[letterpaper,border shrink=5mm]

% FONTS
\usepackage{fontspec}
\def\mainfont{Linux Biolinum O}
\setmainfont[Ligatures={Common,TeX}, Contextuals={NoAlternate}, BoldFont={* Bold}, ItalicFont={* Italic}, Numbers={Proportional}]{\mainfont}
\setmonofont[Scale=MatchLowercase]{Inconsolata} 
\setsansfont[Scale=MatchLowercase]{Linux Biolinum O} 
\usepackage{microtype}

\usepackage{graphicx}
	\graphicspath{%
	{/Users/goby/Pictures/teach/438/lectures/}%
	{/Users/goby/Pictures/teach/common/}%}%
	{img/}} % set of paths to search for images

\usepackage{amsmath,amssymb}

%\usepackage{units}

\usepackage{booktabs}
\usepackage{multicol}
%	\setlength{\columnsep=1em}

%\usepackage{textcomp}
%\usepackage{setspace}
%\usepackage{tikz}
%	\tikzstyle{every picture}+=[remember picture,overlay]

\mode<presentation>
{
  \usetheme{Lecture}
  \setbeamercovered{invisible}
  \setbeamertemplate{items}[square]
}

\usepackage{calc}
\usepackage{hyperref}

\newcommand\HiddenWord[1]{%
	\alt<handout>{\rule{\widthof{#1}}{\fboxrule}}{#1}%
}



\begin{document}
%\lecture{instructor}{instructor}
\lecture{student}{student}

\begin{frame}[t,plain]{Our goal for today is to }

%	\hangpara learn how range size varies with latitude, and
	
	\hangpara interpret and explain \highlight{Rapoport's Rule,} and
	
	\hangpara interpret and explain the \highlight{latitudinal diversity gradient.}
	
%	\vspace{4\baselineskip}\pause
	
	\hangpara We'll begin by identifying a latitudinal pattern in the U.S.

\end{frame}

\lecture{instructor}{instructor}

\begin{frame}[t,plain]
	\begin{center}
		\includegraphics[height=0.95\textheight]{range_size_alabama_fishes}
	\end{center}
\end{frame}

\begin{frame}[t,plain]
	\begin{center}
		\includegraphics[height=0.95\textheight]{range_size_tennessee_fishes}
	\end{center}
\end{frame}

\begin{frame}[t,plain]
	\begin{center}
		\includegraphics[height=0.95\textheight]{range_size_kentucky_fishes}
	\end{center}
\end{frame}

\begin{frame}[t,plain]
	\begin{center}
		\includegraphics[height=0.95\textheight]{range_size_illinois_fishes}
	\end{center}
\end{frame}

\begin{frame}[t,plain]
	\begin{center}
		\includegraphics[height=0.95\textheight]{range_size_wisconsin_fishes}
	\end{center}
\end{frame}

\lecture{student}{student}

{
\usebackgroundtemplate{\includegraphics[width=\paperwidth]{range_size_fishes_pattern}}
\begin{frame}[b,plain]{What is the pattern from south to north?}

\end{frame}
}




\begin{frame}[t,plain]{\highlight{Rapoport's rule} states that range size decreases from south to north.}
	\begin{center}
		\includegraphics[width=\textwidth]{rapoport_range_size}
	\end{center}

\end{frame}

\begin{frame}[b,plain]{How does species richness change from south to north?}
	\begin{center}
		\includegraphics[height=0.85\textheight]{rapoport_fishes_richness}
	\end{center}

\end{frame}

\begin{frame}[t,plain]{\highlight{Rapoport's rule} states that richness decreases from south to north.}
	\begin{center}
		\includegraphics[width=\textwidth]{rapoport_richness}
	\end{center}

\end{frame}


\begin{frame}[t,plain]{A pattern similar to Rapoport's is observed for altitude.}
	\begin{center}
		\includegraphics[width=\textwidth]{rapoport_altitude}
	\end{center}
\end{frame}

{
\usebackgroundtemplate{\includegraphics[width=\paperwidth]{range_size_area_mass}}
\begin{frame}[b,plain]{Range size shows a non-random relation to body mass.}

\end{frame}
}

{
\usebackgroundtemplate{\includegraphics[width=\paperwidth]{range_size_area_abundance}}
\begin{frame}[b,plain]{Range size shows a non-random relation to abundance.}

\end{frame}
}

{
\usebackgroundtemplate{\includegraphics[width=\paperwidth]{range_size_range_shape}}
\begin{frame}[b,plain]{Range shape reflects continental features.}

\end{frame}
}

{
\usebackgroundtemplate{\includegraphics[width=\paperwidth]{topo_us}}
\begin{frame}[b,plain]{\textcolor{white}{North American mountain ranges run north-south.}}

\end{frame}
}

{
\usebackgroundtemplate{\includegraphics[width=\paperwidth]{topo_eurasia}}
\begin{frame}[b,plain]{Eurasian mountain ranges run east-west.}

\end{frame}
}



{
\usebackgroundtemplate{\includegraphics[width=\paperwidth]{diversity_gradient}}
\begin{frame}[b,plain]{Diversity shows a strong latitudinal gradient. Why?}

\end{frame}
}


%\begin{frame}[t,plain]{Several related \highlight{biotic} hypotheses have been proposed.}
%
%	\hangpara Productivity,
%
%	\hangpara Competition, and
%	
%	\hangpara Niche width.
%	
%\end{frame}
%
%\begin{frame}[t,plain]{Several related \highlight{abiotic} hypotheses have also been proposed.}
%
%	\hangpara Tropical antiquity, and
%	
%	\hangpara Tropical stability.
%	
%\end{frame}

\begin{frame}[t,plain]{Environmental gradients contribute to diversity gradient.}
	\begin{center}
		\includegraphics[width=\textwidth]{diversity_gradient_environment}
	\end{center}
	
\end{frame}

\begin{frame}[t,plain]{\highlight{Net primary productivity} is highest in the tropics.}
	\begin{center}
		\includegraphics[width=\textwidth]{diversity_gradient_npp}
	\end{center}
\end{frame}


{
\usebackgroundtemplate{\includegraphics[width=\paperwidth]{diversity_gradient_thermal}}
\begin{frame}[b,plain]{High solar input may cause tropical diversity.}

\end{frame}
}



\begin{frame}[t,plain]{Tropical niche conservation and adaption to new niches may explain diversity gradient.}
	\begin{center}
		\includegraphics[height=0.8\textheight]{diversity_gradient_niche_conservatism}
	\end{center}
	
\end{frame}

{
\usebackgroundtemplate{\includegraphics[width=\paperwidth]{diversity_gradient_niche_conservatism_phylogeny}}
\begin{frame}[t,plain]{Oldest lineages are tropical. Younger lineages are temperate.}

\end{frame}
}

{
\usebackgroundtemplate{\includegraphics[width=\paperwidth]{diversity_gradient_tnc_frog_phylogeny}}
\begin{frame}[b,plain]{Tropical niche conservatism was supported by hylid tree frogs.}

\tiny\hfill Wiens et al. 2006. Am. Nat. 168: 579.
\end{frame}
}


\begin{frame}[t,plain]{Niche conservatism was tested with amphibians.}
	\begin{center}
		\includegraphics[height=0.75\textheight]{diversity_gradient_tnc_amphib1}
	\end{center}
	
	\vskip0pt plus 1filll
	
	\tiny Gómoz-Rodríguez et al. 2015. Global Ecol. Biogeogr. 24: 383.
\end{frame}

\begin{frame}[t,plain]{Most amphibians tended to have a conserved niche.}
	\begin{center}
		\includegraphics[width=\textwidth]{diversity_gradient_tnc_amphib2}
	\end{center}
	
	\vskip0pt plus 1filll
	
	\tiny Gómoz-Rodríguez et al. 2015. Global Ecol. Biogeogr. 24: 383.
\end{frame}




{
\usebackgroundtemplate{\includegraphics[width=\paperwidth]{diversity_gradient_out_of_tropics}}
\begin{frame}[t,plain]{Tropics serve as cradle, museum, and immigration pump.}

\end{frame}
}

\begin{frame}[t,plain]{More marine bivalves tended to evolve first in tropics.}
	\begin{center}
		\includegraphics[width=\textwidth]{diversity_gradient_oot_first_occurrence}
	\end{center}
	
	\vskip0pt plus 1filll
	
	\tiny Jablonski et al. 2013. Proc. Natl. Acad. Sci. 110: 10487.
\end{frame}

\begin{frame}[t,plain]{Marine bivalves tended to appear later in temperate zones.}
	\vspace{-\baselineskip}
	\begin{center}
		\includegraphics[height=0.75\textheight]{diversity_gradient_oot_first_occurrence2}
	\end{center}
	
	\vskip0pt plus 1filll
	
	\tiny Jablonski et al. 2013. Proc. Natl. Acad. Sci. 110: 10487.
\end{frame}






\end{document}
