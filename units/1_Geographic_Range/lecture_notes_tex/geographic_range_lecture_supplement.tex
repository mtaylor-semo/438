%!TEX TS-program = lualatex
%!TEX encoding = UTF-8 Unicode

\documentclass[11pt, hidelinks]{article}
\usepackage{graphicx}
\graphicspath{{/Users/goby/Pictures/teach/438/lab/}} % set of paths to search for images

\usepackage{geometry}
\geometry{letterpaper}                   
\geometry{bottom=1in}
%\geometry{landscape}                % Activate for for rotated page geometry
\usepackage[parfill]{parskip}    % Activate to begin paragraphs with an empty line rather than an indent
\usepackage{amssymb}
%\usepackage{mathtools}
%	\everymath{\displaystyle}

\usepackage{color}
%\pagenumbering{gobble}

\usepackage{fontspec}
\setmainfont[Ligatures={Common, TeX}, BoldFont={* Bold}, ItalicFont={* Italic}, Numbers={Proportional, OldStyle}]{Linux Libertine O}
\setsansfont[Scale=MatchLowercase,Ligatures=TeX]{Linux Biolinum O}
\setmonofont[Scale=MatchLowercase]{Inconsolatazi4}
\usepackage{microtype}

\usepackage{unicode-math}
\setmathfont[Scale=MatchLowercase]{Asana-Math.otf}
%\setmathfont{XITS Math}

% To define fonts for particular uses within a document. For example, 
% This sets the Libertine font to use tabular number format for tables.
%\newfontfamily{\tablenumbers}[Numbers={Monospaced}]{Linux Libertine O}
%\newfontfamily{\libertinedisplay}{Linux Libertine Display O}


\usepackage{booktabs}
\usepackage{longtable}
\usepackage[justification=raggedright, singlelinecheck=off]{caption}
\captionsetup{labelsep=period} % Removes colon following figure / table number.
%\captionsetup{tablewithin=none}  % Sequential numbering of tables and figures instead of
%\captionsetup{figurewithin=none} % resetting numbers within each chapter (Intro, M&M, etc.)
%\captionsetup[table]{skip=0pt}

\usepackage{array}
\newcolumntype{L}[1]{>{\raggedright\let\newline\\\arraybackslash\hspace{0pt}}p{#1}}
\newcolumntype{C}[1]{>{\centering\let\newline\\\arraybackslash\hspace{0pt}}p{#1}}
\newcolumntype{R}[1]{>{\raggedleft\let\newline\\\arraybackslash\hspace{0pt}}p{#1}}

%\usepackage{enumitem}
\usepackage{hyperref}
\usepackage{multicol}

%\usepackage{titling}
%\setlength{\droptitle}{-50pt}
%\posttitle{\par\end{center}}
%\predate{}\postdate{}

\usepackage{hanging}
\usepackage{wrapfig}

\usepackage[sc]{titlesec}

\newcommand{\coursename}{\textsc{bi} 438/638: Biogeography}

\usepackage{fancyhdr}
\fancyhf{}
\pagestyle{fancy}
%\lhead{}
%\chead{}
%\rhead{Name: \rule{5cm}{0.4pt}}
%\renewcommand{\headrulewidth}{0pt}
\setlength{\headheight}{14pt}
\fancyhead[R]{\footnotesize Geographic Range Size \thepage}
\fancyhead[L]{\footnotesize \coursename}

\fancypagestyle{first_page}{%
	\fancyhf{}
	\fancyhead[L]{\coursename}
	\fancyhead[R]{Name: \enspace \rule{2.5in}{0.4pt}}
	\renewcommand{\headrulewidth}{0pt}
}

\newcommand{\bigSpace}{\vspace{5\baselineskip}}

\newlength{\myLength}
\setlength{\myLength}{\parindent}
\title{Unit 1: The Geographic Range}
\author{Biogeography}
%\date{Fall 2017}							% Activate to display a given date or no date
\date{}

\begin{document}
\maketitle
\section{Assigned Reading}
%\subsection{}
\begin{multicols}{2}
4th edition\\
Chapter 4: 83--86, 89 (niche concept)--119. \\
Chapter 15: 621-637.

%\medskip
\columnbreak

5th edition\\
Chapter 4: 71--74, 78--100. \\
Chapter 14: 521-538.
\end{multicols}


The assigned reading from the textbook is fair game for assignments and exams, even if I do not cover the material in  lecture. Figure and page numbers refer to the 5th edition. Numbers in parentheses after figure and page numbers refer to their equivalents in the 4th edition. 

Some of the material in Chapter 4 will be an ecological review for most of you but study these pages to learn how ecology influences the geographic range.  I expect you to know the fundamental ecological concepts (niche, competition, mutualism, etc.) and to be able to explain how they influence the geographic range.  For Figure 4.25 on page 93 (105), can you think of an evolutionary explanation for the origin of non-overlapping distributions?

All hyperlinks in this PDF are live. Click them to be transported to the web site.

\section{The Geographic Range}

You must understand the basic processes that influence the geographic distribution of species because the distribution forms the basis for much of the hypotheses and inferences concerning processes in biotic distributions. The \emph{geographic range}, or simply the range or distribution of a species is the basic observational unit of biogeography. It encompasses the geographic extent of occurrences of a taxon (lineage within species, species, genus, etc) during part or all of its life cycle. 

The range of a species is limited to a specific set of environmental conditions. The species' range may be  extremely localized or, more rarely, may be \emph{cosmopolitan} and found across extremely wide geographic areas. For example, the Devil's Hole pupfish (\emph {Cyprinodon diabolis}) (Figure \ref{cdiabolis}) is found in a single pool of less than 100~m$^2$ in southwestern Nevada. In comparison, the Blue Whale (\emph{Balaenoptera musculus}) has a range estimated to cover 300,000,000~km$^2$.  The Peregrine Falcon (\emph{Falco peregrinus}) is found on all continents except Antarctica. That makes the Peregrine Falcon  the fastest and most widespread dinosaur alive today (Figure \ref{xkcd}).


\begin{figure}[htb]
	\centering
		\includegraphics[width=0.5\textwidth]{Cyprinodon_diabolis}  
		\caption{The Devil's Hole Pupfish, \emph{Cyprinodon diabolis}, is one of several species of \emph{Cyprinodon} that occupy only one or a few springs in Nevada and California.\label{cdiabolis}}
		
\end{figure}


\begin{figure}
	\centering
	\includegraphics[width=4in]{birds_and_dinosaurs.png}
	\caption{Cartoon from xkcd, a must for science geeks.\label{xkcd} \url{http://xkcd.com}}
\end{figure}

\section{Constraints on the Range Size}

The size of the geographic range occupied by a species can be constrained by four broad factors: the ecological niche, habitat disturbance, interactions with other species and historical (evolutionary and geological) factors.  These notes will cover the first three constraints. We'll save historical factors for the next unit.


\subsection{Constraints: The Niche}

In 1957, G.E. Hutchinson conceived of the idea of an \emph{ecological niche}, that is, a multidimensional space that describes the set of conditions which limit the distribution of a species.  The niche occupied by a species is a function of the interaction among many abiotic and biotic factors that influence the persistence of a particular species in a particular area.  Ecologists recognize two forms of the niche, the \emph{fundamental niche} and the \emph{realized niche}. The fundamental niche represents all of the resources that can be potentially used by a species.  The realized niche is the actual resources used by the species. Hutchinson's niche concept is illustrated in Figure \ref{hutch niche}.  The figure shows three axes, each representing a particular resource or environmental gradient.  The location of the species' niche in that space depends on its evolutionary adaptions to the available resources and environment.  Individuals living under close to ideal environmental conditions tend to have greater relative fitness (reproductive output) relative to individuals living in suboptimal conditions.

\begin{figure}[htb]
	\centering
	\includegraphics[width=5in]{niche.jpg}
	\caption{An illustration of Hutchinson's concept of the ecological niche. The area of low fitness represents the fundamental niche. The area of high fitness indicates the realized niche. \label{hutch niche}}
\end{figure}

Similarly, the \emph{realized geographic range} of a species is reduced from the \emph{fundamental geographic range} (Figure~\ref{geoniche}). Other factors further influence the distribution and abundance of individuals within the realized geographic range. In general, the highest abundance of a species is in the central part of the range because environmental conditions and resource availability tend to be high. Thus, relative individual fitness tends to be higher.  Species can persist near the edges of its range but may have reduced fitness because required resources are fewer or the environmental conditions are suboptimal.  Compare figures~\ref{hutch niche} and~\ref{geoniche} below to Figure 4.16 of your text.  

\begin{figure}
	\centering
	\includegraphics[width=1\textwidth]{geoniche.png}
	\caption{Illustration of the relationship between a hypothetical species' geographic distribution (left panel) and its ecological niche (right panel).  x and y refer to geographical coordinates (e.g. longitude and latitude). The niche is defined by two environmental factors, e1 and e2. Crosses represent observed species occurrence records. Grey shading in geographical space represents the species' actual distribution (i.e., those areas that are truly occupied by the species). The solid line in the right panel depicts the species' fundamental niche, which represents the full range of ecological conditions within which the species is viable. In the left panel, the solid lines depict geographic areas with ecological conditions that fall within the fundamental niche; this is the species' potential distribution. Some regions of the potential distribution may not be inhabited by the species due to biotic interactions or dispersal limitations. For example, area B is environmentally suitable for the species, but is not part of the actual distribution, perhaps because the species has been unable to disperse across unsuitable environments to reach this area. Similarly, the non-shaded area around label C is within the species' potential distribution, but is not inhabited, perhaps due to competition from another species. Figure and legend from The American Museum of Natural History, Center for Biodiversity and Conservation, Creative Commons license. \url{http://biodiversityinformatics.amnh.org/index.php?section_id=7.}\label{geoniche}}
\end{figure}

\subsubsection{Habitat Quality}

Although range maps tend to show very distinct boundaries for the distribution of a species, the reality is that range boundaries are irregular and change over time.  Why is that?  Not all habitats within a certain area will have the same character in terms of the environmental factors important in determining a species' range. This variability in habitat quality often results in individual populations being described as source or sink populations. \emph{Source populations} are those living in locations where environmental conditions are close to optimal and resources are abundant (high quality habitat).  As a result, the birth rate exceeds the death rate so source populations tend to produce ``surplus'' individuals that emigrate to other habitats. In contrast, \emph{sink populations} are those living in locations where the environmental conditions are suboptimal and resources are scare (poor habitat quality) so the death rate tends to exceeds birth rate. Sink populations, therefore, are sustained by immigration from source populations. Often, locations with poor habitat quality are found at the periphery of a species' range.  

\subsubsection{Dispersal Limitation}

Dispersal among suitable habitats, especially near the range boundaries, can be limited when favorable habitat is too isolated from the central part of the species' range.  If the species lacks the ability to disperse across unfavorable habitat then the favorable habitat may remain uncolonized for extended periods of time.  In addition to limiting the overall geographic range, unsuitable habitat within the range can create metapopulations. 

\subsubsection{Metapopulation Dynamics}

A \emph{metapopulation} is defined as ``a population consisting of a set of subpopulations that are linked by a cycle of alternating immigration, extinction, and recolonization.''  What that means is that suitable habitat within a species' range may not always be occupied, as described above.  Sampling a particular habitat in one year may yield many individuals but in another year, individuals are few or absent.  Metapopulations may occur in ``patchy'' habitats, where the subpopulations are isolated by areas of unsuitable habitat.  Another possibility is that random factors or environmental variability can cause extinction of local subpopulations, which are subsequently re-established by emigration from other subpopulations. In other words, some habitats may be occupied only intermittently owing to metapopulation dynamics. A patch of habitat may be occupied for a few years, then that subpopulation goes extinct, and the habitat is unoccupied for a few years before it is recolonized by immigration. 

Figure \ref{metapop} shows a metapopulation of the Glanville fritillary butterfly.  The black circles were occupied during one survey but the white circles were unoccupied.  Sampling over many years showed that the occupied and unoccupied areas were not always the same.  Finally, individuals of many species tend to be clumped or aggregated.  Thus, even in the center of the range, in high quality habitat, you may not find any individuals, while a short distance away you may find 100s of individuals.

\begin{figure}[hb]
	\centering
	\includegraphics[width=1\textwidth]{metapopulation.jpg}
	\caption{A metapopulation of the Glanville Fritillary Butterfly. Filled circles indicate occupied locations. Open circles indicate unoccupied locations. Whether a particular location is occupied changes over time. Image based on work by Ilkka Hanski (\url{http://www.helsinki.fi/science/metapop/research/Project_metapop.html}) and colleagues \label{metapop}}
\end{figure}

\subsection{Constraints: Disturbance}

The geographic range is strongly influenced by catastrophic disturbances such as floods, hurricanes, volcanic eruptions, or fires. Disturbances affect the geographic range in both positive and negative ways. For example, healthy growth and maintenance of native grasslands (prairies) were promoted by a natural cycle of fire and regeneration that limited growth of shrub and woody species, leaving plenty of light and nutrients for grass species. The modern practice of forest fire suppression has lead to restrictions in the geographic range of native grasses in many parts of the world.  Conservation managers have learned this and often now only extinguish naturally-caused fires if human populations are threatened. As a negative example, overfishing by humans on Jamaican coral reefs threw the entire ecosystem out of balance.  As a consequence, algae started to dominate the reefs (Figure \ref{reef}), which limited the ability of young corals to recruit into the population.  

\begin{figure}[hb]
	\centering
	\includegraphics[width=0.75\textwidth]{algaereef.jpg}
	\caption{An algae-dominated reef in Jamaica. Algal overgrowth, enabled by the loss of long-spined sea urchins, limits recruitment of new corals.\label{reef}}
\end{figure}

Algal growth was held in check only by long-spined sea urchins.  However, Hurricane Allen in 1980 and widespread disease in 1983 wiped out the sea urchins and algae dominated the reefs.  The entire coral reef ecosystem switched to an algal/rock system.  Only now are the reefs beginning to show any signs of recovery. This idea ties back to metapopulation dynamcs: In 1999, Hurricane Floyd devastated island lizard populations in the Bahamas.  At least 66 island populations were destroyed. However, within 17 months, the islands had been recolonized by overwater dispersal from neighboring islands, either by rafting or swimming.

\subsection{Constraints: Species Interactions}
\subsubsection{Competition}
\emph{Competition} between species with similar requirements for growth and survival (e.g. a food resource) can limit population growth for both species. If niche overlap is too great, one taxon can be excluded from otherwise suitable habitats by the presence of a close competitor.  For example, the Red Squirrel, \emph{Sciurus vulgaris}, was once widely distributed through England, as well as Italy.  The introduction of the Eastern Grey Squirrel (\emph{Sciurus carolinensis}) from North America is causing a decline in the range of the Red Squirrel (Figure \ref{squirrel}).  The larger grey squirrel consumes more food, causing a shortage for the Red Squirrel. 

\begin{figure}[hb]
	\centering
	\includegraphics[width=0.8\textwidth]{squirrel.png}
	\caption{Distribution of the native Red Squirrel (dark shading) and the invasive Eastern Grey Squirrel (grey shading) in England. The range of the Red Squirrel has decreased at the same time that the range of the Eastern Grey Squirrel has increased.\label{squirrel}}
\end{figure}

Also, if you study the ranges of any closely related species, you are very likely to find that many species have adjoining ranges but do not overlap.  Closely related species are very likely to use similar resources, so the ranges remain non-overlapping which reduces competition. 

\subsubsection{Predation}
Distributions of predators may be influenced by the geographic range of their prey species. This can also be said for the distributions of parasites and their hosts. Evidence in favor of this idea would include the complete coincidence of predator/prey (or host/parasite) ranges. It is hard to document how predators limit prey distribution in natural systems, but we have gained clues through accidental or intentional introduction of predators outside of their native range. Introduction of largemouth bass (\emph{Micropterus salmoides}) into the southwestern United States has decimated many of that region's native fishes.  The brown tree snake, \emph{Boiga irregularis}, snake was accidentally introduced  to the island of Guam in the south Pacific and has caused extinction of most of the native bird fauna there (up to 10--12 species) (Figure \ref{brown tree snake}). 

\begin{figure}
	\centering
	\includegraphics[width=1\textwidth]{snake.png}
	\caption{Spread of the Brown Tree Snake on Guam. The large years show the approximate time the snake reached that area of the island. The numbers in the boxes indicate the number of bird species that were surveyed in that area over several years. The decline of birds clearly coincides with the arrival of the brown tree snake in each area of the island.\label{brown tree snake}}
\end{figure}


\subsubsection{Mutualism}

\emph{Mutualism} is a third kind of interspecific interaction and one in which both species benefit from the interaction. In such cases, if the mutualism is very specific, the distribution of one species limits the distribution of the other. Most cases of mutualism involve multiple possible species that can take place in the interaction. For instance, there are eight or so species of hummingbirds in North America which use over 100 species of flowering plants as a food source. Even though the interaction between these birds and the flowers is beneficial to both species, there is little correspondence between the geographic ranges of particular species of hummingbirds and plant species. A good exception is the strong overlap of the ranges of Clark's nutcracker (\emph{Nucifraga columbiana}) and the white bark pine trees (\emph{Pinus albicaulis}) (Figure \ref{nutcracker}). The distribution of the pine (especially) is dependent on the dispersal of seeds by nutcrackers which feed on the cones (and hence release the seeds). On the other hand, the cones/seeds provide an excellent food source for the nutcracker. Both species are found with widely overlapping ranges in the Rocky Mountain sub-alpine habitats.

\begin{figure}[tb]
	\centering
	\includegraphics[width=1\textwidth]{nutcracker.png}
	\caption{Distribution of Clark's Nutcracker, a bird, and the white bark pine tree. The ranges of the two species closely coincide due to the symbiotic interaction (mutualism) between the two species.\label{nutcracker}}
\end{figure}

\section{Rapoport's Rule}

In 1982, Eduardo Rapoport reported a trend in range size that he observed across several different taxonomic groups.  He suggested that species tend to have larger geographic ranges at higher latitudes compared to lower latitudes nearer the equator. Going along with larger geographic range size was decreasing species richness.  Species richness was higher at low latitudes and richness was lower at higher latitudes.  (Figure \ref{raporulelat}). Rapoport also noted a relationship between range size or species richness and elevation (Figure \ref{raporuleelev}).  Species at higher latitudes or higher elevations tended to occupy larger geographic ranges.  In contrast, species richness declined with increasing latitude or elevation.

\begin{figure}
	\centering
		\includegraphics[width=1\textwidth]{raporulelat.png}  
		\caption{Illustration of Rapoport's Rule for range size and latitude, based on range width (degrees latitude, blue line) and species richness (green line) of North American trees and Pacific Coast Molluscs . For both taxonomic groups, average range size increases and species richness decreases with higher latitudes (modified from Brown 1995).\label{raporulelat}}
		
\end{figure}

\begin{figure}
	\centering
		\includegraphics[width=1\textwidth]{raporuleelev.png}  
		\caption{Illustration of Rapoport's Rule for range size and elevation, based on range size (meters, blue line) and species richness (green line) for Costa Rican trees and Venezuelan birds. For both taxonomic groups, average range size increases and species richness decreases with higher altitude (modified from Stevens 1992).\label{raporuleelev}}
\end{figure}

Rapoport's observations are now collectively called \emph{Rapoport's Rule}.  Not all taxa show these trends associated with latitude an elevation so there is considerable debate about whether this rule applies to most taxa. We'll read some papers associated with this debate.  A number of ideas have been proposed to explain this rule, at least for the taxa where the trend has been observed.  Your text discusses this in some detail on pages 526--534 (626--634), so study those pages.  Later in the semester, we will discuss in more depth the hypotheses that have been proposed to explain the latitudinal gradient for species richness.

\section{Body Size, Abundance and Geographic Area}

Figure \ref{areabodysize}  shows the range size of North American birds as a function of body size.  Can you detect any apparent trends? How does range size (Y-axis) differ between small birds (small body mass) compared to large birds? Do the points seem randomly scattered across the entire graph or are they concentrated in some way?  Study the figure and develop some ideas before reading further.

\begin{figure}[tb]
	\centering
		\includegraphics[width=0.9\textwidth]{land_bird_range_size.png}
		\caption{Geographic area as a function of body mass for North American birds.  See the notes for an explanation of the three dashed lines. (Figure 14.11 (15.9) from your text.)\label{areabodysize}}
\end{figure}

If anything, small birds show considerable variation of range size, while large birds are constrained to have large range sizes.  A similar trend is observed for land mammals (see your text).  However, the graphs suggest a bit more than we might first realize.  To understand why, you have to consider the information represented by the three dashed lines, called \emph{constraint lines}.   A constraint is a limitation of some sort, like the constraints on the geographic range discussed in the last set of lecture notes.
 
Two of the constraints should be fairly obvious as to the limitation.  The third will be subtle but it has potentially important consequences from a conservation biology perspective.  This first constraint is the vertical dashed line.  Interpret this in relation to the X-axis. \emph{What do you think is the constraint?} (Think about it before reading further.) The vertical dashed line represents a physiological constraint on body size.  Species have to be of a certain minimum size to function. The second constraint is the horizontal dashed line. Interpret it in relation to the Y-axis. This time the constraint is not related to the organism but to the land.  \emph{What do you think is the second constraint?} (Again, think before peeking ahead.)  The second constraint is the size of the continent or ocean basin.  Most species don't (or can't) extend beyond continents or ocean basins, which sets the upper limit of the range size.

The third constraint is represented by the diagonal dashed line.  It does not relate directly to the two axes but it does begin to explain the non-random relationship between body mass and range size. What do you think is the constraint? This one is rather subtle, so we'll walk through it slowly, but we first have to establish a few basic biological tenets.  For many species, abundance is a negative function of body size.  That is, small species tend to have high abundance, while large species have much lower abundance.  This is not usually a problem to accept because large species need more of the limited resources, so fewer individuals can be supported.  However, studies have shown that species with low abundances are more vulnerable to extinction.  Therefore, large species tend to be much less common and have a higher probability of going extinct.\footnote{If you had BI 300, think back to the lectures on turnover rates and the link between high origination and extinction rates.  If you didn't have my evolution class, think back on a particularly fond childhood memory.  Either way, send me an e-mail saying you read this far to get two extra credit points.}  Thus, for a population of a large species to be sustained, it needs a range size sufficient to support the population.  That's where this diagonal line comes into play, and not just for large species. 
 
Considering the information above, can you now begin to explain what this diagonal constraint line represent?  The area above the diagonal line represents the approximate area needed by a species to sustain a viable population.  Species whose ranges are very small relative to their body size may have a higher probability of extinction.  This information would be critical to know by a conservation biologist, for example, who might then try to increase the available habitat and thereby increase the range size of the species.  Several methods exist to calculate constraint lines, but this is well beyond our goals for this course.  However, the application of this method is one of many analytical tools that may be useful to a conservation biologist charged with protecting threatened or endangered species.

Figure \ref{area and abundance} shows this information in a different way.  Here, range area is plotted with species abundance.  Species with large geographic range sizes generally vary in abundance, but very few species that have small ranges have very high abundances.

\begin{figure}[tb]
	\centering
		\includegraphics[width=0.85\textwidth]{area_abundance.png}  
		\caption{Plot of abundance as a function of geographic area for North American birds.  See the notes for an explanation of the dashed line. (Figure 14.16 (15.10) of your text) \label{area and abundance}}
\end{figure}

\section{Range Shape}

Another trend noted by Rapoport is range shape.  Figure \ref{range shape} plots the range shape for North American and Eurasian birds.  The plot is the size of the range in a north-south direction (y-axis) against the size of the range in an east-west direction (X-axis). If the two dimensions are equal, then the dot falls exactly on the line.  For many species the range is more or less circular.  However, for dots above the diagonal line, the the range is taller (N--S) than wide (E--W).  The reverse is true for dots below the diagonal line.  Notice that some North American birds tend to have taller range shapes while others have a wider range, especially for birds with the largest range size.  In contrast, no European birds have tall ranges and very few have roughly symmetric range shapes. Nearly all species tend to have much wider ranges.  Can you think of why range shape differs between North American and European bird species?

\begin{figure}
	\centering
		\includegraphics[width=0.85\textwidth]{range_shape.png}  
		\caption{Range shape based on N--S vs. E--W length (km) for species of North American birds (panel B) and Eurasian birds (panel C). The black diagonal line indicates a symmetric range. Many distributions are skewed towards N--S or E--W dimensions. (Figure 14.15 of your text) \label{range shape}}
\end{figure}

Continental barriers tend to explain the overall shape.  Large continental barriers, such as the Rocky Mountains and the Appalachian Mountains, tend to run north--south.  In Eurasia, the major mountain barriers tend to run east--west. In North America, species with smaller ranges tend to be constrained in the east--west direction by the barriers, so the ranges tend to stretch into a north--south dimension.  Small North American species may have ranges between the Atlantic coast and the Appalachian Mountains, or between the Appalachian and Rocky mountains, or between the Rockies and the Pacific Coast.  However that Eurasian birds all tend to have large range sizes overall (Figure \ref{range shape}C).  None of the Eurasian birds appear to have small ranges. Can you come up with a possible explanation?  Here's a hint: Pleistocene glaciation.

During the Pleistocene (2.1 million to 12,000 years ago), there were five major glacial advances that extended south into North America and Eurasia. In North America, glaciers advanced as far south as the Missouri River in Missouri and even farther south in Illinois.  Species unable to disperse east to west across mountain barriers could still disperse southward into Mexico or Central America to escape the glaciers and associated cooler climates.  In Eurasia, however, the mountain ranges that run east--west in the southern part of the continent would have limited dispersal south during the glacial advance.  Trapped between a cold place and a high place, many Eurasian species probably went extinct.  The only species that could have survived would have been those that already had the ability to cross over the mountains, and thus would already have larger range sizes. Remember also that the east--west size of the geographic range will be constrained by the edges of the continent.

\section{Range Overlap}

Species exist as part of a community of many species so many species clearly have ranges that overlap. Yet, what may not be obvious is that the ranges of \emph{closely related} species that only recently spectated often do not overlap, as shown for the species of kangaroo rats in the genus \emph{Dipodomys} (Figure \ref{krat range}).  Non-overlapping ranges of closely related species also holds for Caribbean reef fishes (gobies) of the genus \emph{Elacatinus} (Figure \ref{elacatinus}). Closely related species, such as \emph{E. oceanops} and \emph{E. evelynae}, or \emph{E. macrodon} and \emph{E. saucrus} have non-overlapping ranges.  Only species on very long branches, indicating a much more distant relationship, have overlapping ranges.  For example, \emph{E. multifasciatus} and \emph{Ginsburgellus novemlineatus} are sister species as are \emph{E. pallens} and \emph{E. gemmatus} but these sister-pairs diverged long before most other \emph{Elacatinus}. Both of these species-pairs have nearly 100\% overlap of their ranges.

\begin{figure}[tb]
	\centering
		\includegraphics[width=0.75\textwidth]{krart_nonoverlapping_range.png}  
		\caption{Non-overlapping ranges of kangaroo rats, genus \emph{Dipodomys}, in the southwestern United States. (Figure 4.25 of your text.)\label{krat range}}
\end{figure}

\begin{figure}[tb]
	\centering
		\includegraphics[width=1\textwidth]{goby_phylo.png}  
		\caption{Phylogeny showing relationships among Caribbean gobies in the genus \emph{Elacatinus}. Closely related species have non-overlapping ranges. Modified from Taylor and Hellberg 2005.\label{elacatinus}}
\end{figure}

This pattern of non-overlapping ranges among recently diverged species has been observed enough to warrant several comparative studies, but the results are often conflicting.  Thus, the generality of this idea of non-overlap between closely related species remains highly debated.



\end{document}  