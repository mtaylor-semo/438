\documentclass[12pt, oneside]{article}   	% use "amsart" instead of "article" for AMSLaTeX format
\usepackage{geometry}                		% See geometry.pdf to learn the layout options. There are lots.
\geometry{letterpaper}                   		% ... or a4paper or a5paper or ... 
%\geometry{landscape}                		% Activate for for rotated page geometry
\usepackage[parfill]{parskip}    		% Activate to begin paragraphs with an empty line rather than an indent
\usepackage{graphicx}				% Use pdf, png, jpg, or eps§ with pdflatex; use eps in DVI mode
								% TeX will automatically convert eps --> pdf in pdflatex		
\usepackage{amssymb}
\usepackage[margin=10pt, font=small, labelsep=period]{caption}
\usepackage{lmodern}
\usepackage[T1]{fontenc}
%\usepackage{txfonts}
\usepackage{textcomp}
\usepackage[pdftex]{hyperref}

\title{Unit 1: The Geographic Range}
\author{Biogeography}
\date{Fall 2013}							% Activate to display a given date or no date

\begin{document}
\maketitle
\section{Assigned Reading}
%\subsection{}
Chapter 4: 83--86, 89 (niche concept)--119. 

Some of the material will be an ecological review (for some of you) but you should study these pages to learn how ecology influences the geographic range.  I expect you to know the fundamental ecological concepts (niche, competition, mutualism, etc.) and to be able to explain how they influence the geographic range.  For Figure 4.25 on page 105, can you think of an evolutionary explanation for the origin of non-overlapping distributions?

Note: All hyperlinks in the lecture notes are live. Click them to be transported to the web site.

\section{The Geographic Range}

You must understand the basic processes that influence the geographic distribution of species because the distribution forms the basis for much of the hypotheses and inferences concerning processes in biotic distributions. The \emph{geographic range}, or simply the range or distribution of a species is the basic observational unit of biogeography. It encompasses the geographic extent of occurrences of a taxon (lineage within species, species, genus, etc) during part or all of its life cycle. 

The range of a species is limited to a specific set of environmental conditions. The species' range may be  extremely localized or, more rarely, may be \emph{cosmopolitan} and found across extremely wide geographic areas. For example, the Devil's Hole pupfish (\emph {Cyprinodon diabolis}) (Figure \ref{cdiabolis}) is found in a single pool of less than 100~m$^2$ in southwestern Nevada. In comparison, the Blue Whale (\emph{Balaenoptera musculus}) has a range estimated to cover 300,000,000~km$^2$.  The Peregrine Falcon (\emph{Falco peregrinus}) is found on all continents except Antarctica. That makes the Peregrine Falcon  the fastest and most widespread dinosaur alive today (Figure \ref{xkcd}).


\begin{figure}[htb]
	\centering
		\includegraphics[width=0.5\textwidth]{Cyprinodon_diabolis.jpg}  
		\caption{The Devil's Hole Pupfish, \emph{Cyprinodon diabolis}, is one of several species of \emph{Cyprinodon} that occupy only one or a few springs in Nevada and California.\label{cdiabolis}}
		
\end{figure}


\begin{figure}
	\centering
	\includegraphics[width=4in]{birds_and_dinosaurs.png}
	\caption{Cartoon from xkcd, a must for science geeks.\label{xkcd} \url{http://xkcd.com}}
\end{figure}

\section{Constraints on the Range Size}

The size of the geographic range occupied by a species can be constrained by four broad factors: the ecological niche, habitat disturbance, interactions with other species and historical (evolutionary and geological) factors.  These notes will cover the first three constraints. We'll save historical factors for the next unit.


\subsection{Constraints: The Niche}

In 1957, G.E. Hutchinson conceived of the idea of an \emph{ecological niche}, that is, a multidimensional space that describes the set of conditions which limit the distribution of a species.  The niche occupied by a species is a function of the interaction among many abiotic and biotic factors that influence the persistence of a particular species in a particular area.  Ecologists recognize two forms of the niche, the \emph{fundamental niche} and the \emph{realized niche}. The fundamental niche represents all of the resources that can be potentially used by a species.  The realized niche is the actual resources used by the species. Hutchinson's niche concept is illustrated in Figure \ref{hutch niche}.  The figure shows three axes, each representing a particular resource or environmental gradient.  The location of the species' niche in that space depends on its evolutionary adaptions to the available resources and environment.  Individuals living under close to ideal environmental conditions tend to have greater relative fitness (reproductive output) relative to individuals living in suboptimal conditions.

\begin{figure}[htb]
	\centering
	\includegraphics[width=5in]{niche.jpg}
	\caption{An illustration of Hutchinson's concept of the ecological niche. The area of low fitness represents the fundamental niche. The area of high fitness indicates the realized niche. \label{hutch niche}}
\end{figure}

Similarly, the \emph{realized geographic range} of a species is reduced from the \emph{fundamental geographic range} (Figure \ref{geoniche}). Other factors further influence the distribution and abundance of individuals within the realized geographic range. In general, the highest abundance of a species is in the central part of the range because environmental conditions and resource availability tend to be high. Thus, relative individual fitness tends to be higher.  Species can persist near the edges of its range but may have reduced fitness because because required resources are fewer or the environmental conditions are suboptimal.  Compare figures \ref{hutch niche} and \ref{geoniche} to Figure 4.16 on page 95 of your text.  

\begin{figure}
	\centering
	\includegraphics[width=1\textwidth]{geoniche.png}
	\caption{Illustration of the relationship between a hypothetical species' geographic distribution (left panel) and its ecological niche (right panel).  x and y refer to geographical coordinates (e.g. longitude and latitude). The niche is defined by two environmental factors, e1 and e2. Crosses represent observed species occurrence records. Grey shading in geographical space represents the species' actual distribution (i.e., those areas that are truly occupied by the species). The solid line in the right panel depicts the species' fundamental niche, which represents the full range of ecological conditions within which the species is viable. In the left panel, the solid lines depict geographic areas with ecological conditions that fall within the fundamental niche; this is the species' potential distribution. Some regions of the potential distribution may not be inhabited by the species due to biotic interactions or dispersal limitations. For example, area B is environmentally suitable for the species, but is not part of the actual distribution, perhaps because the species has been unable to disperse across unsuitable environments to reach this area. Similarly, the non-shaded area around label C is within the species' potential distribution, but is not inhabited, perhaps due to competition from another species. Figure and legend from The American Museum of Natural History, Center for Biodiversity and Conservation, Creative Commons license. \url{http://biodiversityinformatics.amnh.org/index.php?section_id=7.}\label{geoniche}}
\end{figure}

\subsubsection{Habitat Quality}

Although range maps tend to show very distinct boundaries for the distribution of a species, the reality is that range boundaries are irregular and change over time.  Why is that?  Not all habitats within a certain area will have the same character in terms of the environmental factors important in determining a species' range. This variability in habitat quality often results in individual populations being described as source or sink populations. \emph{Source populations} are those living in locations where environmental conditions are close to optimal and resources are abundant (high quality habitat).  As a result, the birth rate exceeds the death rate so source populations tend to produce ``surplus'' individuals that emigrate to other habitats. In contrast, \emph{sink populations} are those living in locations where the environmental conditions are suboptimal and resources are scare (poor habitat quality) so the death rate tends to exceeds birth rate. Sink populations, therefore, are sustained by immigration from source populations. Often, locations with poor habitat quality are found at the periphery of a species' range.  

\subsubsection{Dispersal Limitation}

Dispersal among suitable habitats, especially near the range boundaries, can be limited when favorable habitat is too isolated from the central part of the species' range.  If the species lacks the ability to disperse across unfavorable habitat then the favorable habitat may remain uncolonized for extended periods of time.  In addition to limiting the overall geographic range, unsuitable habitat within the range can create metapopulations. 

\subsubsection{Metapopulation Dynamics}

A \emph{metapopulation} is defined as ``a population consisting of a set of subpopulations that are linked by a cycle of alternating immigration, extinction, and recolonization.''  What that means is that suitable habitat within a species' range may not always be occupied, as described above.  Sampling a particular habitat in one year may yield many individuals but in another year, individuals are few or absent.  Metapopulations may occur in ``patchy'' habitats, where the subpopulations are isolated by areas of unsuitable habitat.  Another possibility is that random factors or environmental variability can cause extinction of local subpopulations, which are subsequently re-established by emigration from other subpopulations. In other words, some habitats may be occupied only intermittently owing to metapopulation dynamics. A patch of habitat may be occupied for a few years, then that subpopulation goes extinct, and the habitat is unoccupied for a few years before it is recolonized by immigration. 

Figure \ref{metapop} shows a metapopulation of the Glanville fritillary butterfly.  The black circles were occupied during one survey but the white circles were unoccupied.  Sampling over many years showed that the occupied and unoccupied areas were not always the same.  Finally, individuals of many species tend to be clumped or aggregated.  Thus, even in the center of the range, in high quality habitat, you may not find any individuals, while a short distance away you may find 100s of individuals.

\begin{figure}[hb]
	\centering
	\includegraphics[width=1\textwidth]{metapopulation.jpg}
	\caption{A metapopulation of the Glanville Fritillary Butterfly. Filled circles indicate occupied locations. Open circles indicate unoccupied locations. Whether a particular location is occupied changes over time. Image based on work by Ilkka Hanski (\url{http://www.helsinki.fi/science/metapop/research/Project_metapop.html}) and colleagues \label{metapop}}
\end{figure}

\subsection{Constraints: Disturbance}

The geographic range is strongly influenced by catastrophic disturbances such as floods, hurricanes, volcanic eruptions, or fires. Disturbances affect the geographic range in both positive and negative ways. For example, healthy growth and maintenance of native grasslands (prairies) were promoted by a natural cycle of fire and regeneration that limited growth of shrub and woody species, leaving plenty of light and nutrients for grass species. The modern practice of forest fire suppression has lead to restrictions in the geographic range of native grasses in many parts of the world.  Conservation managers have learned this and often now only extinguish naturally-caused fires if human populations are threatened. As a negative example, overfishing by humans on Jamaican coral reefs threw the entire ecosystem out of balance.  As a consequence, algae started to dominate the reefs (Figure \ref{reef}), which limited the ability of young corals to recruit into the population.  

\begin{figure}[hb]
	\centering
	\includegraphics[width=0.75\textwidth]{algaereef.jpg}
	\caption{An algae-dominated reef in Jamaica. Algal overgrowth, enabled by the loss of long-spined sea urchins, limits recruitment of new corals.\label{reef}}
\end{figure}

Algal growth was held in check only by long-spined sea urchins.  However, Hurricane Allen in 1980 and widespread disease in 1983 wiped out the sea urchins and algae dominated the reefs.  The entire coral reef ecosystem switched to an algal/rock system.  Only now are the reefs beginning to show any signs of recovery. This idea ties back to metapopulation dynamcs: In 1999, Hurricane Floyd devastated island lizard populations in the Bahamas.  At least 66 island populations were destroyed. However, within 17 months, the islands had been recolonized by overwater dispersal from neighboring islands, either by rafting or swimming.

\subsection{Constraints: Species Interactions}
\subsubsection{Competition}
\emph{Competition} between species with similar requirements for growth and survival (e.g. a food resource) can limit population growth for both species. If niche overlap is too great, one taxon can be excluded from otherwise suitable habitats by the presence of a close competitor.  For example, the Red Squirrel, \emph{Sciurus vulgaris}, was once widely distributed through England, as well as Italy.  The introduction of the Eastern Grey Squirrel (\emph{Sciurus carolinensis}) from North America is causing a decline in the range of the Red Squirrel (Figure \ref{squirrel}).  The larger grey squirrel consumes more food, causing a shortage for the Red Squirrel. 

\begin{figure}[hb]
	\centering
	\includegraphics[width=0.8\textwidth]{squirrel.png}
	\caption{Distribution of the native Red Squirrel (dark shading) and the invasive Eastern Grey Squirrel (grey shading) in England. The range of the Red Squirrel has decreased at the same time that the range of the Eastern Grey Squirrel has increased.\label{squirrel}}
\end{figure}

Also, if you study the ranges of any closely related species, you are very likely to find that many species have adjoining ranges but do not overlap.  Closely related species are very likely to use similar resources, so the ranges remain non-overlapping which reduces competition. 

\subsubsection{Predation}
Distributions of predators may be influenced by the geographic range of their prey species. This can also be said for the distributions of parasites and their hosts. Evidence in favor of this idea would include the complete coincidence of predator/prey (or host/parasite) ranges. It's hard to document how predators limit prey distribution in natural systems, but we have gained clues through accidental or intentional introduction of predators outside of their native range. Introduction of largemouth bass (\emph{Micropterus salmoides}) into the southwestern United States has decimated many of that region's native fishes.  The brown tree snake, \emph{Boiga irregularis}, snake was accidentally introduced  to the island of Guam in the south Pacific and has caused extinction of most of the native bird fauna there (up to 10--12 species) (Figure \ref{brown tree snake}). 

\begin{figure}
	\centering
	\includegraphics[width=1\textwidth]{snake.png}
	\caption{Spread of the Brown Tree Snake on Guam. The large years show the approximate time the snake reached that area of the island. The numbers in the boxes indicate the number of bird species that were surveyed in that area over several years. The decline of birds clearly coincides with the arrival of the brown tree snake in each area of the island.\label{brown tree snake}}
\end{figure}


\subsubsection{Mutualism}

\emph{Mutualism} is a third kind of interspecific interaction and one in which both species benefit from the interaction. In such cases, if the mutualism is very specific, the distribution of one species limits the distribution of the other. Most cases of mutualism involve multiple possible species that can take place in the interaction. For instance, there are 8 or so species of hummingbirds in North America which use over 100 species of flowering plants as a food source. Even though the interaction between these birds and the flowers is beneficial to both species, there is little correspondence between the geographic ranges of particular species of hummingbirds and plant species. A good exception is the strong overlap of the ranges of Clark's nutcracker (\emph{Nucifraga columbiana}) and the white bark pine trees (\emph{Pinus albicaulis}) (Figure \ref{nutcracker}). The distribution of the pine (especially) is dependent on the dispersal of seeds by nutcrackers which feed on the cones (and hence release the seeds). On the other hand, the cones/seeds provide an excellent food source for the nutcracker. Both species are found with widely overlapping ranges in the Rocky Mountain sub-alpine habitats.

\begin{figure}[hb]
	\centering
	\includegraphics[width=1\textwidth]{nutcracker.png}
	\caption{Distribution of Clark's Nutcracker, a bird, and the white bark pine tree. The ranges of the two species closely coincide due to the symbiotic interaction (mutualism) between the two species.\label{nutcracker}}
\end{figure}




\end{document}  