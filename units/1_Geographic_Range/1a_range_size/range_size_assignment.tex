%!TEX TS-program = lualatex
%!TEX encoding = UTF-8 Unicode

\documentclass[11pt]{article}
\usepackage{graphicx}
	\graphicspath{{/Users/goby/Pictures/teach/438/lab/}} % set of paths to search for images

\usepackage{geometry}
\geometry{letterpaper}                   
\geometry{bottom=1in}
%\geometry{landscape}                % Activate for for rotated page geometry
\usepackage[parfill]{parskip}    % Activate to begin paragraphs with an empty line rather than an indent
\usepackage{amssymb}
%\usepackage{mathtools}
%	\everymath{\displaystyle}

\usepackage{color}
%\pagenumbering{gobble}

\usepackage{fontspec}
\setmainfont[Ligatures={Common, TeX}, BoldFont={* Bold}, ItalicFont={* Italic}, Numbers={Proportional}]{Linux Libertine O}
\setsansfont[Scale=MatchLowercase,Ligatures=TeX]{Linux Biolinum O}
\setmonofont[Scale=MatchLowercase]{Inconsolata}
\usepackage{microtype}

\usepackage{unicode-math}
\setmathfont[Scale=MatchLowercase]{Asana-Math.otf}
%\setmathfont{XITS Math}

% To define fonts for particular uses within a document. For example, 
% This sets the Libertine font to use tabular number format for tables.
%\newfontfamily{\tablenumbers}[Numbers={Monospaced}]{Linux Libertine O}
%\newfontfamily{\libertinedisplay}{Linux Libertine Display O}


\usepackage{booktabs}
\usepackage{longtable}
%\usepackage{tabularx}
%\usepackage{siunitx}
%\usepackage[justification=raggedright, singlelinecheck=off]{caption}
%\captionsetup{labelsep=period} % Removes colon following figure / table number.
%\captionsetup{tablewithin=none}  % Sequential numbering of tables and figures instead of
%\captionsetup{figurewithin=none} % resetting numbers within each chapter (Intro, M&M, etc.)
%\captionsetup[table]{skip=0pt}

\usepackage{array}
\newcolumntype{L}[1]{>{\raggedright\let\newline\\\arraybackslash\hspace{0pt}}p{#1}}
\newcolumntype{C}[1]{>{\centering\let\newline\\\arraybackslash\hspace{0pt}}p{#1}}
\newcolumntype{R}[1]{>{\raggedleft\let\newline\\\arraybackslash\hspace{0pt}}p{#1}}

%\usepackage{enumitem}
%\usepackage{hyperref}
%\usepackage{placeins} %P4ovides \FloatBarrier to flush all floats before a certain point.

\usepackage{titling}
\setlength{\droptitle}{-50pt}
\posttitle{\par\end{center}}
\predate{}\postdate{}

\usepackage{hanging}

\usepackage{fancyhdr}
\fancyhf{}
\pagestyle{fancy}
%\lhead{}
%\chead{}
%\rhead{Name: \rule{5cm}{0.4pt}}
%\renewcommand{\headrulewidth}{0pt}
\setlength{\headheight}{14pt}
\fancyhead[R]{\footnotesize Range Size \thepage}
\fancyhead[L]{\footnotesize Biogeography}

\newcommand{\bigSpace}{\vspace{5\baselineskip}}

\newlength{\myLength}
\setlength{\myLength}{\parindent}

\title{Geographic Range Size}
\author{10 Points}
\date{}                                           % Activate to display a given date or no date

\begin{document}
\maketitle
\thispagestyle{plain}

The goal of this exercise is for you to discover whether most species of
freshwater fishes, mussels and (in the second section) crayfishes have
large, medium or small range sizes, or if range size seems to be a
random mix among species.

Here, you will determine the range size of these species groups at
different geographic scales. You will first determine range size at the
scale of North America (Nearctic) for either fishes or mussels. You will
then determine range size at the state scale (e.g., Missouri, Georgia)
for fishes, mussels or crayfish. The data sets you use will be assigned
to you randomly.

Before you proceed, what is your hypothesis? Do you think most species
have small, medium or large range sizes, or will range sizes be a mix of
sizes at the scale of North America? (Save your ideas about the smaller
state scale for later.) Discuss your ideas with your partner but your
hypothesis does not have to agree with your partner's hypothesis. If
your hypotheses don't agree, glare at your partner with severe
disapproval for not recognizing your genius!

\textbf{1. Write your hypothesis here.} Considering all of North America, tell
whether you think most species in a closely related group of organisms
have large, medium or small range size, or some combination. Be prepared
to discuss your hypothesis and \emph{your justification for your
prediction} with the rest of the class.

\vspace{7\baselineskip}

For the North American scale, each student pair will be assigned a data
set using either fishes or mussels. For the state scale, each student
pair will get a different state and taxon combination. (e.g., Alabama
fishes, Alabama mussels, Missouri crayfishes, etc.). To determine range
size, you will use R to calculate the number of watersheds (rivers)
occupied by each species, and then plot a \textbf{histogram} to evaluate
the range size. A species that occupies only a few watersheds will be
considered to have a small range size, while a species that occupies
most or all watersheds in the state will be considered to have a large
range size.

\vspace{2ex}

\begin{tabular}{@{}ll@{}}
Name of your North American data set: & \rule{2in}{0.4pt}.\\[3ex]

Name of your State and Taxon data set: & \rule{2in}{0.4pt}.\\
\end{tabular}

A \textbf{histogram} is a type of graph that shows the number of
occurrences within categories. For example, a graph that shows the
number of As, Bs, Cs, Ds and Fs for an exam is a type of histogram. To
be certain you know how to interpret a histogram, we'll build one using
the height of the students in class today. \textbf{Calculate your height
in inches, using the following information}:

If you are 5 ft. 6 in., then add the number of inches of your height to
60 (5 ft. = 60 in.). Your height would be 66 inches. If you are 6 ft. 1
in., add the number of inches of your height to 72 (6 ft. = 72 in). Use
48 inches if you are 4 ft.+ and 84 inches if you are 7 ft.+. If you know your height in centimeters, give me that number. 

\vspace{2ex}

What is your height in inches (or centimeters)? \rule{1in}{0.4pt}.
Tell me this number when asked.

One histogram that you create below will show the number of species that
occupy one watershed, two watersheds, and so on, up to the total number
of watersheds in the data set.

\subsection*{North American Analysis}

In the command below, substitute the file name for the North American
data set you were given for the \textbf{filename} part of the command.
For this data set, the file name will be \textbf{NAfishes.csv} or
\textbf{NAmussels.csv}. 

\texttt{spp \textless{}-
read.csv('http://mtaylor4.semo.edu/\textasciitilde{}goby/biogeo/filename.csv', row.names=1)}

Watersheds (major river systems) are rows and species are columns, with presence or absence of
species indicated by 1s and 0s. 1 indicates a species is present in a
watershed and 0 indicates a species is absent from a watershed.

Now, determine how many watersheds are in the data set, and how many
species of your taxon are in the data set. This is easy to do with the
\textbf{dim()} function. (If you don't remember how to use it, refer to
the R Tutorial exercise you did.)

\textbf{dim()} returns the number of rows and columns, in that order so
the first number is the number of watersheds and the second number is
the number of species in your taxon. Remember the dim()
function as you'll use it again for the second data set.

Your primary goal is to calculate the range size for the species in your
data set, represented by the number of watersheds occupied by each
species. You will also calculate the number of species per watershed. R
provides two functions, \textbf{rowSums()} and \textbf{colSums()}, that
sum across a row or down a column, for all rows or columns in your data
set. The table below shows you results from the two functions for a
small data set.

\label{tab:colrowsums}
\begin{longtable}[c]{@{}L{1.5in}cccC{1.5in}@{}}
\toprule
& Species 1 & Species 2 & Species 3 & \textbf{rowSums()}

species per watershed\tabularnewline
\midrule
\endhead
Watershed 1 & 0 & 1 & 1 & \textbf{2}\tabularnewline
Watershed 2 & 1 & 0 & 0 & \textbf{1}\tabularnewline
Watershed 3 & 1 & 1 & 1 & \textbf{3}\tabularnewline
\textbf{colSums()}

watersheds per species & \textbf{2} & \textbf{2} & \textbf{2}
&\tabularnewline
\bottomrule
\end{longtable}

First, calculate the number of watersheds per species.

\texttt{numWatersheds \textless{}- colSums(spp)}\qquad \# Remember that R is case
sensitive.

\texttt{numWatersheds} \qquad \# view the results

Next, calculate the number of species that occur in each watershed.

\texttt{numSpecies \textless{}- rowSums(spp)}

\texttt{numSpecies} \qquad \# view the results

To create a histogram, use the \textbf{hist()} function. The
\textbf{hist()} function, like \textbf{plot()}, has many arguments to
customize your figure, but begin with a simple histogram.

\texttt{hist(numWatersheds, ylab='Number of Species')}

This histogram provides enough detail for you to determine whether most
species occur in many watersheds (large range size) or in just a few
watersheds (small range size).

\textbf{2. Interpret the pattern.} Describe the pattern that you see. Do most
species occur in most watersheds (large range size) or do most species
occur in just a few watersheds (small range size)? Or, is it some
combination of the two? Explain how the observed result supports or
falsifies your hypothesis in question 1. Be prepared to discuss.

\vspace{7\baselineskip}

Now, use the \textbf{hist()} function to graph the number of species per
watershed. You should be able to figure out what command to type. Use 'Number 
of Watersheds' for the ylab part of the command.

\textbf{3. Interpret your second histogram.} How does the information displayed
in this histogram differ from the first histogram? Don't just say that
one is the number of watersheds occupied by each species and the other
is the number of species in each watershed. Interpret what you see.
Explain below and be prepared to discuss.

\vspace{7\baselineskip}

To see the two figures nicely formatted side by side, type

\texttt{source('http://mtaylor4.semo.edu/\textasciitilde{}goby/biogeo/makeprettyNA.r')}

Expand the window to full screen. 

\subsection*{State Analysis}

Perhaps the pattern that you observed is not surprising given the size
of North America. Let's take a look at a much smaller scale: the state.
This is the type of analysis you might perform as a state conservation
manager, for example. The steps are the same as the steps you
performed for North America.

\textbf{4. Write your hypothesis here.} Considering a single state like Missouri
or Georgia, tell whether you think most species in a closely related
group of organisms have large, medium or small range size, or some
combination. Be prepared to discuss your hypothesis and \emph{your
justification for your prediction} with the rest of the class.

\vspace{7\baselineskip}

\textbf{Get the State data.} Substitute the file name of your \textbf{state} data set for filename in
the command. Be sure to use your state file name, which you wrote down
earlier.

\texttt{spp \textless{}-
read.csv('http://mtaylor4.semo.edu/\textasciitilde{}goby/biogeo/filename.csv', row.names=1)}

Calculate the number of watersheds per species and assign it to the
variable \textbf{numWatersheds}. Do the same for number of species per
watershed and assign to \textbf{numSpecies}. Use \textbf{rowSums()} and
\textbf{colSums()} as needed. If you don't remember which function goes
with watersheds and which goes with species, review page \pageref{tab:colrowsums} of this handout.

Use the \textbf{hist()} function on numWatersheds and numSpecies, as you did above,
to determine whether most species at the state scale have large, medium, or
small size ranges, or a mix.  Use ylab = 'Number of Species' when making the histogram
for numWatersheds. 

\textbf{5. Interpret the pattern.} Describe the pattern that you see. Do most
species occur in most watersheds (large range size) or do most species
occur in just a few watersheds (small range size)? Or, is it some
combination of the two? Explain how the observed result supports or
falsifies your hypothesis in question 4.

\vspace{7\baselineskip}
\newpage

\textbf{6. Repeat this for numSpecies.} Interpret the pattern. 
Use ylab='Number of Watersheds' when making the histogram for numSpecies.
How does the information displayed in this histogram differ from the first histogram?
Don't just say that one is the number of watersheds occupied by each
species and the other is the number of species in each watershed.
Interpret what you see. Explain below and be prepared to discuss.

\vspace{7\baselineskip}


To see the two figures nicely formatted side by side, type

\texttt{source('http://mtaylor4.semo.edu/\textasciitilde{}goby/biogeo/makepretty.r')}

Expand the window to full screen. In some cases, the bars of the histogram might be too wide or too skinny. If so, replace \texttt{makepretty.r} with either \texttt{makepretty2.r} or \texttt{makepretty5.r}. Try to get the width of the bars in the right histogram to be about the same width as the left histogram.

Save your state histograms as a PDF file and upload to the Unit 1a: Range Size drop box. If working in pairs, each of you should upload a copy.

\end{document}  