\documentclass[xcolor=svgnames]{beamer}
%\documentclass[xcolor=dvipsnames]{beamer}
%\usetheme{default}
\usepackage{pgf,pgfpages}
\usepackage{graphicx}
\usepackage{units}
\usepackage[utf8]{inputenc}
\usepackage{tikz}

\tikzstyle{every picture}+=[remember picture]


%\mode<handout> 	% Use this to make PDF with all overlays on one slide.
\mode<presentation>
{
  \usetheme{MST}
  \setbeamercovered{transparent}
  \setbeamertemplate{items}[square]
}

\usefonttheme[onlymath]{serif}
\usecolortheme[named=myBlue]{structure}

%% Uncomment the next two lines to make slides with note lines
% \usepackage{handoutWithNotes}
% \pgfpagesuselayout{3 on 1 with notes}[letterpaper,border shrink=5mm]


\title{Pleistocene}

\begin{document}

\section{Biotic Responses}
\subsection{Ecosystem Changes}

\begin{frame}{Climates and ecosystems shifted south during expansion}
	\begin{center}
		\includegraphics[height=0.75\textheight]{img/climate_marine}\hspace*{1cm}
		\includegraphics[height=0.75\textheight]{img/climate_terrestrial}
	\end{center}
	{\hfill\tiny{See textbook for legend.}}
\end{frame}

\begin{frame}{Ecosystems proportions changed}
	\begin{center}
		\includegraphics[height=0.8\textheight]{img/ecosystem_proportions}
	\end{center}
\end{frame}

\begin{frame}{Glaciers altered prevailing wind patterns}
	\begin{center}
		\includegraphics[height=0.8\textheight]{img/laurentide_climate}
	\end{center}
\end{frame}

\section{Terrestrial Responses}
\subsection{Range Expansion}

\begin{frame}{Trees expanded northward with glacial retreat}
	\begin{columns}[T]
		\begin{column}{0.3\textwidth}
			\includegraphics[width=1\textwidth]{img/tree_range_expansion1}
		\end{column}
		\begin{column}{0.65\textwidth}
			Prevailing winds enhanced range expansion of white spruce (below) but not lodgepole pine (left).\vspace*{0.6\baselineskip}
			\includegraphics[height=0.55\textheight]{img/tree_range_expansion2}
		\end{column}
	\end{columns}
\end{frame}

\begin{frame}{Mammals expanded with shifting ecosystems}
	\begin{center}
		\includegraphics[height=0.85\textheight]{img/mammalian_range_expansion}
	\end{center}
\end{frame}

\begin{frame}{Mammalian co-occurrence in communities changed.}
	\begin{center}
		\includegraphics[width=0.9\textwidth]{img/mammalian_community_change}
	\end{center}
\end{frame}

\subsection{Pleistocene Refugia}

\begin{frame}{Unglaciated areas provided refuge from glacial expansion}
		\begin{center}
			\includegraphics[width=1\textwidth]{img/beringia}
		\end{center}
\end{frame}

\begin{frame}{Many glacial refugia were available}
	\begin{center}
		\includegraphics[width=0.65\textwidth]{img/glacial_refugia}
	\end{center}
\end{frame}

\subsection{Range Expansion}

\begin{frame}{Many taxa recolonized from glacial refugia}
		\begin{center}
			\includegraphics[height=0.8\textheight]{img/north_hemi_range_expansion}
		\end{center}
\end{frame}

\subsection{Speciation}

\begin{frame}{New species evolved by vicariance in refugia}
	\vspace*{-1cm}
	\begin{columns}[T]
		\begin{column}{0.5\textwidth}
			\begin{center}
				\includegraphics[width=0.9\textwidth]{img/ground_squirrel_arctic}\\
				Arctic Ground Squirrel\\
				\footnotesize{Beringia refugium}
			\end{center}
		\end{column}
		\begin{column}{0.5\textwidth}
			\begin{center}
				\includegraphics[width=0.9\textwidth]{img/ground_squirrel_columbian}\\
				Columbian Ground Squirrel\\
				\footnotesize{Southern refugium}
			\end{center}
		\end{column}
	\end{columns}
		\begin{block}{}
		Speciation or population genetic divergence also found in: moose, northern and southern red-backed voles, tundra and arctic shrews, etc.
		\end{block}
\end{frame}

\subsection{Testing for Expansion}

\begin{frame}{Can we test for evidence of refugia and range expansion?}
	\vspace*{-0.5cm}
	\begin{columns}[T]
		\begin{column}{0.6\textwidth}
			\includegraphics[width=0.9\textwidth]{img/macgillivray_range} \hspace*{1cm}
		\end{column}
		\begin{column}{0.4\textwidth}
			\includegraphics[width=0.9\textwidth]{img/macgillivray_picture.jpg}
		\end{column}
	\end{columns}
	\begin{tikzpicture}[overlay, line width=2pt]
		\draw [<-] (1.7,1.5) -- (0.3,1.5) ;
	\end{tikzpicture}
	\begin{block}{}
		Is the breeding resident population of MacGillivray's Warbler in Mexico a refugial population?
	\end{block}
\end{frame}

\begin{frame}[plain]
%	Placeholder to work out parsimony network
\end{frame}

\begin{frame}{Parsimony network supports refuge and range expansion for MacGillivray's Warbler}
	\begin{center}
		\includegraphics[width=0.85\textwidth]{img/macgillivray_network}
	\end{center}
\end{frame}

\section{Aquatic Responses}

\subsection{Marine Taxa}

\begin{frame}{Antitropical distributions could form during Pleistocene}
	\begin{center}
		\includegraphics[height=0.4\textheight]{img/pilot_whale_range}\\
		\scriptsize{Range of Pilot Whale (green shading, above)}\\
		\vspace*{0.25cm}
		\includegraphics[height=0.25\textheight]{img/antitropical}
	\end{center}
\end{frame}

\begin{frame}{New diversity evolved in the southern hemisphere}
	\vspace{-1cm}
	\begin{columns}[T]
		\begin{column}{0.45\textwidth}
			\begin{center}
				\includegraphics[height=0.8\textheight]{img/stomatopod.jpg}
			\end{center}
		\end{column}
		\begin{column}{0.45\textwidth}
			\begin{center}
				\includegraphics[height=0.8\textheight]{img/stomatopod_network.png}
			\end{center}
		\end{column}
	\end{columns}
\end{frame}

\subsection{Freshwater Refugia}

\begin{frame}{Western lakes were refugia for freshwater fishes}
	\begin{columns}[T]
		\begin{column}{0.45\textwidth}
			\includegraphics[height=0.7\textheight]{img/western_freshwater_refugia}
		\end{column}
		\begin{column}{0.45\textwidth}
			The pupfishes of the desert southwest represent \textbf{relict distributions} from once larger lakes.\\
			\vspace*{\baselineskip}
			\includegraphics[width=0.8\textwidth]{img/cyprinodon_diabolis.jpg}
		\end{column}
	\end{columns}
\end{frame}

\begin{frame}{Sea levels were much higher before the Pleistocene}
	\begin{center}
		\includegraphics[height=0.45\textheight]{img/prepleisto_sealevels}\\
		\includegraphics[height=0.35\textheight]{img/pleisto_sealevels}\\
	\end{center}
\end{frame}


\begin{frame}{How did glaciers affect most North American fishes?}
	\begin{center}
		\includegraphics[height=0.8\textheight]{img/glacial_fall_line}
	\end{center}
	\begin{tikzpicture}[overlay, line width=2pt]
		\draw [<-] (5.7,2.5) -- (4.5,2.5) ;
		\draw (4.5,2.5) node[anchor=east] {Fall Line} ;

		\draw [<-] (5.5,5.5) -- (4.3,5.5) ;
		\draw (4.3,5.5) node[anchor=east] {Glacial Maxima} ;
	\end{tikzpicture}
\end{frame}

\end{document}
