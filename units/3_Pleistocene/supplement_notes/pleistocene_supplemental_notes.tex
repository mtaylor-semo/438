%!TEX TS-program = lualatex
%!TEX encoding = UTF-8 Unicode

\documentclass{tufte-handout}

\usepackage{graphicx}
\graphicspath{%
	{/Users/goby/Pictures/teach/438/lectures/}
	{/Users/goby/Pictures/teach/438/lab/}}
	
\usepackage{geometry}
\geometry{letterpaper}                   
\geometry{bottom=1in}
%\geometry{landscape}                % Activate for for rotated page geometry
\usepackage[parfill]{parskip}    % Activate to begin paragraphs with an empty line rather than an indent
\usepackage{amssymb}
%\usepackage{mathtools}
%	\everymath{\displaystyle}

\usepackage{color}
%\pagenumbering{gobble}

\usepackage{fontspec}
\setmainfont[Ligatures={Common, TeX}, BoldFont={* Bold}, ItalicFont={* Italic}, Numbers={Proportional, OldStyle}]{Linux Libertine O}
\setsansfont[Scale=MatchLowercase,Ligatures=TeX]{Linux Biolinum O}
\setmonofont[Scale=0.85]{Linux Libertine Mono O}
\usepackage{microtype}

\usepackage{unicode-math}
\setmathfont[Scale=MatchLowercase]{Asana-Math.otf}
%\setmathfont{XITS Math}

% To define fonts for particular uses within a document. For example, 
% This sets the Libertine font to use tabular number format for tables.
%\newfontfamily{\tablenumbers}[Numbers={Monospaced}]{Linux Libertine O}
%\newfontfamily{\libertinedisplay}{Linux Libertine Display O}


\usepackage{booktabs}
\usepackage{longtable}
\usepackage[justification=raggedright, singlelinecheck=off]{caption}
\captionsetup{labelsep=period} % Removes colon following figure / table number.
\captionsetup{font={small}}
%\captionsetup{tablewithin=none}  % Sequential numbering of tables and figures instead of
%\captionsetup{figurewithin=none} % resetting numbers within each chapter (Intro, M&M, etc.)
%\captionsetup[table]{skip=0pt}

\usepackage{array}
\newcolumntype{L}[1]{>{\raggedright\let\newline\\\arraybackslash\hspace{0pt}}p{#1}}
\newcolumntype{C}[1]{>{\centering\let\newline\\\arraybackslash\hspace{0pt}}p{#1}}
\newcolumntype{R}[1]{>{\raggedleft\let\newline\\\arraybackslash\hspace{0pt}}p{#1}}

%\usepackage{enumitem}
\usepackage{hyperref}
\usepackage{multicol}

%\usepackage{titling}
%\setlength{\droptitle}{-50pt}
%\posttitle{\par\end{center}}
%\predate{}\postdate{}

\usepackage{hanging}
\usepackage{wrapfig}

%\usepackage{titling}
%\usepackage[sc]{titlesec}

\newcommand{\coursename}{\textsc{bi} 438/638: Biogeography}

\usepackage{fancyhdr}
\fancyhf{}
\pagestyle{fancy}
%\lhead{}
%\chead{}
%\rhead{Name: \rule{5cm}{0.4pt}}
%\renewcommand{\headrulewidth}{0pt}
\setlength{\headheight}{14pt}
\fancyhead[R]{\footnotesize Geographic Range Size \thepage}
\fancyhead[L]{\footnotesize \coursename}

\fancypagestyle{first_page}{%
	\fancyhf{}
	\fancyhead[L]{\coursename}
	\fancyhead[R]{Name: \enspace \rule{2.5in}{0.4pt}}
	\renewcommand{\headrulewidth}{0pt}
}

\newcommand{\MYA}{\textsc{mya}}
\newcommand{\bigSpace}{\vspace{5\baselineskip}}

\newlength{\myLength}
\setlength{\myLength}{\parindent}

%\setlength{\droptitle}{-50pt}

\title{Unit 4: Pleistocene Glaciation}
\author{Biogeography}
%\date{Fall 2017}							% Activate to display a given date or no date
\date{}

\fboxsep=0.25mm



\begin{document}

\maketitle% this prints the handout title, author, and date


%\printclassoptions

\newthought{The Pleistocene Period}\marginnote{\textbf{Read:} Chap. 9, pages 313–330, 347–357. Skim pages 330–347 to understand how organisms responded to the environmental changes during the Pleistocene.} lasted from about 2.5 million years ago to just 12,000 years ago.  During the Pleistocene, climate and the geological topography of the Earth's surface, especially in the northern hemisphere, changed dramatically and relatively rapidly, due to massive glaciers\sidenote{Glaciers are large ice sheets that move under their own weight.  The Pleistocene glaciers were as much as 2–3 km thick and, at their maximum, covered as much as 1/3 of the northern hemisphere.}  that covered much of the northern hemisphere.  

The Pleistocene was relatively recent so there is a good record of the climatic and geological changes that occurred. The changes were due to at least four major glaciation events and many smaller events.  These events likely affected species diversity through speciation and extinction, as well as determining the biogeographic distribution of most species that are alive today.  Not all species survived the Pleistocene, however because the end of the Pleistocene is marked by the extinction of the mammalian megafauna such as the wooly mammoths and saber-toothed cats.  This unit covers the causes of glaciation and the effect of glaciation on biogeographic distributions and diversity.

\section{Glaciation}\label{sec:glaciation}
\subsection{Milankovitch Cycles}\label{sec:milankovitch}

The environmental changes that occurred during the Pleistocene were probably not due to the movement of the tectonic plates. Most tectonic plates move at a rate of  1–4 cm yr$^{-1}$.  The last major glacial advance was the Wisconsin glacial stage\sidenote{The names of other North American Pleistocene glacial events are the Illinoian (1 event) and pre-Illinoian (11 events).}, which began about 85,000 years ago, reached it's maximum extent about 21,000 ago, and then retreated. During the entire Wisconsin, the North American tectonic plate would have moved about 2.6 km (1.6 mi), at most.  During the entire Pleistocene, the North American plate would have moved only 100 km (62 mi). 

Such little plate movement cannot explain the relatively rapid environmental changes that occurred during the Pleistocene.  Instead, clues can be found by realizing that the advances and retreats of the Pleistocene ice sheets occurred in cycles.  The glacial events are thought to have been caused by cyclical variations in solar input to Earth due to variations of Earth's orbital path around the Sun and fluctuations of Earth's rotation on its axis.  Collectively, these variations are called the \textit{Milankovitch cycles}\sidenote[][-1cm]{Named for Serbian astronomer Milutin Milankovitch, who published many papers on solar radiation and climate between 1910 and 1920.} and they describe the cyclical changes of Earth's orbital path and rotation. The cyclical changes are due to eccentricity, obliquity, and precession, described below.\sidenote{See pages 318–319 for illustrations of the Milankovitch cycles.}

\textit{Eccentricity} describes how the shape of Earth's orbit changes over a 100,000 year cycle. The shape changes from nearly circular to slightly elongated over the cycle. When the orbit is circular, Earth receives a more uniform amount of solar radiation during one year.  When the orbit is more elongated, solar intensity varies much more over the course of one year.  
\begin{marginfigure}%
	\begin{center}
		\includegraphics[width=0.75\linewidth]{milanko_eccentricity}
	\end{center}
	\caption{Earth's orbit varies from nearly circular (solid line) to somewhat more elongated (dashed line). Although the change seems minimal, the maximum change of distance between Earth and the sun is about 8.5 million miles, or nearly 10\% of the average distance of Earth from the Sun. The greater distance decreases the intensity of solar radiation reaching Earth.}
	\label{fig:eccentricity}
\end{marginfigure} 
\textit{Obliquity} describes how the tilt of Earth's axis changes over a 41,000 year cycle. The axis through the poles is not perfectly vertical but instead varies between 22.1$^{\circ}$ and 24.5$^{\circ}$ off vertical over the cycle. When the tilt is more upright, both hemispheres receive a more uniform amount of solar radiation. When the tilt is greater, solar radiation varies much more from season to season. \textit{Precession} describes the wobble of the Earth over a 26,000 year cycle. Earth wobbles on its axis like a spinning top wobbles around its axis of spin. 
\begin{marginfigure}%
	\includegraphics[width=\linewidth]{milanko_topwobble}
	\caption{Earth's precession is similar to a spinning top wobbling on its axis of spin. The “north pole” of the top's axis points in different directions as the wobble changes.}
	\label{fig:topwobble}
\end{marginfigure}
The interaction of all three of these processes would clearly influence the extent of solar radiation striking by the earth at any particular time.  Less radiation striking the earth means less radiation that can be absorbed, leading to cooler temperatures. For instance, assuming maximum eccentricity (the dashed line in Fig.~\ref{fig:eccentricity}, the earth would be farthest from the sun in July.  If obliquity was low (minimum tilt), Earth would experience a cooler summer because Earth is not tilted as closely towards the sun.\sidenote[][0.5cm]{Our summer is when the northern hemisphere is tilted towards the sun.}  Similarly, the coldest winters would occur when precession is maximal, tilt is maximal, and the orbit is most elliptical.

Mean summer temperature may be the critical factor that determines whether glacial ice sheets retreat or grow.  Glacial ice sheets retreat each summer after growing during the winter.  If summers grow cooler over thousands of years as Earth receives less solar radiation, then glaciers would not retreat as much during the summer relative to their winter gain. Each winter would see a net increase in ice sheet coverage and the beginning of glacial advance.  Thus, the Milankovitch cycles are thought to contribute to the cyclical advance and retreat of glaciers during the Pleistocene.

\subsection{Albedo}\label{sec:albedo}

The large ice sheets can further contribute to global cooling through a process called \textit{albedo}.\sidenote{The fraction of solar radiation reflected off Earth back into the atmosphere.} Ice sheets are bright white and reflect solar radiation back into the atmosphere, rather than absorbing it.  As the ice sheets extend southward, the amount of reflective white surface area increase so Earth's aldebo increased.  Less solar radiation would be retained, further enhancing the cooling effects associated with Milankovitch cycles. Conversely, when the ice sheets begin to retreat, albedo decreases, more solar radiation is absorbed, and the ensuing warming trend in enhanced.

\subsection{Glacial Extent and Climate Change}\label{sec:glaciationextent}

The Pleistocene is characterized by as many as 20 glacial advances and retreats but only four or five were major advances that extended far down into the northern hemisphere. Most glacial ice sheets were found in the northern hemisphere because most land surface area is north of the equator. In the southern hemisphere, glaciers were mostly restricted to high elevations at high latitudes, such as the Andes Mountains. Back in the northern hemisphere, the distribution of glacial ice was uneven. As much as 80\% of glacial ice was found in North America although the northern Palearctic was sometimes covered by large ice sheets (Fig.~\ref{fig:glacialextent}).  In North America, the maximum advance of Pleistocene glaciers is marked by the Missouri and Ohio rivers. 
\begin{marginfigure}[-2cm]%
  \includegraphics[width=\linewidth]{pleisto_glacialextent}
  \caption{Maximum extent of northern hemisphere Pleistocene glaciation (white), as viewed from above the North Pole.}
  \label{fig:glacialextent}
\end{marginfigure}
\begin{figure}[h]
  \includegraphics[width=\linewidth]{pleisto_temps}%
  \caption{Change in global mean annual temperatures during the late Pleistocene (upper two panels), estimated from ice cores taken in Antarctica (\textsc{epica}) and northern Russia (Vostock), compared to estimated glacial ice volume (lower panel; note the inverted Y-axis scale). The dashed line represents current mean annual temperature and ice volume.  Temperature changed as much as much as 8$^{\circ}$C cooler between glacial maxima and minima. Data from~\textsc{noaa}.}%
  \label{fig:pleistotemps}%
  \setfloatalignment{b}
\end{figure}
The advance and retreat of the glacial ice sheets caused relatively rapid climate changes (Fig.~\ref{fig:pleistotemps}). Temperatures cooled as much as 8$^{\circ}C$ below current mean annual global temperatures.  The cooler temperatures were not restricted to areas near the ice sheets because equatorial temperatures also dropped 6--8$^{\circ}$.  Somewhat surprising, temperatures at the foot of the ice sheets were relatively mild due to a process called adiabatic warming.  Recall from above that the ice sheets were as much as 3 km thick.  When the cold air above the glacier reached the edge of the ice sheet, the air mass would descend to the unglaciated terrain below.  As the air mass descended, the air pressure would increase, causing the air mass to compress.  The compression would cause the air temperature to warm without external heat being added (e.g., from sunlight). Globally, glaciers tended to moderate temperature swings between summer and winter.  Both seasons would be relative mild. 

Pleistocene glaciers also shaped terrestrial and aquatic features around the globe. The massive ice sheets had strong erosional capability. The advance and retreat of the glaciers eroded much of North America, which is why the upper Midwest and Canada consists of flat and rolling plains.   The five Great Lakes and many smaller lakes\sidenote{Such as Minnesota's ``10,000 lakes'' and New York's Finger Lakes.} are the direct result of Pleistocene glaciation.  


\vskip0pt plus 1fill

\phantom{made you look}\marginnote{You will have to look harder for extra credit.}



\end{document}