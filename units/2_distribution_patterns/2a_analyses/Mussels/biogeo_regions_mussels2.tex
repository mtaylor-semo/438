%!TEX TS-program = lualatex
%!TEX encoding = UTF-8 Unicode

\documentclass[11pt]{article}
\usepackage{graphicx}
	\graphicspath{{/Users/goby/Pictures/teach/438/lab/}} % set of paths to search for images

\usepackage{geometry}
\geometry{letterpaper}                   
\geometry{bottom=1in}
%\geometry{landscape}                % Activate for for rotated page geometry
\usepackage[parfill]{parskip}    % Activate to begin paragraphs with an empty line rather than an indent
\usepackage{amssymb}
%\usepackage{mathtools}
%	\everymath{\displaystyle}

%\usepackage{color}
%\pagenumbering{gobble}

\usepackage{fontspec}
\setmainfont[Ligatures={Common, TeX}, BoldFont={* Bold}, ItalicFont={* Italic}, BoldItalicFont={* BoldItalic}, Numbers={Proportional}]{Linux Libertine O}
\setsansfont[Scale=MatchLowercase,Ligatures=TeX]{Linux Biolinum O}
\setmonofont[Scale=MatchLowercase]{Inconsolata}
\usepackage{microtype}

\usepackage{unicode-math}
\setmathfont[Scale=MatchLowercase]{Asana-Math.otf}
%\setmathfont{XITS Math}

% To define fonts for particular uses within a document. For example, 
% This sets the Libertine font to use tabular number format for tables.
%\newfontfamily{\tablenumbers}[Numbers={Monospaced}]{Linux Libertine O}
%\newfontfamily{\libertinedisplay}{Linux Libertine Display O}


\usepackage{booktabs}
\usepackage{longtable}
\usepackage{multicol}
%\usepackage{tabularx}
%\usepackage{siunitx}
%\usepackage[justification=raggedright, singlelinecheck=off]{caption}
%\captionsetup{labelsep=period} % Removes colon following figure / table number.
%\captionsetup{tablewithin=none}  % Sequential numbering of tables and figures instead of
%\captionsetup{figurewithin=none} % resetting numbers within each chapter (Intro, M&M, etc.)
%\captionsetup[table]{skip=0pt}

\usepackage{array}
\newcolumntype{L}[1]{>{\raggedright\let\newline\\\arraybackslash\hspace{0pt}}p{#1}}
\newcolumntype{C}[1]{>{\centering\let\newline\\\arraybackslash\hspace{0pt}}p{#1}}
\newcolumntype{R}[1]{>{\raggedleft\let\newline\\\arraybackslash\hspace{0pt}}p{#1}}

\usepackage{enumitem}
%\usepackage{hyperref}
%\usepackage{placeins} %P4ovides \FloatBarrier to flush all floats before a certain point.

\usepackage{titling}
\setlength{\droptitle}{-50pt}
\posttitle{\par\end{center}}
\predate{}\postdate{}

\usepackage{hanging}

\usepackage{fancyhdr}
\fancyhf{}
\pagestyle{fancy}
%\lhead{}
%\chead{}
%\rhead{Name: \rule{5cm}{0.4pt}}
%\renewcommand{\headrulewidth}{0pt}
\setlength{\headheight}{14pt}
\fancyhead[R]{\footnotesize Biogeographic Regions \thepage}
\fancyhead[L]{\footnotesize Biogeography}

\newcommand{\bigSpace}{\vspace{5\baselineskip}}

\newlength{\myLength}
\setlength{\myLength}{\parindent}

\title{Biogeographic Regions}
\author{Freshwater Mussels II}
\date{}                                           % Activate to display a given date or no date

\begin{document}
\maketitle
\thispagestyle{empty}


You recently learned that the world's continental landmasses are divided
into seven major biogeographic regions. \textbf{List below the seven
regions, without looking at any source}.

\begin{multicols}{2}
a.\vspace{0.5\baselineskip}

b.\vspace{0.5\baselineskip}

c.\vspace{0.5\baselineskip}

d.\vspace{0.5\baselineskip}

\columnbreak

e.\vspace{0.5\baselineskip}

f.\vspace{0.5\baselineskip}

g.\vspace{0.5\baselineskip}

\end{multicols}

The seven major regions were originally determined in large part because
mammals and birds showed similar patterns of endemism associated with
the different landmasses. Today, you will learn whether three major
groups of freshwater organisms (fishes, mussels, crayfishes) show
patterns of endemism similar to their terrestrial counterparts.

Work together in pairs to answer the following questions.

%1
\begin{enumerate}[leftmargin=*]
\item Do you think the three freshwater groups will show patterns of
endemism associated with the seven biogeographic regions? Explain why or
why not.\vspace{10\baselineskip}

\end{enumerate}

You were lucky enough to draw freshwater mussels for this exercise. You
have been given a table of data that lists the Subfamilies (-inae),
Tribes (-ini) and genera of freshwater mussels of the family Unionidae.
The table also indicates the number of species present for each genus in
each biogeographic region.

%2
\begin{enumerate}[resume, leftmargin=*]
\item How many genera of mussels are found in each biogeographic region?

\begin{multicols}{2}
a.\vspace{0.5\baselineskip}

b.\vspace{0.5\baselineskip}

c.\vspace{0.5\baselineskip}

d.\vspace{0.5\baselineskip}

e.\vspace{0.5\baselineskip}

f.\vspace{0.5\baselineskip}

g.\vspace{0.5\baselineskip}
\end{multicols}

%3
\item  How many species of mussels are found in each biogeographic
region?

\begin{multicols}{2}
a.\vspace{0.5\baselineskip}

b.\vspace{0.5\baselineskip}

c.\vspace{0.5\baselineskip}

d.\vspace{0.5\baselineskip}

e.\vspace{0.5\baselineskip}

f.\vspace{0.5\baselineskip}

g.\vspace{0.5\baselineskip}
\end{multicols}

%4
\item Which region has the greatest number of genera? Which region has the
least number of genera? Which region has the greatest number of species?
The least? Are they the same regions you listed for the genera?

\begin{multicols}{2}

Greatest number of genera: \vspace{2\baselineskip}

Least number of genera: \vspace{2\baselineskip}

 Greatest number of species: \vspace{2\baselineskip}
 
 Least number of species: \vspace{2\baselineskip}
 
 \end{multicols}
 
 \vspace{\baselineskip}
 
%5
\item List the tribes that are found in more than one biogeographic region.
Name the regions for each tribe you list. \textbf{Do not include the
\textit{incertae sedis} genera with this question}.\vspace*{\stretch{1}}

%6
\item Do any genera show a Pangaea, Laurasian, or Gondwanan distribution?
If so, list the genera, and then explain how you determined which
distribution patterns were appropriate.\vspace*{\stretch{1}}

\newpage

%7
\item Do any of the tribes show a Pangaean, Laurasian or Gondwanan
distribution? If so, list the tribes, and then explain how you
determined which distribution patterns were appropriate. \textbf{Do not
include the \textit{incertae sedis} genera with this question}.\vspace*{\stretch{1}}

%8
\item Are there any genera or tribes with a distribution that is \emph{not 
apparently} consistent with plate tectonics, i.e., a distribution
that is not Pangaean, Laurasian or Gondwanan? If so, list the taxa and
regions in which the taxa are located. How might you try to explain this
distribution? (Time to think about the biology of the organisms.)\vspace*{\stretch{1}}

\newpage

%9
\item In what biogeographic region are the \textit{incertae sedis} genera
located? Is this consistent with the distribution of the rest of the
Ambleminae subfamily?\vspace*{\stretch{1}}

\item Carefully study the phylogeny included with this exercise (or shown
on screen). Is the phylogeny consistent with plate tectonics and the
breakup of Pangaea? Explain.\vspace*{\stretch{1}}

\end{enumerate}

Be prepared to come to the front of the class to discuss some of your
results.

\end{document}  