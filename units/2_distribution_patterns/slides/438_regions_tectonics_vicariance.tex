%!TEX TS-program = lualatex
%!TEX encoding = UTF-8 Unicode

\documentclass[t]{beamer}

%%%% HANDOUTS For online Uncomment the following four lines for handout
%\documentclass[t,handout]{beamer}  %Use this for handouts.
%\usepackage{handoutWithNotes}
%\pgfpagesuselayout{3 on 1 with notes}[letterpaper,border shrink=5mm]
%	\setbeamercolor{background canvas}{bg=black!5}


%%% Including only some slides for students.
%%% Uncomment the following line. For the slides,
%%% use the labels shown below the command.
%\includeonlylecture{student}

%% For students, use \lecture{student}{student}
%% For mine, use \lecture{instructor}{instructor}

% FONTS
\usepackage{fontspec}
\def\mainfont{Linux Biolinum O}
\setmainfont[Ligatures={Common,TeX}, Contextuals={NoAlternate}, BoldFont={* Bold}, ItalicFont={* Italic}, Numbers={Proportional}]{\mainfont}
\setmonofont[Scale=MatchLowercase]{Inconsolatazi4} 
\setsansfont[Scale=MatchLowercase]{Linux Biolinum O} 
\usepackage{microtype}

\usepackage{graphicx}
	\graphicspath{%
	{/Users/goby/Pictures/teach/438/lectures/}%
	{/Users/goby/Pictures/teach/common/}%}%
	{img/}} % set of paths to search for images

\usepackage{amsmath,amssymb}

%\usepackage{units}

\usepackage{booktabs}
\usepackage{multicol}
%	\setlength{\columnsep=1em}

%\usepackage{textcomp}
%\usepackage{setspace}
\usepackage{tikz}
	\tikzstyle{every picture}+=[remember picture,overlay]

\mode<presentation>
{
  \usetheme{Lecture}
  \setbeamercovered{invisible}
  \setbeamertemplate{items}[square]
}

\usepackage{calc}
\usepackage{hyperref}

\newcommand\HiddenWord[1]{%
	\alt<handout>{\rule{\widthof{#1}}{\fboxrule}}{#1}%
}

\begin{document}
%\lecture{instructor}{instructor}
\lecture{student}{student}

\begin{frame}[t]{Our goal for today is to }

	\hangpara learn how \highlight{vicariance} and \highlight{dispersal} contribute to the distribution of organisms.
	
\end{frame}
%
\begin{frame}[t]
	\frametitle<1>{The \highlight{geographic template} explains biogeographic patterns.}
	\frametitle<2>{How does \highlight{dispersal} and \highlight{vicariance} explain distribution patterns?}
	
	\begin{center}
		\includegraphics[width=0.95\textwidth]{geographic_template}
	\end{center}

	
	\begin{tikzpicture}
		\onslide<2>{\draw [ultra thick, color=orange6] (7.85,4.5) rectangle (11.9,6.6);}
	\end{tikzpicture}

	\vfilll
	
	\hfill \tiny Fig.~3.1~\copyright Sinauer Assoc., Inc.

\end{frame}
%
{
\usebackgroundtemplate{\includegraphics[width=\paperwidth]{vicariance_repeated}}
\begin{frame}[b]{Are repeated patterns due to independent dispersal or shared geological events?}

\end{frame}
}
%
\begin{frame}[t]{If dispersal, was it by land bridge or other means?}
	\begin{multicols}{2}
	
		\noindent\includegraphics[width=0.49\textwidth]{dispersal_land_bridge}
	
		\columnbreak
		
		\hangpara How far can an organism disperse?
	
		\hangpara Can low \highlight{vagility} organisms disperse large distances?
	
		\hangpara Could dispersal by other means explain repeated patterns?
	
	\end{multicols}

	\vfilll
	
	\hfill \tiny Fig.~2.8 \copyright Sinauer Assoc., Inc.
\end{frame}
%
\lecture{instructor}{instructor}
% See https://whyevolutionistrue.wordpress.com/2013/09/13/nasa-launches-a-frog-and-experimental-biogeograhy/ for background.
{
\usebackgroundtemplate{\includegraphics[width=\paperwidth]{flying_frog1}}
\begin{frame}[b]

	\hfill \tiny \textcolor{white}{Courtesy \href{https://whyevolutionistrue.wordpress.com/2013/09/13/nasa-launches-a-frog-and-experimental-biogeograhy/}{Why Evolution Is True}}
\end{frame}
}
%
{
\usebackgroundtemplate{\includegraphics[width=\paperwidth]{flying_frog2}}
\begin{frame}[b]

	\hfill \tiny \textcolor{white}{Courtesy \href{https://whyevolutionistrue.wordpress.com/2013/09/13/nasa-launches-a-frog-and-experimental-biogeograhy/}{Why Evolution Is True}}
\end{frame}
}
%

\lecture{student}{student}
%
{
\usebackgroundtemplate{\includegraphics[width=\paperwidth]{barbour_darlington_dispersal}}

\begin{frame}[b]{Dispersal: a tale of land bridges, hurricanes, and flying frogs.}

\hfill \tiny \textcolor{white}{Courtesy \href{https://whyevolutionistrue.wordpress.com/2013/09/13/nasa-launches-a-frog-and-experimental-biogeograhy/}{Why Evolution Is True}}

\end{frame}
}
%
\begin{frame}[t]{\highlight{Vicariance} does explain repeated patterns.}

\includegraphics[width=\textwidth]{vicariance_model}
\begin{multicols}{2}
	\includegraphics[width=0.49\textwidth]{vicariance_plants}

	\columnbreak
	
	\includegraphics[width=0.49\textwidth]{vicariance_leaffish}
	
\end{multicols}

	\vfilll
	
	\hfill \tiny \copyright Sinauer Assoc., Inc.
\end{frame}
%
\begin{frame}[t]{The distribution of many taxa is consistent with vicariance.}

	\includegraphics[width=\textwidth]{vicariance_cycads}
	
	{\centering
	Distribution of Cycadales (cycad plants)\par
	}
	
	\vfilll
	
	\hfill \tiny Esculapio, Wikimedia, \ccbysa{3}
\end{frame}
%
\begin{frame}[t]{\highlight{Gondwanan} and \highlight{Laurasian} distributions are due to vicariance.}

	\includegraphics[width=\textwidth]{vicariance_ratites}
	
	{\centering
	Distribution of ratite birds.\par	
	}
	\vfilll
	
	\hfill \tiny Fig.~10.26 \copyright Sinauer Assoc., Inc.
\end{frame}
%
\begin{frame}[t]{\highlight{Plate tectonics} explains vicariant distributions.}

	\vspace*{-0.5\baselineskip}
	
	\includegraphics[width=\textwidth]{vicariance_tectonic_plates}
	
	\vfilll
	
	\hfill \tiny Fig.~8.18 \copyright Sinauer Assoc., Inc.
\end{frame}
%
\begin{frame}[t]{Shared geological features support vicariance.}

	\vspace*{-0.5\baselineskip}
	
	\includegraphics[width=\textwidth]{vicariance_geological_formations}
	
	\vfilll
	
	\hfill \tiny Box~8.1\textsc{a} \copyright Sinauer Assoc., Inc.
\end{frame}
%
\begin{frame}[t]

%	\vspace*{-0.5\baselineskip}
	{\centering
	\includegraphics[width=0.95\textwidth]{vicariance_reptile_mammals}\par
	}
	
	\vfilll
	
	\hfill \tiny Box~8.1\textsc{b} \copyright Sinauer Assoc., Inc.
\end{frame}
%
\begin{frame}[t]

	{\centering
	\includegraphics[width=0.95\textwidth]{vicariance_various}\par
	}
	
	\vfilll
	
	\hfill \tiny Box~8.1\textsc{c} \copyright Sinauer Assoc., Inc.
\end{frame}
%
\begin{frame}[t]

	{\centering
	\includegraphics[height=0.46\textheight]{tectonics_proterozoic}\\[1ex]
	\includegraphics[height=0.46\textheight]{tectonics_silurian}\par
	}
	
	\vfilll
	
	\hfill \tiny Fig.~8.22 \copyright Sinauer Assoc., Inc.
\end{frame}
%
\begin{frame}[t]

	{\centering
	\includegraphics[height=0.46\textheight]{tectonics_triassic}\\[1ex]
	\includegraphics[height=0.46\textheight]{tectonics_cretaceous}\par
	}
	
	\vfilll
	
	\hfill \tiny Fig.~8.22 \copyright Sinauer Assoc., Inc.
\end{frame}
%
\begin{frame}[t]

	{\centering
	\includegraphics[height=0.46\textheight]{tectonics_tertiary}\\[1ex]
	\includegraphics[height=0.46\textheight]{tectonics_miocene}\par
	}
	
	\vfilll
	
	\hfill \tiny Fig.~8.22 \copyright Sinauer Assoc., Inc.
\end{frame}
%
\begin{frame}[t]

	{\centering
	\includegraphics[height=0.46\textheight]{tectonics_pleistocene}\\[1ex]
	\includegraphics[height=0.46\textheight]{tectonics_future}\par
	}
	
	\vfilll
	
	\hfill \tiny Fig.~8.22 \copyright Sinauer Assoc., Inc.
\end{frame}
%
%\begin{frame}[t]{Continental rearrangement caused climate change.}
%
%	{\centering
%	\includegraphics[width=0.9\textwidth]{tectonics_climate_change}\par
%	}
%	
%	\vfilll
%	
%	\tiny Fig.~8.30 \copyright Sinauer Assoc., Inc.
%\end{frame}
%
\begin{frame}[t]{The breakup of Gondwana can be represented as a cladogram (tree).}

	\vspace*{-1\baselineskip}
	
	{\centering
	\includegraphics[width=\textwidth]{gondwana_cladogram}\par
	}
	
	\vfilll
	
%	\tiny Fig.~8.30 \copyright Sinauer Assoc., Inc.
\end{frame}
%
\begin{frame}[t]{Faunal similarity corresponds to tectonic separation.}

	{\centering
	\includegraphics[width=\textwidth]{tectonics_faunal_similarity}\par
	}
	
	\vfilll
	
	\hfill \tiny Fig.~8.32 and Table 8.1 \copyright Sinauer Assoc., Inc.
\end{frame}
%
\begin{frame}[t]{\highlight{Indices of biotic similarity} compare numbers of taxa shared between areas.}


	\begin{multicols}{2}
	
	\begin{center}
	{\LARGE
		$\dfrac{C}{N_1 + N_2 - C}$
	}
	
	\vspace*{\baselineskip}
	
	Jaccard's
	\end{center}
	
	\columnbreak
	
	\begin{center}
	{\LARGE
		$\dfrac{C}{N_1}$
	}
	
	\vspace*{\baselineskip}

	Simpson's
	\end{center}
	\end{multicols}
	
	\hangpara $C$ = number of shared taxa, \\
	$N_1$ = total number of taxa in area 1 (the smaller number)\\
	$N_2$ = total number of taxa in area 2 (the other number)

\end{frame}
%
\begin{frame}[t]{Similarity of global mammalian families.}

	\includegraphics[width=\textwidth]{similarity_mammalian_families}
	
\end{frame}
%
\begin{frame}[t]{Similarity of global mammalian genera.}

	\includegraphics[width=\textwidth]{similarity_mammalian_genera}
	
\end{frame}
%
\begin{frame}[t]{Mammalian similarity can be represented as a cladogram.}

	\includegraphics[width=\textwidth]{similarity_mammalian_cladograms}
	
\end{frame}
%
\begin{frame}[t]

	\includegraphics[width=\textwidth]{similarity_southeast_asia}
	
\end{frame}
%
{
\usebackgroundtemplate{\includegraphics[width=\paperwidth]{similarity_southeast_asia2}}
\begin{frame}[b]

\end{frame}
}
%
\begin{frame}[t]{\highlight{Vagile} and \highlight{volant} species can have active dispersal.}

	\includegraphics[width=\textwidth]{dispersal_vagile_volant}
	
\end{frame}
%
\begin{frame}[t]{Other organisms have passive dispersal.}

	\begin{multicols}{2}
		\includegraphics[width=0.49\textwidth]{dispersal_passive_manowar}
		
		\columnbreak
		
		\includegraphics[width=0.49\textwidth]{dispersal_passive_dandelion}
		
	\end{multicols}
	
\end{frame}
%
\begin{frame}[t]

	{\centering
	\includegraphics[height=0.85\textheight]{dispersal_deer_tickseed}\par
	}
	
	\vfilll
	
	\hfill \tiny Michael S.~Taylor, \ccbysa{3}
\end{frame}
%
\begin{frame}[t]{Range expansion can occur through jump or diffusion dispersal.}

	\vspace*{-0.5\baselineskip}
	{\centering
	\includegraphics[height=0.82\textheight]{dispersal_jump_diffusion}\par
	}
	
	\vfilll
	
%	\hfill \tiny Michael S.~Taylor, \ccbysa{3}
\end{frame}
%
\begin{frame}[t]{How would you explain this distribution for the Camelidae?}

	\vspace*{-0.5\baselineskip}
	{\centering
	\includegraphics[height=0.82\textheight]{camel_distribution}\par
	}
	
	\vfilll
	
%	\hfill \tiny Michael S.~Taylor, \ccbysa{3}
\end{frame}
%
\begin{frame}[t]{\highlight{Secular migration} is range expansion with evolutionary divergence.}

	\vspace*{-0.5\baselineskip}
	{\centering
	\includegraphics[height=0.82\textheight]{camel_distribution_historical}\par
	}
	
	\vfilll
	
%	\hfill \tiny Michael S.~Taylor, \ccbysa{3}
\end{frame}
%
{
\usebackgroundtemplate{\includegraphics[width=\paperwidth]{dispersal_sweepstakes_route}}
\begin{frame}[b]

\end{frame}
}
%
\begin{frame}[t]{Barriers place limits on dispersal.}

	\vspace*{-1\baselineskip}
	
	\begin{columns}
		\begin{column}{0.6\textwidth}
		
		\includegraphics[width=\textwidth]{dispersal_barriers}
		\end{column}
		
		\begin{column}{0.35\textwidth}
		
		\hangpara \highlight{Ecological barriers.}

		\hangpara \highlight{Psychological barriers.}

		\hangpara Barrier effectiveness depends on biotic and abiotic factors.
		\end{column}
	\end{columns}
	
\end{frame}
%

\end{document}
