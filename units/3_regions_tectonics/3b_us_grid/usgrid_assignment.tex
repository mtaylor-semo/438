%!TEX TS-program = lualatex
%!TEX encoding = UTF-8 Unicode

\documentclass[11pt]{article}
\usepackage{graphicx}
	\graphicspath{{/Users/goby/Pictures/teach/438/lab/}} % set of paths to search for images

\usepackage{geometry}
\geometry{letterpaper}                   
\geometry{bottom=1in}
%\geometry{landscape}                % Activate for for rotated page geometry
\usepackage[parfill]{parskip}    % Activate to begin paragraphs with an empty line rather than an indent
\usepackage{amssymb}
%\usepackage{mathtools}
%	\everymath{\displaystyle}

\usepackage{color}
%\pagenumbering{gobble}

\usepackage{fontspec}
\setmainfont[Ligatures={Common, TeX}, BoldFont={* Bold}, ItalicFont={* Italic}, Numbers={Proportional, OldStyle}]{Linux Libertine O}
\setsansfont[Scale=MatchLowercase,Ligatures=TeX]{Linux Biolinum O}
\setmonofont[Scale=0.82]{Linux Libertine Mono O}
\usepackage{microtype}

\usepackage{unicode-math}
\setmathfont[Scale=MatchLowercase]{Asana-Math.otf}
%\setmathfont{XITS Math}


\usepackage{calc}

\usepackage{booktabs}
\usepackage{longtable}
\usepackage{multicol}
\usepackage{listings}
%\usepackage[justification=raggedright, singlelinecheck=off]{caption}
%\captionsetup{labelsep=period} % Removes colon following figure / table number.
%\captionsetup{tablewithin=none}  % Sequential numbering of tables and figures instead of
%\captionsetup{figurewithin=none} % resetting numbers within each chapter (Intro, M&M, etc.)
%\captionsetup[table]{skip=0pt}

\usepackage{array}
\newcolumntype{L}[1]{>{\raggedright\let\newline\\\arraybackslash\hspace{0pt}}p{#1}}
\newcolumntype{C}[1]{>{\centering\let\newline\\\arraybackslash\hspace{0pt}}p{#1}}
\newcolumntype{R}[1]{>{\raggedleft\let\newline\\\arraybackslash\hspace{0pt}}p{#1}}

\usepackage{enumitem}
\setlist[enumerate]{leftmargin=-\parindent}
\setlist[1]{labelindent=\parindent}
%\setlist[enumerate]{label=\textsc{\alph*}.}


%\usepackage{hyperref}
%\usepackage{placeins} %P4ovides \FloatBarrier to flush all floats before a certain point.

\newcommand{\coursename}{\textsc{bi} 438/638: Biogeography}
\newcommand*{\assignmentTitle}{Distribution of species richness}

\usepackage[sc]{titlesec}


\usepackage{fancyhdr}
\fancyhf{}
\pagestyle{fancy}
\setlength{\headheight}{14pt}
\fancyhead[R]{\footnotesize \assignmentTitle{} \thepage}
\fancyhead[L]{\footnotesize \coursename}

\fancypagestyle{first_page}{%
	\fancyhf{}
	\fancyhead[L]{\coursename}
	\fancyhead[R]{Name: \enspace \rule{2.5in}{0.4pt}}
	\renewcommand{\headrulewidth}{0pt}
}

\newcommand{\bigSpace}{\vspace{5\baselineskip}}

\newlength{\myLength}
\setlength{\myLength}{\parindent}


\begin{document}
\thispagestyle{first_page}

\subsection*{\assignmentTitle}

You are going to make a “raster map” for species richness of U.S.~fishes. The data were
obtained by creating a presence / absence matrix for each species of
native fish. Presence or absence was based on 1°\,$\times$\,1° longitude /
latitude grids. If a fish species was present in a grid cell then 1 was
entered for that grid cell. If the species was absent, 0 was entered. Finally, all matrices were summed together to create
the final data matrix.

\begin{enumerate}%[leftmargin=*]
\item \textbf{Make a prediction.} Where in the U.S.~do you think species richness will be the highest? Northeastern U.S.? Southeastern
U.S.? Northwest? Southwest? Midwest? Take a look at the map of U.S.~rivers that was given to you.  Below, name the region and 3--4 rivers in that 
region where you think diversity may be the highest.\vspace{10\baselineskip}
\end{enumerate}

\begin{enumerate}[resume]
\item \textbf{Download all data files and all ``.r'' files} files from the "Pleistocene data files” page of the “Unit 4: Pleistocene” module in Canvas. Move the files to your “biogeo” folder with your “biogeo.Rproj” file.
\end{enumerate}

You will run each ".r” file in RStudio as instructed. If you get an error at
any point, you typed something wrong and will need to reenter the
command. If you do not get an error message, then you can proceed to the
next command. Call Dr.~Taylor if you get stuck.

%The following commands will create a ``raster map'' with shades
%corresponding to the number of species present. 

\begin{enumerate}[resume]
\item \textbf{Set up the R environment for your work.} “setup.r” will install several R packages that you need for today's exercises. \textit{This script may run for a few minutes as it downloads and installs the packages.}
\end{enumerate}

\begin{verbatim}
source("setup.r")
\end{verbatim}

If you wish, you can click the files tab in the lower right pane, find “setup.r”, and then click the file name. The file will open in the upper left panel of RStudio. After the file is open, click the “Source” button  (\,\raisebox{.2\baselineskip-.5\height}{\includegraphics[height=10px]{source_button.png}}\,) above and right of the script to run it.


When the script is finished, you'll be returned to the \texttt{>} prompt. As the script runs, you will see information in the console about any packages being installed. If, by chance, you already had all packages installed, you will see only the prompt.

\begin{enumerate}[resume]
\item \textbf{Plot the map.} Your map will cover
an area between 125°W to 65°W and between 24°N to 49°N. Compare those
coordinates to your watershed map.
\end{enumerate}


\begin{verbatim}
source("nafish_diversity.r") 
\end{verbatim}

\textit{If at any time you do not see a plot, be sure the “Plot” tab is selected in the lower right pane.}

Zoom the map to full size and admire its glory! Your map covers
the area between 125°\textsc{w} to 65°\textsc{w} and between 24°\textsc{n} to 49°\textsc{n.} Compare those
coordinates to your watershed map. 

\textit{Lighter colors represent higher species richness. Darker colors represent lower species richness.}


\begin{enumerate}[resume]
\item \textbf{How does the map compare to your prediction?} Where is the overall
richness the greatest? In what watershed(s) does the species richness appear to be highest? Is it in just one location? More than one location?\vspace{10\baselineskip}
\end{enumerate}

Locations where species richness is very high are often called
\textbf{biodiversity hotspots.} Such hotspots are often areas of conservation
concern. Preserving biodiversity hotspots protects the greatest number
of species, with efficient use of conservation dollars.

\subsubsection*{Your turn to play}

Below is a list of taxonomic groups (mostly families, a couple genera) of North American fishes.  These groups
have relatively large numbers of species with different distribution
patterns across the U.S. You will plot the distribution for each of the 10 groups.

\begin{multicols}{2}
catostomid \\
cottid \\
cyprinid1 \\
cyprinid2 \\
cyprinodont \\
etheostoma \\
fundulus \\
ictalurid \\
percid \\
salmonid
\end{multicols}

\begin{enumerate}[resume]
\item \textbf{Run the command below \emph{once} for each group listed.} (Remember that you can recall
previous commands to plot the map by pressing the up cursor
arrow key to recall previous commands.) You will be prompted to enter one of the group names each time you run the command. Study the distribution of each group as it is mapped.
\end{enumerate}

\begin{verbatim}
source("diversity.r")
\end{verbatim}


\begin{enumerate}[resume]
\item \textbf{On the next page, describe the distribution for each group as you plot it.} Think about the group distibution relative the glacial maximum and the fall line. Are they northern? Eastern U.S.?
Mostly along a coastline? Which of your groups have similar or  different distributions? 

I suggest also that you read the questions at the end to think about answers as you plot the maps. \vspace{\baselineskip}

%How do the distributions relate to the Pleistocene glacial maximum and the Fall Line?

\newpage

This list is alphabetical but you can explore them in any order.

	\begin{enumerate}[label=\alph*., leftmargin=*]

		\item Catostomids (suckers)\vspace{3\baselineskip}

		\item Cottids (sculpins)\vspace{3\baselineskip}

		\item Cyprinids1 (\textit{Notropis} and \textit{Cyprinella})\vspace{3\baselineskip}

		\item Cyprinids2 (other minnow and shiners)\vspace{3\baselineskip}

		\item Cyprinodonts (topminnows, ex.~\textit{Fundulus})\vspace{3\baselineskip}

		\item \textit{Etheostoma} (darters)\vspace{3\baselineskip}

		\item \textit{Fundulus} (topminnows)\vspace{3\baselineskip}

		\item Ictalurids (catfishes)\vspace{3\baselineskip}

		\item Percid (perches ex.~\textit{Etheostoma})\vspace{3\baselineskip}

		\item Salmonids (salmon and trout)\vspace{3\baselineskip}
	\end{enumerate}
\end{enumerate}

\subsubsection*{Questions to answer. Turn in your complete answers before leaving.}

\smallskip

\begin{enumerate}[resume]
	\item \textbf{Which two group have their greatest species richness concentrated well north of the glacial maxima line?} Carefully study the areas where these groups have the highest richness. If these regions were covered by glaciers, where do you think the ancestors of these species might have survived? \textit{Note:} the freshwater rivers to the south are not directly connected to the regions with highest richness.

	\vspace{6\baselineskip}

	\item \textbf{Which group or groups of fishes have greatest species richness in the coastal plain?} Some of the coastal species in this data set extend from the Gulf of Mexico, around peninsular Florida, and part way up the Atlantic Coast. How could these coastal species have such a broad range if the freshwater rivers were never connected? (Hint: Do you think these coastal rivers are strictly freshwater all the way to the their mouths?)
	
	\vspace{6\baselineskip}
	
	\item \textbf{Which groups have the greatest species richness in the interior (eastern and central) highlands?}  Why do you think these regions have very high richness compared to the coastal plain or more northern latitudes? \textit{Be specific.}
	
	\vspace{6\baselineskip}

	\item \textbf{Species richness of catostomids and ictalurids is broadly distributed across the eastern \textsc{u.s.?}} Hypothesize a reason why this might be the case. \textsc{Hint:} I'm sure a hint would be welcomed.

	\vspace{6\baselineskip}
	
	\item \textbf{Grad students only:} Write a question that you think I could have asked about the data that would help your interpretation. You must write a true and serious question.
	

\end{enumerate}

\end{document}  