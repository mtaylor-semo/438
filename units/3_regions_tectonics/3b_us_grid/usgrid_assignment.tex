%!TEX TS-program = lualatex
%!TEX encoding = UTF-8 Unicode

\documentclass[11pt]{article}
\usepackage{graphicx}
	\graphicspath{{/Users/goby/Pictures/teach/438/lab/}} % set of paths to search for images

\usepackage{geometry}
\geometry{letterpaper}                   
\geometry{bottom=1in}
%\geometry{landscape}                % Activate for for rotated page geometry
\usepackage[parfill]{parskip}    % Activate to begin paragraphs with an empty line rather than an indent
\usepackage{amssymb}
%\usepackage{mathtools}
%	\everymath{\displaystyle}

\usepackage{color}
%\pagenumbering{gobble}

\usepackage{fontspec}
\setmainfont[Ligatures={Common, TeX}, BoldFont={* Bold}, ItalicFont={* Italic}, Numbers={Proportional}]{Linux Libertine O}
\setsansfont[Scale=MatchLowercase,Ligatures=TeX]{Linux Biolinum O}
\setmonofont[Scale=0.82]{Linux Libertine Mono O}
\usepackage{microtype}

\usepackage{unicode-math}
\setmathfont[Scale=MatchLowercase]{Asana-Math.otf}
%\setmathfont{XITS Math}

% To define fonts for particular uses within a document. For example, 
% This sets the Libertine font to use tabular number format for tables.
%\newfontfamily{\tablenumbers}[Numbers={Monospaced}]{Linux Libertine O}
%\newfontfamily{\libertinedisplay}{Linux Libertine Display O}


\usepackage{booktabs}
\usepackage{longtable}
\usepackage{multicol}
\usepackage{listings}
%\usepackage[justification=raggedright, singlelinecheck=off]{caption}
%\captionsetup{labelsep=period} % Removes colon following figure / table number.
%\captionsetup{tablewithin=none}  % Sequential numbering of tables and figures instead of
%\captionsetup{figurewithin=none} % resetting numbers within each chapter (Intro, M&M, etc.)
%\captionsetup[table]{skip=0pt}

\usepackage{array}
\newcolumntype{L}[1]{>{\raggedright\let\newline\\\arraybackslash\hspace{0pt}}p{#1}}
\newcolumntype{C}[1]{>{\centering\let\newline\\\arraybackslash\hspace{0pt}}p{#1}}
\newcolumntype{R}[1]{>{\raggedleft\let\newline\\\arraybackslash\hspace{0pt}}p{#1}}

\usepackage{enumitem}
%\usepackage{hyperref}
%\usepackage{placeins} %P4ovides \FloatBarrier to flush all floats before a certain point.



\usepackage[sc]{titlesec}


\usepackage{fancyhdr}
\fancyhf{}
\pagestyle{fancy}
\lhead{}
\chead{}
\rhead{\footnotesize pg.~\thepage }
\renewcommand{\headrulewidth}{0.4pt}

\fancypagestyle{plain}{%
	\fancyhf{}
	\lhead{\textsc{bi}~438/638: Biogeography}
	\rhead{Name: \enspace \makebox[2.5in]{\hrulefill}}
	\renewcommand{\headrulewidth}{0pt}
}

%\usepackage{hanging}

%\usepackage{fancyhdr}
%\fancyhf{}
%\pagestyle{fancy}
%%\lhead{}
%%\chead{}
%%\rhead{Name: \rule{5cm}{0.4pt}}
%%\renewcommand{\headrulewidth}{0pt}
%\setlength{\headheight}{14pt}
%\fancyhead[R]{\footnotesize Distribution of Diversity \thepage}
%\fancyhead[L]{\footnotesize Biogeography}

\newcommand{\bigSpace}{\vspace{5\baselineskip}}

\newlength{\myLength}
\setlength{\myLength}{\parindent}

\title{Distribution of Species Richness}
\author{10 Points}
\date{}                                           % Activate to display a given date or no date

\begin{document}
\thispagestyle{plain}

\subsection*{Distribution of Species Richness}

You are going to plot the species richness for U.S. fishes. The data were
obtained by creating a presence / absence matrix for each species of
native fish. Presence or absence was based on 1° x 1° longitude /
latitude grids. If a fish was present in a grid cell then a 1 was
entered for that grid cell. If the fish was absent, a 0 was entered into
the grid cell. Finally, all the matrices were summed together to create
the final data matrix.

\begin{enumerate}[leftmargin=*]
\item \textbf{Make a prediction.} Where in the U.S. 
do you think species richness will be the highest? Northeast U.S.? Southeast
U.S.? Northwest? Southwest? Midwest? Take a look at the map of U.S. 
Rivers that was given to you?  Below, name the region and 3--4 rivers in that 
region where you think diversity may be the highest.\vspace{10\baselineskip}
\end{enumerate}

The following commands will create a ``filled contour map'' with shades
corresponding to the number of species present. If you get an error at
any point, you typed something wrong and will need to reenter the
command. If you do not get an error message, then you can proceed to the
next command.

\begin{verbatim}
nagrid <- read.csv('http://mtaylor4.semo.edu/~goby/biogeo/nafish_grid.csv')
\end{verbatim}

The next three commands read in the geographic coordinates (longitude
and latitude) necessary to draw state boundaries, rivers, and the US
coast.

\begin{verbatim}
bnd <- read.table('http://mtaylor4.semo.edu/~goby/biogeo/boundaries.txt')
rivers <- read.table('http://mtaylor4.semo.edu/~goby/biogeo/rivers.txt')
coastline <- read.table('http://mtaylor4.semo.edu/~goby/biogeo/coastLoRes.txt')
\end{verbatim}

The next two commands set the longitude (x) and latitude (y) coordinates
for the map. The x values are negative because longitudes west of the
Greenwich Meridian are treated as negative numbers. Your map will cover
an area between 125°W to 65°W and between 24°N to 49°N. Compare those
coordinates to your watershed map.

\begin{verbatim}
x <- -125:-65
y <- 24:49
\end{verbatim}

\newpage

The following commands draw the actual map. You \emph{could} type it exactly as
shown, pressing the Enter key after you have typed the final parenthesis:

% . Hit the return key after you have typed the final parenthesis in
%the command below. If it does not work, you typed something incorrectly.
%It is essentially one long (but weird looking) sentence.

\begin{verbatim}
filled.contour(x, y, t(nagrid), color = terrain.colors, plot.title =
  title(main = "Distribution of Species Richness for U.S. Freshwater Fishes", xlab =
  "Longitude (°W)", ylab = "Latitude (°N)"), plot.axes = { axis(1);
  axis(2); lines(coastline); lines(bnd, col = 'grey35', lty=3);
  lines(rivers, col = 'grey40', lwd='0.75')})
\end{verbatim}

Or, just type:

\begin{verbatim}
source('http://mtaylor4.semo.edu/~goby/biogeo/nafish_diversity.r')
\end{verbatim}

Expand the map to full size and admire its glory. If it didn't work, 
press the ``up'' cursor arrow to recall the full command, and then 
carefully compare it to the command above. Move the cursor to 
the error, correct it, and then press Enter.

The shading corresponds to species richness. Dark green indicates
relatively low richness. Whites indicate high species richness.
Yellowish-green and tans indicate intermediate richness.

\begin{enumerate}[resume, leftmargin=*]
\item \textbf{How does the map compare to your prediction?} Where is the overall
richness the greatest? In what watershed(s) is the species richness
highest? Is it in just one location? More than one location?\vspace{10\baselineskip}
\end{enumerate}

Locations where species richness is very high are often called
biodiversity hotspots. Such hotspots are often areas of conservation
concern. Preserving biodiversity hotspots protects the greatest number
of species, with efficient use of conservation dollars.

\textbf{Now it is your turn to play.} I've divided the North American
fishes into several major groups of fishes, listed below. These groups
have relatively large numbers of species with different distribution
patterns across the U.S. You will be given a selection of five
groups to plot.

\begin{multicols}{2}
catostomid \\
cottid \\
cyprinid1 \\
cyprinid2 \\
cyprinodontiform \\
etheostoma \\
fundulus \\
ictalurid \\
percidae \\
salmonid
\end{multicols}
\newpage

Plot the different data files to see how the
distributions of different groups of fishes differ. Type the following
commands, replacing NAME with one of the names you were assigned, such as
etheostoma or cottid.

\begin{verbatim}
nagrid <- read.csv('http://mtaylor4.semo.edu/~goby/biogeo/NAME.csv')
source('http://mtaylor4.semo.edu/~goby/biogeo/diversity.r')
\end{verbatim}

Repeat this for each group on your list. Remember that you can recall
previous commands to plot the map by pressing the up cursor
arrow key to recall previous commands. Press the up cursor key a couple 
times until you recall the \texttt{nagrid \textless{}- $\ldots$} command, change the NAME,
then press the Enter key to read in the new file. Next, press the up cursor
arrow a couple times to recall the \texttt{source($\ldots$)} command and then
press the Enter key to make the new plot.

\begin{enumerate}[resume, leftmargin=*]
\item \textbf{Below, write the name of each group and
describe the distribution of the fishes.} Are they northern? Eastern US?
Mostly along the coastline? Do any of your groups have similar
distributions? Do any of your groups have different distributions? \vspace{\baselineskip}
%How do the distributions relate to the Pleistocene glacial maximum and the Fall Line?
	\begin{enumerate}[label=\alph*., leftmargin=*]
		\item %\vspace{5\baselineskip}

		\item \vspace{5\baselineskip}

		\item \vspace{5\baselineskip}

		\item \vspace{5\baselineskip}

		\item \vspace{5\baselineskip}
	\end{enumerate}
\end{enumerate}


\end{document}  