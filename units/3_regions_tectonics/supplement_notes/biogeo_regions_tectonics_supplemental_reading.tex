%!TEX TS-program = lualatex
%!TEX encoding = UTF-8 Unicode

\documentclass[11pt, hidelinks]{article}
\usepackage{graphicx}
\graphicspath{{/Users/goby/Pictures/teach/438/lab/}%
			  {/Users/goby/Pictures/teach/438/lectures/}} % set of paths to search for images

\usepackage{geometry}
\geometry{letterpaper}                   
\geometry{bottom=1in}
%\geometry{landscape}                % Activate for for rotated page geometry
\usepackage[parfill]{parskip}    % Activate to begin paragraphs with an empty line rather than an indent
\usepackage{amssymb}
%\usepackage{mathtools}
%	\everymath{\displaystyle}

\usepackage{color}
%\pagenumbering{gobble}

\usepackage{fontspec}
\setmainfont[Ligatures={Common, TeX}, BoldFont={* Bold}, ItalicFont={* Italic}, Numbers={Proportional, OldStyle}]{Linux Libertine O}
\setsansfont[Scale=MatchLowercase,Ligatures=TeX]{Linux Biolinum O}
\setmonofont[Scale=0.85]{Linux Libertine Mono O}
\usepackage{microtype}

\usepackage{unicode-math}
\setmathfont[Scale=MatchLowercase]{Asana-Math.otf}
%\setmathfont{XITS Math}

% To define fonts for particular uses within a document. For example, 
% This sets the Libertine font to use tabular number format for tables.
%\newfontfamily{\tablenumbers}[Numbers={Monospaced}]{Linux Libertine O}
%\newfontfamily{\libertinedisplay}{Linux Libertine Display O}


\usepackage{booktabs}
\usepackage{longtable}
\usepackage[justification=raggedright, singlelinecheck=off]{caption}
\captionsetup{labelsep=period} % Removes colon following figure / table number.
\captionsetup{font={small}}
%\captionsetup{tablewithin=none}  % Sequential numbering of tables and figures instead of
%\captionsetup{figurewithin=none} % resetting numbers within each chapter (Intro, M&M, etc.)
%\captionsetup[table]{skip=0pt}

\usepackage{array}
\newcolumntype{L}[1]{>{\raggedright\let\newline\\\arraybackslash\hspace{0pt}}p{#1}}
\newcolumntype{C}[1]{>{\centering\let\newline\\\arraybackslash\hspace{0pt}}p{#1}}
\newcolumntype{R}[1]{>{\raggedleft\let\newline\\\arraybackslash\hspace{0pt}}p{#1}}

%\usepackage{enumitem}
\usepackage{hyperref}
\usepackage{multicol}

%\usepackage{titling}
%\setlength{\droptitle}{-50pt}
%\posttitle{\par\end{center}}
%\predate{}\postdate{}

\usepackage{hanging}
\usepackage{wrapfig}

\usepackage{titling}
\usepackage[sc]{titlesec}

\newcommand{\coursename}{\textsc{bi} 438/638: Biogeography}

\usepackage{fancyhdr}
\fancyhf{}
\pagestyle{fancy}
%\lhead{}
%\chead{}
%\rhead{Name: \rule{5cm}{0.4pt}}
%\renewcommand{\headrulewidth}{0pt}
\setlength{\headheight}{14pt}
\fancyhead[R]{\footnotesize Geographic Range Size \thepage}
\fancyhead[L]{\footnotesize \coursename}

\fancypagestyle{first_page}{%
	\fancyhf{}
	\fancyhead[L]{\coursename}
	\fancyhead[R]{Name: \enspace \rule{2.5in}{0.4pt}}
	\renewcommand{\headrulewidth}{0pt}
}

\newcommand{\MYA}{\textsc{mya}}
\newcommand{\bigSpace}{\vspace{5\baselineskip}}

\newlength{\myLength}
\setlength{\myLength}{\parindent}

%\setlength{\droptitle}{-50pt}

\title{Unit 3: Biogeographic Regions and Plate Tectonics}
\author{Biogeography}
%\date{Fall 2017}							% Activate to display a given date or no date
\date{}

\fboxsep=0.25mm

\begin{document}
\maketitle
\section{Assigned Reading}

Chapter~10, pages~362-384.

For the material on plate tectonics, carefully study these notes and the following pages of your text. Carefully study the terms and concepts that are emphasized in these notes. I will not cover plate tectonics in lecture but I expect you to know the material from the notes and associated textbook pages for assignments and exams.

Chapter~8, pages 263--269, 272--274, 280--285, 287--290 (the illustrations), 291--296, especially Figure~8.25.

\section{Endemism}
As you learned in the last unit, the geographic range is the basic observation unit of biogeography.  You also learned that although some species, such as Blue Whales and Peregrin Falcons, have \emph{cosmopolitan} distributions, most species have much smaller distributions, often restricted to one region and one region only.  The presence of a particular species or \emph{taxon}\footnote{A taxon is a convenience term that can mean any particular classification category.  For example, a genus, a family, an order, or genetic lineages with a species can all be treated as taxa.  Taxon is singular (e.g., one particular family or genus); taxa is plural (e.g., several families or genera). A former student once remarked that it must be my favorite word because I use it so often.} in one particular area is called \emph{endemism}. Endemism describes the distribution of a taxon that occurs in one place and one place only. For example, the Tasmanian Devil is endemic to the island of Tasmania, of the southeast coast of Australia.

Endemism is relative. A  species may be endemic to only a very small area, such as a spring (remember the Devil's Hole Pupfish?) or small island, or we may consider a larger taxon that is endemic to one particular continent. Endemism was one of the first patterns described by biogeographers, stemming from the observation that islands typically have high levels of endemism.  Australia, for instance, as an island continent has 91\% of its mammal fauna (e.g., kangaroos, koalas, thylacines) as endemic compared to about 19\% for the Nearctic and Palearctic together.  (see below.)  High endemism on islands probably results from the geographic isolation of islands from mainland habitats. The isolation creates a high potential for evolutionary divergence occurring entirely within the island flora and fauna.  For example, several adaptive radiations have occurred within the Hawaiian islands.  The Hawaiian \emph{Drosophila} numbers over 1000 species, the silversword group of plants numbers 50 species, and the Hawaiian honeycreepers (birds) number at least 54 species.  All of these taxa are endemic to the Hawaiian islands. 

Endemism is hierarchical, or nested.  This means that the distribution of lower taxa (e.g., species, genus) is nested within the distribution of the higher taxon (e.g., family, order).  A classic example of nested endemism involves the kangaroo rats (\emph{Dipodomys}) and pocket mice (genus \emph{Chaetodipus}) of the family Heteromyidae.  The Heteromyidae is endemic to the New World, including most of the western United States, Central America, and extreme northwestern South America, as outlined with light brown in Figure~\ref{heteroEndemism} (upper panel).  Genera have smaller ranges nested within the larger range of the family. For example, \emph{Dipodomys} is endemic to the western United States and northwest Mexico. \emph{Microdipodops} is endemic to the Great Basin Area (Great Salt Lake and nearby areas of Nevada, California,  and Arizona). Species have yet smaller ranges nested within the larger range of the genus, as shown for two species of \emph{Microdipodops} (Figure~\ref{heteroEndemism}.  Even within the range of one species, distinct genetic lineages may be endemic to yet smaller regions. For example \emph{M. pallidus} (Figure~\ref{heteroEndemism}, lower left panel, dark green) has two distinct genetic lineages (Figure~\ref{heteroEndemism}, lower right panel).  Each mitochondrial lineage is endemic to a specific region within the overall distribution of the species. Overall, the distributions of lower taxa tend to be mosaics of non-overlapping and partially overlapping distributions that together determine the distribution of the higher taxon. Most taxa display this \emph{hierarchy of endemism}.  There are few taxa (e.g., families) where all of the species in that taxon are equally distributed across the entire range of the taxon.  Species tend to have more localized distributions within the broader range of the higher taxon.


Endemics could have evolved in place, or they could have evolved elsewhere but somehow ended up in their current distribution.  The Hawaiian endemics I mentioned earlier certainly evolved on the islands, perhaps following dispersal of the original ancestor.  This probably applies to most island endemics.  Your text makes a distinction between evolving in place (autochthonous endemism) versus first evolving elsewhere (allochthonous endemism), but such a distinction may be hard to determine in practice without detailed historical evidence, so do not concern yourself with this difference.

\begin{figure}[tbp]
	\centering
		\includegraphics[width=0.76\textwidth]{heteromyidae_hierarchy}
		\caption{Upper panel: Distribution of two heteromyid genera in western North America. Lower taxonomic levels (e.g., genera) are nested within the larger endemic distribution of higher taxonomic levels (e.g., families). Lower panel: Non-overlapping ranges of two species nested within the genus \emph{Microdipodops} (left panel). Populations within species can be genetic subdivided but still nested within a species (right panel). The genetic subdivisions are endemic to specific regions within the overall distribution of the species.(Fig. 10.3 of your text.)\label{heteroEndemism}}
		
\end{figure}


\section{Global Biogeographic Regions}

Early biogeographers noted that the continents tended to have very high levels of endemism.  We already noted the especially high (91\%) endemism of Australia.  Other continents also have large numbers of endemic taxa, although not to the extent of Australia.  This observation of high continental endemism led early biogeographers to divide the globe into seven distinct biogeographic regions or realms.  Two versions, very similar to each other, were developed by Philip Sclater in 1858 and by Alfred Russel Wallace in 1876. The map drawn by Wallace is inside the front cover of your text. Both scientists arrived at similar conclusions based primarily on the distribution of mammals and birds. The regions defined by Sclater and Wallace are still used today, with little modification, although some of the names of the regions have changed (Figure~\ref{biogeoregions}).  The seven terrestrial regions are: 

\begin{itemize}
\item Nearctic --- North America
\item Neotropical --- South America
\item Palearctic --- Europe, northern Asia
\item Afrotropical --- Africa
\item Indomalaysia --- India and other parts of southern Asia down through the Malaysian Peninsula and Indonesia.
\item Australian --- Australia
\item Antarctic --- Antarctica, New Zealand, plus the southern tips of South America and Africa
\end{itemize}

You may see other names for Afrotropical (Ethiopian) and Indomalaysian (Oriental), as shown in the Wallace figure in your text.  These names are common in the older literature but you should use the names listed above. The islands of the Pacific Ocean, including Hawaii, are sometimes placed into an eighth region called Oceania (Figure~\ref{biogeoregions}). A different but similar arrangement of the biogeographic regions was recently proposed by Olson~et~al.~(2001) (Figure~\ref{Olson Biogeo Regions}). The primary difference is the continental regions that compose the Antarctic region.  Yet another arrangement of the regions is provided by the authors of your text on page~383.  For this course, you should learn the arrangement shown here in Figure~\ref{biogeoregions}.

\begin{figure}[tb]
	\centering
		\fbox{\includegraphics[width=0.95\textwidth]{regions_terrestrial_oceania}}
		\caption{Eight biogeographic regions described by Sclater and Wallace, with updated names.  Oceania is sometimes used to include the many Pacific islands.\label{biogeoregions}}
		
\end{figure}

\begin{figure}[tb]
	\centering
		\fbox{\includegraphics[width=0.95\textwidth]{biogeo_regions_olson}}
		\caption{A different delineation of the eight biogeographic regions described by Olson~et~al.,~2001. Terrestrial ecoregions of the world: A new map of life on earth. Bioscience 51:\,933-938.\label{Olson Biogeo Regions}}
		
\end{figure}


The oceans themselves are generally divided into four regions. The four regions are, from the equator to the poles, Tropical, Warm Temperate, Cold Temperate, and Polar. Unless the terrestrial regions, the marine regions are clearly associated with temperature.  You must know the seven terrestrial regions, plus Oceania and be able to draw them on a map, as they are critical to our discussions of biogeography.

\section{Plate Tectonics}

Since the 1600s, many scientists noted that some continents, especially South America and Africa, seemed to fit together like pieces of a puzzle, but none had proposed a satisfactory model of how the continents could move. The first comprehensive model of continental drift was published in 1912 by Alfred Wegener. Wegener had little evidence, however, so his ideas were not widely accepted until the 1960s, when geological exploration of the ocean floors yielded considerable evidence. However, Wegener's model turned out to be inaccurate, even if his overall conception was sound.

\begin{figure}[tb]
	\centering
		\includegraphics[width=\textwidth]{tectonic_plates}  
		\caption{The eight major plates that form the earth's crust are the Eurasian Plate, the North American Plate, the Australian Plate, the Pacific Plate, the South American Plate, the African Plate and the Indian Plate, and the Antarctic Plate. Note that some references treat the Australian and Indian plates as a single Indoaustralian Plate. Some of the minor plates shown include the Caribbean Plate, the Arabian Plate, the Filipino Plate (aka the Philippines Plate), the Nazca Plate, and the Scotia Plate. The red arrows indicate the direction of movement for the plate boundaries. You are not required to memorize the plates, but you should be familiar with the major plates.\label{tectonic plates}}
		
\end{figure}

The current model of \emph{plate tectonic theory} holds that the earth's crust is composed of 7--8  major plates, such as the North American and Pacific plates, plus many more smaller plates, such as the Caribbean and Nazca plates (Figure~\ref{tectonic plates}). The plates can be composed of less dense \emph{continental crust} or more dense \emph{oceanic crust}.  The plates move on convective currents created by the magma that lies beneath the plates. If two plates collide, one of three processes might occur. If two plates of continental crust collide, then mountain ranges can form. If two oceanic plates collide, they can sink to form deep trenches. The most common occurrence, however, is when the less dense continental crust slides above the denser, sinking oceanic crust. The sinking oceanic crust forms a \emph{subduction zone} (see Fig. 8.20 of your text) as it is pulled down (\emph{slab pull}) beneath the continental crust. Subduction zones cause high levels of seismic activity. For example, the Pacific and Nazca plates, which form most of the Pacific Ocean sea floor, are sinking beneath the North American, South American, Philippine and Australian plates (Figure~\ref{tectonic plates}), causing numerous earthquakes and volcanic eruptions. 

Slab pull separates plates at their other boundary, creating ridges such as the Mid-Atlantic Ridge (Figure~\ref{mid atlantic ridge}). As the plates are pulled apart, magma rises through through the created gap, and then cools to form new crust. This process is called \emph{seafloor spreading}. As the magma rises, it pushes up the edges of the two plates, creating a high ridge (Figure~\ref{mid atlantic ridge}). The weight of ridge, combined with slab pull, continues to move the new crust away from the ridge (\emph{ridge push}), allowing magma to continue to rise and form new crust.

\begin{figure}[tb]
	\centering
		\fbox{\includegraphics[width=0.75\textwidth]{mid_atlantic_ridge}}
		\caption{The Mid-Atlantic Ridge is formed by seafloor spreading. The ridge is the dark-grey uplifted area that snakes north-south between the continental landmasses.\label{mid atlantic ridge}}
		
\end{figure}

\section{Plate Tectonics and Biogeography}

The tectonic history of the earth has had an enormous influence on the geographic distribution of organisms, especially higher taxa\footnote{If you list Linneaus' classification hierarchy with Domain and Kingdom at the top, and Genus and Species at the bottom, then relative terms like higher and lower taxa make intuitive sense. Higher taxa are high on the list, lower taxa are at the bottom of the list.} like families and orders. To understand how the moving continents have influenced the biogeographic distribution of taxa, you need to know three major landmasses that existed as part of earth's geological history. The landmasses are Pangaea, Laurasia, and Gondwana (sometimes called Gondwanaland in the older literature). \emph{Pangaea} (Figure~\ref{Pangaea}, upper panel) was formed about 250 million years ago (\MYA) when all of the landmasses were united into a single supercontinent. By the mid-Jurassic, about 170–160 \MYA, Pangaea broke up into two large landmasses, the northern Laurasia landmass and the southern Gondwana (Figure~\ref{Pangaea}, lower panel). \emph{Laurasia} was composed of landmasses that eventually formed most of North America and Eurasia. \emph{Gondwana} was composed of landmasses that included South America, Africa, India (including Madagascar), Australia, New Zealand, and Antarctica.

\begin{figure}[tbp]
	\centering
		\fbox{\includegraphics[width=0.85\textwidth]{permian_jurassic}}
		\caption{Upper panel: Early stages of Pangaea formation during the late Permian period (255~\MYA). Compare this figure with Fig. 8.22\textsc{e} on page~288 of your text. Lower panel: Formation of Laurasia and Gondwana during the Jurassic period (152~\MYA). Compare this figure with Fig. 8.22\textsc{f} and~\textsc{g} on pages~288–289 of your text. Images from The Paleomap Project. \url{http://www.scotese.com} \label{Pangaea}}
		
\end{figure}

The modern distribution of higher taxa often provides clues about the earlier distribution of their ancestors. For taxa found on nearly all continents, their ancestors may have been spread across the Pangaean supercontinent. These taxa have a Pangaean distribution. If a taxon is limited to North America and Eurasia, their ancestors may have evolved in Laurasia sometime after the breakup of Pangaea. This taxon would have a Laurasian distribution. Similarly, if a taxon is present on some combination of South America, Africa, India, Australia or Antarctica, their ancestors may have evolved on Gondwana after the breakup of Pangaea. This taxon has a Gondwanan distribution. Study, for example, the figures shown in Box~8.1 to see several examples of fossil and modern taxa that have a Gondwanan distribution. 

\section{Pangaea and the Biogeographic Regions}

You can now relate this information to the biogeographic regions. Recall that the seven major biogeographic regions were determined based primarily on high levels of endemic taxa. Much of this endemism can be explain by plate tectonics. 

For example, a higher taxon, such as a family, that evolved on Gondwana after the breakup of Pangaea will not be present in Laurasia, except via uncommon instances of dispersal (which you'll learn about later). If the higher taxon continues to diversify (the evolution of new genera and species), the lower taxa will only be found on Gondwanan landmasses. As Gondwana continued to break up into the continents we now recognize, evolutionary processes would continue to form new genera and species endemic to each of these continents. For example, carefully study Figure~12.5 on page~467 of your text. Notice the repeated diversification of several genera of flies (Chironomidae) on Australia, South America and New Zealand. These genera show a pattern of endemism restricted to Gondawnan landmasses, which you  now know as parts of the Neotropical, Australian and Antarctica biogeographic regions.

More broadly, Gondwanan taxa may be present in the Neotropical, Afrotropical, Indomalaysian, Australian or Antarctic biogeographic regions (Figure~\ref{Gondwana Distributions}), or some combination of these regions, as you saw with the flies in the text. The specific continents occupied by a taxon will depend on when and where they evolved relative to the break up of Gondwana, as well as subsequent evolutionary events, such as extinction.  Similarly, a Laurasian taxon would be found in the Nearctic and the Palearctic, such as the Betulaceae (birch trees and relatives) (Figure~\ref{Betulaceae Map}). The major biogeographic regions were recognized long before any concept of continental movement was accepted.  In fact, these Pangaean, Laurasian, and Gondwana distributions for fossil and living taxa provided strong evidence for the eventual acceptance of plate tectonic theory. 

\begin{figure}[tb]
	\centering
		\fbox{\includegraphics[width=0.75\textwidth]{gondwana_distribs}}
		\caption{Gondwanan distributions of four fossil taxa. Fossils of the reptiles \emph{Cynognathus} and \emph{Mesosaurus} have only been found on South America and Africa. \emph{Lystrosaurus} fossils have been found on Africa, India and Antarctica. The fossil fern \emph{Glossopteris} has been found on all five continents that composed Gondwana.\label{Gondwana Distributions}}
		
\end{figure}

\begin{figure}[tb]
	\centering
		\fbox{\includegraphics[width=0.75\textwidth]{betulaceae_map}}
		\caption{Laurasian distributions of the Betulaceae, which include birch, hazel, and hornbeam trees, and other relatives. The earliest fossils are from the late Cretaceous, consistent with a Laurasian distribution. The distribution of modern taxa in the Neotropical and Indomalaysian biogeographic regions is likely due to later dispersal.\label{Betulaceae Map}}
		
\end{figure}
 
The distributions of some organisms suggest an association with a particular supercontinent (such as Pangaea) but may not be. Figure~\ref{Bear Distribution} shows the distribution of the bear family Ursidae.  The four species in the genus \emph{Ursus} (the Brown, Black, Polar and Asian bears) are distributed primarily in the northern hemisphere. (Any distribution of organisms that is restricted to the northern hemisphere is called a \emph{holarctic distribution}).  The bear species of the other four genera (\emph{Melursus}, \emph{Helarctos}, \emph{Ailuropoda} and \emph{Tremarctos}) are located in India, China, the Malaysian Peninsula, and South America. The distribution of the Ursidae suggests that the earliest ancestral bears may have occupied some part of Pangaea before it separated but this hypothesis is not supported by the fossil evidence. The earliest ancestral fossil bears are from North America during the late Eocene period about 38~\MYA, long after the breakup of Laurasia (Figure~\ref{Eocene Map}). Fossil bears did not appear in Eurasia until the early Oligocene about 30--33~\MYA, perhaps after crossing a Bering land bridge that formed about 37~\MYA.  Fossil bears do not appear in South America until the Pleistocene, about 1~\MYA.  Thus, the distribution of modern taxa across the continents provides clues about their evolutionary origins and subsequent biogeographic history, the distributions should be matched with the fossil record, if available, to assemble an accurate evolutionary history.  

\begin{figure}[tb]
	\centering
		\fbox{\includegraphics[width=0.90\textwidth]{bear_distribution}}
		\caption{Distribution of the bear family Ursidae. The family appears to have a Pangaean distribution but the family didn't evolve until the Eocene, well after the breakup of the Pangaean supercontinents. For practice, name the biogeographic region associated with each genus and species.\label{Bear Distribution}}
		
\end{figure}

\begin{figure}[tb]
	\centering
		\includegraphics[width=0.85\textwidth]{eocene}  
		\caption{Distribution of continental landmasses during the early Eocene (50~\MYA). Ancestral bears did not appear until 12 million years later.\label{Eocene Map}}
		
\end{figure}


You should be able to relate the seven terrestrial biogeographic regions to the tectonic history of Pangaea, Laurasia and Gondwana. For practice, and without looking at what you just read, list the biogeographic regions that belong to Pangaea, then the regions that belong to Laurasia, and then the regions that belong to Gondwana. For 2 extra credit points, e-mail this list to me \emph{before} the vicariance lecture.


\end{document}  