%!TEX TS-program = lualatex
%!TEX encoding = UTF-8 Unicode

\documentclass[t]{beamer}

%%%% HANDOUTS For online Uncomment the following four lines for handout
%\documentclass[t,handout]{beamer}  %Use this for handouts.
%\usepackage{handoutWithNotes}
%\pgfpagesuselayout{3 on 1 with notes}[letterpaper,border shrink=5mm]

%\includeonlylecture{student}

%% For students, use \lecture{student}{student}
%% For mine, use \lecture{instructor}{instructor}

% FONTS
\usepackage{fontspec}
\def\mainfont{Linux Biolinum O}
\setmainfont[Ligatures={Common,TeX}, Contextuals={NoAlternate}, BoldFont={* Bold}, ItalicFont={* Italic}, Numbers={Proportional}]{\mainfont}
\setmonofont[Scale=MatchLowercase]{Inconsolatazi4} 
\setsansfont[Scale=MatchLowercase]{Linux Biolinum O} 
\usepackage{microtype}

\usepackage{graphicx}
	\graphicspath{%
	{/Users/goby/Pictures/teach/438/lab/}%
	{/Users/goby/Pictures/teach/common/}%}%
	{img/}} % set of paths to search for images

\usepackage{amsmath,amssymb}

%\usepackage{units}

\usepackage{booktabs}
\usepackage{multicol}
%	\setlength{\columnsep=1em}

%\usepackage{textcomp}
%\usepackage{setspace}
%\usepackage{tikz}
%	\tikzstyle{every picture}+=[remember picture,overlay]

\mode<presentation>
{
  \usetheme{Lecture}
  \setbeamercovered{invisible}
  \setbeamertemplate{items}[square]
}

\usepackage{calc}
\usepackage{hyperref}

\newcommand\HiddenWord[1]{%
	\alt<handout>{\rule{\widthof{#1}}{\fboxrule}}{#1}%
}

\begin{document}
%\lecture{instructor}{instructor}
\lecture{student}{student}

\begin{frame}[t]
	\includegraphics[width=\textwidth]{global_continents_outline}
\end{frame}
%
\begin{frame}[t]
	\includegraphics[height=0.95\textheight]{mussel_families}
\end{frame}
%
\begin{frame}[t]
	\includegraphics[width=\textwidth]{mussel_distribution_global}
\end{frame}
%
\begin{frame}[t]
	\centering
	\includegraphics[width=0.95\textwidth]{mussel_distribution_hyriidae}\\
	\includegraphics[width=0.95\textwidth]{mussel_distribution_mycetopodidae}
\end{frame}
%
\begin{frame}[t]
	\includegraphics[width=\textwidth]{mussel_distribution_unioninae}
\end{frame}
%
\begin{frame}[t]
	\includegraphics[width=\textwidth]{mussel_distribution_ambleminae}
\end{frame}
%
\begin{frame}[t]
	\includegraphics[width=\textwidth]{crayfish_distribution_global}
\end{frame}
%
\begin{frame}[t]{Are these crayfish phylogenies consistent with a Pangean breakup?}
	\includegraphics[width=\textwidth]{crayfish_phylogenies}
\end{frame}
%
\begin{frame}[t]{Are these crayfish phylogenies consistent with a Pangean breakup?}
	\includegraphics[width=\textwidth]{crayfish_phylogenies}
\end{frame}
%
{
\usebackgroundtemplate{\includegraphics[width=\paperwidth]{cichlidae_distribution}}
\begin{frame}[t]
\end{frame}
}
%
\begin{frame}[t]{Do these phylogenies predict the order of the continental breakup of Gondwana?}
	\centering
	\vspace{-0.5\baselineskip}
	\includegraphics[height=0.8\textheight]{cichlid_phylogenies}
\end{frame}
%
\begin{frame}[t]
	\includegraphics[width=\textwidth]{cichlid_gondwana_breakup}
\end{frame}
%

%
%
%{
%\usebackgroundtemplate{\includegraphics[width=\paperwidth]{regions_polyodon}}
%\begin{frame}[b]{\highlight{Relict distributions} can arise from extinction.}
%
%	\tiny\hfill\textit{Polyodon spatula}, The Earth Society, Wikimedia Commons.
%\end{frame}
%}
%

%
\end{document}
