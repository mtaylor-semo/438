%!TEX TS-program = lualatex
%!TEX encoding = UTF-8 Unicode

\documentclass[11pt]{article}
\usepackage{graphicx}
	\graphicspath{{/Users/goby/Pictures/teach/438/lab/}} % set of paths to search for images

\usepackage{geometry}
\geometry{letterpaper}                   
\geometry{bottom=1in}
%\geometry{landscape}                % Activate for for rotated page geometry
\usepackage[parfill]{parskip}    % Activate to begin paragraphs with an empty line rather than an indent
\usepackage{amssymb}
%\usepackage{mathtools}
%	\everymath{\displaystyle}

\usepackage{color}
%\pagenumbering{gobble}

\usepackage{fontspec}
\setmainfont[Ligatures={Common, TeX}, BoldFont={* Bold}, ItalicFont={* Italic}, Numbers={Proportional}]{Linux Libertine O}
\setsansfont[Scale=MatchLowercase,Ligatures=TeX]{Linux Biolinum O}
\setmonofont[Scale=0.82]{Linux Libertine Mono O}
\usepackage{microtype}

\usepackage{unicode-math}
\setmathfont[Scale=MatchLowercase]{Asana-Math.otf}
%\setmathfont{XITS Math}

% To define fonts for particular uses within a document. For example, 
% This sets the Libertine font to use tabular number format for tables.
%\newfontfamily{\tablenumbers}[Numbers={Monospaced}]{Linux Libertine O}
%\newfontfamily{\libertinedisplay}{Linux Libertine Display O}


\usepackage{booktabs}
\usepackage{longtable}
\usepackage{multicol}
\usepackage{listings}
%\usepackage[justification=raggedright, singlelinecheck=off]{caption}
%\captionsetup{labelsep=period} % Removes colon following figure / table number.
%\captionsetup{tablewithin=none}  % Sequential numbering of tables and figures instead of
%\captionsetup{figurewithin=none} % resetting numbers within each chapter (Intro, M&M, etc.)
%\captionsetup[table]{skip=0pt}

\usepackage{array}
\newcolumntype{L}[1]{>{\raggedright\let\newline\\\arraybackslash\hspace{0pt}}p{#1}}
\newcolumntype{C}[1]{>{\centering\let\newline\\\arraybackslash\hspace{0pt}}p{#1}}
\newcolumntype{R}[1]{>{\raggedleft\let\newline\\\arraybackslash\hspace{0pt}}p{#1}}

\usepackage{enumitem}
%\usepackage{hyperref}
%\usepackage{placeins} %P4ovides \FloatBarrier to flush all floats before a certain point.

\usepackage[sc]{titlesec}

%\usepackage{titling}
%\setlength{\droptitle}{-50pt}
%\posttitle{\par\end{center}}
%\predate{}\postdate{}

%\usepackage{hanging}

\newcommand{\coursename}{\textsc{bi} 438/638: Biogeography}

\usepackage{fancyhdr}
\fancyhf{}
\pagestyle{fancy}
%\lhead{}
%\chead{}
%\rhead{Name: \rule{5cm}{0.4pt}}
%\renewcommand{\headrulewidth}{0pt}
\setlength{\headheight}{14pt}
\fancyhead[R]{\footnotesize Geographic Range Size \thepage}
\fancyhead[L]{\footnotesize \coursename}

\fancypagestyle{first_page}{%
	\fancyhf{}
	\fancyhead[L]{\coursename}
	\fancyhead[R]{Name: \enspace \rule{2.5in}{0.4pt}}
	\renewcommand{\headrulewidth}{0pt}
}

\newcommand{\bigSpace}{\vspace{5\baselineskip}}

\newlength{\myLength}
\setlength{\myLength}{\parindent}

\title{Biogeographic Provinces}
\author{}
\date{}                                           % Activate to display a given date or no date

\begin{document}
%\maketitle
\thispagestyle{first_page}

\subsection*{Biogeographic provinces}

You recently learned about the eight major biogeographic regions. 
You also learned that the regions are divided into provinces, each with
endemic taxa.  Today, you will use two different analytical techniques to explore 
differences among aquatic provinces based on the fishes present in each 
province.

The first technique you will use is called \textbf{cluster analysis.} Clusters
group together objects (such as biogeographic provinces or river watersheds)
based on similarity of taxa. We'll use the fishes of Montana to show you
how to interpret the results of a cluster analysis.  The figure below shows the
major watersheds of Montana.  The thick line shows the location of the
continental divide. Rivers west of the continental divide (left of the line)
flow into the Pacific Ocean. Rivers east of the continental divide (right
of the line) flow into the Gulf of Mexico.  All of the rivers east of the divide
in the figure are part of the larger Missouri River watershed.

\begin{center}
	\includegraphics[width=\linewidth]{montana_watersheds}
\end{center}

\begin{enumerate}[leftmargin=*]
\item \textbf{Make a prediction.} If you compared the fish fauna from 
rivers west of the continental divide  to rivers east of the divide, do you think
they will he very similar or very different? Explain.%\vspace{8\baselineskip}
\end{enumerate}


\newpage

The cluster analysis of the Montana fishes began with a presence/absence
grid. A species was given a 1 if present in a watershed and a zero if absent. 
The grid is converted to a similarity matrix. Watersheds that have different
fish faunas will have low similarity values. Watersheds that have similar fish 
faunas will have high similarity values.  

The similarity matrix is then used for the cluster analysis to create a \textbf{dendrogram}.
A dendrogram looks similar to a phylogenetic tree but a phylogenetic tree is
based on evolutionary relationships. The dendrogram is based only on similarity
of the fauna. Below is the dendrogram from the cluster analysis of Montana fishes.

%\begin{center}
	\includegraphics[width=\linewidth]{montana_cluster}
%\end{center}

Notice the branches are a series of hierarchical clusters. Intepret this dendrogram
by following each cluster of branches from the top down. The first division of the dendrogram 
identifies two groups that appear to be most similar:

\begin{multicols}{2}
\textbf{Left cluster}\\
Milk\\
Lower Yellowstone\\
Lower Missouri\\
Musselshell\\
Bighorn\\
Upper Yellowstone
\columnbreak

\textbf{Right cluster}\\
Kootenai\\
Clark Fork\\
Saskatachewan\\
Upper Missouri
\end{multicols}

\begin{enumerate}[resume,leftmargin=*]
\item \textbf{Does the dendrogram support your prediction?} Does the dendrogram appear
to support the prediction that the rivers west of the continental divide are very different
from rivers east of the divide?%\vspace{10\baselineskip}
\end{enumerate}

\newpage

You may have been surprised to see that the Kootenai and Clark Fork watersheds 
(west of the divide) appear to be similar to Saskatchewan and the Upper Missouri 
(east of the divide). More careful thought suggests that the left cluster are all rivers in
the prairie regions (flat lands) of Montana. The right cluster contains watersheds in
the mountainous region of Montana. Thus, the mountainous watersheds may actually
have a more similar fauna due to similar habitat, despite the presence of the continental
divide.

We'll use the second technique to explore these data another way, which may provide 
further insight into the results of our dendrogram.  You will use a technique called 
\textbf{non-metric multidimensional scaling} (\textsc{nmds}). The Montana fishes data set
is too small for a reliable \textsc{nmds} analysis so I performed a principal
coordinates analysis (\textsc{pco}). Both \textsc{nmds} 
and \textsc{pco} attempt to represent the similarity of the faunas in either a 2- or 3-
dimensional space (like a scatterplot) and present similar results, but \textsc{nmds} is generally the better approach.

Here is the result of the Montana \textsc{pco}. A 
color version is shown on screen. The colors correspond to the numbers next to each
watershed on the dendrogram above. Watersheds with the same number on the dendrogram
have the same color on the \textsc{pco} figure.

\begin{center}
	\includegraphics[width=1\textwidth]{montana_pco}
\end{center}

Compare this plot to the dendrogram shown above. Watersheds with similar fauna cluster
together in the figure. Notice that the Kootenai and Clark Fork (watersheds west of the 
continental divide) are clearly isolated from the rivers east of the divide. Furthermore, 
the Saskatchewan and Upper Missouri are almost as different from each other as they are 
from the other watersheds east of the divide. Although the Musselshell clustered with the 
Upper Yellowstone and Bighorn watersheds in the dendrogram, the \textsc{pco} results 
suggest it has a fauna more similar to the Milk, Lower Yellowstone, and Lower Missouri.

Why do these discrepancies exist between the cluster and \textsc{pco} analyses? The
differences show why you must inspect your data carefully. Montana has relatively few fish
species, so just 2--3 species shared between two watersheds will increase their apparent similarity, 
influencing the cluster results. The math that underlies \textsc{pco} and \textsc{nmds}
is much better at identifying and representing the differences. So, you will use both cluster and \textsc{nmds} analyses
today.

%\subsubsection*{Instructions}

You will be given the name of a data set. Next to your data set name in parentheses is a number.
The number will determine  the number of clusters to be identified for your cluster analysis and establish the number
of colors to be displayed for your \textsc{nmds} results.

\textbf{Read this section if you are using your own computer.} If not, skip to the next section. 
If you are using your own computer, you need to install a package called \textsc{vegan} (\textsc{veg}etation
\textsc{an}alysis) that is not part of the R default installation. Run this command.  Hopefully, you will not have any problems. Holler at me if you do.

\begin{verbatim}
source('http://mtaylor4.semo.edu/~goby/biogeo/install_vegan.r')
\end{verbatim}

\textbf{Begin here} by setting up the conditions for your analysis by 
entering the following command.  This will establish the default color 
set that will be used for your analyses, prompt you for the name of your data set,
and the cluster cutoff number (the one by your data set name). Enter the data set
name exactly as given to you. \emph{Do not enter the number or parentheses as part of the file name.} 
Do enter the number when prompted for the cut off number.

\begin{verbatim}
source('http://mtaylor4.semo.edu/~goby/biogeo/setup.r')
\end{verbatim}

Next, perform the cluster analysis to obtain your dendrogram. Instead of typing the entire command,
press your up cursor key, change ``setup.r'' to ``cluster.r'' and press Enter.

\begin{verbatim}
source('http://mtaylor4.semo.edu/~goby/biogeo/cluster.r')
\end{verbatim}

You have been given a map of the watersheds for your state.  Use that map, along with the large U.S. watershed map to 
get a sense of your results. %\emph{Here is where I ran out of time. Ask for help.} %

\subsubsection*{Thought questions to guide your interpretion}

What is the first split in your cluster? Do you find differences for rivers that run into the Atlantic or Gulf of Mexico? Rivers that flow north or east into the Tennessee River watershed or the Atlantic Ocean? Northern Missouri vs Southern Missouri?  Upper parts of rivers versus lower parts of rivers? Try to establish why particular watersheds clustered together and others did not.

Save a copy of the dendrogram so that you can compare it to the \textsc{nmds} results, which you will perform now.
Do that up cursor key thang again.

\begin{verbatim}
source('http://mtaylor4.semo.edu/~goby/biogeo/nmds.r')
\end{verbatim}

Interpret away. Write stuff below. I will be around to help. When you think you understand your results, get my attention. Upon approval, I will give you the next assignment.


\end{document}  