
%!TEX TS-program = lualatex
%!TEX encoding = UTF-8 Unicode

\documentclass[11pt]{article}
\usepackage{graphicx}
	\graphicspath{{/Users/goby/Pictures/teach/438/lab/}} % set of paths to search for images

\usepackage{geometry}
\geometry{letterpaper}                   
\geometry{bottom=1in}
%\geometry{landscape}                % Activate for for rotated page geometry
\usepackage[parfill]{parskip}    % Activate to begin paragraphs with an empty line rather than an indent
\usepackage{amssymb}
%\usepackage{mathtools}
%	\everymath{\displaystyle}

\usepackage{color}
%\pagenumbering{gobble}

\usepackage{fontspec}
\setmainfont[Ligatures={Common, TeX}, BoldFont={* Bold}, ItalicFont={* Italic}, BoldItalicFont={* Bold Italic}, Numbers={Proportional, OldStyle}]{Linux Libertine O}
\setsansfont[Scale=MatchLowercase,Ligatures=TeX]{Linux Biolinum O}
\setmonofont[Scale=MatchLowercase]{Linux Libertine Mono O}
%\setmonofont[Scale=MatchLowercase]{Inconsolatazi4}
\usepackage{microtype}

\usepackage{unicode-math}
\setmathfont[Scale=MatchLowercase]{Asana Math}
%\setmathfont{XITS Math}

% To define fonts for particular uses within a document. For example, 
% This sets the Libertine font to use tabular number format for tables.
%\newfontfamily{\tablenumbers}[Numbers={Monospaced}]{Linux Libertine O}
%\newfontfamily{\libertinedisplay}{Linux Libertine Display O}


\usepackage{booktabs}
\usepackage{longtable}
%\usepackage{tabularx}
%\usepackage{siunitx}
%\usepackage[justification=raggedright, singlelinecheck=off]{caption}
%\captionsetup{labelsep=period} % Removes colon following figure / table number.
%\captionsetup{tablewithin=none}  % Sequential numbering of tables and figures instead of
%\captionsetup{figurewithin=none} % resetting numbers within each chapter (Intro, M&M, etc.)
%\captionsetup[table]{skip=0pt}

\usepackage{array}
\newcolumntype{L}[1]{>{\raggedright\let\newline\\\arraybackslash\hspace{0pt}}p{#1}}
\newcolumntype{C}[1]{>{\centering\let\newline\\\arraybackslash\hspace{0pt}}p{#1}}
\newcolumntype{R}[1]{>{\raggedleft\let\newline\\\arraybackslash\hspace{0pt}}p{#1}}

\usepackage{enumitem}
\setlist{leftmargin=*}
\setlist[1]{labelindent=\parindent}
%\setlist[enumerate]{label=\textsc{\alph*}.}
%\usepackage{hyperref}

\usepackage[sc]{titlesec}

\newcommand{\coursename}{\textsc{bi} 438/638: Biogeography}

\usepackage{fancyhdr}
\fancyhf{}
\pagestyle{fancy}
%\lhead{}
%\chead{}
%\rhead{Name: \rule{5cm}{0.4pt}}
%\renewcommand{\headrulewidth}{0pt}
\setlength{\headheight}{14pt}
\fancyhead[R]{\footnotesize Rapoport's Rule \thepage}
\fancyhead[L]{\footnotesize \coursename}

\fancypagestyle{first_page}{%
	\fancyhf{}
	\fancyhead[L]{\coursename}
	\fancyhead[R]{Name: \enspace \rule{2.5in}{0.4pt}}
	\renewcommand{\headrulewidth}{0pt}
}

\newcommand{\bigSpace}{\vspace{5\baselineskip}}

\newlength{\myLength}
\setlength{\myLength}{\parindent}


\title{Rapoport's Rule}
\author{10 Points}
\date{}                                           % Activate to display a given date or no date

\begin{document}

\thispagestyle{first_page}

\subsection*{Range Size: Rapoport's Rule (10 points)}

\textbf{Save copies of your figures today. You will need them for the
writing assignment.}

Now that you understand Rapoport's Rule, you will explore whether the 
rules apply to primary freshwater fishes in the U.S., plus parts of 
northern Mexico and southern Canada. First, you will calculate the 
mean and maximum species richness per degree of latitude and plot that 
against latitude. Then, you will determine whether species range size 
increases with higher latitude.

The geographic area was divided into a grid of 1° latitude $\times$ 1°
longitude cells (see figure below). If a species was collected in one 
of the 1° cells, then presence was recorded with a 1. Absence was 
recorded with a 0. The presence/absence data for each cell was 
recorded for 529 species of North American freshwater fishes.

\begin{center}
	\includegraphics[width=\textwidth]{na_grid}
\end{center}

On one line, enter the following:

\begin{verbatim}
nagrid <- read.csv('http://mtaylor4.semo.edu/~goby/biogeo/nagrid.csv', 
  row.names=1)
\end{verbatim}

\texttt{head(nagrid)} \qquad \# View the first six rows of the data set.

The file is a grid of latitude (rows) by longitude (columns). Each
number greater than zero is the number of species, or \emph{species
richness}, that was collected from that 1° grid cell.

\subsection*{Species Richness}

To test Rapoport's rule for species richness, you will compare mean and
maximum species richness to degrees of latitude. You calculate the
mean and maximum values with the \texttt{apply()} function. The apply
function applies other functions to each row or column 
of your data set. Here, you will apply the \texttt{mean()} and
\texttt{max()} functions to each row of your data set.

\texttt{meansp \textless{}- apply(nagrid,1,mean)} \quad \# Mean 
richness for each degree of latitude (row).

\texttt{maxsp \textless{}- apply(nagrid,1,max)} \qquad \# Maximum 
richness for each degree of latitude (row).

Before you can plot the results, you need to create a latitude axis 
that runs from 24 to 49. In the past I have had you use the 
\texttt{seq()} function. Here is another way, as long as you want 
default increments of 1.

\texttt{lat \textless{}- 24:49}  \qquad \# lat \textless{}- 
seq(24,49,1) does the same thing.

\texttt{lat} \qquad \# view the results

Finally, create two graphs side by side.
As before, I will give you the basic framework for the \texttt{plot()} 
function, then let you \texttt{source} a file of commands to make it 
pretty.

\texttt{op \textless{}- par(mfrow=c(1,2))} 

\texttt{plot(lat\textasciitilde{}meansp)}

\texttt{plot(lat\textasciitilde{}maxsp)}

\texttt{par(op)}

With the plot window still open, bring the console to the front and 
enter:

\texttt{source('http://mtaylor4.semo.edu/\textasciitilde{}goby/biogeo/rapo\_pretty.r')}

Save your graph as a \textsc{png} or \textsc{pdf} file and upload to 
the Rapoport drop box. You \emph{must} do this.

\begin{enumerate}
	\item At what latitudes, approximately, are mean and maximum 
	species richness the highest? 

	\vspace{5\baselineskip}

	\item Does species richness follow Rapoport's Rule? Explain.

%	\vspace{7\baselineskip}
	
	\newpage

	\item Why do you think species richness is relatively low between
	24–28°N compared to between 30–38°N? A map should be on screen to 
	help you think about this.

	\vspace{7\baselineskip}

\end{enumerate}

\subsection*{Range Size}

You need a different data set to evaluate Rapoport's Rule for range 
size and latitude. Type the following command on one line.

\begin{verbatim}
fishArea <- read.csv('http://mtaylor4.semo.edu/~goby/biogeo/fish_area.csv',
  row.names=1)
\end{verbatim}

\texttt{fishArea} \qquad \# view the data file

Latitude (\texttt{lat}) is one column in the data set. Area
(\texttt{area}) is another column in the data set. Plot a quality
graph by typing this code on one line:

\begin{verbatim}
plot(fishArea$lat ~ fishArea$area, las = 1, xlab = 'Relative Area',
  ylab = 'Latitude (degrees N)',
  main='Relative area occupied by U.S. freshwater fishes')
\end{verbatim}

Save your graph as a \textsc{png} or \textsc{pdf} file on your flash 
drive or upload it to the Rapoport drop box for safekeeping. You are 
\emph{not required} to upload this file to the drop box.

\begin{enumerate}[resume]
	\item Based on this graph, do these data support Rapoport's rule 
	for range size and latitude? Explain.

	\vspace{8\baselineskip}

\end{enumerate}

Next, create a graph that displays the results in a different way, 
using what I call a “champaign graph.” This graph uses ``bubbles'' to 
plot the relative area occupied by each species, overlaid on a map of 
the U.S., complete with state boundaries and major rivers.  The center 
of each bubble is located at the mean latitude and longitude for each 
species.

To save typing, I have written all the code for you (you are welcome), 
which you can run at once by typing:

\begin{verbatim}
source('http://mtaylor4.semo.edu/~goby/biogeo/rapo_champaign.r')
\end{verbatim}

Zoom this graph to full size and save it as a \textsc{png} or 
\textsc{pdf} file on your flash drive or upload it to the Rapoport 
drop box for safekeeping. You are \emph{not required} to upload this 
file to the drop box. Then, answer the following questions. (Try 
to ignore the circles located in the oceans. The algorithm I wrote is 
rather poor but it serves our needs.)

\begin{enumerate}[resume]
	\item Based on this graph, combined with the previous graph, do
	U.S. freshwater fishes follow Rapoport's Rule for geographic range 
	size? Explain.

	\vspace{9\baselineskip}

	\item Considering the geographic ranges east of the Rocky
	Mountains, what region of the U.S. seems to have most of the
	smaller bubbles? Why do you think this is?

	\vspace{9\baselineskip}

	\item What geographic region west of the Rocky Mountains seems to
	have the fewest bubbles? (Or, most of the smallest bubbles?) Why 
	do you think this is? 
\end{enumerate}

\end{document}  