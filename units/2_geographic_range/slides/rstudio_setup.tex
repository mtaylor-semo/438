%!TEX TS-program = lualatex
%!TEX encoding = UTF-8 Unicode

\documentclass[t]{beamer}

% FONTS
\usepackage{fontspec}
\def\mainfont{Linux Biolinum O}
\setmainfont[Ligatures={Common,TeX}, Contextuals={NoAlternate}, BoldFont={* Bold}, ItalicFont={* Italic}, Numbers={Proportional}]{\mainfont}
%\setmonofont[Scale=MatchLowercase]{Inconsolata} 
\setsansfont[Scale=MatchLowercase]{Linux Biolinum O} 
\usepackage{microtype}

\usepackage{graphicx}
	\graphicspath{%
	{/Users/goby/Pictures/teach/438/lectures/}%
	{/Users/goby/Pictures/teach/common/}}%
%	{img/}} % set of paths to search for images

\usepackage{amsmath,amssymb}

%\usepackage{units}

\usepackage{booktabs}
\usepackage{multicol}
%	\setlength{\columnsep=1em}

\usepackage{textcomp}
\usepackage{setspace}
\usepackage{tikz}
	\tikzstyle{every picture}+=[remember picture,overlay]

\mode<presentation>
{
  \usetheme{Lecture}
  \setbeamercovered{invisible}
  \setbeamertemplate{items}[default]
}





\begin{document}

\begin{frame}[t]{Create a Project file in R Studio.}

\begin{enumerate}
\item Make a folder called “biogeo” in your Documents folder
\begin{itemize}
\item Mac: \textasciitilde/Documents/biogeo
\item PC: My PC/Documents

\end{itemize}

	\item In RStudio, choose “New Project…” from the File menu.
	
	\item Click on “Existing Directory,” and then click the “Browse” button. 
	
	\item Navigate to your new “biogeo” folder, select it, and then press the “Open” button.
	
	\item Press the “Create Project” button. This will create a file called “biogeo.Rproj” inside your “biogeo” folder.

\end{enumerate}
	
	\hangpara \textit{!All files for today and the future go into the “biogeo” folder!`}
	
	\hangpara \textit{!Always open this project file when starting RStudio in this class!`}
	
\end{frame}

%
%
\end{document}
