%!TEX TS-program = lualatex
%!TEX encoding = UTF-8 Unicode

\documentclass[t]{beamer}

%%%% HANDOUTS For online Uncomment the following four lines for handout
%\documentclass[t,handout]{beamer}  %Use this for handouts.
%\includeonlylecture{student}
%\usepackage{handoutWithNotes}
%\pgfpagesuselayout{3 on 1 with notes}[letterpaper,border shrink=5mm]


%% For students, use \lecture{student}{student}
%% For mine, use \lecture{instructor}{instructor}


%\usepackage{pgf,pgfpages}
%\pgfpagesuselayout{4 on 1}[letterpaper,border shrink=5mm]

% FONTS
\usepackage{fontspec}
\def\mainfont{Linux Biolinum O}
\setmainfont[Ligatures={Common,TeX}, Contextuals={NoAlternate}, BoldFont={* Bold}, ItalicFont={* Italic}, Numbers={Proportional}]{\mainfont}
%\setmonofont[Scale=MatchLowercase]{Inconsolata} 
\setsansfont[Scale=MatchLowercase]{Linux Biolinum O} 
\usepackage{microtype}

\usepackage{graphicx}
	\graphicspath{%
	{/Users/goby/Pictures/teach/438/lectures/}%
	{/Users/goby/Pictures/teach/common/}}%
%	{img/}} % set of paths to search for images

\usepackage{amsmath,amssymb}

%\usepackage{units}

\usepackage{booktabs}
\usepackage{multicol}
%	\setlength{\columnsep=1em}

\usepackage{textcomp}
\usepackage{setspace}
\usepackage{tikz}
	\tikzstyle{every picture}+=[remember picture,overlay]

\mode<presentation>
{
  \usetheme{Lecture}
  \setbeamercovered{invisible}
  \setbeamertemplate{items}[square]
}

\usepackage{calc}
\usepackage{hyperref}

\newcommand\HiddenWord[1]{%
	\alt<handout>{\rule{\widthof{#1}}{\fboxrule}}{#1}%
}



\begin{document}
%\lecture{instructor}{instructor}
\lecture{student}{student}

\begin{frame}[t]{Our goals for today are to }

%	\hangpara whether most organisms have small, medium, or large range sizes, and
	
	\hangpara compare geographic range size patterns of terrestrial organisms to those you found for aquatic organisms, and
	
	\hangpara explore the processes that regulate the geographic range.

\end{frame}

{
\usebackgroundtemplate{\includegraphics[width=\paperwidth]{range_range_size}}
\begin{frame}[t]{These figures show range size frequency. Interpret them.}

\end{frame}
}

%\begin{frame}[t]{In groups, discuss the following questions.}
%
%%	\hangpara Do most plants and animals have small, medium or large range sizes?
%
%	\hangpara What factors do you think determine range size?
%
%\end{frame}


\begin{frame}[t]{In groups, discuss the factors that you think constrain geographic range size.}

	\begin{center}
		\includegraphics[width=\textwidth]{range_pinus_catarrhines}
	\end{center}
	
	\hangpara Write 3–4 hypotheses to explain why most species have small range size. Be prepared to discuss them with the class.
	
\end{frame}

\lecture{instructor}{instructor}

\begin{frame}[t]{Factors that influence geographic range size include}

	\hangpara \alt<handout>{\rule{2in}{0.4pt}}{the ecological niche,} 
	
	\hangpara \alt<handout>{\rule{2in}{0.4pt}}{disturbance,}
	
	\hangpara \alt<handout>{\rule{2in}{0.4pt}}{species interactions, and}

	\hangpara \alt<handout>{\rule{2in}{0.4pt}}{historical processes.}
	
		
\end{frame}

\lecture{student}{student}

\begin{frame}[t]{The geographic range reflects the species’ niche.}

	\hangpara What is the \highlight{fundamental niche} vs. the \highlight{realized niche?}
	\pause
	
	\begin{center}
		\includegraphics[width=\textwidth]{range_niche}
	\end{center}
	\pause
	
	\hangpara What is the \highlight{fundamental geographic range} vs the \highlight{realized geographic range?}

\end{frame}

{
\usebackgroundtemplate{\includegraphics[width=\paperwidth]{range_groundhog}}
\begin{frame}[b]
\tiny Wikimedia Commons.
\end{frame}
}

\begin{frame}[t]{Abundance and density is highest near the center of the range. Why?}
	\begin{center}
		\includegraphics[width=\textwidth]{range_scissortail}
	\end{center}
\end{frame}


{
\usebackgroundtemplate{\includegraphics[width=\paperwidth]{range_juniper}}
\begin{frame}[b]{\highlight{Habitat quality} influences abundance and density.}

\end{frame}
}
%
{
\usebackgroundtemplate{\includegraphics[width=\paperwidth]{range_size_point_conception}}
\begin{frame}[b]
\end{frame}
}
%
\begin{frame}[t]{\highlight{Metapopulations} have cycles of extinction and colonization.}
	\begin{center}
		\includegraphics[width=\textwidth]{range_metapopulation}	
	\end{center}
\end{frame}

{
\usebackgroundtemplate{\includegraphics[width=\paperwidth]{range_disturbance_flood}}
\begin{frame}[b]{Floods restore sediment and nutrients.}
\tiny\hfill\textcolor{white}{2008 St. Louis area flood, Jocelyn Augustino, Wikimedia Commons.}
\end{frame}
}

\lecture{instructor}{instructor}

{
\usebackgroundtemplate{\includegraphics[width=\paperwidth]{range_flood_birdspoint}}
\begin{frame}[b]{}

\end{frame}
}

{
\usebackgroundtemplate{\includegraphics[width=\paperwidth]{range_louisiana_land_loss}}
\begin{frame}[b]{}

\end{frame}
}

\lecture{student}{student}
%
{
\usebackgroundtemplate{\includegraphics[width=\paperwidth]{range_disturbance_fire}}
\begin{frame}[b]{\textcolor{white}{Fires maintain forests and grasslands.}}

\end{frame}
}
%
{
\usebackgroundtemplate{\includegraphics[width=\paperwidth]{range_disturbance_fire_regime}}
\begin{frame}[b]
	\tiny\hfill USDA, Wikimedia Commons.
\end{frame}
}
%
{
\usebackgroundtemplate{\includegraphics[width=\paperwidth]{range_hurricane}}
\begin{frame}[b]{Hurricanes can alter reef ecosystems.}
	\tiny\hfill Hurricane Sandy, NOAA, Wikimedia Commons.
\end{frame}
}
%
{
\usebackgroundtemplate{\includegraphics[width=\paperwidth]{range_urchins}}
\begin{frame}[b]
	\tiny\textcolor{white}{NOAA, Flickr Creative Commons.}
\end{frame}
}
%
{
\usebackgroundtemplate{\includegraphics[width=\paperwidth]{range_ginsburgellus}}
\begin{frame}[b]{\textcolor{white}{The range of some species depends on the range of other species.}}
	\tiny\textcolor{white}{Nineline Goby, \copyright André de Molenaar.}
\end{frame}
}
%
\begin{frame}[t]{Whitebark Pine uses Clark’s Nutcracker for seed dispersal.}
	\begin{center}
		\includegraphics[width=\textwidth]{range_nutcracker}
	\end{center}
\end{frame}
%
{
\usebackgroundtemplate{\includegraphics[width=\paperwidth]{range_squirrels}}
\begin{frame}[b]{Native Red Squirrels have lost range to invasive Grey Squirrels.}	
	\tiny\hfill Red Squirrel Survival Trust
\end{frame}
}
%
{
\usebackgroundtemplate{\includegraphics[width=\paperwidth]{range_brown_tree_snake}}
\begin{frame}[b]{Native bats and birds have lost range to invasive snake.}
\end{frame}
}
%
{
\usebackgroundtemplate{\includegraphics[width=\paperwidth]{range_historical}}
\begin{frame}[b]{Historical factors influence range size.}
\end{frame}
}
%
\lecture{instructor}{instructor}
{
\usebackgroundtemplate{\includegraphics[width=\paperwidth]{range_groundhog}}
\begin{frame}[b]
\tiny Wikimedia Commons.
\end{frame}
}
%
\lecture{student}{student}

{
\usebackgroundtemplate{\includegraphics[width=\paperwidth]{range_marmota}}
\begin{frame}[b]
	\tiny Modified from Wikimedia Commons.
\end{frame}
}
%
{
\usebackgroundtemplate{\includegraphics[width=\paperwidth]{range_allopatric}}
\begin{frame}[t]{Closely related species have \highlight{allopatric} distributions.}
\end{frame}
}
%
\begin{frame}[t]{Closely related species have fewer overlapping ranges than expected.}

	\begin{center}
		\hangpara Observed vs. expected number of \\\hspace{0.5em} overlapping ranges for desert rodents.

		\begin{tabular}{@{}lrr@{}}
		\toprule
		Taxon & Observed & Expected \tabularnewline
		\midrule
		Same genus & 111 & 146.3 \tabularnewline
		Diff. genus & 785 & 749.7 \tabularnewline
		Similar size & 217 & 242.2 \tabularnewline
		Diff. size & 697 & 653.8 \tabularnewline
		\bottomrule
		\end{tabular}
	\end{center}
\end{frame}
%
{
\usebackgroundtemplate{\includegraphics[width=\paperwidth]{range_goby_phylo}}
\begin{frame}[b]
\end{frame}
}
%
%
\end{document}
