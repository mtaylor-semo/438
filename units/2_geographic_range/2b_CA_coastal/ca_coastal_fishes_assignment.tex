%!TEX TS-program = lualatex
%!TEX encoding = UTF-8 Unicode

\documentclass[11pt]{article}
\usepackage{graphicx}
\graphicspath{{/Users/mtaylor/Pictures/teach/438/lab/}} % set of paths to search for images

\usepackage{geometry}
\geometry{letterpaper}                   
\geometry{bottom=1in}
%\geometry{landscape}                % Activate for for rotated page geometry
\usepackage[parfill]{parskip}    % Activate to begin paragraphs with an empty line rather than an indent
\usepackage{amssymb}
%\usepackage{mathtools}
%	\everymath{\displaystyle}

\usepackage{color}
%\pagenumbering{gobble}

\usepackage{fontspec}
\setmainfont[Ligatures={Common, TeX}, BoldFont={* Bold}, ItalicFont={* Italic}, Numbers={Proportional, OldStyle}]{Linux Libertine O}
\setsansfont[Scale=MatchLowercase,Ligatures=TeX]{Linux Biolinum O}
\setmonofont[Scale=MatchLowercase]{Inconsolatazi4}
\usepackage{microtype}

\usepackage{unicode-math}
\setmathfont[Scale=MatchLowercase]{Asana-Math.otf}
%\setmathfont{XITS Math}

% To define fonts for particular uses within a document. For example, 
% This sets the Libertine font to use tabular number format for tables.
%\newfontfamily{\tablenumbers}[Numbers={Monospaced}]{Linux Libertine O}
%\newfontfamily{\libertinedisplay}{Linux Libertine Display O}


\usepackage{booktabs}
\usepackage{longtable}
%\usepackage[justification=raggedright, singlelinecheck=off]{caption}
%\captionsetup{labelsep=period} % Removes colon following figure / table number.
%\captionsetup{tablewithin=none}  % Sequential numbering of tables and figures instead of
%\captionsetup{figurewithin=none} % resetting numbers within each chapter (Intro, M&M, etc.)
%\captionsetup[table]{skip=0pt}

\usepackage{array}
\newcolumntype{L}[1]{>{\raggedright\let\newline\\\arraybackslash\hspace{0pt}}p{#1}}
\newcolumntype{C}[1]{>{\centering\let\newline\\\arraybackslash\hspace{0pt}}p{#1}}
\newcolumntype{R}[1]{>{\raggedleft\let\newline\\\arraybackslash\hspace{0pt}}p{#1}}

%\usepackage{enumitem}
%\usepackage{hyperref}

%\usepackage{titling}
%\setlength{\droptitle}{-50pt}
%\posttitle{\par\end{center}}
%\predate{}\postdate{}

\usepackage{hanging}
\usepackage{wrapfig}

\usepackage[sc]{titlesec}

\newcommand{\coursename}{\textsc{bi} 438/638: Biogeography}

\usepackage{fancyhdr}
\fancyhf{}
\pagestyle{fancy}
%\lhead{}
%\chead{}
%\rhead{Name: \rule{5cm}{0.4pt}}
%\renewcommand{\headrulewidth}{0pt}
\setlength{\headheight}{14pt}
\fancyhead[R]{\footnotesize Coastal Californa Range Size \thepage}
\fancyhead[L]{\footnotesize \coursename}

\fancypagestyle{first_page}{%
	\fancyhf{}
	\fancyhead[L]{\coursename}
	\fancyhead[R]{Name: \enspace \rule{2.5in}{0.4pt}}
	\renewcommand{\headrulewidth}{0pt}
}

\newcommand{\bigSpace}{\vspace{5\baselineskip}}

\newlength{\myLength}
\setlength{\myLength}{\parindent}

\title{Geographic Range Size}
\author{10 Points}
\date{}                                           % Activate to display a given date or no date

\begin{document}
	\thispagestyle{first_page}

\subsection*{Range Size: California Coastal Marine Fishes (10 points)}

In a previous exercise, you determined the range size for freshwater
organisms. Fishes, mussels, and some crayfishes can't cross land so it seems
reasonable that they may be restricted to just a few watersheds.
However, coastal marine fishes do not have obvious limitations to their
distribution. Adults of many coastal marine fishes are able to swim very
long distances. Thus, at least potentially, more marine fishes may have
relatively large range sizes compared to those with small ranges. Your
goal for this exercise is to determine whether coastal marine fishes
have large range sizes.

For this exercise, all students will use the same data set, which comes
from Horn et al. (2006, in \emph{The Ecology of Marine Fishes:
California and Adjacent Waters}). The data set includes distribution
information for 516 species of coastal marine fishes that occur in
California. The authors determined the range size of each species using
the number of degrees of latitude (°S to N) that were occupied. The
range could (and usually does) extend outside of California. The only
requirement was that some part of the species' range had to occur within
the coastal waters of California. For each degree of latitude, a species
was assigned 1 if present, and 0 if absent. The full latitude range
extends from 30°S (central Chile in South America) to 68°N (north of the
Arctic Circle in Alaska).

You will calculate the number of degrees latitude occupied by each
species. You will also calculate the number of species present in each
degree of latitude. In this data set, species are rows and degrees of
latitude are columns. You will plot histograms of the results from these
two calculations. Finally, you will learn a new trick with the
\texttt{plot()} function to identify the region with the highest
diversity of coastal marine fishes.

\subsubsection*{Obtain the Data Set}

Double-click your project file to launch R~Studio. Alternatively, launch R~Studio, then open the project file. \textbf{Do this before any step below.}

Download "california\_marine\_fishes.csv" from the course Canvas page. \textit{Place the file in the same folder as your project file.}

In the R~Studio console, enter the following on one line:

\texttt{cafish \textless{}-
read.csv("california\_marine\_fishes.csv", row.names = 1)}

Calculate how many degrees of latitude are covered by each
species to determine range size. Repeat for the number of species per
degree of latitude.

\texttt{rangeSize \textless{}- rowSums(cafish)}

\texttt{numSpecies \textless{}- colSums(cafish)}

Calculate some basic statistics to fill in the table below. I've shown
you what to do for range size. Repeat for the number of species present
in each degree of latitude. Fill in the table on the next page.

\texttt{max(rangeSize)} \qquad \# maximum number of degrees latitude occupied

\texttt{min(rangeSize)} \qquad \# minimum number of degrees latitude occupied

\texttt{mean(rangeSize)} \qquad \# mean number of degrees latitude occupied

\begin{tabular}[l]{@{}lL{1.6in}L{1.6in}@{}}
\toprule
& Degrees of Latitude Occupied & Number of Species per Degree\tabularnewline
\midrule
 & & \\[1ex]
Maximum & \rule{1.5in}{0.4pt} & \rule{1.5in}{0.4pt} \tabularnewline[2ex]
Minimum & \rule{1.5in}{0.4pt} & \rule{1.5in}{0.4pt} \tabularnewline[2ex]
Mean & \rule{1.5in}{0.4pt} & \rule{1.5in}{0.4pt} \tabularnewline
\bottomrule
\end{tabular}

\bigskip

\textbf{1. What is your hypothesis?} Given that the total latitudinal range in
the data set covers 99° of latitude, from 30°S to 68°N, does the mean
number of degrees latitude occupied suggest that most coastal marine
fishes likely have large or small range size? Explain.

\vspace{9\baselineskip}

Create a histogram of \texttt{rangeSize} to show the distribution of range sizes for California coastal marine fishes. Type the following code on one line.\footnote{Or, you can enter \texttt{source("makeprettyCA.r")} into the console.}

\texttt{hist(rangeSize, breaks = 20, xlim = c(0, 100), las = 1, xlab = "Degrees of Latitude\newline Occupied", ylab = "Number of Species")}

%What do the results suggest about range size?

\bigskip\bigskip

\textbf{2. Was your hypothesis supported for range size?} Did most coastal marine
fishes have relatively small or relatively large range sizes, especially
in relation to the actual distance covered by the data set? Explain.

%\vspace{6\baselineskip}

%\textbf{3. Describe the pattern for number of species per degree latitude.} Study the histogram
%for you made for \texttt{numSpecies}. A few
%degrees of latitude seem to have very few species. Can you think of
%reasons why? \textsc{Hint 1}: What was the minimum number of species per
%degree latitude in the table on the first page. Do you really think
%there is a degree of latitude along the west coast of North America that has only one
%species of fish? \textsc{Hint 2}: The data set encompasses only those species
%that occur in California. Use this information to try to explain the
%very high peak with very low numbers of species and the secondary peak
%with more reasonable numbers of species.

\newpage

%When you have finished, save the file as a PDF and upload to the Unit
%1b: California course drop box before you leave today.

\subsubsection*{Where is the richness of coastal marine fishes the greatest?}

Species richness is the number of species present at a given location.
%The second histogram showed that most degrees of latitude had fewer than
%half of the 519 species, and some degrees of latitude had far less than
%one-fifth of all species in the data set. Thus, most degrees of latitude
%had relatively low species richness (relative to the total data set).
%However, a few degrees of latitude had nearly or more than 400 species,
%which indicates very high species richness. Those were the degrees of
%latitude at the very right side of your histogram. Are those few degrees
%of latitude in the same area or are they randomly scattered? If the
%degrees of latitude are in the same area, they may suggest some type of
%biogeographic hotspot with high overall species diversity. 
Latitutdes with high species richness might indicate a \textbf{biogeographic hotspot.} Biogeographic
hotspots are often targets for conservation management to ensure
maintenance of diversity. The area of high richness may also represent a
boundary between two biogeographically distinct regions. Regions that
are biogeographically distinct tend to have distinct groups of taxa,
with relatively few organisms shared between regions. You will learn
more about biogeographic regions in the next unit. For now, you will
explore which degrees of latitude have the highest overall species
richness.

To do so, use the \texttt{plot()} function to plot the
\texttt{numSpecies} results.

\texttt{plot(numSpecies)}

Clearly, there is a peak in species richness, but at what latitude?
The Y-axis is the richness per degree latitude. The X-axis, labeled
Index, is simply the order of the results in the \texttt{numSpecies}
data set, so you can't easily tell which latitudes have the highest
species richness. However, the order of records corresponds to the order
of latitude from 30°S to 68°N.

To get degrees of latitude on the X-axis, you need to generate a
sequence of numbers that correspond to the degrees of latitude. Use the \texttt{seq()} function to generate a
sequence of numbers from $\minus$30 to 68, incrementing by 1. Assign the
results to a variable called \texttt{lat} for \emph{lat}itude.

\texttt{lat <- seq(-30, 68, 1)}

Now you can plot species richness against latitude. \texttt{xlim{}} limits create a balanced distribution of tick marks along the X-axis. That is a tilde after \texttt{numSpecies.}

\texttt{plot(numSpecies \textasciitilde{} lat, xlim = c(-40, 80))}

If you compare the points that represent the greatest richness, you
get a sense that the highest diversity is somewhere between 25°\textsc{n} and
40°\textsc{n}. %This is useful but it would be nice to know precisely which
%degrees of latitude have the highest species richness.

\textbf{3. Estimate the approximate latitude of greatest species richness from your graph. Write the number below. Compare your estimate to the map of California on the next page. Look at your notes and the “range size” slides from last lecture. What is the name of this location in California?}

\vspace{3\baselineskip}


%R provides a way to interact with your plots, using the
%\texttt{identify()} function. The \texttt{identify()} function lets you
%click on points in your plot. It will identify each point you click, up
%to a specifiable limit. You have to use the \texttt{identify()} function
%immediately after you use the \texttt{plot()} function because
%\texttt{identify()} works only in the currently active plot window.
%
%After you type or recall the plot command below, move the plot window to the back and type the \texttt{identify{}} command, then bring the plot window back to the front.
%
%\texttt{plot(numSpecies\textasciitilde{}lat, xlim=c(-40,80))}\\
%\texttt{identify(numSpecies\textasciitilde{}lat, labels=lat, cex=0.8)} \qquad \# cex reduces the character size.
%
%Now, click once on the highest point. The number that should appear is the latitude of that point. Remember that a negative number is °S (south of the equator). Click 3–4 more of the highest points. Also click the lowest leftmost point and the
%lowest rightmost point. Press the \texttt{Esc} key when you are finished
%to stop the interaction. \emph{RStudio users may have to click the points, then press \texttt{Esc} to see
%the numbers.} If numbers do not appear at all, try the commands again. If you are still not
%successful, ask for help.
%
%The lowest leftmost point should be $\minus$30 and the lowest rightmost point
%should be 68. The highest points are in the low- to mid-30s. Notice also that
%the numbers that appeared on your graph also appear in the R command
%window after you pressed the \texttt{Esc} key. You could assign the
%results to a variable for later use but it is not necessary to do so
%here.
%
%You can now easily tell that the region with the highest species
%richness occurs in the same general area in the northern hemisphere. The
%question is, where is this?

\textbf{4. Discuss with others at your table} some potential reasons why this area might have such high
species richness compared to surrounding higher and lower latitudes.
List some of your reasons and be prepared to share them with the class.

\newpage

\begin{center}
	\includegraphics[width=\textwidth]{california_with_latitude}
\end{center}


\newpage


\end{document}  