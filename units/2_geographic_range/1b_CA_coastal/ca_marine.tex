In the last exercise, you determined the range size for freshwater
organisms. Fishes, crayfishes and mussels can't cross land so it seems
reasonable that they may be restricted to just a few watersheds.
However, coastal marine fishes do not have obvious limitations to their
distribution. Adults of many coastal marine fishes are able to swim very
long distances. Thus, at least potentially, more marine fishes may have
relatively large range sizes compared to those with small ranges. Your
goal for this exercise is to determine whether coastal marine fishes
have large range sizes.

For this exercise, all students will use the same data set, which comes
from Horn et al. (2006, in \emph{The Ecology of Marine Fishes:
California and Adjacent Waters}). The data set includes distribution
information for 516 species of coastal marine fishes that occur in
California. The authors determined the range size of each species using
the number of degrees of latitude (°S to °N) that were occupied. The
range could (and usually does) extend outside of California. The only
requirement was that some part of the species' range had to occur within
the coastal waters of California. For each degree of latitude, a species
was assigned 1 if present, and 0 if absent. The full latitude range
extends from 30°S (central Chile in South America) to 68°N (north of the
Arctic Circle in Alaska).

You will calculate the number of degrees latitude occupied by each
species. You will also calculate the number of species present in each
degree of latitude. In this data set, species are rows and degrees of
latitude are columns. You will plot histograms of the results from these
two calculations. Finally, you will learn a new trick with the
\textbf{plot()} function to identify the region with the highest
diversity of coastal marine fishes.

\textbf{Obtain the Data Set}. Type the entire command on one line.

cafish \textless{}-
read.csv('http://mtaylor4.semo.edu/\textasciitilde{}goby/biogeo/california\_marine\_fishes.csv',
header=TRUE, row.names=1)

First, calculate how many degrees of latitude are covered by each
species to determine range size. Repeat for the number of species per
degree of latitude.

rangeSize \textless{}- rowSums(cafish)

numSpecies \textless{}- colSums(cafish)

Calculate some basic statistics to fill in the table below. I've shown
you what to do for range size. Repeat for the number of species present
in each degree of latitude.

max(rangeSize) \# \emph{max}imum number of degrees latitude occupied

min(rangeSize) \# \emph{min}imum number of degrees latitude occupied

mean(rangeSize) \# mean number of degrees latitude occupied

\begin{longtable}[c]{@{}lll@{}}
\toprule
& \textbf{Degrees of Latitude Occupied} & \textbf{Number of Species per
Degree}\tabularnewline
\midrule
\endhead
\textbf{Maximum} & &\tabularnewline
\textbf{Minimum} & &\tabularnewline
\textbf{Mean} & &\tabularnewline
\bottomrule
\end{longtable}

1) What is your hypothesis? Given that the total latitudinal range in
the data set covers 99° of latitude, from 30°S to 68°N, does the mean
number of degrees latitude occupied suggest that most coastal marine
fishes likely have large or small range size? Explain.

Create a histogram of \textbf{rangeSize} and \textbf{numSpecies}. What
do the result suggest about range size?

2) Was your hypothesis supported for range size? Did most coastal marine
fishes have relatively small or relatively large range sizes, especially
in relation to the actual distance covered by the data set? Explain.

3) Describe the pattern for number of species per degree latitude. A few
degrees of latitude seem to have very few species. Can you think of
reasons why? Hint 1: Look back at the minimum number of species per
degree latitude in the table on the first page. Do you really think
there is a degree of latitude anywhere along the coast that has only one
species of fish? Hint 2: The data set encompasses only those species
that occur in California. Use this information to try to explain the
very high peak with very low numbers of species and the secondary peak
with more reasonably numbers of species.

\textbf{Presentation Quality Figures}. Each student should write the
code for one of the histograms.

Plot the two histograms side by side. I've given you the basic code,
including the breaks (bin size) and limits of the X-axis. See the
information below the code to use to make presentation quality
histograms.

op \textless{}- par(mfrow=c(1,2))

hist(rangeSize, breaks=25, xlim=c(0,100))

hist(numSpecies, breaks=50, xlim=c(0,500))

par(op)

To increase the presentation quality of your histograms, modify the
\textbf{hist()} commands above to include the following details:

\textbf{Main titles. Apply each title to the appropriate histogram.*}

Frequency Distribution of Range Size for California Coastal Marine
Fishes

Frequency Distribution of Number of Species per Degree Latitude

* These titles are very long so replace one of the spaces between words
in the title with ``\textbackslash{}n'' (no quotes) to split the title
across two lines. You decide where but aim for balance between the two
lines. Include each your initials in the histogram that you create. You
used this same technique in the first tutorial to add your name to one
figure.

\textbf{X-Axis Label. Apply each label to the appropriate histogram.}

Degrees of Latitude Occupied

Species Per Degree Latitude

You may play adjust \textbf{col} and \textbf{border} if you wish.

When you have finished, save the file as a PDF and upload to the Unit
1b: California course drop box before you leave today.

\textbf{Where is the richness of coastal marine fishes the highest?}

Species richness is the number of species present at a given location.
The second histogram showed that most degrees of latitude had fewer than
half of the 519 species, and some degrees of latitude had far less than
one-fifth of all species in the data set. Thus, most degrees of latitude
had relatively low species richness (relative to the total data set).
However, a few degrees of latitude had nearly or more than 400 species,
which indicates very high species richness. Those were the degrees of
latitude at the very right side of your histogram. Are those few degrees
of latitude in the same area or are they randomly scattered? If the
degrees of latitude are in the same area, they may suggest some type of
biogeographic hotspot with high overall species diversity. Biogeographic
hotspots are often targets for conservation management to ensure
maintenance of diversity. The area of high richness may also represent a
boundary between two biogeographically distinct regions. Regions that
are biogeographically distinct tend to have distinct groups of taxa,
with relatively few organisms shared between regions. You will learn
more about biogeographic regions in the next unit. For now, you will
explore which degrees of latitude have the highest overall species
richness.

To do so, use the \textbf{plot()} function to plot the
\textbf{numSpecies} results.

plot(numSpecies)

Clearly, there is a peak in species richness, but at which latitudes?
The Y-axis is the richness per degree latitude. The X-axis, labeled
Index, is simply the order of the results in the \textbf{numSpecies}
data set, so you can't easily tell which latitudes have the highest
species richness. However, the order of records corresponds to the order
of latitude from 30°S to 68°N.

To get degrees of latitude on the X-axis, you need to generate a
sequence of numbers that correspond to the degrees of latitude. As you
did in the last exercise, use the \textbf{seq()} function to generate a
sequence of numbers from -30 to 68, incrementing by 1. Type
\textbf{?seq} if you don't remember how to use this function. Assign the
results to a variable called \textbf{lat} for \emph{lat}itude.

Now you can plot species richness against latitude.

plot(numSpecies\textasciitilde{}lat, xlim=c(-40,80))

\# These limits create a balanced distribution of tick marks

Now, if you compare the points that represent the greatest richness, you
get a sense that the highest diversity is somewhere between 25°N and
40°N. This is useful but it would be nice to know precisely which
degrees of latitude have the highest species richness.

R provides a way to interact with your plots, using the
\textbf{identify()} function. The \textbf{identify()} function lets you
click on points in your plot. It will identify each point you click, up
to a specifiable limit. You have to use the \textbf{identify()} function
immediately after you use the \textbf{plot()} function because
\textbf{identify()} works only in the currently active plot window.

After you type the plot command below, move the window to the back and
type the identify command, then bring the plot window back to the front.

plot(numSpecies\textasciitilde{}lat) \# Skip xlim for now.

identify(numSpecies\textasciitilde{}lat)

Now, click once on the highest point. A number should appear. Click 3-4
more of the highest points. Also click the lowest leftmost point and the
lowest rightmost point. Press the \textbf{Esc} key when you are finished
to stop the interaction.

(If numbers do not appear, try the commands again. If you are still not
successful, ask for help.)

The lowest leftmost point should be 0 and the lowest rightmost point
should be 99. The highest points are in the mid-60s. Notice also that
the numbers that appeared on your graph also appear in the R command
window after you pressed the \textbf{Esc} key. You could assign the
results to a variable for later use but it is not necessary to do so
here.

The numbers that appeared on your graph are the record numbers
indicating the order in the data set, hardly useful for your needs.
Helpfully, the \textbf{identify()} function lets you specify a
vector\footnote{A vector is one line of numeric \emph{or} text data (not
  both). Think of it as one row of information from a data set. You
  created a vector when you used the \textbf{seq()} function.} of
information that can be used as labels. A suitable label would be the
actual degree of latitude. You can extract the labels from the header
row of the data set, which contains the degrees of latitude as column
names. Type \textbf{head(cafish)} to see. S in the column header stands
for latitudes \emph{s}outh of the equator and N stands for latitudes
\emph{n}orth of the equator. The \textbf{colnames()} function extracts
the column names for you.

latlabels \textless{}- colnames(cafish)

latlabels \# View the result

Now, plot the species richness and identify the points. The numbers that
appeared on the previous plot were somewhat large so use the cex
argument in \textbf{identify()} to reduce the character size.
(\textbf{cex} is a generic character expansion argument that allows you
to reduce (values less than 1) or increase (values greater than 1) the
size of symbols and text.

plot(numSpecies\textasciitilde{}lat, xlim=c(-40,80))

identify(numSpecies\textasciitilde{}lat, labels = latlabels, cex=0.8)

As before, click several of the highest points, as well as the lowest
leftmost and rightmost points. Press the \textbf{Esc} key when finished.

You can now easily tell that the region with the highest species
richness occurs in the same general area in the northern hemisphere. The
question is, where is this?

4) Identify the area with the map on the next page. Discuss with your
partner some potential reasons why this area might have such high
species richness compared to surrounding higher and lower latitudes.
List some of your reasons and be prepared to share them with the class.

\includegraphics{media/image1.jpeg}
