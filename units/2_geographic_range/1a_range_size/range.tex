The goal of this exercise is for you to discover whether most species of
freshwater fishes, mussels and (in the second section) crayfishes have
large, medium or small range sizes, or if range size seems to be a
random mix among species.

Here, you will determine the range size of these species groups at
different geographic scales. You will first determine range size at the
scale of North America (Nearctic) for either fishes or mussels. You will
then determine range size at the state scale (e.g., Missouri, Georgia)
for fishes, mussels or crayfish. The data sets you use will be assigned
to you randomly.

Before you proceed, what is your hypothesis? Do you think most species
have small, medium or large range sizes, or will range sizes be a mix of
sizes at the scale of North America? (Save your ideas about the smaller
state scale for later.) Discuss your ideas with your partner but your
hypothesis does not have to agree with your partner's hypothesis. If
your hypotheses don't agree, glare at your partner with severe
disapproval for not recognizing your genius!

1) Write your hypothesis here. Considering all of North America, tell
whether you think most species in a closely related group of organisms
have large, medium or small range size, or some combination. Be prepared
to discuss your hypothesis and \emph{your justification for your
prediction} with the rest of the class.

For the North American scale, each student pair will be assigned a data
set using either fishes or mussels. For the state scale, each student
pair will get a different state and taxon combination. (e.g., Alabama
fishes, Alabama mussels, Missouri crayfishes, etc.). To determine range
size, you will use R to calculate the number of watersheds (rivers)
occupied by each species, and then plot a \textbf{histogram} to evaluate
the range size. A species that occupies only a few watersheds will be
considered to have a small range size, while a species that occupies
most or all watersheds in the state will be considered to have a large
range size.

\textbf{Name of your North American data set
\_\_\_\_\_\_\_\_\_\_\_\_\_\_\_\_\_\_\_\_\_\_\_\_\_\_\_\_\_\_\_\_\_\_\_\_\_\_\_\_\_.}

\textbf{Name of your State/Taxon data set
\_\_\_\_\_\_\_\_\_\_\_\_\_\_\_\_\_\_\_\_\_\_\_\_\_\_\_\_\_\_\_\_\_\_\_\_\_\_\_\_\_.}

A \textbf{histogram} is a type of graph that shows the number of
occurrences within categories. For example, a graph that shows the
number of As, Bs, Cs, Ds and Fs for an exam is a type of histogram. To
be certain you know how to interpret a histogram, we'll build one using
the height of the students in class today. \textbf{Calculate your height
in inches, using the following information}:

If you are 5 ft. 6 in., then add the number of inches of your height to
60 (5 ft. = 60 in.). Your height would be 66 inches. If you are 6 ft. 1
in., add the number of inches of your height to 72 (6 ft. = 72 in). Use
48 inches if you are 4 ft.+ and 84 inches if you are 7 ft.+.

\textbf{What is your height in inches? \_\_\_\_\_\_\_\_\_\_\_\_\_\_\_.}
Tell me this number when asked.

One histogram that you create below will show the number of species that
occupy one watershed, two watersheds, and so on, up to the total number
of watersheds in the data set.

\textbf{Get the North American data set}

In the command below, substitute the file name for the North American
data set you were given for the \textbf{filename} part of the command.
For this data set, the file name will be \textbf{NAfishes.csv} or
\textbf{NAmussels.csv}.

spp \textless{}-
read.csv('http://mtaylor4.semo.edu/\textasciitilde{}goby/biogeo/\emph{filename}.csv',
header=TRUE, row.names=1)

Watersheds are rows and species are columns, with presence or absence of
species indicated by 1s and 0s. 1 indicates a species is present in a
watershed and 0 indicates a species is absent from a watershed.

Now, determine how many watersheds are in the data set, and how many
species of your taxon are in the data set. This is easy to do with the
\textbf{dim()} function. (If you don't remember how to use it, refer to
the R Tutorial exercise you did.)

\textbf{dim()} returns the number of rows and columns, in that order so
the first number is the number of watersheds and the second number is
the number of species in your taxon. Remember the \textbf{dim()}
function as you'll use it again for the second data set.

Your primary goal is to calculate the range size for the species in your
data set, represented by the number of watersheds occupied by each
species. You will also calculate the number of species per watershed. R
provides two functions, \textbf{rowSums()} and \textbf{colSums()}, that
sum across a row or down a column, for all rows or columns in your data
set. The table below shows you results from the two functions for a
small data set.

\begin{longtable}[c]{@{}lllll@{}}
\toprule
& Species 1 & Species 2 & Species 3 & \textbf{rowSums()}

Species per watershed\tabularnewline
\midrule
\endhead
Watershed 1 & 0 & 1 & 1 & \textbf{2}\tabularnewline
Watershed 2 & 1 & 0 & 0 & \textbf{1}\tabularnewline
Watershed 3 & 1 & 1 & 1 & \textbf{3}\tabularnewline
\textbf{colSums()}

Watersheds per species & \textbf{2} & \textbf{2} & \textbf{2}
&\tabularnewline
\bottomrule
\end{longtable}

First, calculate the number of watersheds per species.

numWatersheds \textless{}- colSums(spp) \# Remember that R is case
sensitive.

numWatersheds \# view the results

Next, calculate the number of species that occur in each watershed.

numSpecies \textless{}- rowSums(spp)

numSpecies \# view the results

To create a histogram, use the \textbf{hist()} function. The
\textbf{hist()} function, like \textbf{plot()}, has many arguments to
customize your figure, but begin with a simple histogram.

hist(numWatersheds)

This histogram provides enough detail for you to determine whether most
species occur in many watersheds (large range size) or in just a few
watersheds (small range size).

2) Interpret the pattern. Describe the pattern that you see. Do most
species occur in most watersheds (large range size) or do most species
occur in just a few watersheds (small range size)? Or, is it some
combination of the two? Explain how the observed result supports or
falsifies your hypothesis in question 1. Be prepared to discuss.

Now, use the \textbf{hist()} function to graph the number of species per
watershed. You should be able to figure out what command to type.

3) Interpret your second histogram. How does the information displayed
in this histogram differ from the first histogram? Don't just say that
one is the number of watersheds occupied by each species and the other
is the number of species in each watershed. Interpret what you see.
Explain below and be prepared to discuss.

\textbf{State Analysis}

Perhaps the pattern that you observed is not surprising given the size
of North America. Let's take a look at a much smaller scale: the state.
This is the type of analysis you might perform as a state conservation
manager, for example. The steps are the same as you the steps you
performed for North America.

4) Write your hypothesis here. Considering a single state like Missouri
or Georgia, tell whether you think most species in a closely related
group of organisms have large, medium or small range size, or some
combination. Be prepared to discuss your hypothesis and \emph{your
justification for your prediction} with the rest of the class.

\textbf{Get the State data.}

Substitute the file name of your \textbf{state} data set for filename in
the command. Be sure to use your state file name, which you wrote down
earlier.

spp \textless{}-
read.csv('http://mtaylor4.semo.edu/\textasciitilde{}goby/biogeo/filename.csv',
header=TRUE, row.names=1)

dim(spp) \# do you remember which number represents watersheds

\begin{quote}
\# and which represents species? If not, review above.
\end{quote}

Write down the number of watersheds and species. You'll use the number
of watersheds later in this exercise.

\textbf{Number of watersheds \_\_\_\_\_\_\_\_\_\_\_\_\_\_\_\_. Number of
species \_\_\_\_\_\_\_\_\_\_\_\_\_\_\_\_.}

Calculate the number of watersheds per species and assign it to the
variable \textbf{numWatersheds}. Do the same for number of species per
watershed and assign to \textbf{numSpecies}. Use \textbf{rowSums()} and
\textbf{colSums()} as needed. If you don't remember which function goes
with watersheds and which goes with species, review page 3.

Determine whether most species at the state scale have large, medium, or
small size ranges, or a mix. You should know what function to use.

5) Interpret the pattern. Describe the pattern that you see. Do most
species occur in most watersheds (large range size) or do most species
occur in just a few watersheds (small range size)? Or, is it some
combination of the two? Explain how the observed result supports or
falsifies your hypothesis in question 4.

6) Repeat this for numSpecies. Interpret the pattern. How does the
information displayed in this histogram differ from the first histogram?
Don't just say that one is the number of watersheds occupied by each
species and the other is the number of species in each watershed.
Interpret what you see. Explain below and be prepared to discuss.

Time to Get Fancy

Now that you have viewed your two state-scale histograms in rough form,
you'll improve the presentation quality\footnote{While you already know
  the answers for this exercise, that is not enough. As a scientist, you
  will always need to generate presentation quality figures or tables.
  The exercises in this course will always stress this necessary and
  valuable skill.}. The arguments you will use include \textbf{xlab},
\textbf{xlim}, \textbf{main}, and \textbf{breaks}. You should be
familiar with the first three arguments from the R Tutorial but
\textbf{?hist} will help you refresh your memory. The \textbf{breaks}
argument tells the \textbf{hist()} function how many categories, or
bins, on the X-axis to use for the histogram. Setting the number of
breaks appropriately is important for creating an easy to interpret
histogram.

First, you need to set the breaks for your \textbf{numWatersheds} data.
This one is rather simple. Create bins for 1 watershed, 2 watersheds,
and so on up to the number of watersheds in your data set (remember that
value from above? If not, look it up.). For example, if you have 17
watersheds in your data set, you would use 17 breaks, with one break for
each watershed.

hist(numWatersheds, breaks=17)

However, when \textbf{breaks} is specified in this manner,
\textbf{hist()} treats it as a suggestion, but may and probably will
override your command. To work around this, generate a sequence of
numbers from 0 to 17 in 0.1 increments (use the number of watersheds in
your data set instead of 17). Assign the result to the variable
\textbf{wsbreaks} (for \emph{w}ater\emph{s}hed breaks).

wsbreaks \textless{}- seq(0, 17, 1) \# ?seq. Only need to do this once.

wsbreaks \# view the results

View the histogram using your defined sequence of breaks.

hist(numWatersheds, breaks=wsbreaks) \# hist() will obey this time.

\textbf{Coding note}: You could actually combine the two lines of code
into one:

hist(numWatersheds, breaks = seq(0, 17, 1)) \# Whichever works for you.

Determining the number of breaks for the second histogram (number of
species per watershed) will be a bit trickier. I want you to create bins
of 10. That is, the first column in the histogram will contain
watersheds with 1-10 species, the second with 11-20 species, and so on.
First, find out the maximum number of species in any watershed.

max(numSpecies)

Round that number \emph{up} to the largest multiple of 10. For example,
if you have a watershed with 117 species, round up to 120. If you have a
watershed with 83 species, round up to 90. Now, divide that number by
10. That result will be the number of breaks. This will create even
breaks that should display nicely (I hope! I haven't tried all of the
data sets.)

Use the \textbf{seq()}function to generate a sequence of numbers from 1
to the number you just calculated, in increments of 10. Assign the
sequence to a variable called \textbf{spbreaks} (for \emph{sp}ecies
breaks). Then, create a histogram of \textbf{numSpecies} using the
\textbf{spbreaks} argument.

\textbf{Important Hint}: In the R Tutorial, I had you type the commands
repeatedly to help drill them into your memory. However, you can save
yourself some time in the next section by typing your commands into
Notepad (PC), Wordpad (PC), TextEdit (Mac), or other text editor, then
copy and paste your commands into R. You can copy and paste many
commands at once into the R command window. This will save retyping your
commands as you add your arguments and correct any mistakes. \emph{Do
not use Microsoft Word to type your commands, as Word will try to turn
your quote marks into ``curly quotes'' which R does not recognize}.

Onward! Build the two histograms, one at a time, using the following
parameters. Each student should build one of the histograms, and then
use his or her initials in the main title, as indicated below. Don't
forget to use quotes around the text you use for the labels and titles.

\textbf{\emph{Number of Watersheds Per Species (numWatersheds)}}

X-Axis Label: Number of Watersheds per Species

Main Title*: Frequency Distribution of Range Size\textbackslash{}nfor
\textless{}state taxon\textgreater{}. Change to your state and taxonomic
group, and add your initials in parentheses. For example, ``Frequency
Distribution of Range Size for Vermont Mussels (MST)''. \textbf{(See *
next page)}

X-axis dimensions: 0 to the maximum number of watersheds in your data
set.

Look up the \textbf{label} argument in \textbf{hist()} to place labels
above each bar.

\textbf{\emph{Number of Species Per Watershed (numSpecies)}}

X-Axis Label: Proportion of Species per Watershed

Main Title*: Frequency Distribution of Species
Richness\textbackslash{}nfor \textless{}taxon\textgreater{} in
\textless{}state\textgreater{} Watersheds (initials). For example,
``Frequency Distribution of Species Richness for Mussels in Vermont
Watersheds (MST)''.

X-axis dimensions: Will probably be set properly based on your breaks.

Use the \textbf{label} argument to place labels above each bar.

* These titles are very long so replace one of the spaces between words
in the title with ``\textbackslash{}n'' (no quotes) to split the title
across two lines. You decide where but aim for balance between the two
lines. You used this same technique in the first tutorial to add your
name to one figure.

If you wish, you may play with the \textbf{col} and \textbf{border}
arguments to specify fill colors and border colors for the bars in your
histogram. Remember though that presentation quality, not psychedelic
flashbacks, is your primary goal.

To plot both histograms side by side in one figure, type the following:

op \textless{}- par(mfrow=c(1,2)) \# Change default \emph{par}ameters to
plot one row

\# with 2 columns. One graph per column.

hist(numWatersheds) \# plot the first histogram in the first column

hist(numSpecies) \# plot the second histogram in the second column

\# Substitute your commands for each histogram!

par(op) \# reset the default parameters. Good habit.

If necessary, expand the window to full screen. Save your histograms as
a PDF file and upload to the Unit 1a: Range Size drop box.
