%!TEX TS-program = lualatex
%!TEX encoding = UTF-8 Unicode

\documentclass[11pt]{article}
\usepackage{graphicx}
	\graphicspath{{/Users/goby/Pictures/teach/438/lab/}} % set of paths to search for images

\usepackage{geometry}
\geometry{letterpaper}                   
\geometry{bottom=1in}
%\geometry{landscape}                % Activate for for rotated page geometry
\usepackage[parfill]{parskip}    % Activate to begin paragraphs with an empty line rather than an indent
\usepackage{amssymb}
%\usepackage{mathtools}
%	\everymath{\displaystyle}

\usepackage{color}
%\pagenumbering{gobble}

\usepackage{fontspec}
\setmainfont[Ligatures={Common, TeX}, BoldFont={* Bold}, ItalicFont={* Italic}, BoldItalicFont={* Bold Italic}, Numbers={Proportional, OldStyle}]{Linux Libertine O}
\setsansfont[Scale=MatchLowercase,Ligatures=TeX]{Linux Biolinum O}
\setmonofont[Scale=MatchLowercase]{Linux Libertine Mono O}
\usepackage{microtype}

\usepackage{unicode-math}
\setmathfont[Scale=MatchLowercase]{Asana-Math.otf}
%\setmathfont{XITS Math}

% To define fonts for particular uses within a document. For example, 
% This sets the Libertine font to use tabular number format for tables.
%\newfontfamily{\tablenumbers}[Numbers={Monospaced}]{Linux Libertine O}
%\newfontfamily{\libertinedisplay}{Linux Libertine Display O}


\usepackage{booktabs}
\usepackage{longtable}
%\usepackage{tabularx}
%\usepackage{siunitx}
%\usepackage[justification=raggedright, singlelinecheck=off]{caption}
%\captionsetup{labelsep=period} % Removes colon following figure / table number.
%\captionsetup{tablewithin=none}  % Sequential numbering of tables and figures instead of
%\captionsetup{figurewithin=none} % resetting numbers within each chapter (Intro, M&M, etc.)
%\captionsetup[table]{skip=0pt}

\usepackage{array}
\newcolumntype{L}[1]{>{\raggedright\let\newline\\\arraybackslash\hspace{0pt}}p{#1}}
\newcolumntype{C}[1]{>{\centering\let\newline\\\arraybackslash\hspace{0pt}}p{#1}}
\newcolumntype{R}[1]{>{\raggedleft\let\newline\\\arraybackslash\hspace{0pt}}p{#1}}

\usepackage{enumitem}
\setlist{leftmargin=*}
\setlist[1]{labelindent=\parindent}
\setlist[enumerate]{label=\textsc{\alph*}.}
%\setlist[itemize]{label=\color{gray}\textbullet}
%\usepackage{hyperref}

\usepackage[sc]{titlesec}

\newcommand{\coursename}{\textsc{bi} 438/638: Biogeography}

\usepackage{fancyhdr}
\fancyhf{}
\pagestyle{fancy}
%\lhead{}
%\chead{}
%\rhead{Name: \rule{5cm}{0.4pt}}
%\renewcommand{\headrulewidth}{0pt}
\setlength{\headheight}{14pt}
\fancyhead[R]{\footnotesize Rapoport's Rule Abstract Assignment\thepage}
\fancyhead[L]{\footnotesize \coursename}

\fancypagestyle{first_page}{%
	\fancyhf{}
	\fancyhead[L]{\coursename}
	\fancyhead[R]{Name: \enspace \rule{2.5in}{0.4pt}}
	\renewcommand{\headrulewidth}{0pt}
}

\newcommand{\bigSpace}{\vspace{5\baselineskip}}

\newlength{\myLength}
\setlength{\myLength}{\parindent}



\begin{document}
\thispagestyle{first_page}

\subsection*{Rapoport's Rule: Abstract Assignment (30 points)}

An important role as a scientist is to share your results with other
scientists. One way to share your results is via oral presentation or
poster presentation at a national conference. Prior to giving your
presentation, you have to submit an abstract that will be published in
the conference proceedings. (Bet you thought for a moment you might have
to give an oral presentation.) An abstract is a concise summary of the
purpose of your project, the methods, results and the broader
implications of your results. A typical abstract is usually limited to
250 words.

You must write a 230 word (mininimum) to 250 word (maximum) abstract 
that includes the following information.

\begin{enumerate}
	\item The title and author. Follow the format of the examples on the next page.
	\textit{These do not count towards your word count limit.}

	\item The background and motivation for your project. What
	is the question you attempted to answer and why did it need to be
	answered? You have read about Rapoport's Rule and you have a sense of
	the debate about whether Rapoport's Rule applies universally to all
	taxa. Thus, you have background information and a motivation for
	performing your study.

	\item Your methods. I have provided a basic outline of the methods
	in the handouts. \textit{Transcribe them to your own words. Do not copy mine.} Feel free to ask questions.

	\item Your results. You have them. Study and interpret them
	concisely and correctly.

	\item Implication of your results, particularly as it relates to
	the background and motivation. How do your results compare with the
	predictions of the rules? What would be the implications of your results
	for other aquatic groups in North America, such as crayfishes or
	mussels, or for freshwater fishes in Europe and Asia or South America or
	Africa? This is often very brief. I am not as concerned about your
	specific statements as I am that you demonstrate some depth of thought.
	Think about why or how Rapoport's Rule does or does not apply to fishes,
	and relate those ideas to other groups or regions.

\end{enumerate}

On the other side of this handout are two example abstracts from the
2013 proceedings of the International Biogeographic Society biannual
conference for you to use as a guide. The course website also contains a
link to recent conference program filled with abstracts that you can 
browse to get a sense of proper abstract structure.%
\footnote{Scientists are human. Some turn in abstracts
before they have the full results to meet the deadline. You may
stumble across an abstract that suggests, ``Results will be
discussed.'' You must not do this for the assignment. You must include
results and a brief discussion.}

Upload your abstract to the Rapoport's Rule Abstracts 
drop box by the due date.

Abstracts that are below the 230 word limit, exceed the 250 word limit, 
or that are written poorly will be returned for further editing 
(and a 20\% deduction). Be concise. Be professional.

\newpage

\subsubsection*{Example abstracts}

\textbf{Species-to-genus ratios reveal latitudinal asymmetry in the
diversification of New World bats}

Hector T. Arita

The tropical niche conservatism hypothesis (TNCH) has been proposed to
explain the high species richness in Neotropical vertebrates. Under the
hypothesis, most clades have a tropical origin and fail to disperse to
extratropical areas; clades of temperate origin, in contrast, tend to
occur also in the tropics, thus contributing to the diversity there.
Here I test the hypothesis for the New World fauna of bats (Chiroptera)
by analyzing the latitudinal gradient of diversity of the nine families.
The gradient of richness for the 341 species shows a pattern consistent
with the TNCH, but the gradient for the 89 genera reveals subtler
patterns: (a) higher than expected species-to-genus ratios both at the
southern and northern extremes of the continent; (b) a low ratio in
Central America and Mexico, but not in the corresponding latitudes in
South America; (c) a symmetrical Rapoport pattern for genera (more
genera with small ranges near the equator); but (d) a highly
asymmetrical pattern for species (with an over-representation of
restricted species in the extreme south); (e) a rather uniform
distribution of endemic species; but (f) a higher than expected number
of endemic genera in North America but not in South America. In general,
results support the TNCH, but also reveal the effect of the geological
history of the area. In particular, I hypothesize that some of the
genera traditionally considered of South American origin might have
originated in North America prior to the Great American Biotic
Interchange (GABI).

\textbf{Do Rapoport effects apply to epibiont aquatic invertebrates? The
case of ostracod Entocytheridae commensal on freshwater crustaceans}

Alexandre Mestre, Juan S. Monrós and Francesc Mesquita-Joanes

The Rapoport pattern, observed in a wide variety of taxonomic groups,
states that species having geographic distributions with higher mean
latitude or altitude would exhibit wider latitudinal or altitudinal
ranges, respectively. One of the main explanations proposed for this
pattern suggests that at higher altitudes/latitudes, the reduction in
species richness facilitates increased ranges because of reduced
competition. Studies testing Rapoport's rule are mainly based on
free-living organisms. The distribution of non free-living species,
however, may be strongly cued to particular factors such as host
specificity, richness or dispersal abilities. To check for Rapoport
effects on a group of epibiont organisms, we built a geographic database
of Entocytherid ostracods, commensal on other crustaceans, which
constituted of \textgreater{}2100 sites located between 15 and 50° of
latitude, and altitudes of 0–3000 m a.s.l., including 220 ostracod
species and 243 hosts, mostly crayfishes. The altitudinal Rapoport
relationship seems to hold both for commensals and their hosts, parallel
to an expected reduction in species richness. However, although the
hosts seem to loosely follow the latitudinal pattern of Rapoport's rule,
their commensals do not show a clear pattern. This could be due to a
peculiar latitudinal distribution of species richness in both groups,
which present a peak at intermediate latitudes (30–40°). The
biogeographic history of crayfish hosts, together with strong dispersal
barriers such as Central American dessert areas, could strongly affect
the geographic distribution of the entocytherids.


\end{document}  