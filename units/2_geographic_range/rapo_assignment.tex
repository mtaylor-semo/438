%!TEX TS-program = lualatex
%!TEX encoding = UTF-8 Unicode

\documentclass[11pt]{article}
\usepackage{graphicx}
	\graphicspath{{/Users/goby/Pictures/teach/438/lab/}} % set of paths to search for images

\usepackage{geometry}
\geometry{letterpaper}                   
\geometry{bottom=1in}
%\geometry{landscape}                % Activate for for rotated page geometry
\usepackage[parfill]{parskip}    % Activate to begin paragraphs with an empty line rather than an indent
%\usepackage{amssymb}
%\usepackage{mathtools}
%	\everymath{\displaystyle}

%\pagenumbering{gobble}

\usepackage{fontspec}
\setmainfont[Ligatures={Common, TeX}, BoldFont={* Bold}, ItalicFont={* Italic}, Numbers={Proportional}]{Linux Libertine O}
\setsansfont[Scale=MatchLowercase,Ligatures=TeX]{Linux Biolinum O}
\setmonofont[Scale=MatchLowercase]{Linux Libertine Mono O}
\usepackage{microtype}

\usepackage{unicode-math}
\setmathfont[Scale=MatchLowercase]{Asana-Math.otf}
%\setmathfont{XITS Math}

% To define fonts for particular uses within a document. For example, 
% This sets the Libertine font to use tabular number format for tables.
%\newfontfamily{\tablenumbers}[Numbers={Monospaced}]{Linux Libertine O}
%\newfontfamily{\libertinedisplay}{Linux Libertine Display O}


\usepackage{booktabs}
\usepackage{longtable}
%\usepackage{tabularx}
%\usepackage{siunitx}
%\usepackage[justification=raggedright, singlelinecheck=off]{caption}
%\captionsetup{labelsep=period} % Removes colon following figure / table number.
%\captionsetup{tablewithin=none}  % Sequential numbering of tables and figures instead of
%\captionsetup{figurewithin=none} % resetting numbers within each chapter (Intro, M&M, etc.)
%\captionsetup[table]{skip=0pt}

\usepackage{array}
\newcolumntype{L}[1]{>{\raggedright\let\newline\\\arraybackslash\hspace{0pt}}p{#1}}
\newcolumntype{C}[1]{>{\centering\let\newline\\\arraybackslash\hspace{0pt}}p{#1}}
\newcolumntype{R}[1]{>{\raggedleft\let\newline\\\arraybackslash\hspace{0pt}}p{#1}}

\usepackage{enumitem}
\setlist{leftmargin=*}
\setlist[1]{labelindent=\parindent}
\setlist[enumerate]{label=\textsc{\alph*}., ref=\textsc{\alph*}}

\usepackage{hyperref}
%\usepackage{placeins} %P4ovides \FloatBarrier to flush all floats before a certain point.

\usepackage[sc]{titlesec}

\usepackage{hanging}

\newcommand{\assignmentTitle}{Rapoport's Rule: U.S. Freshwater Fishes}

\usepackage{fancyhdr}
\fancyhf{}
\pagestyle{fancy}
%\lhead{}
%\chead{}
%\rhead{Name: \rule{5cm}{0.4pt}}
%\renewcommand{\headrulewidth}{0pt}
\setlength{\headheight}{14pt}
\fancyhead[LE,RO]{\footnotesize \assignmentTitle\ \thepage}

\fancypagestyle{firstpage}
{
   \fancyhf{}
   \lhead{\textsc{bi}~438/638 Biogeography}
   \rhead{Name: \rule{5cm}{0.4pt}}
   \renewcommand{\headrulewidth}{0pt}
%   \fancyfoot[C]{\footnotesize Page \thepage\ of \pageref{LastPage}}
}

\newcommand{\bigSpace}{\vspace{5\baselineskip}}

\newlength{\myLength}
\setlength{\myLength}{\parindent}

%\title{Tutorial: Climate and Distribution}
%\author{10 Points}
%\date{}                                           % Activate to display a given date or no date

\begin{document}
%\maketitle
\thispagestyle{firstpage}

\subsection*{\assignmentTitle\ (10 points)}

Rapoport's Rules were based originally on land birds and mammals. You have 
already explored how the rules fit coastal marine fishes and, to a lesser extent,
freshwater fishes. 

For this exercise, you will explore whether the rules apply to 
primary freshwater fishes in the U.S., plus parts of northern Mexico and southern 
Canada. First, you will explore mean and maximum richness as a function of latitude.
Second, you will explore whether range size increases with latitude.

The geographic area was divided into a grid of 1° latitude x 1° longitude cells
(see figure below). If a species was collected in one of the 1° cells, then
presence was recorded with a 1. Absence was recorded with a 0. The 
presence/absence data for each cell was recorded for 529 species of North American
freshwater fishes.

\begin{center}
	\includegraphics[width=\textwidth]{na_grid}
\end{center}

\subsubsection*{Important}

\textbf{Think carefully about your predictions.} Is the U.S. uniform from
south to north? What about east to west? Think about where the U.S. tends
to be wetter (more precipitation) or more arid (drier). Do some regions
have more rivers than other regions? Will the U.S. have any regional climate
effects due to rain shadows or deserts? How might this influence freshwater
fishes and Rapoport's Rule?"

\newpage

\subsubsection*{Predictions}

\begin{enumerate}
	\item Open a web browser on your computer and go to \url{https://semobio.shinyapps.io/rapo}. 
	You will also find a “Rapoport's Rule analysis site” link on our Canvas page (in the Unit 2: Geographic Range module) that you can click instead.
	
	\item Read the Introduction page, then click the Predictions tab to begin. 
	
	\item Enter your name and enter your predictions. Think about information learned
	from lecture and previous exercises as you form your predictions.
	
	\item Press the Next button when finished with this page.
	
\end{enumerate}

\subsubsection*{Species Richness and Area Occupied.}

\begin{enumerate}[resume]
	\item Study both the Species Richness and Area Occupied plots, and then answer the three questions below the plots.
	
	\item When finished, press the Next button for the next tab.
		
\end{enumerate}

\subsubsection*{U.S. Range Size}

%	\textbf{Part 1: state range size}

\begin{enumerate}[resume]
	
	\item Study this plot carefully. Each circle represents the range of one species.
	The diameter of the circle is proportional to the total range area occupied by the species.
	The center of the circle is located at the approximate mid-point of the latitude and 
	longitude of the range. (The algorithm is crude so some circles end up in the ocean but
	this is due to the slant of the east coast of the U.S.)
	
	\item Things to think about: Why are there more species in the eastern U.S. compared
	to the western U.S. Why are the range sizes so small in the southwestern U.S.? What
	about the many small ranges in the south eastern U.S. Where would you say overall
	species richness is highest in the U.S. Why bigeographic process might explain this pattern?
	
	\item Answer the three questions below the figure, then press the Next button.

\end{enumerate}

\subsubsection*{Summarize what you learned}

\begin{enumerate}[resume]
	
	\item Enter a summary of the lessons you learned from this exercise, then click the Download button.
	
	\emph{Leave your browser open until the message says it is okay to close it}.
	
	\item After your \textsc{pdf} report has downloaded,  upload it to the Rapoport's Rule (Freshwater Fishes) drop box on our Canvas page.
\end{enumerate}


\end{document}  