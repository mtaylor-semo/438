%!TEX TS-program = lualatex
%!TEX encoding = UTF-8 Unicode

\documentclass[t]{beamer}

%%%% HANDOUTS For online Uncomment the following four lines for handout
%\documentclass[t,handout]{beamer}  %Use this for handouts.
%\includeonlylecture{student}
%\usepackage{handoutWithNotes}
%\pgfpagesuselayout{3 on 1 with notes}[letterpaper,border shrink=5mm]


% FONTS
\usepackage{fontspec}
\def\mainfont{Linux Biolinum O}
\setmainfont[Ligatures={Common,TeX}, Contextuals={NoAlternate}, BoldFont={* Bold}, ItalicFont={* Italic}, Numbers={Proportional}]{\mainfont}
\setmonofont[Scale=MatchLowercase]{Inconsolatazi4} 
\setsansfont[Scale=MatchLowercase]{Linux Biolinum O} 
\usepackage{microtype}

\usepackage{graphicx}
	\graphicspath{%
	{/Users/goby/Pictures/teach/438/lectures/}%
	{/Users/goby/Pictures/teach/common/}%}%
	{img/}} % set of paths to search for images

\usepackage{amsmath,amssymb}

%\usepackage{units}

\usepackage{booktabs}
\usepackage{multicol}
%	\setlength{\columnsep=1em}

%\usepackage{textcomp}
%\usepackage{setspace}
\usepackage{tikz}
	\tikzstyle{every picture}+=[remember picture,overlay]

\mode<presentation>
{
  \usetheme{Lecture}
  \setbeamercovered{invisible}
  \setbeamertemplate{items}[square]
}

\usepackage{calc}
\usepackage{hyperref}

\newcommand\HiddenWord[1]{%
	\alt<handout>{\rule{\widthof{#1}}{\fboxrule}}{#1}%
}



\begin{document}
%\lecture{instructor}{instructor}
\lecture{student}{student}
%
\begin{frame}[t]{Two competing hypotheses proposed to explain early human biogeography.}
\begin{multicols}{2}
	\includegraphics[height=0.72\textheight]{human_multiregional}

	\includegraphics[height=0.72\textheight]{human_ooafrica}
\end{multicols}
\end{frame}
%
\begin{frame}{Human mitochondrial DNA supports the \highlight{replacement} hypothesis.}
	\begin{center}
		\includegraphics[width=\textwidth]{human_ooafrica_genetic_evidence}
	\end{center}
\end{frame}
%
\begin{frame}[t]{Replacement and range expansion was complex.}
	\vspace{-\baselineskip}
	\begin{center}
		\includegraphics[height=0.85\textheight]{human_complex_expansion}
	\end{center}

	\vfilll
	
	\hfill \tiny Fig.~16.30 \copyright Sinauer Assoc., Inc.
\end{frame}
%
{
\usebackgroundtemplate{\includegraphics[width=\paperwidth]{human_african_populations} }
\begin{frame}[b]{}

\hfill\tiny Gunz et al. 2009. PNAS 106: 6094-6098.
\end{frame}
}
%
\begin{frame}{Global expansion of modern humans took \textasciitilde{}200,000 years.}
	\vspace{-\baselineskip}
	\begin{center}
		\includegraphics[width=\textwidth]{human_range_expansion}
	\end{center}

	\vfilll
	
	\hfill \tiny Fig.~16.32 \copyright Sinauer Assoc., Inc.
	
\end{frame}
%
\begin{frame}{\textit{Homo erectus} expanded to Europe and Asia.}
	\vspace{-\baselineskip}
	\begin{center}
		\includegraphics[width=0.92\textwidth]{human_homo_erectus_range}
	\end{center}

	\vfilll
	\hfill \tiny Fig.~16.31 \copyright Sinauer Assoc., Inc.

\end{frame}
%
{
\usebackgroundtemplate{\includegraphics[width=\paperwidth]{human_indonesia} }
\begin{frame}[b]{}

	\vfilll
	
	\tiny Fig.~16.33 \copyright Sinauer Assoc., Inc.

\end{frame}}
%
\begin{frame}{Sidebar: Insular species tend to change mean body size relative to mainland.}
	\vspace{-\baselineskip}
	\begin{center}
		\includegraphics[width=\textwidth]{island_rule_table}
	\end{center}

	\vfilll
	
	\hfill \tiny Table 14.5 \copyright Sinauer Assoc., Inc.
\end{frame}
%
\begin{frame}{Sidebar: The \highlight{island rule} explains body size change for insular species.}
	\vspace{-\baselineskip}
	\begin{center}
		\includegraphics[width=0.85\textwidth]{island_rule_figure}
	\end{center}

	\vfilll
	
	\hfill \tiny Fig.~14.26 \copyright Sinauer Assoc., Inc.
	
\end{frame}
%
\begin{frame}{\textit{H. floresiensis} body size follows the island rule.}
	\vspace{-\baselineskip}
	\begin{center}
		\includegraphics[width=\textwidth]{human_floresiensis_dwarfism}
	\end{center}

	\vfilll

	\hfill \tiny Fig.~14.28 \copyright Sinauer Assoc., Inc.
	
\end{frame}
%
{
\usebackgroundtemplate{\includegraphics[width=\paperwidth]{human_neanderthal_expansion} }
\begin{frame}[b]{}

\tiny Stewart and Stringer 2012. Science 335: 1317–1321.
\end{frame}}
%
{
\usebackgroundtemplate{\includegraphics[width=\paperwidth]{human_neanderthal_climate_refugia} }
\begin{frame}[b]{}

\tiny Finlayson 2008.~Quatern.~Sci.~Rev.~27:2246–2252. \hfill Finlayson 2005.~TREE 20:457–463.
\end{frame}
}
%
\begin{frame}{\textit{H. sapiens} entered North America \emph{at least} once.}
	\vspace{-\baselineskip}
	\begin{center}
		\includegraphics[width=\textwidth]{human_range_expansion}
	\end{center}

	\vfilll
	
	\hfill \tiny Fig.~16.32 \copyright Sinauer Assoc., Inc.
\end{frame}
%
\begin{frame}{Linguistic analysis suggest humans colonized \textsc{n.a.} in three waves. What about the genetic evidence?}
	\vspace{-0.5\baselineskip}
	\begin{center}
		\includegraphics[height=0.83\textheight]{human_language_groups}
	\end{center}
\end{frame}
%
\begin{frame}[t]{Beringian refuge may explain observed pattern.}
	\vspace{-\baselineskip}
	\begin{center}
		\includegraphics[width=\textwidth]{human_language_hypotheses}
	\end{center}
\end{frame}
%
\begin{frame}[t]{Megafauna extinction increased with human expansion.}
	\vspace{-\baselineskip}
	\begin{center}
		\includegraphics[width=\textwidth]{human_expansion_animal_extinction}
	\end{center}

	\vfilll
	
	\hfill \tiny Fig.~16.37 \copyright Sinauer Assoc., Inc.
	
\end{frame}
%
\begin{frame}{The southwest Pacific islands were the last places colonized.}
	\vspace{-\baselineskip}
	\begin{center}
		\includegraphics[width=\textwidth]{human_pacific_islands}
	\end{center}
\end{frame}
%
\begin{frame}{Sidebar: Lizards came along for the ride.}
	\vspace{-\baselineskip}
	\begin{center}
		\includegraphics[height=0.85\textheight]{human_pacific_lizards}
	\end{center}

	\hfill \tiny Austin 1999.~Nature 397:113-114.
\end{frame}
%
\begin{frame}{Linguistic diversity on islands fits the IB model.}
	\vspace{-\baselineskip}
	\begin{center}
		\includegraphics[height=0.85\textheight]{human_insular_language_diversity}
	\end{center}

	\vfilll
	
	\hfill \tiny Fig.~16.35 \copyright Sinauer Assoc., Inc.
\end{frame}
%
\begin{frame}{Insular extinction increased with human colonization.}
	\vspace{-0.5\baselineskip}
	\begin{center}
		\includegraphics[height=0.82\textheight]{human_insular_animal_extinction}
	\end{center}

	\vfilll

	\tiny Fig.~16.38 \copyright Sinauer Assoc., Inc.

\end{frame}
%
\end{document}
