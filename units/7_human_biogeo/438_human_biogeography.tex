%!TEX TS-program = lualatex
%!TEX encoding = UTF-8 Unicode

\documentclass[t]{beamer}

%%%% HANDOUTS For online Uncomment the following four lines for handout
%\documentclass[t,handout]{beamer}  %Use this for handouts.
%\usepackage{handoutWithNotes}
%\pgfpagesuselayout{3 on 1 with notes}[letterpaper,border shrink=5mm]
%	\setbeamercolor{background canvas}{bg=black!5}


%%% Including only some slides for students.
%%% Uncomment the following line. For the slides,
%%% use the labels shown below the command.
%\includeonlylecture{student}

%% For students, use \lecture{student}{student}
%% For mine, use \lecture{instructor}{instructor}


%\usepackage{pgf,pgfpages}
%\pgfpagesuselayout{4 on 1}[letterpaper,border shrink=5mm]

% FONTS
\usepackage{fontspec}
\def\mainfont{Linux Biolinum O}
\setmainfont[Ligatures={Common,TeX}, Contextuals={NoAlternate}, BoldFont={* Bold}, ItalicFont={* Italic}, Numbers={Proportional}]{\mainfont}
\setmonofont[Scale=MatchLowercase]{Inconsolatazi4} 
\setsansfont[Scale=MatchLowercase]{Linux Biolinum O} 
\usepackage{microtype}

\usepackage{graphicx}
	\graphicspath{%
	{/Users/goby/Pictures/teach/438/lectures/}%
	{/Users/goby/Pictures/teach/common/}%}%
	{img/}} % set of paths to search for images

\usepackage{amsmath,amssymb}

%\usepackage{units}

\usepackage{booktabs}
\usepackage{multicol}
%	\setlength{\columnsep=1em}

%\usepackage{textcomp}
%\usepackage{setspace}
\usepackage{tikz}
	\tikzstyle{every picture}+=[remember picture,overlay]

\mode<presentation>
{
  \usetheme{Lecture}
  \setbeamercovered{invisible}
  \setbeamertemplate{items}[square]
}

\usepackage{calc}
\usepackage{hyperref}

\newcommand\HiddenWord[1]{%
	\alt<handout>{\rule{\widthof{#1}}{\fboxrule}}{#1}%
}



\begin{document}
%\lecture{instructor}{instructor}
\lecture{student}{student}



\begin{frame}[t]{Two competing hypothesis proposed to explain early human biogeography.}
\begin{multicols}{2}
	\includegraphics[height=0.72\textheight]{human_multiregional}

	\includegraphics[height=0.72\textheight]{human_ooafrica}
\end{multicols}
\end{frame}

\begin{frame}{Human mitochondrial DNA supports the \highlight{replacement} hypothesis.}
	\begin{center}
		\includegraphics[width=\textwidth]{human_ooafrica_genetic_evidence}
	\end{center}
\end{frame}

\begin{frame}[t]{Replacement and range expansion was complex.}
	\vspace{-\baselineskip}
	\begin{center}
		\includegraphics[height=0.9\textheight]{human_complex_expansion}
	\end{center}
\end{frame}

{
\usebackgroundtemplate{\includegraphics[width=\paperwidth]{human_african_populations} }
\begin{frame}[b]{}

\hfill\tiny Gunz et al. 2009. PNAS 106: 6094-6098.
\end{frame}}

\begin{frame}{Global expansion of modern humans took \textasciitilde{}200,000 years.}
	\vspace{-\baselineskip}
	\begin{center}
		\includegraphics[width=\textwidth]{human_range_expansion}
	\end{center}
\end{frame}

\begin{frame}{\textit{Homo erectus} expanded to Europe and Asia.}
	\vspace{-\baselineskip}
	\begin{center}
		\includegraphics[width=\textwidth]{human_homo_erectus_range}
	\end{center}
\end{frame}

{
\usebackgroundtemplate{\includegraphics[width=\paperwidth]{human_indonesia} }
\begin{frame}[b]{}

\end{frame}}

\begin{frame}{Sidebar: Insular species tend to change mean body size relative to mainland.}
	\vspace{-\baselineskip}
	\begin{center}
		\includegraphics[width=\textwidth]{island_rule_table}
	\end{center}
\end{frame}

\begin{frame}{Sidebar: The \highlight{island rule} explains body size change for insular species.}
	\vspace{-\baselineskip}
	\begin{center}
		\includegraphics[width=0.9\textwidth]{island_rule_figure}
	\end{center}
\end{frame}

\begin{frame}{\textit{H. floresiensis} body size follows the island rule.}
	\vspace{-\baselineskip}
	\begin{center}
		\includegraphics[width=\textwidth]{human_floresiensis_dwarfism}
	\end{center}
\end{frame}




{
\usebackgroundtemplate{\includegraphics[width=\paperwidth]{human_neanderthal_expansion} }
\begin{frame}[b]{}

\tiny Stewart and Stringer 2012. Science 335: 1317–1321.
\end{frame}}

{
\usebackgroundtemplate{\includegraphics[width=\paperwidth]{human_neanderthal_climate_refugia} }
\begin{frame}[b]{}

\hfill\tiny Finlayson 2005. TREE 20: 457–463. Finlayson 2008. Quatern. Sci. Rev. 27: 2246–2252.
\end{frame}}

\begin{frame}{\textit{H. sapiens} entered North America \emph{at least} once.}
	\vspace{-\baselineskip}
	\begin{center}
		\includegraphics[width=\textwidth]{human_range_expansion}
	\end{center}
\end{frame}

\begin{frame}{Linguistic analysis suggest humans colonized N.A. in three waves. What about the genetic evidence?}
	\vspace{-0.5\baselineskip}
	\begin{center}
		\includegraphics[height=0.83\textheight]{human_language_groups}
	\end{center}
\end{frame}

\begin{frame}[t]{Berengian refuge may explain observed pattern.}
	\vspace{-\baselineskip}
	\begin{center}
		\includegraphics[width=\textwidth]{human_language_hypotheses}
	\end{center}
\end{frame}

\begin{frame}[t]{Megafauna extinction increased with human expansion.}
	\vspace{-\baselineskip}
	\begin{center}
		\includegraphics[width=\textwidth]{human_expansion_animal_extinction}
	\end{center}
\end{frame}

\begin{frame}{The southwest Pacific islands were the last places colonized.}
	\vspace{-\baselineskip}
	\begin{center}
		\includegraphics[width=\textwidth]{human_pacific_islands}
	\end{center}
\end{frame}

\begin{frame}{Sidebar: Lizards came along for the ride.}
	\vspace{-\baselineskip}
	\begin{center}
		\includegraphics[height=0.85\textheight]{human_pacific_lizards}
	\end{center}
\end{frame}


\begin{frame}{Linguistic diversity on islands fits the IB model.}
	\vspace{-\baselineskip}
	\begin{center}
		\includegraphics[height=0.85\textheight]{human_insular_language_diversity}
	\end{center}
\end{frame}

\begin{frame}{Insular extinction increased with human colonization.}
	\vspace{-0.5\baselineskip}
	\begin{center}
		\includegraphics[height=0.85\textheight]{human_insular_animal_extinction}
	\end{center}
\end{frame}



\end{document}
