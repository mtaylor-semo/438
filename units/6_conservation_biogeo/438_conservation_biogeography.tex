%!TEX TS-program = lualatex
%!TEX encoding = UTF-8 Unicode

\documentclass[t]{beamer}

%%%% HANDOUTS For online Uncomment the following four lines for handout
%\documentclass[t,handout]{beamer}  %Use this for handouts.
%\includeonlylecture{student}
%\usepackage{handoutWithNotes}
%\pgfpagesuselayout{3 on 1 with notes}[letterpaper,border shrink=5mm]
%	\setbeamercolor{background canvas}{bg=black!5}


%%% Including only some slides for students.
%%% Uncomment the following line. For the slides,
%%% use the labels shown below the command.

%% For students, use \lecture{student}{student}
%% For mine, use \lecture{instructor}{instructor}

% FONTS
\usepackage{fontspec}
\def\mainfont{Linux Biolinum O}
\setmainfont[Ligatures={Common,TeX}, Contextuals={NoAlternate}, BoldFont={* Bold}, ItalicFont={* Italic}, Numbers={Proportional}]{\mainfont}
\setmonofont[Scale=MatchLowercase]{Inconsolatazi4} 
\setsansfont[Scale=MatchLowercase]{Linux Biolinum O} 
\usepackage{microtype}

\usepackage{graphicx}
	\graphicspath{%
	{/Users/goby/Pictures/teach/438/lectures/}%
	{/Users/goby/Pictures/teach/common/}}%
%	{img/}} % set of paths to search for images

\usepackage{amsmath,amssymb}

%\usepackage{units}

\usepackage{booktabs}
\usepackage{multicol}
%	\setlength{\columnsep=1em}

%\usepackage{textcomp}
%\usepackage{setspace}
\usepackage{tikz}
	\tikzstyle{every picture}+=[remember picture,overlay]

\mode<presentation>
{
  \usetheme{Lecture}
  \setbeamercovered{invisible}
%  \setbeamertemplate{items}[square]
}

\usepackage{calc}
\usepackage{hyperref}

\newcommand\HiddenWord[1]{%
	\alt<handout>{\rule{\widthof{#1}}{\fboxrule}}{#1}%
}

\begin{document}

\begin{frame}[plain]{\highlight{Biodiversity hotspots} have high endemism and many threats.}
	\centering%
		\includegraphics[width=1\textwidth]{global_hotspots}\\
\end{frame}

\begin{frame}[plain]{Areas of high endemism for African birds and amphibians.}
	\centering%
		\includegraphics[width=1\textwidth]{africa_birds_amphibs}\\
\end{frame}

\begin{frame}[plain]{Areas of high endemism for African mammals and plants.}
	\centering%
		\includegraphics[width=1\textwidth]{africa_mammals_plants}\\
\end{frame}

\begin{frame}[plain]{African hotspots have high endemism for many taxa.}
	\centering
		\includegraphics[height=0.8\textheight]{africa_hotspots}\\
\end{frame}

\begin{frame}[plain]{Diversity for marine fishes, corals and snails is highest in the Indo-Pacific.}
	\centering
		\includegraphics[width=\textwidth]{marine_endemism}
		
	Fishes (top), corals (middle), and snails (bottom).
\end{frame}

\begin{frame}[plain]{Marine hotspots have high endemism and are threatened.}
	\centering
		\includegraphics[width=\textwidth]{marine_hotspots2}

	Threats (upper) and endemism (lower).
		
\end{frame}

{
\usebackgroundtemplate{\includegraphics[width=\paperwidth]{philippine_eagle}}
\begin{frame}[plain]

\vfilll

\hfill \tiny \textcolor{white}{\copyright\,Klaus Nigge, National Geographic}
\end{frame}
}

\begin{frame}[plain]{Apply island biogeography and metapopulation concepts to conserve diversity.}
	\begin{columns}[T]
		\begin{column}{0.6\textwidth}%
			\centering
			\includegraphics[height=0.6\textheight]{insular_distribution_function_top}\\
		\end{column}
		\begin{column}{0.4\textwidth}%
			\begin{itemize}%
				\item How are organisms distributed?
				\item Protect the distribution.
				\item Proceed despite limited knowledge.
			\end{itemize}
		\end{column}		
	\end{columns}
\end{frame}

{
\usebackgroundtemplate{\includegraphics[width=\paperwidth]{florida_panther_headshot.jpg}
}
\begin{frame}[plain]
\end{frame}
}

\begin{frame}[plain]{\textit{Puma concolor} is distributed widely.}
	\centering%
		\includegraphics[height=0.9\textheight]{puma_range}
\end{frame}

\begin{frame}[plain]{Fewer than 200 panthers live in South Florida.}
	\centering%
		\includegraphics[height=0.85\textheight]{florida_panther_range}
\end{frame}

\begin{frame}[plain]{Roadkills cause many fatalities every year.}
	\begin{columns}[T]%
		\begin{column}{0.4\textwidth}%
			\centering%
			\includegraphics[height=0.7\textheight]{florida_panther_roadkill.jpg}
		\end{column}
		\begin{column}{0.55\textwidth}%
			\centering%
			\includegraphics[height=0.7\textheight]{panther_roadkill_map}
		\end{column}
	\end{columns}
\end{frame}


{
\usebackgroundtemplate{\includegraphics[width=\paperwidth]{florida_panther_kittens}
}
\begin{frame}[plain]
\end{frame}
}

\begin{frame}[plain]{How can you increase the Florida Panther range up to the Okeefenokee Swamp in Georgia?}
	\centering%
		\includegraphics[height=0.8\textheight]{florida_corridor1}
\end{frame}

\begin{frame}[plain]{Florida agencies are proposing a series of habitat corridors.}
	\centering%
		\includegraphics[height=0.8\textheight]{florida_corridor2}
\end{frame}


\begin{frame}[plain]{Bridges and tunnels may increase safety and dispersal of panthers.}
	\centering%
		\includegraphics[height=0.8\textheight]{panther_corridors_proposed}
\end{frame}

{
\usebackgroundtemplate{\includegraphics[width=\paperwidth]{mesophotic_coral.jpg}
}
\begin{frame}[b,plain]{\textcolor{white}{Use biogeographic techniques to predict distribution and identify suitable habitat of mesophotic corals.}}
\hfill \textcolor{white}{\textit{Leptoseris} is one genus of mesophotic coral.}
\end{frame}
}

\begin{frame}[t,plain]{Predictive biogeography was used to identify suitable habitat in Hawaii}
	\centering%
		\includegraphics[height=0.8\textheight]{mesophotic_hawaii.jpg}\par
		
	\vskip0pt plus 1fill
		
\hfill\tiny Costa B.M., et al. 2012. NOAA Technical Report 149.
\end{frame}

\begin{frame}[t,plain]{34 parameters were used to predict the distribution.}
	\centering%
		\includegraphics[height=0.85\textheight]{mesophotic_predictors.jpg}\par

	\vskip0pt plus 1fill

\hfill\tiny Costa B.M., et al. 2012. NOAA Technical Report 149.
\end{frame}

\begin{frame}[t,plain]{Depth, light availability and distance from shore were most important.}
	\centering%
		\includegraphics[height=0.8\textheight]{mesophotic_results.jpg}\par

	\vskip0pt plus 1fill

\hfill\tiny Costa B.M., et al. 2012. NOAA Technical Report 149.
\end{frame}

\begin{frame}{Predicting the future is harder.}
\includegraphics[width=\textwidth]{forest_shifts}

\tinyfill Environmental Protection Agency, public domain
\end{frame}

\begin{frame}{Increasing temperatures put ecosystems at risk.}
\includegraphics[width=\textwidth]{grimm_fig3a}

\tinyfill Grimm et al.\ 2013. Front.\ Ecol.\ Environ.\ 11: 474
\end{frame}

\begin{frame}{The distributions of ecosystems will change. But how?}
\includegraphics[width=\textwidth]{grimm_fig3b}

\tinyfill Grimm et al.\ 2013. Front.\ Ecol.\ Environ.\ 11: 474
\end{frame}

\begin{frame}{Assess \highlight{exposure} and \highlight{resilience.}}
\centering
\includegraphics[height=0.85\textheight]{vulnerability_index}

\tinyfill Comer et al.\ 2022. Land 11: 302.
\end{frame}

\begin{frame}{An example with central mixedgrass prairies.}

\includegraphics[width=\textwidth]{mixed_grass_historical}

\onslide*<2>{
\begin{tikzpicture}
\draw[ultra thick, orange] (4.7,1.6) rectangle (6,6);
\end{tikzpicture}
}

\tinyfill Comer et al.\ 2022. Land 11: 302.
\end{frame}

\begin{frame}{Modeled exposure and resilience.}

\includegraphics[width=\textwidth]{exposure_resilience}

\tinyfill Comer et al.\ 2022. Land 11: 302.
\end{frame}

\begin{frame}{Mixedgrass prairies are vulnerable to climate change.}

\centering
\includegraphics[height=0.85\textheight]{vulnerability_mixedgrass}

\tinyfill Comer et al.\ 2022. Land 11: 302.
\end{frame}

\begin{frame}

\includegraphics[width=\textwidth]{grasshopper_sparrow}

\tinyfill eBird data
\end{frame}

{
\usebackgroundtemplate{\includegraphics[width=\paperwidth]{citizen_science}
}
\begin{frame}[b,plain]{Get involved \textit{now} via citizen science.}
\end{frame}
}


\begin{frame}
\includegraphics[width=\textwidth]{house_wren_trends}
\end{frame}


\begin{frame}
\includegraphics[width=\textwidth]{sage_grouse_trends}
\end{frame}

\begin{frame}
\includegraphics[width=\textwidth]{monarch_population_mexico}
\end{frame}

\end{document}
