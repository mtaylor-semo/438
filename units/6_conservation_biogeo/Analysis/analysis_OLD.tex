\documentclass[xcolor=svgnames]{beamer}
%\documentclass[xcolor=dvipsnames]{beamer}
%\usetheme{default}
\usepackage{pgf,pgfpages}
\usepackage{amsmath}
\usepackage{graphicx}
	\graphicspath{{img/}} % set of paths to search for images
\usepackage{units}
\usepackage{booktabs}
%\usepackage[utf8]{inputenc}
\usepackage[T1]{fontenc}
\usepackage{helvet}
\usepackage{fancyvrb}
\usepackage{tikz}
	\tikzstyle{every picture}+=[remember picture,overlay]


%\mode<handout> 	% Use this to make PDF with all overlays on one slide.
\mode<presentation>
{
  \usetheme{MST}
  \setbeamercovered{invisible}
  \setbeamertemplate{items}[square]
}

\usefonttheme[onlymath]{serif}
\usecolortheme[named=myBlue]{structure}

%% Uncomment the next two lines to make slides with note lines
%\usepackage{handoutWithNotes}
%\pgfpagesuselayout{3 on 1 with notes}[letterpaper,border shrink=5mm]


\title{Conservation Biogeography}

\begin{document}

\section{section}
\subsection{no section}

\begin{frame}[plain]{Set up Georgia crayfishes and fishes for analyses.}
\begin{enumerate}
	\item Read in the presence / absence data
	\item Calculate similarity of watersheds based on shared species
\end{enumerate}
\pause

\begin{block}{Type}
	source(`http://mtaylor4.semo.edu/~goby/biogeo/gacrayfish.r') \pause
	source(`http://mtaylor4.semo.edu/~goby/biogeo/gafish.r')
\end{block}
\end{frame}


\begin{frame}[plain]{Perform a cluster analysis.}
Watersheds will be clustered together based on similarity. 
\pause

\begin{block}{Type}
	source(`http://mtaylor4.semo.edu/~goby/biogeo/gacrayfishcluster.r')
\end{block}


\end{frame}

{
\usebackgroundtemplate{\includegraphics[width=\paperwidth]{ga_crayfish_cluster.png}
}
\begin{frame}[plain]
\end{frame}
}

\begin{frame}[plain]{Perform a principal coordinates analysis.}
Ordination procedure that reduces the number of dimensions while maintaining Euclidean distances. Plot the first two dimensions.
\pause

\begin{block}{Type}
	source(`http://mtaylor4.semo.edu/~goby/biogeo/crayfishpco.r')
\end{block}


\end{frame}

{
\usebackgroundtemplate{\includegraphics[width=\paperwidth]{ga_crayfish_pco.pdf}
}
\begin{frame}[plain]
\end{frame}
}

\begin{frame}[plain]
	\centering
	\includegraphics[height=1\textheight]{georgia_watersheds_crayfishes}
\end{frame}


\begin{frame}[plain]{Now, plot 3 dimensions at the same time.}

\begin{block}{Type}
	source(`http://mtaylor4.semo.edu/~goby/biogeo/crayfish3d.r')
\end{block}

Expand the window to full size. Right-click to zoom in and out. Left-click to rotate.

\end{frame}

\begin{frame}[plain]{Now, the fishes. First, perform a cluster analysis.}

\begin{block}{Type}
	source(`http://mtaylor4.semo.edu/~goby/biogeo/gafishcluster.r')
\end{block}
\end{frame}

{
\usebackgroundtemplate{\includegraphics[width=\paperwidth]{ga_fish_cluster.pdf}
}
\begin{frame}[plain]
\end{frame}
}


\begin{frame}[plain]{Next, the principal coordinates analysis for fishes.}

Study the two dimensional plot for patterns.
\begin{block}{Type}
	source(`http://mtaylor4.semo.edu/~goby/biogeo/fishpco.r')
\end{block}
\end{frame}


{
\usebackgroundtemplate{\includegraphics[width=\paperwidth]{ga_fish_pco.pdf}
}
\begin{frame}[plain]
\end{frame}
}

\begin{frame}[plain]
	\centering
	\includegraphics[height=1\textheight]{georgia_watersheds_fishes}
\end{frame}


\begin{frame}[plain]{Next, the principal coordinates analysis for fishes.}

Now, 3-D.

\begin{block}{Type}
	source(`http://mtaylor4.semo.edu/~goby/biogeo/fish3d.r')
\end{block}
\end{frame}


\begin{frame}[plain]{Do watersheds show similar distributions in ordinated space when compared between fishes and crayfishes?}

Perform a Procrustes analysis.

\begin{block}{Type}
	source(`http://mtaylor4.semo.edu/~goby/biogeo/procrustes.r')
\end{block}
\end{frame}


{
\usebackgroundtemplate{\includegraphics[width=\paperwidth]{ga_procrustes.pdf}
}
\begin{frame}[plain]
\end{frame}
}

\begin{frame}[plain]{Do watersheds show similar distributions in ordinated space when compared between fishes and crayfishes?}

Are the results significant?  If so, then the alignment of the two PCO analysis in ordination space is not random. The two are similar.

\begin{block}{Type}
	protest(fish.sc, crayfish.sc, permutations = 9999)
\end{block}
\end{frame}

\begin{frame}[plain]{Now, let's do Missouri crayfishes, then fishes.}
\begin{enumerate}
	\item Read in the presence / absence data
	\item Calculate similarity of watersheds based on shared species
\end{enumerate}
\pause

\begin{block}{Type}
	source(`http://mtaylor4.semo.edu/~goby/biogeo/mocrayfish.r') \pause
	source(`http://mtaylor4.semo.edu/~goby/biogeo/mofish.r')
\end{block}
\end{frame}

\begin{frame}[plain]{Perform the cluster analyses for both taxa.}
Watersheds will be clustered together based on similarity. 
\pause

\begin{block}{Type}
	source(`http://mtaylor4.semo.edu/~goby/biogeo/mocrayfishcluster.r')
	\pause
	source(`http://mtaylor4.semo.edu/~goby/biogeo/mofishcluster.r')
\end{block}
\end{frame}


{
\usebackgroundtemplate{\includegraphics[width=\paperwidth]{mo_crayfish_cluster.pdf}
}
\begin{frame}[plain]
\end{frame}
}

{
\usebackgroundtemplate{\includegraphics[width=\paperwidth]{mo_fish_cluster.pdf}
}
\begin{frame}[plain]
\end{frame}
}

\begin{frame}[plain]{Next, the principal coordinates analysis for crayfishes.}

Study the two dimensional plot for patterns.
\begin{block}{Type}
	source(`http://mtaylor4.semo.edu/~goby/biogeo/fishpco.r')
\end{block}
\end{frame}


{
\usebackgroundtemplate{\includegraphics[width=\paperwidth]{mo_crayfish_pco.pdf}
}
\begin{frame}[plain]
\end{frame}
}

\begin{frame}[plain]
	\centering
	\includegraphics[height=1\textheight]{mo_watersheds_crayfishes}
\end{frame}


\begin{frame}[plain]{Next, the principal coordinates analysis for fishes.}

Study the two dimensional plot for patterns.
\begin{block}{Type}
	source(`http://mtaylor4.semo.edu/~goby/biogeo/fishpco.r')
\end{block}
\end{frame}


{
\usebackgroundtemplate{\includegraphics[width=\paperwidth]{mo_fish_pco.pdf}
}
\begin{frame}[plain]
\end{frame}
}

\begin{frame}[plain]
	\centering
	\includegraphics[height=1\textheight]{mo_watersheds_fishes}
\end{frame}


\begin{frame}[plain]{Perform the Procrustes analysis.}


\begin{block}{Type}
	source(`http://mtaylor4.semo.edu/~goby/biogeo/procrustes.r')
	\pause
	protest(fish.sc, crayfish.sc, permutations = 9999)
\end{block}
\end{frame}

{
\usebackgroundtemplate{\includegraphics[width=\paperwidth]{mo_procrustes.pdf}
}
\begin{frame}[plain]
\end{frame}
}


\end{document}
