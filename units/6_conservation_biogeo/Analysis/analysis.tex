%!TEX TS-program = lualatex
%!TEX encoding = UTF-8 Unicode

\documentclass[t]{beamer}

%%%% HANDOUTS For online Uncomment the following four lines for handout
%\documentclass[t,handout]{beamer}  %Use this for handouts.
%\usepackage{handoutWithNotes}
%\pgfpagesuselayout{3 on 1 with notes}[letterpaper,border shrink=5mm]
%	\setbeamercolor{background canvas}{bg=black!5}


%%% Including only some slides for students.
%%% Uncomment the following line. For the slides,
%%% use the labels shown below the command.
%\includeonlylecture{student}

%% For students, use \lecture{student}{student}
%% For mine, use \lecture{instructor}{instructor}


%\usepackage{pgf,pgfpages}
%\pgfpagesuselayout{4 on 1}[letterpaper,border shrink=5mm]

% FONTS
\usepackage{fontspec}
\def\mainfont{Linux Biolinum O}
\setmainfont[Ligatures={Common,TeX}, Contextuals={NoAlternate}, BoldFont={* Bold}, ItalicFont={* Italic}, Numbers={Proportional}]{\mainfont}
\setmonofont[Scale=MatchLowercase]{Inconsolatazi4} 
\setsansfont[Scale=MatchLowercase]{Linux Biolinum O} 
\usepackage{microtype}

\usepackage{graphicx}
	\graphicspath{%
	{/Users/goby/Pictures/teach/438/lectures/}%
	{/Users/goby/Pictures/teach/common/}%}%
	{img/}} % set of paths to search for images

\usepackage{amsmath,amssymb}

%\usepackage{units}

\usepackage{booktabs}
\usepackage{multicol}
%	\setlength{\columnsep=1em}

%\usepackage{textcomp}
%\usepackage{setspace}
\usepackage{tikz}
	\tikzstyle{every picture}+=[remember picture,overlay]

\mode<presentation>
{
  \usetheme{Lecture}
  \setbeamercovered{invisible}
  \setbeamertemplate{items}[square]
}

\usepackage{calc}
\usepackage{hyperref}

\newcommand\HiddenWord[1]{%
	\alt<handout>{\rule{\widthof{#1}}{\fboxrule}}{#1}%
}
\begin{document}

\begin{frame}[plain]{Mac users need to install XQuartz.}

	\hangpara Otherwise, you'll miss out on the groovy 3-D plots.

\begin{block}{Download XQuartz 2.7.8}
	\includegraphics[height=1cm]{xlogo} \quad http://www.xquartz.org 
\end{block}
\end{frame}





\begin{frame}[plain]{Do the preliminary set-up for your analysess.}

	\hangpara Load / install R packages needed, and set default colors for plots.

\begin{block}{Type}
	source(`http://mtaylor4.semo.edu/~goby/biogeo/color.r') 
\end{block}
\end{frame}



\begin{frame}[plain]{Set up Georgia crayfishes and fishes for analyses.}

	\hangpara Read in the presence / absence data.
	
	\hangpara Calculate similarity of watersheds based on shared species.

\begin{block}{Type}
	source(`http://mtaylor4.semo.edu/~goby/biogeo/gacrayfish.r') \pause
	source(`http://mtaylor4.semo.edu/~goby/biogeo/gafish.r')
\end{block}
\end{frame}


\begin{frame}[plain]{Perform a cluster analysis.}

\hangpara Watersheds will be clustered together based on similarity. 

\begin{block}{Type}
	source(`http://mtaylor4.semo.edu/~goby/biogeo/gacrayfishcluster.r')
\end{block}

\end{frame}

{
\usebackgroundtemplate{\includegraphics[width=\paperwidth]{ga_crayfish_cluster}
}
\begin{frame}[plain]
\end{frame}
}

\begin{frame}[plain]{Perform a principal coordinates analysis.}

\hangpara Ordination procedure that reduces the number of dimensions while maintaining Euclidean distances. Plot the first two dimensions.

\begin{block}{Type}
	source(`http://mtaylor4.semo.edu/~goby/biogeo/crayfishpco.r')
\end{block}


\end{frame}

{
\usebackgroundtemplate{\includegraphics[width=\paperwidth]{ga_crayfish_pco.pdf}
}
\begin{frame}[plain]
\end{frame}
}

\begin{frame}[plain]
	\centering
	\includegraphics[height=1\textheight]{georgia_watersheds_crayfishes}
\end{frame}


\begin{frame}[plain]{Now, plot 3 dimensions at the same time.}

\begin{block}{Type}
	source(`http://mtaylor4.semo.edu/~goby/biogeo/crayfish3d.r')
\end{block}

	\hangpara Expand the window to full size. 
	
	\hangpara Right-click to zoom in and out. Left-click to rotate.

\end{frame}

\begin{frame}[plain]{Now, the fishes. Perform a cluster analysis.}

\begin{block}{Type}
	source(`http://mtaylor4.semo.edu/~goby/biogeo/gafishcluster.r')
\end{block}
\end{frame}

{
\usebackgroundtemplate{\includegraphics[width=\paperwidth]{ga_fish_cluster.pdf}
}
\begin{frame}[plain]
\end{frame}
}


\begin{frame}[plain]{Next, the principal coordinates analysis for fishes.}

\hangpara Study the two dimensional plot for patterns.

\begin{block}{Type}
	source(`http://mtaylor4.semo.edu/~goby/biogeo/fishpco.r')
\end{block}
\end{frame}


{
\usebackgroundtemplate{\includegraphics[width=\paperwidth]{ga_fish_pco.pdf}
}
\begin{frame}[plain]
\end{frame}
}

\begin{frame}[plain]
	\centering
	\includegraphics[height=1\textheight]{georgia_watersheds_fishes}
\end{frame}


\begin{frame}[plain]{Next, the principal coordinates analysis for fishes.}

	\hangpara Now, 3-D.

\begin{block}{Type}
	source(`http://mtaylor4.semo.edu/~goby/biogeo/fish3d.r')
\end{block}
\end{frame}


{
\usebackgroundtemplate{\includegraphics[width=\paperwidth]{procrustes}
}
\begin{frame}[plain]
\end{frame}
}

\begin{frame}[plain]{Do watersheds show similar distributions in ordinated space when fishes and crayfishes are compared?}

	\hangpara Perform a Procrustes analysis.

\begin{block}{Type}
	source(`http://mtaylor4.semo.edu/~goby/biogeo/procrustes.r')
\end{block}
\end{frame}


{
\usebackgroundtemplate{\includegraphics[width=\paperwidth]{ga_procrustes.pdf}
}
\begin{frame}[plain]
\end{frame}
}

\begin{frame}[plain]{Do watersheds show similar distributions in ordinated space when compared between fishes and crayfishes?}

	\hangpara Are the results significant?  If so, then the alignment of the two PCO analyses in ordination space is not random. The two are similar.

\begin{block}{Type}
	protest(fish.sc, crayfish.sc, permutations = 9999)
\end{block}
\end{frame}

\begin{frame}[plain]{Now, let's do Missouri crayfishes, then fishes.}

	\hangpara Read in the presence / absence data.
	
	\hangpara Calculate similarity of watersheds based on shared species.

\begin{block}{Type}
	source(`http://mtaylor4.semo.edu/~goby/biogeo/mocrayfish.r') \pause
	source(`http://mtaylor4.semo.edu/~goby/biogeo/mofish.r')
\end{block}
\end{frame}

\begin{frame}[plain]{Perform the cluster analyses for both taxa.}

	\hangpara Watersheds will be clustered together based on similarity. 

\begin{block}{Type}
	source(`http://mtaylor4.semo.edu/~goby/biogeo/mocrayfishcluster.r')
	\pause
	source(`http://mtaylor4.semo.edu/~goby/biogeo/mofishcluster.r')
\end{block}
\end{frame}


{
\usebackgroundtemplate{\includegraphics[width=\paperwidth]{mo_crayfish_cluster.pdf}
}
\begin{frame}[plain]
\end{frame}
}

{
\usebackgroundtemplate{\includegraphics[width=\paperwidth]{mo_fish_cluster.pdf}
}
\begin{frame}[plain]
\end{frame}
}

\begin{frame}[plain]{Next, the principal coordinates analysis for crayfishes.}

	\hangpara Study the two dimensional plot for patterns.

\begin{block}{Type}
	source(`http://mtaylor4.semo.edu/~goby/biogeo/fishpco.r')
\end{block}
\end{frame}


{
\usebackgroundtemplate{\includegraphics[width=\paperwidth]{mo_crayfish_pco.pdf}
}
\begin{frame}[plain]
\end{frame}
}

\begin{frame}[plain]
	\centering
	\includegraphics[height=1\textheight]{mo_watersheds_crayfishes}
\end{frame}


\begin{frame}[plain]{Next, the principal coordinates analysis for fishes.}

	\hangpara Study the two dimensional plot for patterns.

\begin{block}{Type}
	source(`http://mtaylor4.semo.edu/~goby/biogeo/fishpco.r')
\end{block}
\end{frame}


{
\usebackgroundtemplate{\includegraphics[width=\paperwidth]{mo_fish_pco.pdf}
}
\begin{frame}[plain]
\end{frame}
}

\begin{frame}[plain]
	\centering
	\includegraphics[height=1\textheight]{mo_watersheds_fishes}
\end{frame}


\begin{frame}[plain]{Perform the Procrustes analysis.}


\begin{block}{Type}
	source(`http://mtaylor4.semo.edu/~goby/biogeo/procrustes.r')
	\pause

	protest(fish.sc, crayfish.sc, permutations = 9999)
\end{block}
\end{frame}

{
\usebackgroundtemplate{\includegraphics[width=\paperwidth]{mo_procrustes.pdf}
}
\begin{frame}[plain]
\end{frame}
}


\end{document}
