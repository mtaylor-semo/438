\documentclass[letterpaper]{tufte-handout}

%\geometry{showframe} % display margins for debugging page layout

\usepackage{graphicx} % allow embedded images
  \setkeys{Gin}{width=\linewidth,totalheight=\textheight,keepaspectratio}
  \graphicspath{{img/}} % set of paths to search for images
\usepackage{amsmath}  % extended mathematics
\usepackage{booktabs} % book-quality tables
\usepackage{units}    % non-stacked fractions and better unit spacing
%\usepackage{multicol} % multiple column layout facilities
\usepackage{microtype}   % filler text
%\usepackage{fancyvrb} % extended verbatim environments
%  \fvset{fontsize=\normalsize}% default font size for fancy-verbatim environments

% Standardize command font styles and environments
\newcommand{\doccmd}[1]{\texttt{\textbackslash#1}}% command name -- adds backslash automatically
\newcommand{\docopt}[1]{\ensuremath{\langle}\textrm{\textit{#1}}\ensuremath{\rangle}}% optional command argument
\newcommand{\docarg}[1]{\textrm{\textit{#1}}}% (required) command argument
\newcommand{\docenv}[1]{\textsf{#1}}% environment name
\newcommand{\docpkg}[1]{\texttt{#1}}% package name
\newcommand{\doccls}[1]{\texttt{#1}}% document class name
\newcommand{\docclsopt}[1]{\texttt{#1}}% document class option name
\newenvironment{docspec}{\begin{quote}\noindent}{\end{quote}}% command specification environment


\title{Conservation Biogeography}

\author[Biogeography]{Biogeography}

\date{Fall 2013} % without \date command, current date is supplied

\begin{document}

\maketitle	% this prints the handout title, author, and date


%\printclassoptions

\newthought{Biogeogeography has made} many important contributions to conservation biology.\sidenote{Skim pages 697-740 for information of interest to you, especially if your career goals are conservation oriented. The information you are required to know is in this document.} Global climate change and other human impacts will have an enormous, yet difficult to predict, effect on the biogeographic distribution of diversity.  Humans have an an enormously detrimental impact on diversity, with a significant increase in extinctions due to human activities. Some studies have estimated that the current extinction rate is anywhere from 100-1000X higher than the background extinction rate over evolutionary time scales.

These issues have led to an increased demand for the conservation of biological diversity yet many conservation measures are often been reduced to basic protection.  For example, a conservation biologist may be tasked with protecting the many small patches where an endangered animal is found. However, little thought is given to the ability of the animal to disperse between patches, or the intervening habitat to facilitate dispersal. Increasingly, conservation managers are recognizing the importance of basic biogeographic principles and analyses to help design and implement conservation management plans.


\section{Biodiversity Hotspots}
\begin{marginfigure}%
	\includegraphics{global_hotspots}
\end{marginfigure} 

Diversity is not even, but is clumped.  Some regions show extraordinarily high diversity while adjacent regions may have diversity more typical for the region. These regions of extra high diversity are known as biodiversity hotspots.  Hotspots are areas with extremely high local endemism.  Because these  endemic species are found nowhere else, they tend to be very rare species, and are often threatened or endangered.  Thus, biodiversity hotspots are often targets of conservation management.  Conserving diversity hotspots gives you good bang for the buck by protecting the greatest number of species in a relatively small area. As a conservation biologist, for example, you may have a goal of identifying global hotspots that may warrant protection.

How might you go about identifying a hotspot?  Faunal and floral surveys can reveal regions with the greatest co-occurrence of endemic taxa.  One or a few endemic taxa may not be sufficient to identify an area as a hotspot but areas with high endemism for many different taxa are hotspot candidates.  In addition to high endemism, hotspots also tend to be areas where the threat of extinction is high due to loss of habitat through human impact. 

\begin{marginfigure}%
	\includegraphics{africa_hotspots}
\end{marginfigure} 

One example is shown for Africa.  The darkly shaded regions have have high endemism for birds, amphibians, mammals and plants.   Subsaharan Africa shows distinct regions of endemism for these four taxonomic groups.  When all are overlaid, you can readily identify the areas with the highest endemicity for the greatest number of taxa, which can again be the focus of concentrated conservation efforts.

\section{Marine Endemism and Hotspots}
\begin{marginfigure}%
	\includegraphics{marine_endemism}
	\includegraphics{marine_hotspots}
Marine Hotspots
\end{marginfigure} 

We see similar hotspots in the marine realm, again concentrated in the tropics, and especially in the area centered around Indonesia. These panels show the relative species richness for three major marine taxa, the fishes, corals, and snails.  The darker the square, the higher the richness.  Although this indicates richness, many of the species are are endemic to small regions within the area. When the endemism for these taxa, and a few others, are overlaid, we identify the hotspots, shown by the darkest squares. Again, we see the area between the Philippines and New Guinea showing the greatest diversity and endemism, and thus perhaps in need of the greatest conservation effort. This becomes especially apparent when the human mediated threats such as over harvesting are also mapped.  The areas of highest threat from human activities have nearly 100\% overlap with the hotspots.

This is not a good combination for the protection of biological diversity.  Many of these areas are extremely overpopulated and impoverished. Indeed, we see that many of the diversity hotspots, marine and terrestrial, are in areas of poverty, overcrowding, or both. The people in these areas are generally more concerned with feeding the family and themselves than concerning themselves whether some obscure little bird or plant or fish or other animal goes extinct. It is hard to blame somebody for not caring about biological conservation when they are trying only to survive. However, even beyond impoverished areas, the rate of human-induced extinction, and consequent need for the conservation of biodiversity is greater than ever. 

\section{Extinction-Prone Species}
\begin{marginfigure}%
	\centering%
	\includegraphics{philippine_eagle.jpg}\\
	Philippine Eagle
\end{marginfigure} 

When considering biodiversity hotspots and the threat of human activity, one question that must arise is which species are most vulnerable to extinction?  Some species are more prone to extinction than others, even in natural conditions.  Read the following list and ask yourself which traits seem to be shared between species that are threatened or endangered.  

\begin{itemize}
	\item Body size,
	\item trophic position (low or high trophic levels),
	\item generalist or specialist,
	\item generation time,
	\item dispersal ability,
	\item population size,
	\item and geographic range size.
\end{itemize}

While you can think about each item in the list above individually, you should be able to recognize how they relate to each other to explain why some species are more vulnerable to extinction.  Extinction, according to the MacArthur-Wilson model of island biogeography, is related to island area.  However, regular immigration can offset extinction (recall both target and rescue effects). 

\section{Conservation Biogeography}
\begin{marginfigure}%
	\includegraphics{insular_distribution_function_top}
\end{marginfigure} 

Within the past several decades, researchers have realized that they can use island biogeographic principles to protect species that are vulnerable to extinction. This realization led to the development of conservation biogeography as discipline, which is built around two simple but essential concepts.  First, if we as a people are going to adequately and properly conserve biological diversity and not just individual species, we need to understand how organisms are distributed in nature and the processes that explain this distribution. This is much of what we have talked about this semester.  We need to know why taxa occur where they occur, and then use that information to develop plans that maintain the greatest diversity. For example, a hot topic right now is trying to predict biome shifts due to global warming.  We can use information on past biome shifts during and following the Pleistocene.  We can look at the rates of dispersal in plants and animals, and try to predict the impact given various rates of temperature increase.

The second concept is how we think about diversity. While a goal might be to protect a particular species, we can't consider that species alone.  Species are not isolated entities but instead have evolved as part of a complex ecological community.  Survival depends on all of those interactions in the community.  Changing even one small interaction may greatly influence community dynamics across many trophic levels. You also have to consider the habitat patch size occupied by the community.  Small patches of suitable habitat may not be adequate to sustain the community.  What is an adequate patch size?  If you recall, you can use the insular distribution function to determine the necessary sizes of ``insular habitat'' necessary to maintain the populations that make up the community. The insular distribution function shows the combinations of patch (island) size and isolation necessary to sustain a population.  We can then expand to account for the ability of the taxa to disperse among the isolated patches.

\section{Protecting the Florida Panther}
\begin{marginfigure}%
	\includegraphics{florida_corridor2.jpg}
\end{marginfigure} 

In class, we worked through some ideas on how to apply island biogeography ideas to protect the Florida panther, \textit{Puma concolor}.\sidenote{Recent studies suggest the Florida panther is same same subspecies as the mountain lion / cougar.  However, the Florida population is so geographically isolated that many researchers still treat them as a separate subspecies to highlight the need for protection} Several important concepts emerged.  First, remember that patch area determines population sustainability.  Increasing the size of suitable habitat patches provides more resources and so can sustain a larger population.  Second, conservation ``corridors''\sidenote{Read page 194 for more information about corridors.} can increase the rate of migration among patches.  The corridors must still conform to basic island biogeographic principles.  The corridors must contain suitable habitat and be sufficiently large to provide resources needed during migration.  In essence, the corridors connecting habitat patches increase the overall patch size, and therefore reducing the possibility of extinction.

\section{Predictive Biogeography}
Another aspect of biogeographic analysis and conservation biogeography is to use Geographic Information Systems, or GIS, to quantify the types of habitats needed by species of concern.  If a species is present, then you can measure as many environmental parameters as reasonably possible, such as annual precipitation, seasonal temperature variation, soil types, and prevailing wind direction. The collected data then goes into a database, which can be queried by GIS software to identify other areas with similar habitat parameters, which can be surveyed for the presence of the species of concern. Alternatively, these may be areas where the species of interest can be introduced and successfully established.  Predictive biogeography has recently blossomed into a powerful conservation tool.  
\begin{marginfigure}%
	\includegraphics{mesophotic_results.jpg}
\end{marginfigure} 
In class, we walked through an example of applying predictive biogeography to establish the habitat conditions most suitable for species of mesophilic corals.

\end{document}