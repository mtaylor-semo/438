%!TEX TS-program = lualatex
%!TEX encoding = UTF-8 Unicode

\documentclass[11pt, addpoints]{exam}

%\printanswers

\usepackage{graphicx}
	\graphicspath{{/Users/goby/Pictures/teach/300/exercises/}} % set of paths to search for images

\usepackage{geometry}
\geometry{letterpaper, bottom=1in}                   
%\geometry{landscape}                % Activate for for rotated page geometry
\usepackage[parfill]{parskip}    % Activate to begin paragraphs with an empty line rather than an indent
\usepackage{amssymb, amsmath}
\usepackage{mathtools}
	\everymath{\displaystyle}

\usepackage{fontspec}
\setmainfont[Ligatures={TeX}, BoldFont={* Bold}, ItalicFont={* Italic}, BoldItalicFont={* BoldItalic}, Numbers={Proportional}]{Linux Libertine O}
\setsansfont[Scale=MatchLowercase,Ligatures=TeX]{Linux Biolinum O}
\setmonofont[Scale=MatchLowercase]{Inconsolatazi4}
\usepackage{microtype}

\usepackage{unicode-math}
\setmathfont[Scale=MatchLowercase]{Asana Math}
%\setmathfont[Scale=MatchLowercase]{XITS Math}

% To define fonts for particular uses within a document. For example, 
% This sets the Libertine font to use tabular number format for tables.
\newfontfamily{\tablenumbers}[Numbers={Monospaced}]{Linux Libertine O}
\newfontfamily{\libertinedisplay}{Linux Libertine Display O}
 
\usepackage{booktabs}
%\usepackage{tabularx}
\usepackage{longtable}
%\usepackage{siunitx}
\usepackage{array}
\newcolumntype{L}[1]{>{\raggedright\let\newline\\\arraybackslash\hspace{0pt}}p{#1}}
\newcolumntype{C}[1]{>{\centering\let\newline\\\arraybackslash\hspace{0pt}}p{#1}}
\newcolumntype{R}[1]{>{\raggedleft\let\newline\\\arraybackslash\hspace{0pt}}p{#1}}

\usepackage{enumitem}
\setlist{leftmargin=*}
\setlist[1]{labelindent=\parindent}
\setlist[enumerate]{label=\textsc{\alph*}.}
\setlist[itemize]{label=\color{gray}\textbullet}

\usepackage{hyperref}
%\usepackage{placeins} %PRovides \FloatBarrier to flush all floats before a certain point.
%\usepackage{hanging}

\usepackage[sc]{titlesec}

\renewcommand{\solutiontitle}{\noindent}
\unframedsolutions
\SolutionEmphasis{\bfseries}

\pagestyle{headandfoot}
\firstpageheader{BI 438: Biogeography}{}{\ifprintanswers\textbf{KEY}\else Name: \enspace \makebox[2.5in]{\hrulefill}\fi}
\runningheader{}{}{\footnotesize{pg. \thepage}}
\footer{}{}{}
\runningheadrule

\unframedsolutions
\renewcommand{\solutiontitle}{}
\SolutionEmphasis{\bfseries}

\newcommand*\AnswerBox[3]{%
	\parbox[t][#1]{0.92\textwidth}{%
		\begin{solution}#3\end{solution}}
	\vspace*{\stretch{#2}}
}

\newenvironment{AnswerPage}[2]
{\begin{minipage}[t][#1]{0.92\textwidth}%
		\begin{solution}}
		{\end{solution}\end{minipage}
	\vspace*{\stretch{1}}}

\newlength{\basespace}
\setlength{\basespace}{4\baselineskip}

\begin{document}

The goal of this exercise is for you to discover how Pleistocene
glaciation and the fall line shaped the distribution of freshwater
fishes in the United State.

\textbf{Important!} This exercise is different than the other online exercises you have completed. You will explore the data online but
the questions are listed below. Type your answers in to a Word document
and upload it to the drop box. You will not enter your answers online nor will you download a finished report.

The exercise online is divided into two parts. Only one part will be 
visible at a time to save space in the tab bar but all parts will remain 
accessible. When you start Part 2, the tabs for Part 1 will be hidden. 
You can press the “Prev” buttons to view the Part 1 tabs again. If you 
go back to Part 1, then Part 2 tabs will be hidden. Just press the “Next”
buttons to view them again.

Your goal for Part 1 is to explore the distribution of species richness 
of freshwater fishes relative to the Pleistocene glacial 
maximum and the fall line (see image online). The fall line is the
boundary that separates the coastal plain from uplifted interior highlands (central and eastern highlands). 

Your goal for Part 2 is to explore river watersheds that cross the fall 
line to learn whether the distribution of freshwater fishes is affected by this important biogeographic feature.

\subsection*{Part 1: Interior Highlands}

\begin{enumerate}
	\item Open a web browser on your computer and go to \url{https://semobio.shinyapps.io/provinces}. You will also find a “Glacial Maxima and the Fall Line” link on our Canvas page that you can click instead.
	
	\item Read the Introduction page, then click the Next button to begin 
	Part 1. Study the information carefully to learn more about what you
	will see as you explore the data. 
\end{enumerate}

\subsubsection*{Part 1 predictions}

\begin{questions}

\question
\textbf{What do you predict?} Where in the U.S. will species richness be highest? Will it be north of the glacial maximum? Will richness be highest in the interior highlands? The eastern or central highlands? Will richness be highest on the coastal plain? Explain why you think so. (Enter your answers in a separate document. You do not need to type the question. Just enter the question number and your answer.)


\fullwidth{

\subsubsection*{North America}

\begin{enumerate}[resume]
\item Press the Next button to view a relief map of species richness for
freshwater fishes. Dark areas indicate low species richness. Brighter
areas indicate higher species richness. This is a large data set so the
map may take a few moments to plot.
\end{enumerate}

}

\question
\textbf{Do the results agree with your predictions?} Describe with 
reasonable detail where species richness is highest. Further, describe 
the distribution of species richness relative to the interior highlands,
the coastal plain, and north of the glacial maxima.


\fullwidth{

\subsubsection*{Family Richness}
\begin{enumerate}[resume]
\item Press the Next button to view species richness for selected seven
families of freshwater fishes. Use the drop down menu to view the 
distribution for each family.
\end{enumerate}
}

\question
List the families that have the highest richness in the interior highlands (generally the eastern highlands). This list can include families that have high richness outside of the interior
highlands. \textit{Remember that the brightest color indicates the 
highest richness.}

\question
List the one family that has very low richness outside of the eastern
highlands. This family has very high diversity in a comparibly small 
region. Write a hypothesis that might explain why diversity of this 
family is so high in a relatively small geographic area. (Hint: do you
think the rivers and streams in this unglaciated area have always
had the same configuration they have now?)

\question
List the two families that have high richness outside of the highlands? Based on the family information, why do you think these families might
be so widely distributed.


\question
Which family has the greatest richness on the coastal plain? 
Read the description for this family. From this information, why do you
think the richness of this family is coastal.


\question
List the two families have the greatest richness in the Great Lakes or 
northwestern U.S. Read the descriptions for these two families. What do 
they have in common? How does that fit with the richness shown on their 
distribution maps?


\fullwidth{
\subsection*{Part 2: Fall Line}

\begin{enumerate}[resume]
\item Press the Next button to begin Part 2. The tabs for Part 1 will be 
hidden but you can press the Prev buttons to view them again.

\item Read the information carefully. The Fall Line separates two 
major biogeographic regions: the coastal plain and the interior 
highlands. 

The coastal plain is very flat so the streams and rivers tend to be sluggish, warm, and more turbid (more sediment in the water). The stream bottoms tend to be sandy or muddy. The interior highlands are mountainous regions so the streams and rivers tend to have faster moving, cooler water, with gravel bottoms in the faster stretches.

Many southeastern rivers start in the interior highlands above the fall line but flow across the fall line to the coastal plain before reaching the ocean.

You will use both cluster and \textsc{nmds} analyses for this part.
Review the exercise on Montana (you kept a copy of your report, right?)
to recall how to interpret cluster and \textsc{nmds} plots. Remember
that cluster and \textsc{nmds} analyses are based on species similarity;
that is the species that are shared among the watersheds.

\end{enumerate}
}

\question
\textbf{What do you predict?} Do you think the fishes found in the watersheds \emph{above} the fall line will be more similar to the fishes in the \emph{same river below} the fall line? Or, will they be more similar to the fishes in \emph{nearby rivers above} the fall line?
Explain your reasoning.

\fullwidth{
\subsubsection*{Cluster and nmds}

For the cluster and \textsc{nmds} tabs, you will see a map of each state
showing the smaller watersheds and the fall line. The fall line is an important biogeographic feature. Also important is larger watersheds where the smaller watersheds terminate. Some rivers flow east to 
the Atlantic Ocean (Atlantic Slope). Some flow south to the Gulf of
Mexico. All the other smaller watersheds flow west to the Mississippi 
River, the Tennessee River, or the Ohio River. In the state of Missouri, 
the smaller watersheds flow to the Missouri River, the White River, and 
the Arkansas River. Each state map lists the smaller watersheds grouped 
by the larger watershed. Each tab also includes a regional reference map 
of these larger watersheds to help you stay oriented.

(I hope the above becomes apparent you view the different states.)

\begin{enumerate}[resume]
\item Press the Next button to view a cluster analysis of Virginia watersheds. Notice that the Atlantic Slope rivers are the Potamac,
Rappahannock, York, James, Chowan, and Roanoake rivers. The Clinch/Powell and Holston are part of the Tennessee River watershed. The
New and Big Sandy flow into the Ohio River watershed. 

\item Press the Next button to view the \textsc{nmds} plot for Virginia.
Notice the cluster plot is below the \textsc{nmds} plot. \emph{You will 
not have to return to the cluster tab at any point during the rest of
this exercise} (it's too late for me to reprogram now).

\item Compare the arrangement of points in the \textsc{nmds} plot to 
the cluster plot. Recall that points closer together are more similar.

\item  View the \textsc{nmds} and clusters plot for each state in this 
order: Virgina, North Carolina, South Carolina, Georgia, Alabama, 
Mississippi, Missouri. You will follow the fall line from the northeast 
to the south, and then up to Missouri. 

\item As you view each state, look at the plots with reference to the
fall line \emph{and} the major watersheds. The questions below are
presented in the recommended order of viewing.
\end{enumerate}
}

\question

\textbf{Virginia:} I'll guide you for the Virginia plots so you have
an idea how to approach subsequent questions. \emph{You do not have to 
type an answer for this question. Use my questions below to help guide 
your thinking for the other questions.} Virginia has three major 
watersheds. Does the cluster plot suggest differences in species 
similarity among the major watersheds?  Look at the first split of the 
cluster plot. The upper cluster contains the Holston, the Clinch-Powell, 
the New, and the Big Sandy. Which watersheds are this rivers a part of? 
Where do all the other rivers in Virginia end up? 

This initial split is consistent with the rivers flowing west to the
Tennessee or Ohio versus rivers of the Atlantic Slope. Compare the 
cluster plot to the \textsc{nmds} plot. Notice the western rivers are 
quite distinct from the rivers that flow to the Atlantic (x-axis). 
Although the Ohio and Tennessee rivers are in the same cluster, notice 
that they are  separated along the y-axis of the \textsc{nmds} plot.

Finally, notice that the stretches of rivers above the fall line 
tend to cluster together, such as the Upper Chowan, Roanoke, and Upper 
James, This is repeated for the upper Potomac and Rappahannock
rivers. The lower reaches of the rivers below the fall line 
also tend to cluster together. Does this relationship hold when viewed
in the context of \textsc{nmds?}

\question
\textbf{North Carolina:} Interpret the cluster and \textsc{nmds} plots
as we did above and summarize your interpretation of the results. You can
safely disregard the Watauga, Savannah, and New rivers. They barely enter
in to the state so their results are not reliable. Also and 
unfortunately, the distribution of fishes above and below the fall line
were not available for North Carolina so you do not have to address
the fall line for this state. However, notice that the Lumber,
White Oak, and Chowan-Paquatonk cluster together and separate slightly
from the other Atlantic Slope rivers. This may be that these rivers 
barely extend above the fall line. The other Atlantic Slope rivers 
extend much farther west above the fall line.

\question
\textbf{South Carolina:} All rivers are Atlantic Slope so you only need
to compare rivers relative to the fall line. Unfortunately most of the
rivers overlap on the \textsc{nmds} plot but you should be able to 
determine which rivers are plotted at that point and why.

\question
\textbf{Georgia:} This state has three watersheds. Mobile Bay is part of 
the Gulf of Mexico but is highlighted to point out that the Coosa and
Tallapoosa rivers are distinct from other Georgia rivers that flow into 
the Gulf. The results here are very good for the watersheds and the fall 
line. Write a thorough and clear summary of the results.

\question
\textbf{Alabama:} Another data set with clear results so write another
thorough and clear summary.

\question
\textbf{Mississippi:} This state lies entirely below the fall line so
your summary here should focus on the watersheds.

\question
\textbf{Missouri:} In many ways, Missouri has the most complex pattern
although the results are clear.  A few notes: The Arkansas River 
watershed begins in southwest Missouri, flows west and south through 
Kansas and Oklahoma before cutting back through central Arkansas to meet 
the Mississippi River. The White River watershed flows east across 
northern Arkansas to meet the Mississippi River.  Of interest is the
Salt, Chariton, and Grand rivers. The Salt flows east directly to the 
Mississippi River but the Chariton and Grand both flow south to the 
Missouri River. Write a hypothesis why you think these three rivers might
be similar. (Hint: remember that all of Missouri north of the Missouri
River is north of the Pleistocene glacial maximum.)

\end{questions}

\end{document}  