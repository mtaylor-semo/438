\documentclass[xcolor=svgnames]{beamer}
%\documentclass[xcolor=dvipsnames]{beamer}
%\usetheme{default}
\usepackage{pgf,pgfpages}
\usepackage{graphicx}
\usepackage{units}
\usepackage[utf8]{inputenc}
\usepackage{tikz}
\usetheme{Boadilla}


%% Uncomment the next two lines to make slides with note lines
%\usepackage{handoutWithNotes}
%\pgfpagesuselayout{3 on 1 with notes}[letterpaper,border shrink=5mm]


\definecolor{myBlue}{RGB}{0, 0, 100}

%\mode<handout> 	% Use this to make PDF with all overlays on one slide.
\mode<presentation>

\usefonttheme[onlymath]{serif}
\usecolortheme[named=myBlue]{structure}

%\usecolortheme{seahorse}

\makeatother
\setbeamertemplate{footline}
{
  \leavevmode%
  \hbox{%
  \begin{beamercolorbox}[wd=.5\paperwidth,ht=2.25ex,dp=1ex,center]{section in head/foot}%
    \usebeamerfont{section in head/foot}\insertsection
  \end{beamercolorbox}%
  \begin{beamercolorbox}[wd=.5\paperwidth,ht=2.25ex,dp=1ex,center]{subsection in head/foot}%
    \usebeamerfont{subsection in head/foot}\insertsubsection\hspace*{3em}
%    \insertframenumber{} / \inserttotalframenumber\hspace*{1ex}
  \end{beamercolorbox}}%
  \vskip0pt%
}
\makeatletter\setbeamertemplate{navigation symbols}{} 

\begin{document}

\section{Pleistocene}
\subsection{Biogeography}

\begin{frame}{The Pleistocene}
	\begin{center}
		\begin{block}{}
			\includegraphics[width=1.0\textwidth]{img/iceage}
		\end{block}
		\begin{block}{}
			The Pleistocene (2.6 MYA to 12 KYA) had a profound influence on the biogeography of modern taxa.
		\end{block}
	\end{center}
\end{frame}

\subsection{Background}
\begin{frame}{Extent of Glaciation}
	\begin{columns}[T]
		\begin{column}{0.45\textwidth}
			\includegraphics[width=1\textwidth]{img/north_hemi_overhead}
		\end{column}
		\begin{column}{0.45\textwidth}
			\begin{itemize}
				\item Four major advances in northern hemisphere
					\begin{itemize}
						\item Nebraska
						\item Kansan
						\item Illinois
						\item Wisconsin
					\end{itemize}
				\item Wisconsin was 85,000 -- 12,000 years bp
			\end{itemize}
		\end{column}
	\end{columns}
\end{frame}


\subsection{Glacier Formation}
\begin{frame}{Why do glaciers advance and retreat?}
	\begin{center}
		\includegraphics[width=1.0\textwidth]{img/wowl.jpg}
	\end{center}
\end{frame}

\begin{frame}{Milankovitch Cycles: Eccentricity}
	\begin{center}
		\includegraphics[width=1\textwidth]{img/eccentricity}
	\end{center}
	\begin{block}{}
		Earth's orbit deviates from less to more elliptical every 100,000 years.
	\end{block}
\end{frame}

\begin{frame}{Milankovitch Cycles: Obliquity}
	\begin{center}
		\includegraphics[width=1\textwidth]{img/obliquity}
	\end{center}
	\begin{block}{}
		Earth's tilt angle of Earth changes every 41,000 years.
	\end{block}
\end{frame}

\begin{frame}{Milankovitch Cycles: Precession}
	\begin{center}
		\includegraphics[width=0.7\textwidth]{img/precession}
	\end{center}
	\begin{block}{}
		Earth's axis ``wobbles'' like a spinning top every 22,000 years.
	\end{block}
\end{frame}

\begin{frame}{Milankovitch: Present}
	\begin{center}
		\includegraphics[width=1\textwidth]{img/milankopresent}
	\end{center}
\end{frame}

\begin{frame}{Milankovitch: Past}
	\begin{center}
		\includegraphics[width=1\textwidth]{img/milankopast}
	\end{center}
\end{frame}

\begin{frame}{Albedo Effect Enhances Glaciation}
	\begin{columns}[T]
		\begin{column}{0.5\textwidth}
			\begin{center}
				\includegraphics[width=1\textwidth]{img/albedo}
			\end{center}
		\end{column}
		\begin{column}{0.45\textwidth}
			\begin{itemize}
				\item	\textcolor{myBlue}{\textbf{Albedo}} is a measure of reflected sunlight. 
				\item Glaciers reflect much more sunlight back to the atmosphere. 
			\end{itemize}
		\end{column}
	\end{columns}
\end{frame}

\subsection{Environmental Change}

\begin{frame}{Environmental Change}
		\begin{center}
			\includegraphics[width=0.7\textwidth]{img/glacial_pirate_ship.jpg}
			\begin{block}{}
				What changes to the environment did Pleistocene glaciation cause? Think about terrain (topography), sea levels, and overall climate.
			\end{block}
		\end{center}
\end{frame}

\begin{frame}{Environmental Change}
	\includegraphics[width=1\textwidth]{img/glacial_land_coverage}
	\begin{block}{}
		About 1/3 of land covered by ice sheets. \pause 
		Where did the water come from?
	\end{block}
\end{frame}

\begin{frame}{Environmental Change}
	\includegraphics[width=1\textwidth]{img/pleisto_sealevels}
	\begin{block}{}
		New land exposed by sea level reduction.
	\end{block}
\end{frame}


{
\usebackgroundtemplate{\includegraphics[width=\paperwidth]{img/NorthAmerica.jpg}}
\setbeamercolor{frametitle}{fg=DarkOrange}
\begin{frame}{Ice Sheets Altered North American Topography}
\end{frame}
}
\begin{frame}{Terrestrial Change}
		\begin{center}
			\includegraphics[width=0.58\textwidth]{img/riverpatterns.jpg}
			\begin{block}{}
				North of the Missouri River, glaciated rivers run parallel.
			\end{block}
		\end{center}
		{\hfill\tiny{Image: Bruce Railsback, Dept. of Geology, Univ. Georgia}}
\end{frame}

\begin{frame}{Terrestrial Change}
		\begin{center}
			\includegraphics[width=0.7\textwidth]{img/coteau.jpg}
			\begin{block}{}
				Coteau des Prairies Plateau in the upper Midwest.
			\end{block}
		\end{center}
		{\hfill\tiny{Image: Bruce Railsback, Dept. of Geology, Univ. Georgia}}
\end{frame}

\begin{frame}{Pleistocene Lake Formation}
		\begin{center}
			\includegraphics[width=0.7\textwidth]{img/lake_agassiz.png}
			\begin{block}{}
				Meltwater from retreating ice sheets formed many large lakes, such as \\
				Lake Agassiz in Canada...
			\end{block}
		\end{center}
		{\hfill\tiny{Image: Bruce Railsback, Dept. of Geology, Univ. Georgia}}
\end{frame}

\begin{frame}{Pleistocene Lake Formation}
	\begin{columns}
		\begin{column}{0.5\textwidth}
			\begin{center}
				\includegraphics[height=7cm]{img/bonneville_lakes.png}
			\end{center}
		\end{column}
		\begin{column}{0.45\textwidth}
			\begin{block}{}
			...and Lake Bonneville, Lake Lahonta, and many others in the western U.S.
			\end{block}
		\end{column}
	\end{columns}
\end{frame}

\begin{frame}{Pleistocene Sea Level Drop}
		\begin{columns}[T]
			\begin{column}{0.45\textwidth}
				\includegraphics[width=1\textwidth]{img/caribbean}
			\end{column}
			\begin{column}{0.45\textwidth}
				\begin{itemize}
					\item Mean sea level ca. 130 km below present
					\item 52 million km$^3$ of sea water in ice
					\item \textcolor{myBlue}{\textbf{Eustatic}} and \textcolor{myBlue}{\textbf{Isostatic}} sea level changes
				\end{itemize}
			\end{column}
		\end{columns}
\end{frame}

\begin{frame}{Pleistocene Land Bridges}
		\begin{center}
			\includegraphics[width=1\textwidth]{img/beringia}
			\begin{block}{}
				Beringia was a land bridge between Palearctic and Nearctic.
			\end{block}
		\end{center}
\end{frame}

\begin{frame}{Pleistocene Land Bridges}
		\begin{center}
			\includegraphics[width=0.75\textwidth]{img/wallacia}
			\begin{block}{}
				Wallacia allowed dispersal to Indomalaysian and Australian islands.
			\end{block}
		\end{center}
\end{frame}

\subsection{Climate Change}

\begin{frame}{Climates Were Cooler}
		\begin{center}
			\includegraphics[width=0.9\textwidth]{img/pleistotemps}
			\begin{block}{}
				Mean Pleistocene temperatures were as much as $-$8$^{\circ}$C below current mean temps.
			\end{block}
		\end{center}
\end{frame}

\begin{frame}{Seasonal Temperature Swings Were Moderate}
		\begin{center}
			\includegraphics[width=0.9\textwidth]{img/adiabatic}
			\begin{block}{}
				Seasonal temperatures did not differ as much due to \textcolor{myBlue}{\textbf{adiabatic warming}} and altered climate patterns.
			\end{block}
		\end{center}
\end{frame}




\end{document}
