%!TEX TS-program = lualatex
%!TEX encoding = UTF-8 Unicode

\documentclass[t]{beamer}

%%%% HANDOUTS For online Uncomment the following four lines for handout
%\documentclass[t,handout]{beamer}  %Use this for handouts.
%\usepackage{handoutWithNotes}
%\pgfpagesuselayout{3 on 1 with notes}[letterpaper,border shrink=5mm]
%	\setbeamercolor{background canvas}{bg=black!5}


%%% Including only some slides for students.
%%% Uncomment the following line. For the slides,
%%% use the labels shown below the command.
%\includeonlylecture{student}

%% For students, use \lecture{student}{student}
%% For mine, use \lecture{instructor}{instructor}


%\usepackage{pgf,pgfpages}
%\pgfpagesuselayout{4 on 1}[letterpaper,border shrink=5mm]

% FONTS
\usepackage{fontspec}
\def\mainfont{Linux Biolinum O}
\setmainfont[Ligatures={Common,TeX}, Contextuals={NoAlternate}, BoldFont={* Bold}, ItalicFont={* Italic}, Numbers={Proportional}]{\mainfont}
\setmonofont[Scale=MatchLowercase]{Inconsolatazi4} 
\setsansfont[Scale=MatchLowercase]{Linux Biolinum O} 
\usepackage{microtype}

\usepackage{graphicx}
	\graphicspath{%
	{/Users/goby/Pictures/teach/438/lectures/}%
	{/Users/goby/Pictures/teach/common/}%}%
	{img/}} % set of paths to search for images

\usepackage{amsmath,amssymb}

%\usepackage{units}

\usepackage{booktabs}
\usepackage{multicol}
%	\setlength{\columnsep=1em}

%\usepackage{textcomp}
%\usepackage{setspace}
%\usepackage{tikz}
%	\tikzstyle{every picture}+=[remember picture,overlay]

\mode<presentation>
{
  \usetheme{Lecture}
  \setbeamercovered{invisible}
  \setbeamertemplate{items}[square]
}

\usepackage{calc}
\usepackage{hyperref}

\newcommand\HiddenWord[1]{%
	\alt<handout>{\rule{\widthof{#1}}{\fboxrule}}{#1}%
}



\begin{document}
%\lecture{instructor}{instructor}
\lecture{student}{student}

\begin{frame}{The \highlight{Pleistocene} (2.6 MYA to 12 KYA) had a profound influence on the biogeography of modern taxa.}
	\begin{center}
			\includegraphics[width=0.95\textwidth]{pleistocene_iceage}
			
	\end{center}
\end{frame}

\begin{frame}{Four major glacial advances occurred in the northern hemisphere.}
	\begin{columns}[T]
		\begin{column}{0.45\textwidth}
			\includegraphics[width=1\textwidth]{pleistocene_north_hemi_overhead}
		\end{column}
		\begin{column}{0.45\textwidth}
			\hangpara\highlight{Nebraska}
			
			\hangpara\highlight{Kansan}
			
			\hangpara\highlight{Illinois}
			
			\hangpara\highlight{Wisconsin:} 85,000–12,000 KYA.
			
		\end{column}
	\end{columns}
\end{frame}


\begin{frame}{Why do glaciers advance and retreat?}
	\begin{center}
		\includegraphics[width=1.0\textwidth]{pleistocene_wowl}
	\end{center}
\end{frame}

\begin{frame}{\highlight{Eccentricity:} Earth’s orbit deviates from less to more elliptical every 100,000 years.}
	\begin{center}
		\includegraphics[width=1\textwidth]{milankovitch_eccentricity}
	\end{center}
\end{frame}

\begin{frame}{\highlight{Obliquity:} Earth’s tilt angle changes every 41,000 years.}
	\begin{center}
		\includegraphics[width=1\textwidth]{milankovitch_obliquity}
	\end{center}
\end{frame}

\begin{frame}{\highlight{Precession:} Earth’s axis “wobbles” like a spinning top every 22,000 years.}
	\begin{center}
		\includegraphics[width=0.7\textwidth]{milankovitch_precession}
	\end{center}
\end{frame}

\begin{frame}{Milankovitch Cycle: Present}
	\begin{center}
		\includegraphics[width=1\textwidth]{milankopresent}
	\end{center}
\end{frame}

\begin{frame}{Milankovitch Cycle: 11,000 years ago.}
	\begin{center}
		\includegraphics[width=1\textwidth]{milankopast}
	\end{center}
\end{frame}

\begin{frame}{\highlight{Albedo} effect enhances glaciation by reflecting sunlight.}
	\begin{columns}[T]
		\begin{column}{0.5\textwidth}
			\begin{center}
				\includegraphics[width=1\textwidth]{pleistocene_albedo}
			\end{center}
		\end{column}
		\begin{column}{0.45\textwidth}
			\hangpara Albedo is a measure of reflected sunlight. 
			
			\hangpara Glaciers reflect sunlight back to the atmosphere. 
		\end{column}
	\end{columns}
\end{frame}

\begin{frame}{What changes to the environment did Pleistocene glaciation cause? }
		\vspace{-\baselineskip}
		\begin{center}
			\includegraphics[width=0.75\textwidth]{pleistocene_glacial_pirate_ship}
		\end{center}
		\vspace{-\baselineskip}
		\qquad Think about terrain (topography), sea levels, and overall climate.
\end{frame}

\begin{frame}{About 1/3 of land covered by ice sheets. Where did the water come from?}
	\includegraphics[width=1\textwidth]{pleistocene_glacial_land_coverage}
\end{frame}

\begin{frame}{New land was exposed by sea level reduction.}
	\includegraphics[width=1\textwidth]{pleistocene_sealevels}
\end{frame}


{
\usebackgroundtemplate{\includegraphics[width=\paperwidth]{NorthAmerica.jpg}}
\begin{frame}[t]{\textcolor{white}{Ice sheets altered North American topography.}}
\end{frame}
}
\begin{frame}{North of the Missouri River, glaciated rivers run parallel.}
	\vspace{-1\baselineskip}
	\begin{center}
		\includegraphics[width=0.9\textheight]{missouri_riverpatterns}
	\end{center}

	\vskip0pt plus 1fill

	\hfill\tiny{Image: Bruce Railsback, Dept. of Geology, Univ. Georgia}
\end{frame}

\begin{frame}{Coteau des Prairies Plateau in the upper Midwest was formed by glaciers.}
		\vspace{-1\baselineskip}
		\begin{center}
			\includegraphics[width=1\textheight]{missouri_coteau}
		\end{center}
	\vskip0pt plus 1fill
	\hfill\tiny{Image: Bruce Railsback, Dept. of Geology, Univ. Georgia}
\end{frame}

\begin{frame}[t]{Meltwater from retreating ice sheets formed many large lakes, such as Lake Agassiz in Canada\dots}
		\vspace{-1\baselineskip}
		\begin{center}
			\includegraphics[width=1\textheight]{pleistocene_lake_agassiz}
		\end{center}
	\vskip0pt plus 1fill
	
	\hfill\tiny{Image: Bruce Railsback, Dept. of Geology, Univ. Georgia}
\end{frame}

\begin{frame}{\dots and Lake Bonneville, Lake Lahonta, and many others in the western U.S.}
	\begin{center}
		\includegraphics[height=0.8\textheight]{pleistocene_bonneville_lakes}
	\end{center}
\end{frame}

\begin{frame}{Pleistocene glaciers caused \highlight{eustatic} and \highlight{isostatic} sea level changes.}
	\vspace{-\baselineskip}
	\begin{center}
		\includegraphics[width=0.95\textwidth]{pleistocene_caribbean}
	\end{center}
	\vskip0pt plus 1fill
	\hfill\tiny Darekk2, Wikimedia Commons.
\end{frame}

\begin{frame}{\highlight{Beringia} allowed dispersal between Palearctic and Nearctic regions.}
		\includegraphics[width=1\textwidth]{pleistocene_beringia}
\end{frame}

\begin{frame}{\highlight{Wallacea} allowed dispersal betweel Indomalaysian to Australian regions.}
	\vspace{-\baselineskip}
	\begin{center}
		\includegraphics[width=0.85\textwidth]{pleistocene_wallacia}
	\end{center}
\end{frame}

\begin{frame}{Mean Pleistocene temperatures were as much as $-$8°C below current mean temps.}
	\includegraphics[width=1\textwidth]{pleistocene_temps}
\end{frame}

\begin{frame}{Seasonal temperatures did not vary much due to \highlight{adiabatic warming} and altered climate patterns.}
	\includegraphics[width=1\textwidth]{pleistocene_adiabatic}
\end{frame}

\end{document}
