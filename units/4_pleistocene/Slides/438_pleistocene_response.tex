%!TEX TS-program = lualatex
%!TEX encoding = UTF-8 Unicode

\documentclass[t]{beamer}

%%%% HANDOUTS For online Uncomment the following four lines for handout
%\documentclass[t,handout]{beamer}  %Use this for handouts.
%\includeonlylecture{student}
%\usepackage{handoutWithNotes}
%\pgfpagesuselayout{3 on 1 with notes}[letterpaper,border shrink=5mm]

%% For students, use \lecture{student}{student}
%% For mine, use \lecture{instructor}{instructor}


% FONTS
\usepackage{fontspec}
\def\mainfont{Linux Biolinum O}
\setmainfont[Ligatures={Common,TeX}, Contextuals={NoAlternate}, BoldFont={* Bold}, ItalicFont={* Italic}, Numbers={Proportional}]{\mainfont}
%\setmonofont[Scale=MatchLowercase]{Inconsolata} 
\setsansfont[Scale=MatchLowercase]{Linux Biolinum O} 
\usepackage{microtype}

\usepackage{graphicx}
	\graphicspath{%
	{/Users/goby/Pictures/teach/438/lectures/}%
	{/Users/goby/Pictures/teach/common/}%}%
	{img/}} % set of paths to search for images

\usepackage{amsmath,amssymb}

%\usepackage{units}

\usepackage{booktabs}
\usepackage{multicol}
%	\setlength{\columnsep=1em}

%\usepackage{textcomp}
%\usepackage{setspace}
\usepackage{tikz}
	\tikzstyle{every picture}+=[remember picture,overlay]

\mode<presentation>
{
  \usetheme{Lecture}
  \setbeamercovered{invisible}
  \setbeamertemplate{items}[square]
}

\usepackage{calc}
\usepackage{hyperref}

\newcommand\HiddenWord[1]{%
	\alt<handout>{\rule{\widthof{#1}}{\fboxrule}}{#1}%
}



\begin{document}
%
\lecture{student}{student}

\begin{frame}[t]{Climates and ecosystems shifted south during expansion.}
	\vspace{-\baselineskip}
	\begin{center}
		\includegraphics[height=0.8\textheight]{pleistocene_climate_marine}\hspace*{1cm}
		\includegraphics[height=0.8\textheight]{pleistocene_climate_terrestrial}
	\end{center}
	
	\vfilll
	
	\tiny \hfill Figs.~9.12 \& 9.13 \copyright Sinauer Assoc., Inc. See textbook for legends.
\end{frame}
%
\begin{frame}{Ecosystems proportions changed.}
	
	{\centering
		\includegraphics[width=\textwidth]{pleistocene_ecosystem_proportions}\par
	}

	\vfilll
	
	\hfill \tiny Fig.~9.17 \copyright Sinauer Assoc., Inc.
\end{frame}
%
\begin{frame}{Glaciers altered prevailing wind patterns.}

	{\centering
		\includegraphics[height=0.8\textheight]{pleistocene_laurentide_climate}\par
	}

	\vfilll

	\hfill \tiny Fig.~9.7 \copyright Sinauer Assoc., Inc.
\end{frame}
%
\begin{frame}{Trees expanded northward with glacial retreat.}
	\begin{columns}[T]
		\begin{column}{0.3\textwidth}
			\includegraphics[width=1\textwidth]{pleistocene_tree_range_expansion1}
		\end{column}
		\begin{column}{0.65\textwidth}
			Prevailing winds enhanced range expansion of white spruce (below) but not lodgepole pine (left).\vspace*{0.6\baselineskip}
			\includegraphics[height=0.55\textheight]{pleistocene_tree_range_expansion2}
		\end{column}
	\end{columns}

	\vfilll

	\hfill \tiny Fig.~9.18 \copyright Sinauer Assoc., Inc.

\end{frame}
%
\begin{frame}[t]{Mammal ranges expanded with shifting ecosystems.}

	{\centering
		\includegraphics[height=0.83\textheight]{pleistocene_mammalian_range_expansion}\par
	}

	\vfilll

	\hfill \tiny Fig.~9.19 \copyright Sinauer Assoc., Inc.
\end{frame}
%
\begin{frame}[t]{Mammalian co-occurrence in communities changed.}

	{\centering
		\includegraphics[width=\textwidth]{pleistocene_mammalian_community_change}\par
	}

	\vfilll	

	\hfill \tiny Fig.~9.20 \copyright Sinauer Assoc., Inc.
\end{frame}
%
\begin{frame}{Unglaciated areas were \highlight{refugia} from glacial expansion.}

	{\centering
		\includegraphics[width=\textwidth]{pleistocene_beringia}\par
	}

	\vfilll

	\hfill \tiny Fig.~9.9\textsc{a} \copyright Sinauer Assoc., Inc.
\end{frame}
%
\begin{frame}{Many glacial refugia were available.}

	{\centering
		\includegraphics[height=0.85\textheight]{pleistocene_glacial_refugia}\par
	}

	\vfilll

	\hfill \tiny Fig.~9.25\textsc{a} \copyright Sinauer Assoc., Inc.

\end{frame}
%
\begin{frame}{Many taxa recolonized from glacial refugia.}
	\vspace{-\baselineskip}
	\begin{center}
		\includegraphics[height=0.9\textheight]{pleistocene_north_hemi_range_expansion}
	\end{center}
\end{frame}
%
\begin{frame}{New species evolved by vicariance in refugia.}
	\begin{columns}[T]
		\begin{column}{0.5\textwidth}
			\vspace{-\baselineskip}
			\begin{center}
				\includegraphics[width=0.9\textwidth]{pleistocene_ground_squirrel_arctic}\\
				Arctic Ground Squirrel\\
				\footnotesize{Beringia refugium}
			\end{center}
		\end{column}
		\begin{column}{0.5\textwidth}
			\vspace{-\baselineskip}
			\begin{center}
				\includegraphics[width=0.9\textwidth]{pleistocene_ground_squirrel_columbian}\\
				Columbian Ground Squirrel\\
				\footnotesize{Southern refugium}
			\end{center}
		\end{column}
	\end{columns}
	\vspace{\baselineskip}

	Speciation or population genetic divergence also found in moose, northern and southern red-backed voles, tundra and arctic shrews, etc.

\end{frame}
%
\begin{frame}{Where might we find refugia for freshwater fishes?}
	\vspace{-\baselineskip}
	\begin{center}
		\includegraphics[height=0.8\textheight]{pleistocene_glacial_fall_line}
	\end{center}
	\begin{tikzpicture}[overlay, line width=2pt]
		\draw [<-] (5.7,2.5) -- (4.5,2.5) ;
		\draw (4.5,2.5) node[anchor=east] {Fall Line} ;

		\draw [<-] (5.5,5.5) -- (4.3,5.5) ;
		\draw (4.3,5.5) node[anchor=east] {Glacial Maxima} ;
	\end{tikzpicture}
\end{frame}
%
{
\usebackgroundtemplate{\includegraphics[width=\paperwidth]{pleistocene_us_fish_diversity}}
\begin{frame}
\end{frame}
}
%
\begin{frame}{}

	{\centering
		\includegraphics[height=0.95\textheight]{pleistocene_modern_fish_distribution}\par
	}

\end{frame}
%

%\begin{frame}{Western lakes were refugia for freshwater fishes.}
%	\begin{columns}[T]
%		\begin{column}{0.45\textwidth}
%			\includegraphics[height=0.7\textheight]{pleistocene_western_freshwater_refugia}
%		\end{column}
%		\begin{column}{0.45\textwidth}
%			The desert southwest pupfishes represent \highlight{relict distributions} of once larger lakes.\\
%			\vspace*{\baselineskip}
%			\reflectbox{\includegraphics[width=0.8\textwidth]{cyprinodon_diabolis}}
%		\end{column}
%	\end{columns}
%\end{frame}
%
\begin{frame}{The desert southwest pupfishes have \highlight{relict distributions.}}
	\begin{columns}[T]
		\begin{column}{0.5\textwidth}
			\vspace{-\baselineskip}
			\begin{center}
				\includegraphics[height=0.8\textheight]{pleistocene_western_freshwater_refugia}
			\end{center}
		\end{column}
		\begin{column}{0.45\textwidth}
			\vspace{\baselineskip}
			\begin{center}
				\reflectbox{\includegraphics[width=0.8\textwidth]{cyprinodon_diabolis}}
			\end{center}
		\end{column}
	\end{columns}

	\vfilll
	
	\tiny Fig.~9.24 \copyright Sinauer Assoc., Inc.
\end{frame}
%
\begin{frame}{Can we test for evidence of refugia and range expansion?}
	\begin{columns}[T]
		\begin{column}{0.6\textwidth}
			\includegraphics[width=0.9\textwidth]{pleistocene_macgillivray_range} %\hspace*{1cm}
		\end{column}
		\begin{column}{0.4\textwidth}
			\includegraphics[width=0.9\textwidth]{pleistocene_macgillivray_picture}
		\end{column}
	\end{columns}
	\begin{tikzpicture}[overlay, line width=2pt]
		\draw [<-] (1.7,1.5) -- (0.3,1.5) ;
	\end{tikzpicture}
	\vspace{\baselineskip}
	
	Is the breeding resident population of MacGillivray's Warbler in Mexico a refugial population?
	
\end{frame}
%
\lecture{instructor}{instructor}
{\setbeamercolor{background canvas}{bg=black}
\begin{frame}[plain]
%	Placeholder to work out parsimony network
\end{frame}}
%
\lecture{student}{student}
\begin{frame}{\highlight{Parsimony network} supports refuge and range expansion for MacGillivray's Warbler.}
	\vspace{-\baselineskip}
	\begin{center}
		\includegraphics[width=0.9\textwidth]{pleistocene_macgillivray_network}
	\end{center}
\end{frame}
%
\begin{frame}[t]{Sea levels were much higher before the Pleistocene.}

	{\centering
		\includegraphics[height=0.45\textheight]{pleistocene_prepleisto_sealevels}\\
		\includegraphics[height=0.35\textheight]{pleistocene_sealevels}\par
	}

	\vfilll
	
	\hfill \tiny Fig.~9.10 \copyright Sinauer Assoc., Inc.
\end{frame}
%
\begin{frame}[t]{New diversity evolved in the southern hemisphere.}
	\begin{columns}[T]
		\begin{column}{0.45\textwidth}
			\vspace{-\baselineskip}
			\begin{center}
				\includegraphics[height=0.85\textheight]{pleistocene_stomatopod}
			\end{center}
		\end{column}
		\begin{column}{0.45\textwidth}
			\vspace{-\baselineskip}
			\begin{center}
				\includegraphics[height=0.85\textheight]{pleistocene_stomatopod_network.png}
			\end{center}
		\end{column}
	\end{columns}
\end{frame}
%
\begin{frame}{\highlight{Antitropical distributions} could form during Pleistocene. How?}

	{\centering
		\includegraphics[width=\textwidth]{pleistocene_pilot_whale_range}\\
		Range of Pilot Whale (green shading)\par
	}

\end{frame}
%
\lecture{instructor}{instructor}
\begin{frame}{Tropical waters were cooler during the Pleistocene.}

	{\centering
		\includegraphics[width=0.9\textwidth]{pleistocene_antitropical}\par
	}

	\vfilll

	\hfill \tiny Fig.~9.10 \copyright Sinauer Assoc., Inc.
\end{frame}
%
\end{document}
