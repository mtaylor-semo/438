%!TEX TS-program = lualatex
%!TEX encoding = UTF-8 Unicode

\documentclass[t]{beamer}

%%%% HANDOUTS For online Uncomment the following four lines for handout
%\documentclass[t,handout]{beamer}  %Use this for handouts.
%\usepackage{handoutWithNotes}
%\pgfpagesuselayout{3 on 1 with notes}[letterpaper,border shrink=5mm]
%	\setbeamercolor{background canvas}{bg=black!5}


%%% Including only some slides for students.
%%% Uncomment the following line. For the slides,
%%% use the labels shown below the command.
%\includeonlylecture{student}

%% For students, use \lecture{student}{student}
%% For mine, use \lecture{instructor}{instructor}

% FONTS
\usepackage{fontspec}
\def\mainfont{Linux Biolinum O}
\setmainfont[Ligatures={Common,TeX}, Contextuals={NoAlternate}, BoldFont={* Bold}, ItalicFont={* Italic}, Numbers={Proportional}]{\mainfont}
\setmonofont[Scale=MatchLowercase]{Inconsolatazi4} 
\setsansfont[Scale=MatchLowercase]{Linux Biolinum O} 
\usepackage{microtype}

\usepackage{graphicx}
	\graphicspath{%
	{/Users/goby/Pictures/teach/438/lectures/}%
	{/Users/goby/Pictures/teach/common/}%}%
	{img/}} % set of paths to search for images

\usepackage{amsmath,amssymb}

%\usepackage{units}

\usepackage{booktabs}
\usepackage{multicol}
%	\setlength{\columnsep=1em}

%\usepackage{textcomp}
%\usepackage{setspace}
%\usepackage{tikz}
%	\tikzstyle{every picture}+=[remember picture,overlay]

\mode<presentation>
{
  \usetheme{Lecture}
  \setbeamercovered{invisible}
  \setbeamertemplate{items}[square]
}

\usepackage{calc}
\usepackage{hyperref}

\newcommand\HiddenWord[1]{%
	\alt<handout>{\rule{\widthof{#1}}{\fboxrule}}{#1}%
}

\begin{document}
%\lecture{instructor}{instructor}
\lecture{student}{student}

\begin{frame}[t,plain]{Our goal for today is to }

	\hangpara interpret and explain \highlight{endemism,} and
	
	\hangpara relate endemism to biogeographic \highlight{regions} and \highlight{provinces.}
	
\end{frame}


{
\usebackgroundtemplate{\includegraphics[width=\paperwidth]{regions_cosmopolitan}}
\begin{frame}[b,plain]{\highlight{Cosmopolitan distributions} are uncommon.}

\end{frame}
}




\begin{frame}[t,plain]{Several geographic patterns have been recognized.}
\begin{multicols}{2}

	\noindent\includegraphics[width=0.49\textwidth]{regions_blueyes}

	\columnbreak
	
	\hangpara\highlight{Endemism,}
	
	\hangpara\highlight{relict distributions,}
	
	\hangpara\highlight{disjunct distributions,} and

	\hangpara\highlight{provincialism.}

\end{multicols}
\end{frame}

\begin{frame}[t,plain]{\highlight{Endemism} means a taxon is found only in one place.}
\begin{multicols}{2}

	\includegraphics[width=0.45\textwidth]{regions_wallflower}

	\columnbreak
	
	\hfil\includegraphics[height=0.6\textheight]{wallflower_pic}\hfill

	\vfill
		
\end{multicols}

	\vskip0pt plus 1fill
	
	\tiny\hfill San Francisco Wallflower, Eric in SF, Wikimedia Commons.

\end{frame}

\begin{frame}[t,plain]{Endemism is hierarchical within taxa. Why?}

	\begin{center}
		\includegraphics[height=0.82\textheight]{regions_krats}
	\end{center}

	\vskip0pt plus 1fill
	
	\tiny\hfill Tipton Kangaroo Rat, USFWS, Wikimedia Commons.
\end{frame}


\begin{frame}[t,plain]{Islands have high levels of endemism.}
\begin{multicols}{2}

	\includegraphics[width=0.46\textwidth]{regions_silversword}

	\columnbreak

	\vspace*{2\baselineskip}	

	\noindent Over 50 species of silverswords are endemic to the Hawaiian islands.
	
\end{multicols}

	\vskip0pt plus 1fill
	
	\tiny\hfill\textit{Argyroxiphium sandwicense}, Wikimedia Commons.

\end{frame}

{
\usebackgroundtemplate{\includegraphics[width=\paperwidth]{regions_polyodon}}
\begin{frame}[b,plain]{\highlight{Relict distributions} can arise from extinction.}

\tiny\hfill\textit{Polyodon spatula}, The Earth Society, Wikimedia Commons.
\end{frame}
}

{
\usebackgroundtemplate{\includegraphics[width=\paperwidth]{regions_bristlecone}}
\begin{frame}[b,plain]{\highlight{Relict distributions} can arise from habitat loss.}

\tiny\hfill Bristlecone Pine, dcrjst, Wikimedia Commons.
\end{frame}
}

{
\usebackgroundtemplate{\includegraphics[width=\paperwidth]{regions_terrestrial}}
\begin{frame}[b,plain]{Terrestrial \highlight{regions} reflect endemism across many taxa.}

\end{frame}
}

\begin{frame}[t,plain]{Marine \highlight{regions} also reflect taxonomic endemism.}

%	\begin{center}
		\includegraphics[width=\textwidth]{regions_marine}
%	\end{center}

\end{frame}

{
\usebackgroundtemplate{\includegraphics[width=\paperwidth]{regions_amphitropical}}
\begin{frame}[b,plain]{\highlight{Amphitropical} distributions reflect temperature as barriers.}

\tiny\hfill Long-finned Pilot Whale, Barney Moss, Wikimedia Commons.
\end{frame}
}

{
\usebackgroundtemplate{\includegraphics[width=\paperwidth]{regions_provinces}}
\begin{frame}[b,plain]{\highlight{Provinces} are often biodiversity hotspots and may be centers of origin.}

\end{frame}
}

{
\usebackgroundtemplate{\includegraphics[width=\paperwidth]{regions_north_american_provinces}}
\begin{frame}[b,plain]

\end{frame}
}

{
\usebackgroundtemplate{\includegraphics[width=\paperwidth]{regions_freshwater_provinces}}
\begin{frame}[b,plain]

\end{frame}
}

\begin{frame}[t,plain]{What explains high regional and provincial endemism?}
\begin{multicols}{2}

	\includegraphics[width=0.46\textwidth]{regions_madagascar}

	\columnbreak

	\noindent Madagascar has\\ \quad eight plant families, \\ \quad four bird families, and,\\ \quad five primate families.
	
	\vspace{\baselineskip}
	
	\noindent The Seychelles has one amphibian family (Sooglossidae).
	
	\vspace{\baselineskip}
	
\end{multicols}

\end{frame}


{
\usebackgroundtemplate{\includegraphics[width=\paperwidth]{regions_terrestrial}}
\begin{frame}[b,plain]{Regional endemism separated by sharp or broad \highlight{“lines.”}}

\end{frame}
}



{
\usebackgroundtemplate{\includegraphics[width=\paperwidth]{regions_wallaces_line}}
\begin{frame}[b,plain]{\highlight{Wallace’s line} marks sharp faunal changes.}

\end{frame}
}

\begin{frame}[b,plain]{\highlight{Bond’s line} separates continental and insular faunas.}
	\begin{center}
		\includegraphics[height=0.9\textheight]{regions_bonds_line}
	\end{center}
\end{frame}

{
\usebackgroundtemplate{\includegraphics[width=\paperwidth]{regions_broad_lines}}
\begin{frame}[b,plain]

\end{frame}
}

\begin{frame}[t,plain]

	\hangpara Read the online notes and assigned reading for plate tectonics.
	
	\hangpara I will not cover the material in lecture but it is fair game for exams.
	
	\hangpara Next lecture will require tectonic understanding.
	
\end{frame}




\end{document}
