%!TEX TS-program = lualatex
%!TEX encoding = UTF-8 Unicode

\documentclass[11pt, addpoints]{exam}

%\printanswers


%\usepackage{graphicx}
%	\graphicspath{{/Users/goby/Pictures/teach/300/}
%	{img/}} % set of paths to search for images

\usepackage{geometry}
\geometry{letterpaper, bottom=1in}                   
%\geometry{landscape}                % Activate for for rotated page geometry
\usepackage[parfill]{parskip}    % Activate to begin paragraphs with an empty line rather than an indent
%\usepackage{amssymb, amsmath}
%\usepackage{mathtools}
%	\everymath{\displaystyle}

\usepackage{fontspec}
\setmainfont[Ligatures={Common,TeX}, BoldFont={* Bold}, ItalicFont={* Italic}, BoldItalicFont={* BoldItalic}, Numbers={Proportional}]{Linux Libertine O}
\setsansfont[Scale=MatchLowercase,Ligatures=TeX]{Linux Biolinum O}
%\setmonofont[Scale=MatchLowercase]{Inconsolata}
\usepackage{microtype}

\usepackage{unicode-math}
\setmathfont[Scale=MatchLowercase]{Asana Math}

\newfontfamily{\tablenumbers}[Numbers={Monospaced}]{Linux Libertine O}
\newfontfamily{\libertinedisplay}{Linux Libertine Display O}

\usepackage{hanging}

%\usepackage{booktabs}
%\usepackage{tabularx}
%\usepackage{longtable}
%\usepackage{siunitx}
\usepackage{array}
\newcolumntype{L}[1]{>{\raggedright\let\newline\\\arraybackslash\hspace{0pt}}p{#1}}
\newcolumntype{C}[1]{>{\centering\let\newline\\\arraybackslash\hspace{0pt}}p{#1}}
\newcolumntype{R}[1]{>{\raggedleft\let\newline\\\arraybackslash\hspace{0pt}}p{#1}}

\usepackage{enumitem}

\usepackage{titling}
\setlength{\droptitle}{-60pt}
\posttitle{\par\end{center}}
\predate{}\postdate{}

\newlength{\myLength}
\setlength{\myLength}{\parindent}

\renewcommand{\solutiontitle}{\noindent}
\unframedsolutions
\SolutionEmphasis{\bfseries}

\pagestyle{headandfoot}
\firstpageheader{BI 438: Biogeography}{}{\ifprintanswers\textbf{KEY\ \numpoints~points}\else Conservation Biogeography\fi}
\runningheader{}{}{\footnotesize{pg. \thepage}}
\footer{}{}{}
\runningheadrule

\begin{document}

Read carefully the papers listed below and then type your answers to the following questions.
Type the question number and then your answer. You do not need to retype the question. Hand-written
assignments will not be accepted. You must have your completed answers
with you in class or your assignment will be considered late. Assignments e-mailed to me after the \emph{start} of class will be
considered late. \emph{Interpret} what you read; do not copy answers directly from the text (that would
be plagiarism). Be prepared to discuss your answers in small groups and as a class. Failure
to discuss these questions may result in a pop quiz with a point value
equal to this assignment.

\textbf{Failure to attend} \textbf{class} \textbf{on the due date} will
result in an automatic 50\% deduction.

\begin{hangparas}{1.5em}{1}
Spatz, D.\,R.~et al.~2014.~The biogeography of globally threatened seabirds and 
island conservation opportunities. Conservation Biogeography 28: 1282–1290.

Jones, H.\,P.~et al.~2016.~Invasive mammal eradication on islands results in 
substantial conservation gains. Proc.~Natl.~Acad.~Sci. 113: 4033–4038. 
\end{hangparas}

The papers will be referred to as Spatz and Jones, respectively. These papers did not lend themselves easily to questions with relatively concrete answers so most questions are “thought” questions. Show me that you put some thought into your answers and you will be fine, even if you and I come up with different answers.

\begin{questions}
\question[5]
What are ecosystem services? Why, in general, should we care about them? (Cite external sources you use to answer this question.)

\begin{solution}
	Ecosystem services are the benefits that humans receive from ecosystem functions.
\end{solution}

\question[5]
Spatz did not analyze subspecies. They instead included them within the broader species. From a biogeographic perspective, do you think this was a valid way to handle subspecies? Explain why or why not.

\begin{solution}
	I have concerns. Subspecies, although somewhat arbitrary, can indicate genetic differences among the groups. A genetically unique subspecies may be threatened but the threat could go unrecognized, even if the species is widespread on several islands. This could risk the evolutionary potential and genetic flexibility of the population.
\end{solution}

\question[5] \label{ques:subspecies}
Jones suggests that reintroduction of threatened species to islands without invasive species may be feasible. Do you agree?  What would you have to consider? (Hint: think about the previous question.)

\begin{solution}
	I would be concerned. Selection for conditions on one island would increase favorable alleles in that island population. Individuals from that population introduced to a different island might not be able to survive well if the conditions are different.
\end{solution}


\question[5]
Spatz states that islands with \textgreater1000 people were not feasible for total eradication of invasive species, even though 13\% of threatened seabird species had their entire population on these islands. Yet, Jones suggests that strategies can be developed for populated islands to aid recovery of threatened species. Are these two ideas in conflict?  Why or why not. (For fun, compare the author lists. Most of the authors of the Spatz paper are on the Jones paper, and Croll is the senior author of both papers. Are they contradicting themselves?)

%Why do you think this is not feasible? For the seed of one idea, what were the most comonly eradicated invasive species, according to Jones? How would this tie back to the number of people on the island?

\begin{solution}
	Feasibility does not mean impossible but usually means not economically effective or otherwise difficult. For example, may be difficult to maintain a mammal-free area in an inhabitated area due to roaming pets, etc. Reintroduction may work but see subspecies question above (question~\ref{ques:subspecies}). Spatz's comment is page 6 (of my copy). Jones' comment is page 4035. Both mention total eradication not possible. Jones is suggesting partial eradication in a subset of the island.
%	People bring animals and pets (goats, cats) that can be hard to eradicate.
\end{solution}


\question[5]
Jones mentions the use of poisoned bait to eradicate invasive species. Suppose you were tasked with using baits to eradicate invasive species on a human-populated island. What are the potential risks?

\begin{solution}
	Jones et al.~discussed how baits might cause a temporarily reduction of seabird populations but that they ultimately rebounded. However, poisoned baits could be consumed by pets, livestock, etc. which would upset the local population.
	
	Other ideas?
\end{solution}


\question[5]
Compare Figure 1 of Spatz to the upper panel of Figure 1 of Jones. Can you identify two regions that have a high concentration of islands with extirpated populations that have neither demonstrated or predicted benefits from invasive eradication? (That is, can you identify two areas that have a concentration of black dots but do not have a corresponding concentration of green, yellow, or blue dots?) Why do you think this is?

\begin{solution}
	The areas I am thinking of is around the East China Sea/Philippine Sea and the Persian Gulf. One possibility is that the seabird threats on islands in those ares not threatened by mammals. I find this unlikely, especially for the heavily populated China/Philippine seas. I suspect those islands would have a strong mammalian presence. 
	
	Another possibility is that those areas have not been tested or modeled. This may be most likely.
	
	Other ideas?
\end{solution}


\end{questions}

\end{document}  