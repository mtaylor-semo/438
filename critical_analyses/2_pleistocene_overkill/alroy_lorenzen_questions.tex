%!TEX TS-program = lualatex
%!TEX encoding = UTF-8 Unicode

\documentclass[11pt]{exam}
%\usepackage{graphicx}
%	\graphicspath{{/Users/goby/Pictures/teach/300/}
%	{img/}} % set of paths to search for images

\usepackage{geometry}
\geometry{letterpaper, bottom=1in}                   
%\geometry{landscape}                % Activate for for rotated page geometry
\usepackage[parfill]{parskip}    % Activate to begin paragraphs with an empty line rather than an indent
%\usepackage{amssymb, amsmath}
%\usepackage{mathtools}
%	\everymath{\displaystyle}

\usepackage{fontspec}
\setmainfont[Ligatures={Common,TeX}, BoldFont={* Bold}, ItalicFont={* Italic}, BoldItalicFont={* BoldItalic}, Numbers={Proportional}]{Linux Libertine O}
\setsansfont[Scale=MatchLowercase,Ligatures=TeX]{Linux Biolinum O}
%\setmonofont[Scale=MatchLowercase]{Inconsolata}
\usepackage{microtype}

\usepackage{unicode-math}
\setmathfont[Scale=MatchLowercase]{Asana Math}

\newfontfamily{\tablenumbers}[Numbers={Monospaced}]{Linux Libertine O}
\newfontfamily{\libertinedisplay}{Linux Libertine Display O}

\usepackage{hanging}

%\usepackage{booktabs}
%\usepackage{tabularx}
%\usepackage{longtable}
%\usepackage{siunitx}
\usepackage{array}
\newcolumntype{L}[1]{>{\raggedright\let\newline\\\arraybackslash\hspace{0pt}}p{#1}}
\newcolumntype{C}[1]{>{\centering\let\newline\\\arraybackslash\hspace{0pt}}p{#1}}
\newcolumntype{R}[1]{>{\raggedleft\let\newline\\\arraybackslash\hspace{0pt}}p{#1}}

\usepackage{enumitem}

\usepackage{titling}
\setlength{\droptitle}{-60pt}
\posttitle{\par\end{center}}
\predate{}\postdate{}

\newlength{\myLength}
\setlength{\myLength}{\parindent}

\renewcommand{\solutiontitle}{\noindent}
\unframedsolutions
\SolutionEmphasis{\bfseries}

\pagestyle{headandfoot}
\firstpageheader{BI 438: Biogeography}{}{\ifprintanswers\textbf{KEY}\else Pleistocene Extinction\fi}
\runningheader{}{}{\footnotesize{pg. \thepage}}
\footer{}{}{}
\runningheadrule

\printanswers

\begin{document}

Read carefully the papers listed below and then type your answers to the following questions.
Type the question number and then your answer. You do not need to retype the question. Hand-written
assignments will not be accepted. You must have your completed answers
with you in class or your assignment will be considered late. Assignments e-mailed to me after the \emph{start} of class will be
considered late. \emph{Interpret} what you read; do not copy answers directly from the text (that would
be plagiarism). Be prepared to discuss your answers in small groups and as a class. Failure
to discuss these questions may result in a pop quiz with a point value
equal to this assignment.

\textbf{Failure to attend} \textbf{class} \textbf{on the due date} will
result in an automatic 50\% deduction.

\begin{hangparas}{1.5em}{1}
Alroy, J. 2001. A multispecies overkill simulation of the
End-Pleistocene megafaunal mass extinction. Science 292: 1893-1896.

Lorenzen, E.D. et al. 2011. Species-specific responses of Late
Quaternary megafauna to climate and humans. Nature 479: 359-364. 
\end{hangparas}


\begin{questions}

\question[5]
Does Alroy 2001 conclusively eliminate End-Pleistocene climate change
as the cause of the megafaunal mass extinction? Briefly, but clearly and
accurately, state how he does or why he does not.

\ifprintanswers
No, Alroy does not eliminate climate change as a possibility. His
computer model is designed only to test whether the observed extinctions
are consistent with particular assumptions of human hunting abilities,
human population increase and spread, human meat consumptions (large
animals only), etc. He in no way tests for climate change as a
possibility. He mentions on page 1896 (top of second column) that
climate change may be a contributing factor that exacerbates the loss of
megafauna initiated through human predation.
\else\vspace*{\stretch{1}}
\fi

\question[5]
Alroy states that human hunting of large herbivores may have
significantly disrupted ecosystem function. Describe the specific
evidence he uses, including specific numbers from the relevant column(s)
of Table 1, and then provide a biologically sound description of how the
evidence he cites might lead to ecosystem collapse.
  
\ifprintanswers
On page 1896, Alroy states that ``\dots these figures imply a major
disruption of ecosystem function at the continental scale\dots''. Trial
8 is the ``best-fit scenario'' for his model (pg. 1893, 3rd column, 3rd
paragraph). In Table 1, the column labeled \textbf{Energy Use} refers to
the amount of primary producers consumed by herbivores (primary
consumers), relative to the maximum (I presume 1.00 is maximum). Trial 7
shows an energy use of 0.762, which is approximately \textbf{25\% less}
than 1.00. Trial 8, the best-fit, is about 2/3 of maximum. Trial 9 has
an energy use of 0.515, which is approximately 1/2 of the relative
maximum. These numbers match his statement of human predation depressing
herbivory by 1/4 to 1/2.

Types of ecosystem function disruption:

\begin{itemize}
\itemsep1pt\parskip0pt\parsep0pt
\item
  Vegetational structure: Herbivores will eat certain types of
  producers. If those producers are not consumed due to lack of
  herbivores, then the community composition of plants will be
  different.
\item
  Vegetational carbon sinks: Plants are carbon sinks in the sense that
  they take up and store carbon atmospheric carbon. Different species of
  plants take up and store carbon at different rates. By changing the
  community structure, you change the amount of carbon that will (or
  will not) be stored.
\item
  Watershed dynamics: watersheds are influenced by the surrounding
  vegetation, from erosion to canopy cover.
\item
  Insect and small vertebrate dynamics: the vegetative community is
  often what we think of as habitat. By changing the community
  composition, you will change the ``habitat.'' Changing the habitat
  will change the types of organisms present.
\end{itemize}
\else\vspace*{\stretch{1}}
\fi

\question[5]
Alroy lists several nonhuman ecological factors that could complicate
his model, but he claims most of them would further increase the
extinction rates predicted by his model. Pick any one of these factors
and then write a biologically sound explanation of how it would (or
would not) further increase the extinction rate of some megafauna.

\ifprintanswers
Bottom of page 1896, first column, Alroy lists the following factors:

\begin{itemize}
\itemsep1pt\parskip0pt\parsep0pt
\item
  Demographic stochasticity
\item
  Genetic ``meltdown'' at small population sizes (loss of genetic
  variability; bottleneck; drift)
\item
  Short-term environmental stochasticity
\item
  Variation among geographic regions on the availability of secondary
  food sources
\item
  Ecological feedbacks induced by populations of nonhuman predators
\item
  Habitat change induced by extinction of keystone herbivore species,
  and
\item
  Direct biological effects of long-term climate change.
\end{itemize}
\else\vspace*{\stretch{1}}
\fi


\question[5]
Lorenzen et al. based their study on six megafaunal herbivores, two
of which are now extinct. Do you think these six species are a
sufficient representation of the entire Pleistocene megafauna to
distinguish between human overkill and climate change as the primary
cause of the End-Pleistocene megafauna extinction? Explain why or why
not. (Hint: what is the primary distribution of these six species ca.
30–40,000 years ago?)

\ifprintanswers
Open-ended question: The distributions shown in Figure 1 show the
species had a predominately Palearctic distribution about 30--40
thousand years ago. Thus, it may underrepresent taxa with a North
American distribution. On the other hand, the ranges may accurately
reflect the ranges for most Pleistocene megafauna. Many, many more than
6 species went extinct during the Pleistocene and not just in the
northern hemisphere (e.g., Australia).
\else
	\vspace*{\stretch{1}}
\fi

\question[5]
Clearly explain how Lorenzen et al.'s conclusions differ from that of
Alroy. Do they clearly demonstrate that a mix of human overkill and
climate change is the best explanation for the megafauna extinction?
Justify your answer. (I'm not asking you whether you think a mix of the
two is most likely. I'm asking whether you think the authors clearly
\emph{demonstrate} that this is the most likely explanation.)

\ifprintanswers
Alroy's model shows that the megafaunal extinction could be driven
entirely through human predation, without including climate change as at
least a contributing factor. In contrast, Lorenzen et al. shows that extinction
of at least some of the megafauna is entirely consistent with climate change. 

\begin{itemize}
\item Musk Ox and Wooly Rhino: Climate 

\item Horse and Steppe Bison: Climate and overkill

\item Woolly Mammoth: Inconclusive

\item Reindeer: relatively unaffected
\end{itemize}

For example, that the decrease of the bison is entirely consistent with climate change
due to the gradual decrease in range size and gradual loss of genetic
diversity (see page 5, first paragraph). They also argue that decrease of the
bison may have been exacerbated by competition with moose and elk. After
16 kya, bison appeared at many archaeological sites, suggesting sudden
and increased usage by early humans.

In the end, they argue that one hypothesis (climate vs overkill) is not sufficient to 
explain all of the Pleistocene mass extinctions. They argue that different species
responded differently to the changing climate and overkill pressure.
\else\vspace*{\stretch{1}}
\fi

\question[5]
Figures 2 and 3 are central to Lorenzen et al.'s conclusions.
Carefully interpret and compare these two figures. Describe how each
figure is used to support the arguments made by the authors. How do
these two figures together lead the authors to their conclusions? Use
the North American Bison in your description to demonstrate your
understanding of the figures. 

\ifprintanswers
Figure 2 suggests that Bison occupied a large geographic area (ca.
10,000,000 km\^{}2) around 30 kya, then reduced in size by about 1/5
(0.17 in Fig 3), equivalent to a log km\textsuperscript{2} of about 6.2
(Fig 2), reducing N\textsubscript{e} to less than 10k (Fig. 3) about 10
kya. The decrease in population size coincides with a loss of genetic
variation, as shown in the skyline plot on the lower panel of of Fig. 2.
Genetic variation was highest when the geographic range size was
highest. Genetic variation decreased gradually as population
size/geographic range size decreased.
\else\vspace*{\stretch{1}}
\fi

\end{questions}

\end{document}  