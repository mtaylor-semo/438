%!TEX TS-program = lualatex
%!TEX encoding = UTF-8 Unicode

\documentclass[letterpaper]{tufte-handout}

%\geometry{showframe} % display margins for debugging page layout

\usepackage{fontspec}
\def\mainfont{Linux Libertine O}
\setmainfont[Ligatures={Common,TeX}, Contextuals={NoAlternate}, BoldFont={* Bold}, ItalicFont={* Italic}, Numbers={OldStyle}]{\mainfont}
\setsansfont[Scale=MatchLowercase, Numbers={OldStyle}]{Linux Biolinum O} 
\setmonofont{Linux Libertine O}
\usepackage{microtype}

\usepackage{graphicx} % allow embedded images
  \setkeys{Gin}{width=\linewidth,totalheight=\textheight,keepaspectratio}
  \graphicspath{{/Users/goby/Documents/teach/438/lectures/}} % set of paths to search for images
\usepackage{amsmath}  % extended mathematics
\usepackage{booktabs} % book-quality tables
\usepackage{units}    % non-stacked fractions and better unit spacing
\usepackage{multicol} % multiple column layout facilities
%\usepackage{fancyvrb} % extended verbatim environments
%  \fvset{fontsize=\normalsize}% default font size for fancy-verbatim environments

\usepackage{enumitem}
\usepackage{mhchem}

\makeatletter
% Paragraph indentation and separation for normal text
\renewcommand{\@tufte@reset@par}{%
  \setlength{\RaggedRightParindent}{1.0pc}%
  \setlength{\JustifyingParindent}{1.0pc}%
  \setlength{\parindent}{1pc}%
  \setlength{\parskip}{0pt}%
}
\@tufte@reset@par

% Paragraph indentation and separation for marginal text
\renewcommand{\@tufte@margin@par}{%
  \setlength{\RaggedRightParindent}{0pt}%
  \setlength{\JustifyingParindent}{0.5pc}%
  \setlength{\parindent}{0.5pc}%
  \setlength{\parskip}{0pt}%
}
\makeatother

% Set up the spacing using fontspec features
   \renewcommand\allcapsspacing[1]{{\addfontfeatures{LetterSpace=15}#1}}
   \renewcommand\smallcapsspacing[1]{{\addfontfeatures{LetterSpace=10}#1}}


\title{{\scshape bi} 438/638 Exam 1 Study Guide}

\date{} % without \date command, current date is supplied

\begin{document}

\maketitle	% this prints the handout title, author, and date

%\printclassoptions
\section*{Vocabulary}

\vspace{-1\baselineskip}
\begin{multicols}{2}
geographic template \\
climate \\
range shadow \\
geographic range  \\
restricted distribution \\
cosmopolitan distribution \\
niche \\
fundamental geographic range \\
realized geographic range \\
habitat quality \\
metapopulations \\
source population \\
sink population \\
allopatric distribution \\
Rapoport's rule \\
latitudinal diversity gradient \\
trophic niche conservatism \\
out-of-the-tropics model \\
\end{multicols}

\section*{Concepts}

\begin{enumerate}

\item   Do most species have large range sizes or small range sizes? For
  species with large range sizes, do they tend to be generalists or
  specialists? Argue for both cases.
  
\item   How does an organism's niche relate to it's geographic range?\marginnote{Geographic distribution is synonymous with geographic range.} Does the
  niche perfectly define the range? Why or why not? What factors can
  also influence the size of an organism's range?
  
\item  List some of factors that you think would determine the range boundary
  of a species. Briefly explain why you think each would influence the
  boundary?
  
\item   You come across a perfectly good patch of habitat that could be
  occupied by your favorite species. After exhaustive searching, you
  ascertain that your species is not present. What reasons could you
  give to explain its absence?
  
\item  What is meant by metapopulation dynamics? Are metapopulations static
  systems?
  
\item  Provide examples of how disturbance can positively and negatively
  affect a species' geographic range. Do \emph{not} use the examples
  given in class.
  
\item  What types of species interactions may affect the distribution of a
  species? Describe how species interactions may affect the range of one
  or both of the interaction species. Provide examples, preferably those
  not used in class.
  
\item  What are source and sink populations? How does this relate to habitat
  quality, range edges, the niche, and other facets of the geographic
  range. How might you relate source and sink populations to
  metapopulation dynamics?
  
\item  How do historical factors impact a species's range? Can you provide
  examples in addition to those used in class?

\item  Explain Rapoport's Rule, for both species richness and range size. 
Does it apply universally to all higher taxonomic groups (assume family 
level and higher)? Does it apply universally
to all continental land masses? Explain.

\item  We considered the relationship between range size (area) and body size
  in birds and mammals. We identified three different constraints that,
  at least in part, explain the overall relationship. You should be able
  to identify the approximate locations of the three constraints on a
  figure. You should also be able to explain each of the constraints:
  What is the constraint and how does it affect the relationship between
  range size and body size?
  
\item   Explain why North American ranges of mammals and birds tend to be
  elongated in a north to south direction, while Eurasian distributions
  of these organisms are elongated in an east-west direction. Consider
  overall available area and potential biogeographic barriers.
  
\item  North American birds and mammals have various range sizes, from very
  small to very large. In contrast, Eurasian birds and mammals tend to
  have only very large range sizes. Explain the hypothesis that has been
  proposed to account for this discrepancy.  
  
\item Compare and contrast the tropical niche conservatism model and the
out-of-the-tropics model as models that explain the latitudinal diversity gradient.

\item Explain what cradle, museum, and immigration pump refer to in the 
out-of-the-tropics model.

\end{enumerate}

\end{document}